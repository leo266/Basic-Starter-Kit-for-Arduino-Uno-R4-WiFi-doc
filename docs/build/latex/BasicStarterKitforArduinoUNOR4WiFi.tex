%% Generated by Sphinx.
\def\sphinxdocclass{report}
\documentclass[a4paper,11pt,english]{sphinxmanual}
\ifdefined\pdfpxdimen
   \let\sphinxpxdimen\pdfpxdimen\else\newdimen\sphinxpxdimen
\fi \sphinxpxdimen=.75bp\relax
\ifdefined\pdfimageresolution
    \pdfimageresolution= \numexpr \dimexpr1in\relax/\sphinxpxdimen\relax
\fi
%% let collapsible pdf bookmarks panel have high depth per default
\PassOptionsToPackage{bookmarksdepth=5}{hyperref}

\PassOptionsToPackage{booktabs}{sphinx}
\PassOptionsToPackage{colorrows}{sphinx}

\PassOptionsToPackage{warn}{textcomp}
\usepackage[utf8]{inputenc}
\ifdefined\DeclareUnicodeCharacter
% support both utf8 and utf8x syntaxes
  \ifdefined\DeclareUnicodeCharacterAsOptional
    \def\sphinxDUC#1{\DeclareUnicodeCharacter{"#1}}
  \else
    \let\sphinxDUC\DeclareUnicodeCharacter
  \fi
  \sphinxDUC{00A0}{\nobreakspace}
  \sphinxDUC{2500}{\sphinxunichar{2500}}
  \sphinxDUC{2502}{\sphinxunichar{2502}}
  \sphinxDUC{2514}{\sphinxunichar{2514}}
  \sphinxDUC{251C}{\sphinxunichar{251C}}
  \sphinxDUC{2572}{\textbackslash}
\fi
\usepackage{cmap}
\usepackage[T1]{fontenc}
\usepackage{amsmath,amssymb,amstext}
\usepackage{babel}



\usepackage{tgtermes}
\usepackage{tgheros}
\renewcommand{\ttdefault}{txtt}



\usepackage[Bjarne]{fncychap}
\usepackage{sphinx}

\fvset{fontsize=auto}
\usepackage{geometry}

        \usepackage{graphicx}
        \usepackage{booktabs}
        \usepackage{multirow}
        \usepackage{xcolor}
    

% Include hyperref last.
\usepackage{hyperref}
% Fix anchor placement for figures with captions.
\usepackage{hypcap}% it must be loaded after hyperref.
% Set up styles of URL: it should be placed after hyperref.
\urlstyle{same}


\usepackage{sphinxmessages}
\setcounter{tocdepth}{1}



\title{Basic Starter Kit for Arduino UNO R4 WiFi}
\date{Sep 24, 2025}
\release{}
\author{Bot}
\newcommand{\sphinxlogo}{\vbox{}}
\renewcommand{\releasename}{}
\makeindex
\begin{document}

\ifdefined\shorthandoff
  \ifnum\catcode`\=\string=\active\shorthandoff{=}\fi
  \ifnum\catcode`\"=\active\shorthandoff{"}\fi
\fi

\pagestyle{empty}

        \maketitle
        \tableofcontents
    
\pagestyle{plain}
\sphinxtableofcontents
\pagestyle{normal}
\phantomsection\label{\detokenize{index::doc}}


\sphinxstepscope


\chapter{Get Started with Arduino}
\label{\detokenize{Get_Started_with_Arduino/Get_Started_with_Arduino:get-started-with-arduino}}\label{\detokenize{Get_Started_with_Arduino/Get_Started_with_Arduino::doc}}
\sphinxAtStartPar
Follow these steps to learn how to use Arduino from zero!

\sphinxstepscope


\section{Install Arduino IDE}
\label{\detokenize{Get_Started_with_Arduino/Install_Arduino_IDE:install-arduino-ide}}\label{\detokenize{Get_Started_with_Arduino/Install_Arduino_IDE:install-arduino}}\label{\detokenize{Get_Started_with_Arduino/Install_Arduino_IDE::doc}}
\sphinxAtStartPar
The Arduino IDE, known as Arduino Integrated Development Environment, provides all the software support needed to complete an Arduino project. It is a programming software specifically designed for Arduino, provided by the Arduino team, that allows us to write programs and upload them to the Arduino board.

\sphinxAtStartPar
The Arduino IDE 2.0 is an open\sphinxhyphen{}source project. It is a big step from its sturdy predecessor, Arduino IDE 1.x, and comes with revamped UI, improved board \& library manager, debugger, autocomplete feature and much more.

\sphinxAtStartPar
In this tutorial, we will show how to download and install the Arduino IDE 2.0 on your Windows, Mac, or Linux computer.


\subsection{Requirements}
\label{\detokenize{Get_Started_with_Arduino/Install_Arduino_IDE:requirements}}\begin{itemize}
\item {} 
\sphinxAtStartPar
Windows \sphinxhyphen{} Win 10 and newer, 64 bits

\item {} 
\sphinxAtStartPar
Linux \sphinxhyphen{} 64 bits

\item {} 
\sphinxAtStartPar
Mac OS Intel \sphinxhyphen{} Version 10.14: “Mojave” or newer, 64 bits

\item {} 
\sphinxAtStartPar
Mac OS Apple Silicon \sphinxhyphen{} Version 11: “Big Sur” or newer, 64 bits

\end{itemize}


\subsection{Download the Arduino IDE 2.x.x}
\label{\detokenize{Get_Started_with_Arduino/Install_Arduino_IDE:download-the-arduino-ide-2-x-x}}\begin{enumerate}
\sphinxsetlistlabels{\arabic}{enumi}{enumii}{}{.}%
\item {} 
\sphinxAtStartPar
Visit \sphinxhref{https://www.arduino.cc/en/software}{Download Arduino IDE} page.

\item {} 
\sphinxAtStartPar
Download the IDE for your OS version.

\noindent\sphinxincludegraphics{{Install_Arduino_IDE_1}.png}

\end{enumerate}

\begin{sphinxadmonition}{note}{Note:}
\sphinxAtStartPar
Note: Uploading code to the Arduino UNO R4 requires Arduino IDE version 2.2 or higher. If your version is older, please upgrade to the latest version.
\end{sphinxadmonition}


\subsection{Installation}
\label{\detokenize{Get_Started_with_Arduino/Install_Arduino_IDE:installation}}

\subsubsection{Windows}
\label{\detokenize{Get_Started_with_Arduino/Install_Arduino_IDE:windows}}\begin{enumerate}
\sphinxsetlistlabels{\arabic}{enumi}{enumii}{}{.}%
\item {} 
\sphinxAtStartPar
Double click the \sphinxcode{\sphinxupquote{arduino\sphinxhyphen{}ide\_xxxx.exe}} file to run the downloaded file.

\item {} 
\sphinxAtStartPar
Read the License Agreement and agree it.

\noindent\sphinxincludegraphics{{Install_Arduino_IDE_2}.png}

\item {} 
\sphinxAtStartPar
Choose installation options.

\noindent\sphinxincludegraphics{{Install_Arduino_IDE_3}.png}

\item {} 
\sphinxAtStartPar
Choose install location. It is recommended that the software be installed on a drive other than the system drive.

\noindent\sphinxincludegraphics{{Install_Arduino_IDE_4}.png}

\item {} 
\sphinxAtStartPar
Then Finish.

\noindent\sphinxincludegraphics{{Install_Arduino_IDE_5}.png}

\end{enumerate}


\subsubsection{macOS}
\label{\detokenize{Get_Started_with_Arduino/Install_Arduino_IDE:macos}}
\sphinxAtStartPar
Double click on the downloaded \sphinxcode{\sphinxupquote{arduino\_ide\_xxxx.dmg}} file and follow the instructions to copy the \sphinxstylestrong{Arduino IDE.app} to the \sphinxstylestrong{Applications} folder, you will see the Arduino IDE installed successfully after a few seconds.

\noindent\sphinxincludegraphics[width=800\sphinxpxdimen]{{Install_Arduino_IDE_6}.png}


\subsubsection{Linux}
\label{\detokenize{Get_Started_with_Arduino/Install_Arduino_IDE:linux}}
\sphinxAtStartPar
For the tutorial on installing the Arduino IDE 2.0 on a Linux system, please refer \sphinxhref{https://docs.arduino.cc/software/ide-v2/tutorials/getting-started/ide-v2-downloading-and-installing\#linux}{Linux\sphinxhyphen{}Install Arduino IDE}


\subsection{Open the IDE}
\label{\detokenize{Get_Started_with_Arduino/Install_Arduino_IDE:open-the-ide}}\begin{enumerate}
\sphinxsetlistlabels{\arabic}{enumi}{enumii}{}{.}%
\item {} 
\sphinxAtStartPar
When you first open Arduino IDE 2.0, it automatically installs the Arduino AVR Boards, built\sphinxhyphen{}in libraries, and other required files.

\noindent\sphinxincludegraphics{{Install_Arduino_IDE_7}.png}

\item {} 
\sphinxAtStartPar
In addition, your firewall or security center may pop up a few times asking you if you want to install some device driver. Please install all of them.

\noindent\sphinxincludegraphics{{Install_Arduino_IDE_8}.png}

\item {} 
\sphinxAtStartPar
Now your Arduino IDE is ready!

\begin{sphinxadmonition}{note}{Note:}
\sphinxAtStartPar
In the event that some installations didn’t work due to network issues or other reasons, you can reopen the Arduino IDE and it will finish the rest of the installation. The Output window will not automatically open after all installations are complete unless you click Verify or Upload.
\end{sphinxadmonition}

\end{enumerate}

\sphinxstepscope


\section{Creating, Opening, and Saving Sketches}
\label{\detokenize{Get_Started_with_Arduino/Creating_Opening_Saving_Sketches:creating-opening-and-saving-sketches}}\label{\detokenize{Get_Started_with_Arduino/Creating_Opening_Saving_Sketches::doc}}

\subsection{Panel of Arduino IDE}
\label{\detokenize{Get_Started_with_Arduino/Creating_Opening_Saving_Sketches:panel-of-arduino-ide}}
\noindent\sphinxincludegraphics{{Creat_Skerch0}.png}
\begin{enumerate}
\sphinxsetlistlabels{\arabic}{enumi}{enumii}{}{.}%
\item {} 
\sphinxAtStartPar
\sphinxstylestrong{Verify}: Compile your code. Any syntax problem will be prompted with errors.

\item {} 
\sphinxAtStartPar
\sphinxstylestrong{Upload}: Upload the code to your board. When you click the button, the RX and TX LEDs on the board will flicker fast and won’t stop until the upload is done.

\item {} 
\sphinxAtStartPar
\sphinxstylestrong{Debug}: For line\sphinxhyphen{}by\sphinxhyphen{}line error checking.

\item {} 
\sphinxAtStartPar
\sphinxstylestrong{Select Board}: Quick setup board and port.

\item {} 
\sphinxAtStartPar
\sphinxstylestrong{Serial Plotter}: Check the change of reading value.

\item {} 
\sphinxAtStartPar
\sphinxstylestrong{Serial Monitor}: Click the button and a window will appear. It receives the data sent from your control board. It is very useful for debugging.

\item {} 
\sphinxAtStartPar
\sphinxstylestrong{File}: Click the menu and a drop\sphinxhyphen{}down list will appear, including file creating, opening, saving, closing, some parameter configuring, etc.

\item {} 
\sphinxAtStartPar
\sphinxstylestrong{Edit}: Click the menu. On the drop\sphinxhyphen{}down list, there are some editing operations like \sphinxstylestrong{Cut}, \sphinxstylestrong{Copy}, \sphinxstylestrong{Paste}, \sphinxstylestrong{Find}, and so on, with their corresponding shortcuts.

\item {} 
\sphinxAtStartPar
\sphinxstylestrong{Sketch}: Includes operations like \sphinxstylestrong{Verify}, \sphinxstylestrong{Upload}, \sphinxstylestrong{Add} files, etc. A more important function is \sphinxstylestrong{Include Library} \textendash{} where you can add libraries.

\item {} 
\sphinxAtStartPar
\sphinxstylestrong{Tool}: Includes some tools \textendash{} the most frequently used Board (the board you use) and Port (the port your board is at). Every time you want to upload the code, you need to select or check them.

\item {} 
\sphinxAtStartPar
\sphinxstylestrong{Help}: If you’re a beginner, you may check the options under the menu and get the help you need, including operations in IDE, introduction information, troubleshooting, code explanation, etc.

\item {} 
\sphinxAtStartPar
\sphinxstylestrong{Output Bar}: Switch the output tab here.

\item {} 
\sphinxAtStartPar
\sphinxstylestrong{Output Window}: Print information.

\item {} 
\sphinxAtStartPar
\sphinxstylestrong{Board and Port}: Here you can preview the board and port selected for code upload. You can select them again by \sphinxstylestrong{Tools} \sphinxhyphen{}\textgreater{} \sphinxstylestrong{Board} / \sphinxstylestrong{Port} if any is incorrect.

\item {} 
\sphinxAtStartPar
The editing area of the IDE. You can write code here.

\item {} 
\sphinxAtStartPar
\sphinxstylestrong{Sketchbook}: For managing sketch files.

\item {} 
\sphinxAtStartPar
\sphinxstylestrong{Board Manager}: For managing board driver.

\item {} 
\sphinxAtStartPar
\sphinxstylestrong{Library Manager}: For managing your library files.

\item {} 
\sphinxAtStartPar
\sphinxstylestrong{Debug}: Help debugging code.

\item {} 
\sphinxAtStartPar
\sphinxstylestrong{Search}: Search the codes from your sketches.

\end{enumerate}


\subsection{Creating, Saving Sketches}
\label{\detokenize{Get_Started_with_Arduino/Creating_Opening_Saving_Sketches:creating-saving-sketches}}\begin{enumerate}
\sphinxsetlistlabels{\arabic}{enumi}{enumii}{}{.}%
\item {} 
\sphinxAtStartPar
When you open the Arduino IDE for the first time or create a new sketch, you will see a page like this, where the Arduino IDE creates a new file for you, which is called a “sketch”.

\noindent\sphinxincludegraphics{{Creat_Skerch1}.png}

\sphinxAtStartPar
These sketch files have a regular temporary name, from which you can tell the date the file was created. \sphinxcode{\sphinxupquote{sketch\_oct14a.ino}} means October 14th first sketch, \sphinxcode{\sphinxupquote{.ino}} is the file format of this sketch.

\item {} 
\sphinxAtStartPar
Now let’s try to create a new sketch. Copy the following code into the Arduino IDE to replace the original code.

\noindent\sphinxincludegraphics{{Creat_Skerch2}.png}

\begin{sphinxVerbatim}[commandchars=\\\{\}]
\PYG{k+kr}{void}\PYG{+w}{ }\PYG{n+nb}{setup}\PYG{p}{(}\PYG{p}{)}\PYG{+w}{ }\PYG{p}{\PYGZob{}}
\PYG{+w}{    }\PYG{c+c1}{// put your setup code here, to run once:}
\PYG{+w}{    }\PYG{n+nf}{pinMode}\PYG{p}{(}\PYG{l+m+mi}{13}\PYG{p}{,}\PYG{k+kr}{OUTPUT}\PYG{p}{)}\PYG{p}{;}
\PYG{p}{\PYGZcb{}}

\PYG{k+kr}{void}\PYG{+w}{ }\PYG{n+nb}{loop}\PYG{p}{(}\PYG{p}{)}\PYG{+w}{ }\PYG{p}{\PYGZob{}}
\PYG{+w}{    }\PYG{c+c1}{// put your main code here, to run repeatedly:}
\PYG{+w}{    }\PYG{n+nf}{digitalWrite}\PYG{p}{(}\PYG{l+m+mi}{13}\PYG{p}{,}\PYG{k+kr}{HIGH}\PYG{p}{)}\PYG{p}{;}
\PYG{+w}{    }\PYG{n+nf}{delay}\PYG{p}{(}\PYG{l+m+mi}{500}\PYG{p}{)}\PYG{p}{;}
\PYG{+w}{    }\PYG{n+nf}{digitalWrite}\PYG{p}{(}\PYG{l+m+mi}{13}\PYG{p}{,}\PYG{k+kr}{LOW}\PYG{p}{)}\PYG{p}{;}
\PYG{+w}{    }\PYG{n+nf}{delay}\PYG{p}{(}\PYG{l+m+mi}{500}\PYG{p}{)}\PYG{p}{;}
\PYG{p}{\PYGZcb{}}
\end{sphinxVerbatim}

\item {} 
\sphinxAtStartPar
Press \sphinxcode{\sphinxupquote{Ctrl+S}} or click \sphinxstylestrong{File} \sphinxhyphen{}\textgreater{} \sphinxstylestrong{Save}. The Sketch is saved in: \sphinxcode{\sphinxupquote{C:\textbackslash{}Users\textbackslash{}\{your\_user\}\textbackslash{}Documents\textbackslash{}Arduino}} by default, you can rename it or find a new path to save it.

\noindent\sphinxincludegraphics{{Creat_Skerch3}.png}

\item {} 
\sphinxAtStartPar
After successful saving, you will see that the name in the Arduino IDE has been updated.

\noindent\sphinxincludegraphics{{Creat_Skerch4}.png}

\end{enumerate}


\subsection{Opening a Sketch}
\label{\detokenize{Get_Started_with_Arduino/Creating_Opening_Saving_Sketches:opening-a-sketch}}\begin{enumerate}
\sphinxsetlistlabels{\arabic}{enumi}{enumii}{}{.}%
\item {} 
\sphinxAtStartPar
Click on \sphinxcode{\sphinxupquote{File}} in the top menu bar.

\item {} 
\sphinxAtStartPar
Select \sphinxcode{\sphinxupquote{Open}} from the dropdown menu.

\item {} 
\sphinxAtStartPar
Navigate to the folder where your sketch is saved (Arduino sketches typically have a \sphinxcode{\sphinxupquote{.ino}} file extension).

\end{enumerate}

\sphinxAtStartPar
4. Select the sketch file and click \sphinxcode{\sphinxupquote{Open}}.
You can also quickly open recently used sketches by going to \sphinxcode{\sphinxupquote{File \textgreater{} Sketchbook}} and selecting a sketch from the list.

\noindent\sphinxincludegraphics{{Creat_Skerch5}.png}

\sphinxAtStartPar
Please continue with the next section to learn how to upload this created sketch to your Arduino board.

\sphinxstepscope


\section{How to upload Sketch to the Board?}
\label{\detokenize{Get_Started_with_Arduino/How_to_Upload_Sketch:how-to-upload-sketch-to-the-board}}\label{\detokenize{Get_Started_with_Arduino/How_to_Upload_Sketch::doc}}
\sphinxAtStartPar
In this section, you will learn how to upload the sketch created previously to the Arduino board, as well as learn about some considerations.


\subsection{1. Choose Board and port}
\label{\detokenize{Get_Started_with_Arduino/How_to_Upload_Sketch:choose-board-and-port}}
\sphinxAtStartPar
Arduino development boards usually come with a USB cable. You can use it to connect the board to your computer.

\sphinxAtStartPar
Select the correct \sphinxstylestrong{Board} and \sphinxstylestrong{Port} in the Arduino IDE. Normally, Arduino boards are recognized automatically by the computer and assigned a port, so you can select it here.
\begin{quote}

\noindent\sphinxincludegraphics[width=0.900\linewidth]{{Upload_Sketch1}.png}
\end{quote}

\sphinxAtStartPar
If your board is already plugged in, but not recognized, check if the \sphinxstylestrong{INSTALLED} logo appears in the \sphinxstylestrong{Arduino UNO R4 Boards} section of the \sphinxstylestrong{Boards Manager}, if not, please scroll down a bit and click on \sphinxstylestrong{INSTALL}.

\sphinxAtStartPar
Search \sphinxstylestrong{“UNO R4”} in \sphinxstylestrong{Boards Manager} and check if the corresponding library is installed.
\begin{quote}

\noindent\sphinxincludegraphics[width=0.900\linewidth]{{Upload_Sketch2}.png}
\end{quote}

\sphinxAtStartPar
Reopening the Arduino IDE and re\sphinxhyphen{}plugging the Arduino board will fix most of the problems. You can also click \sphinxstylestrong{Tools} \sphinxhyphen{}\textgreater{} \sphinxstylestrong{Board} or \sphinxstylestrong{Port} to select them.
\begin{quote}

\noindent\sphinxincludegraphics[width=0.900\linewidth]{{Upload_Sketch1-1}.png}

\noindent\sphinxincludegraphics[width=0.900\linewidth]{{Upload_Sketch1-2}.png}
\end{quote}


\subsection{2. Verify the Sketch}
\label{\detokenize{Get_Started_with_Arduino/How_to_Upload_Sketch:verify-the-sketch}}
\sphinxAtStartPar
After clicking the Verify button, the sketch will be compiled to see if there are any errors.
\begin{quote}

\noindent\sphinxincludegraphics[width=0.900\linewidth]{{Upload_Sketch3}.png}
\end{quote}

\sphinxAtStartPar
You can use it to find mistakes if you delete some characters or type a few letters by mistake. From the message bar, you can see where and what type of errors occurred.
\begin{quote}

\noindent\sphinxincludegraphics[width=0.900\linewidth]{{Upload_Sketch4}.png}
\end{quote}

\sphinxAtStartPar
If there are no errors, you will see a message like the one below.
\begin{quote}

\noindent\sphinxincludegraphics[width=0.900\linewidth]{{Upload_Sketch5}.png}
\end{quote}


\subsection{3. Upload sketch}
\label{\detokenize{Get_Started_with_Arduino/How_to_Upload_Sketch:upload-sketch}}
\sphinxAtStartPar
After completing the above steps, click the \sphinxstylestrong{Upload} button to upload this sketch to the board.
\begin{quote}

\noindent\sphinxincludegraphics[width=0.900\linewidth]{{Upload_Sketch6}.png}
\end{quote}

\sphinxAtStartPar
If successful, you will be able to see the following prompt.
\begin{quote}

\noindent\sphinxincludegraphics[width=0.900\linewidth]{{Upload_Sketch7}.png}
\end{quote}

\sphinxAtStartPar
At the same time, the on\sphinxhyphen{}board LED blink.

\noindent{\hspace*{\fill}\sphinxincludegraphics[width=400\sphinxpxdimen]{{Upload_Sketch8}.png}\hspace*{\fill}}



\sphinxAtStartPar
The Arduino board will automatically run the sketch after power is applied after the sketch is uploaded. The running program can be overwritten by uploading a new sketch.

\sphinxstepscope


\section{Arduino IDE Sketch Writing Rules}
\label{\detokenize{Get_Started_with_Arduino/Sketch_Writing_Rules:arduino-ide-sketch-writing-rules}}\label{\detokenize{Get_Started_with_Arduino/Sketch_Writing_Rules::doc}}
\sphinxAtStartPar
Arduino is an easy\sphinxhyphen{}to\sphinxhyphen{}use open\sphinxhyphen{}source hardware platform, suitable for beginners to develop electronic projects. Following some basic rules and best practices when writing Arduino code (called “sketch”) can help improve code readability, maintainability, and efficiency. This article will introduce these rules and best practices and provide detailed explanations.


\subsection{1. Structured Code}
\label{\detokenize{Get_Started_with_Arduino/Sketch_Writing_Rules:structured-code}}
\sphinxAtStartPar
A typical Arduino sketch consists of two main parts: the \sphinxcode{\sphinxupquote{setup()}} function and the \sphinxcode{\sphinxupquote{loop()}} function.
\begin{itemize}
\item {} 
\sphinxAtStartPar
The \sphinxcode{\sphinxupquote{setup()}} function: used for initialization settings, such as setting pin modes, starting serial communication, etc. This function is executed once at the start of the program.

\begin{sphinxVerbatim}[commandchars=\\\{\}]
\PYG{k+kr}{void}\PYG{+w}{ }\PYG{n+nb}{setup}\PYG{p}{(}\PYG{p}{)}\PYG{+w}{ }\PYG{p}{\PYGZob{}}
\PYG{+w}{  }\PYG{c+c1}{// Initialization settings}
\PYG{p}{\PYGZcb{}}
\end{sphinxVerbatim}

\item {} 
\sphinxAtStartPar
The \sphinxcode{\sphinxupquote{loop()}} function: contains the main program logic, which is executed repeatedly.

\begin{sphinxVerbatim}[commandchars=\\\{\}]
\PYG{k+kr}{void}\PYG{+w}{ }\PYG{n+nb}{loop}\PYG{p}{(}\PYG{p}{)}\PYG{+w}{ }\PYG{p}{\PYGZob{}}
\PYG{+w}{  }\PYG{c+c1}{// Main program logic}
\PYG{p}{\PYGZcb{}}
\end{sphinxVerbatim}

\end{itemize}


\subsection{2. Meaningful Naming}
\label{\detokenize{Get_Started_with_Arduino/Sketch_Writing_Rules:meaningful-naming}}
\sphinxAtStartPar
Use meaningful variable names, function names, and constant names to make the code easy to understand and maintain.
\begin{itemize}
\item {} 
\sphinxAtStartPar
Variable names should describe their purpose, such as \sphinxcode{\sphinxupquote{int sensorValue;}}.

\item {} 
\sphinxAtStartPar
Function names should describe their function, such as \sphinxcode{\sphinxupquote{void readSensor();}}.

\item {} 
\sphinxAtStartPar
Constant names are usually in uppercase letters and underscores, such as \sphinxcode{\sphinxupquote{const int LED\_PIN = 13;}}.

\end{itemize}


\subsection{3. Comments}
\label{\detokenize{Get_Started_with_Arduino/Sketch_Writing_Rules:comments}}
\sphinxAtStartPar
Adding comments to the code can help yourself and others understand the code logic. Comments should be concise and clear.
\begin{itemize}
\item {} 
\sphinxAtStartPar
Use single\sphinxhyphen{}line comments to explain the function of variables, functions, or code blocks.

\begin{sphinxVerbatim}[commandchars=\\\{\}]
\PYG{k+kr}{int}\PYG{+w}{ }\PYG{n}{sensorValue}\PYG{p}{;}\PYG{+w}{ }\PYG{c+c1}{// Sensor reading}
\end{sphinxVerbatim}

\item {} 
\sphinxAtStartPar
Use block comments to explain complex logic or algorithms.

\begin{sphinxVerbatim}[commandchars=\\\{\}]
\PYG{c+cm}{/*}
\PYG{c+cm}{ * Read sensor data and calculate the average}
\PYG{c+cm}{ */}
\PYG{k+kr}{void}\PYG{+w}{ }\PYG{n+nf}{readSensor}\PYG{p}{(}\PYG{p}{)}\PYG{+w}{ }\PYG{p}{\PYGZob{}}
\PYG{+w}{  }\PYG{c+c1}{// Code implementation}
\PYG{p}{\PYGZcb{}}
\end{sphinxVerbatim}

\end{itemize}


\subsection{4. Use Constants}
\label{\detokenize{Get_Started_with_Arduino/Sketch_Writing_Rules:use-constants}}
\sphinxAtStartPar
For values that do not change, use the \sphinxcode{\sphinxupquote{const}} keyword to define constants, which can improve code readability and security.

\begin{sphinxVerbatim}[commandchars=\\\{\}]
\PYG{k+kr}{const}\PYG{+w}{ }\PYG{k+kr}{int}\PYG{+w}{ }\PYG{n}{LED\PYGZus{}PIN}\PYG{+w}{ }\PYG{o}{=}\PYG{+w}{ }\PYG{l+m+mi}{13}\PYG{p}{;}\PYG{+w}{ }\PYG{c+c1}{// LED connected to digital pin 13}
\end{sphinxVerbatim}


\subsection{5. Avoid Magic Numbers}
\label{\detokenize{Get_Started_with_Arduino/Sketch_Writing_Rules:avoid-magic-numbers}}
\sphinxAtStartPar
“Magic numbers” refer to numerical constants used directly in the code. Use meaningful named constants instead to improve code readability.

\begin{sphinxVerbatim}[commandchars=\\\{\}]
\PYG{k+kr}{const}\PYG{+w}{ }\PYG{k+kr}{int}\PYG{+w}{ }\PYG{n}{MAX\PYGZus{}SPEED}\PYG{+w}{ }\PYG{o}{=}\PYG{+w}{ }\PYG{l+m+mi}{255}\PYG{p}{;}
\PYG{n+nf}{analogWrite}\PYG{p}{(}\PYG{n}{motorPin}\PYG{p}{,}\PYG{+w}{ }\PYG{n}{MAX\PYGZus{}SPEED}\PYG{p}{)}\PYG{p}{;}
\end{sphinxVerbatim}


\subsection{6. Handle Errors}
\label{\detokenize{Get_Started_with_Arduino/Sketch_Writing_Rules:handle-errors}}
\sphinxAtStartPar
When writing code, consider possible error conditions and handle them appropriately. For example, check function return values to ensure device connections are normal.

\begin{sphinxVerbatim}[commandchars=\\\{\}]
\PYG{k}{if}\PYG{+w}{ }\PYG{p}{(}\PYG{n+nf}{WiFi}\PYG{p}{.}\PYG{n}{status}\PYG{p}{(}\PYG{p}{)}\PYG{+w}{ }\PYG{o}{!}\PYG{o}{=}\PYG{+w}{ }\PYG{n}{WL\PYGZus{}CONNECTED}\PYG{p}{)}\PYG{+w}{ }\PYG{p}{\PYGZob{}}
\PYG{+w}{  }\PYG{c+c1}{// Handle WiFi not connected case}
\PYG{p}{\PYGZcb{}}
\end{sphinxVerbatim}


\subsection{7. Format Code}
\label{\detokenize{Get_Started_with_Arduino/Sketch_Writing_Rules:format-code}}
\sphinxAtStartPar
Good code formatting helps improve readability. Maintain consistent indentation style, appropriate spacing, and alignment.
\begin{itemize}
\item {} 
\sphinxAtStartPar
Each code block should be indented by two or four spaces.

\item {} 
\sphinxAtStartPar
Leave a blank line between functions.

\item {} 
\sphinxAtStartPar
Leave spaces on both sides of operators.

\end{itemize}


\subsection{8. Decompose Tasks}
\label{\detokenize{Get_Started_with_Arduino/Sketch_Writing_Rules:decompose-tasks}}
\sphinxAtStartPar
Decompose complex tasks into multiple small functions, each performing a distinct function. This makes the code simpler, easier to debug, and maintain.

\begin{sphinxVerbatim}[commandchars=\\\{\}]
\PYG{k+kr}{void}\PYG{+w}{ }\PYG{n+nb}{setup}\PYG{p}{(}\PYG{p}{)}\PYG{+w}{ }\PYG{p}{\PYGZob{}}
\PYG{+w}{  }\PYG{n}{initializePins}\PYG{p}{(}\PYG{p}{)}\PYG{p}{;}
\PYG{+w}{  }\PYG{n}{initializeSerial}\PYG{p}{(}\PYG{p}{)}\PYG{p}{;}
\PYG{p}{\PYGZcb{}}

\PYG{k+kr}{void}\PYG{+w}{ }\PYG{n+nf}{initializePins}\PYG{p}{(}\PYG{p}{)}\PYG{+w}{ }\PYG{p}{\PYGZob{}}
\PYG{+w}{  }\PYG{n+nf}{pinMode}\PYG{p}{(}\PYG{n}{LED\PYGZus{}PIN}\PYG{p}{,}\PYG{+w}{ }\PYG{k+kr}{OUTPUT}\PYG{p}{)}\PYG{p}{;}
\PYG{p}{\PYGZcb{}}

\PYG{k+kr}{void}\PYG{+w}{ }\PYG{n+nf}{initializeSerial}\PYG{p}{(}\PYG{p}{)}\PYG{+w}{ }\PYG{p}{\PYGZob{}}
\PYG{+w}{  }\PYG{n+nf}{Serial}\PYG{p}{.}\PYG{n+nf}{begin}\PYG{p}{(}\PYG{l+m+mi}{9600}\PYG{p}{)}\PYG{p}{;}
\PYG{p}{\PYGZcb{}}
\end{sphinxVerbatim}


\subsection{Conclusion}
\label{\detokenize{Get_Started_with_Arduino/Sketch_Writing_Rules:conclusion}}
\sphinxAtStartPar
Following the above rules and best practices can help you write high\sphinxhyphen{}quality Arduino code. Good programming habits not only improve code readability and maintainability but also reduce errors and enhance development efficiency. Hopefully, these rules will help you achieve better results in Arduino development.

\sphinxstepscope


\section{How to Build the Circuit}
\label{\detokenize{Get_Started_with_Arduino/How_to_Build_the_Circuit:how-to-build-the-circuit}}\label{\detokenize{Get_Started_with_Arduino/How_to_Build_the_Circuit::doc}}
\sphinxAtStartPar
Many of the things you use every day are powered by electricity, like the lights in your house and the computer you’re reading.

\sphinxAtStartPar
To use electricity, you must build an electrical circuit. Basically, a circuit is a path through which electricity flows, or an electronic circuit, and is made up of electrical devices and components (appliances) that are connected in a certain way, such as resistors, capacitors, power supplies, and switches.

\noindent{\hspace*{\fill}\sphinxincludegraphics[width=250\sphinxpxdimen]{{Bulid_Circuit1}.png}\hspace*{\fill}}



\sphinxAtStartPar
A circuit is a closed path in which electrons move to create an electric current. To flow current, there must be a conducting path between the positive terminal of the power supply and the negative terminal, which is called a closed circuit (if it is broken, it is called an open circuit.) .

\sphinxAtStartPar
The Arduino Board has some power output pins (positive) and some ground pins (negative).
You can use these pins as the positive and negative sides of the power supply by plugging the power source into the board.

\noindent{\hspace*{\fill}\sphinxincludegraphics[width=0.700\linewidth]{{Bulid_Circuit2}.png}\hspace*{\fill}}

\sphinxAtStartPar
With electricity, you can create works with light, sound, and motion.
You can light up an LED by connecting the long pin to the positive terminal and the short pin to the negative terminal.
However, doing this directly can quickly damage not just the LED but also risk harming the pins of your UNO R4 board. To avoid this, it’s essential to add a 1kΩ resistor into the circuit, protecting both the LED and the UNO R4’s pins.

\sphinxAtStartPar
The circuit they form is shown below.

\noindent{\hspace*{\fill}\sphinxincludegraphics[width=0.650\linewidth]{{Bulid_Circuit3}.png}\hspace*{\fill}}



\sphinxAtStartPar
You may have questions this time: how do I build this circuit? Hold the wires by hand, or tape the pins and wires?

\sphinxAtStartPar
In this situation, solderless breadboards will be your strongest allies.


\subsection{Hello, Breadboard!}
\label{\detokenize{Get_Started_with_Arduino/How_to_Build_the_Circuit:hello-breadboard}}\label{\detokenize{Get_Started_with_Arduino/How_to_Build_the_Circuit:bc-bb}}
\sphinxAtStartPar
A breadboard is a rectangular plastic plate with a bunch of small holes.
These holes allow us to easily insert electronic components and build electronic circuits.
Breadboards do not permanently fix electronic components, so we can easily repair a circuit and start over if something goes wrong.

\begin{sphinxadmonition}{note}{Note:}
\sphinxAtStartPar
There is no need for special tools to use breadboards. However, many electronic components are very small, and a pair of tweezers can help us to pick up small parts better.
\end{sphinxadmonition}

\sphinxAtStartPar
On the Internet, we can find a lot of information about breadboards.
\begin{itemize}
\item {} 
\sphinxAtStartPar
\sphinxhref{https://www.sciencebuddies.org/science-fair-projects/references/how-to-use-a-breadboard\#pth-smd}{How to Use a Breadboard \sphinxhyphen{} Science Buddies}

\item {} 
\sphinxAtStartPar
\sphinxhref{https://cdn.makezine.com/uploads/2012/10/breadboardworkshop.pdf}{What is a BREADBOARD? \sphinxhyphen{} Makezine}

\end{itemize}

\sphinxAtStartPar
Here are some things you should know about breadboards.

\noindent{\hspace*{\fill}\sphinxincludegraphics[width=0.850\linewidth]{{Bulid_Circuit4}.png}\hspace*{\fill}}


\begin{enumerate}
\sphinxsetlistlabels{\arabic}{enumi}{enumii}{}{.}%
\item {} 
\sphinxAtStartPar
Each half\sphinxhyphen{}row group (such as column A\sphinxhyphen{}E in row 1 or column F\sphinxhyphen{}J in row 3) is connected. Therefore, if an electrical signal flows in from A1, it can flow out from B1, C1, D1, E1, but not from F1 or A2.

\item {} 
\sphinxAtStartPar
In most cases, both sides of the breadboard are used as power buses, and the holes in each column (about 50 holes) are connected together. As a general rule, positive power supplies are connected to the holes near the red wire, and negative power supplies are connected to the holes near the blue wire.

\end{enumerate}

\begin{sphinxadmonition}{note}{Note:}
\sphinxAtStartPar
G1 and G2, V1 and V2 in the breadboard are not connected. Usually you need to use jumper wires to connect them.
\end{sphinxadmonition}

\noindent{\hspace*{\fill}\sphinxincludegraphics[width=0.850\linewidth]{{Bulid_Circuit4-4}.png}\hspace*{\fill}}



\sphinxAtStartPar
\sphinxstylestrong{Let us follow the direction of the current to build the circuit!}

\noindent{\hspace*{\fill}\sphinxincludegraphics[width=0.600\linewidth]{{Bulid_Circuit5}.png}\hspace*{\fill}}


\begin{enumerate}
\sphinxsetlistlabels{\arabic}{enumi}{enumii}{}{.}%
\item {} 
\sphinxAtStartPar
In this circuit, we use the 5V pin of the board to power the LED. Use a male\sphinxhyphen{}to\sphinxhyphen{}male (M2M) jumper wire to connect it to the red power bus.

\item {} 
\sphinxAtStartPar
To protect the LED and the UNO R4’s pins, the current must pass through a 1k ohm resistor. Connect one end (either end) of the resistor to the red power bus, and the other end to the free row of the breadboard.

\begin{sphinxadmonition}{note}{Note:}
\sphinxAtStartPar
The color ring of the 1000 ohm {\hyperref[\detokenize{Components_Kit/component_resistor:component-resistor}]{\sphinxcrossref{\DUrole{std}{\DUrole{std-ref}{Resistor}}}}} is red, black, black, brown and brown.
\end{sphinxadmonition}

\item {} 
\sphinxAtStartPar
If you pick up the LED, you will see that one of its leads is longer than the other. Connect the longer lead to the same row as the resistor, and the shorter lead to the other row.

\begin{sphinxadmonition}{note}{Note:}
\sphinxAtStartPar
The longer lead is the anode, which represents the positive side of the circuit; the shorter lead is the cathode, which represents the negative side.

\sphinxAtStartPar
The anode needs to be connected to the GPIO pin through a resistor; the cathode needs to be connected to the GND pin.
\end{sphinxadmonition}

\item {} 
\sphinxAtStartPar
Using a male\sphinxhyphen{}to\sphinxhyphen{}male (M2M) jumper wire, connect the LED short pin to the breadboard’s negative power bus.

\item {} 
\sphinxAtStartPar
Connect the GND pin of board to the negative power bus using a jumper.

\end{enumerate}


\subsection{Beware of short circuits}
\label{\detokenize{Get_Started_with_Arduino/How_to_Build_the_Circuit:beware-of-short-circuits}}
\sphinxAtStartPar
Short circuits can occur when two components that shouldn’t be connected are “accidentally” connected.
This kit includes resistors, transistors, capacitors, LEDs, etc. that have long metal pins that can bump into each other and cause a short. Some circuits are simply prevented from functioning properly when a short occurs. Occasionally, a short circuit can damage components permanently, especially between the power supply and the ground bus, causing the circuit to get very hot, melting the plastic on the breadboard and even burning the components!

\sphinxAtStartPar
Therefore, always make sure that the pins of all the electronics on the breadboard are not touching each other.


\subsection{Direction of the circuit}
\label{\detokenize{Get_Started_with_Arduino/How_to_Build_the_Circuit:direction-of-the-circuit}}
\sphinxAtStartPar
There is an orientation to circuits, and the orientation plays a significant role in certain electronic components. There are some devices with polarity, which means they must be connected correctly based on their positive and negative poles. Circuits built with the wrong orientation will not function properly.

\noindent{\hspace*{\fill}\sphinxincludegraphics[width=0.600\linewidth]{{Bulid_Circuit6}.png}\hspace*{\fill}}



\sphinxAtStartPar
If you reverse the LED in this simple circuit that we built earlier, you will find that it no longer works.

\sphinxAtStartPar
In contrast, some devices have no direction, such as the resistors in this circuit, so you can try inverting them without affecting the LEDs’ normal operation.

\sphinxAtStartPar
Most components and modules with labels such as “+”, “\sphinxhyphen{}”, “GND”, “VCC” or have pins of different lengths must be connected to the circuit in a specific way.


\subsection{Protection of the circuit}
\label{\detokenize{Get_Started_with_Arduino/How_to_Build_the_Circuit:protection-of-the-circuit}}
\sphinxAtStartPar
Current is the rate at which electrons flow past a point in a complete electrical circuit. At its most basic, current = flow. An ampere (AM\sphinxhyphen{}pir), or amp, is the international unit used for measuring current. It expresses the quantity of electrons (sometimes called “electrical charge”) flowing past a point in a circuit over a given time.

\sphinxAtStartPar
The driving force (voltage) behind the flow of current is called voltage and is measured in volts (V).

\sphinxAtStartPar
Resistance (R) is the property of the material that restricts the flow of current, and it is measured in ohms (Ω).

\sphinxAtStartPar
According to Ohm’s law (as long as the temperature remains constant), current, voltage, and resistance are proportional.
A circuit’s current is proportional to its voltage and inversely proportional to its resistance.

\sphinxAtStartPar
Therefore, current (I) = voltage (V) / resistance (R).
\begin{itemize}
\item {} 
\sphinxAtStartPar
\sphinxhref{https://en.wikipedia.org/wiki/Ohm\%27s\_law}{Ohm’s law \sphinxhyphen{} Wikipedia}

\end{itemize}

\sphinxAtStartPar
About Ohm’s law we can do a simple experiment.

\noindent\sphinxincludegraphics[width=0.550\linewidth]{{Bulid_Circuit7}.png}

\sphinxAtStartPar
By changing the wire connecting 5V to 3.3V , the LED gets dimmer.
If you change the resistor from 1000 ohm to 2000 ohm (color ring: red, black, black, brown, and brown), you will notice that the LED becomes dimmer than before. The larger the resistor, the dimmer the LED.

\sphinxAtStartPar
Most packaged modules only require access to the proper voltage (usually 3.3V or 5V), such as ultrasonic module.

\sphinxAtStartPar
However, in your self\sphinxhyphen{}built circuits, you need to be aware of the supply voltage and resistor usage for electrical devices.

\sphinxAtStartPar
As an example, LEDs usually consume 20mA of current, and their voltage drop is about 1.8V. According to Ohm’s law, if we use 5V power supply, we need to connect a minimum of 160ohm ((5\sphinxhyphen{}1.8)/20mA) resistor in order not to burn out the LED.


\subsection{Control circuit with Arduino}
\label{\detokenize{Get_Started_with_Arduino/How_to_Build_the_Circuit:control-circuit-with-arduino}}
\sphinxAtStartPar
Now that we have a basic understanding of Arduino programming and electronic circuits, it’s time to face the most critical question: How to control circuits with Arduino?

\sphinxAtStartPar
Simply put, the way Arduino controls a circuit is by changing the level of the pins on the board. For example, when controlling an on\sphinxhyphen{}board LED, it is writing a high or low level signal to pin 13.

\sphinxAtStartPar
Now let’s try to code the Arduino board to control the blinking LED on the breadboard. Build the circuit so that the LED is connected to pin 9.

\noindent{\hspace*{\fill}\sphinxincludegraphics[width=400\sphinxpxdimen]{{Bulid_Circuit8}.png}\hspace*{\fill}}

\sphinxAtStartPar
Next, upload this sketch to the Arduino development board.

\begin{sphinxVerbatim}[commandchars=\\\{\}]
\PYG{k+kt}{int}\PYG{+w}{ }\PYG{n}{ledPin}\PYG{+w}{ }\PYG{o}{=}\PYG{+w}{ }\PYG{l+m+mi}{9}\PYG{p}{;}
\PYG{k+kt}{int}\PYG{+w}{ }\PYG{n}{delayTime}\PYG{+w}{ }\PYG{o}{=}\PYG{+w}{ }\PYG{l+m+mi}{500}\PYG{p}{;}

\PYG{k+kt}{void}\PYG{+w}{ }\PYG{n+nf}{setup}\PYG{p}{(}\PYG{p}{)}\PYG{+w}{ }\PYG{p}{\PYGZob{}}
\PYG{+w}{    }\PYG{n}{pinMode}\PYG{p}{(}\PYG{n}{ledPin}\PYG{p}{,}\PYG{n}{OUTPUT}\PYG{p}{)}\PYG{p}{;}
\PYG{p}{\PYGZcb{}}

\PYG{k+kt}{void}\PYG{+w}{ }\PYG{n+nf}{loop}\PYG{p}{(}\PYG{p}{)}\PYG{+w}{ }\PYG{p}{\PYGZob{}}
\PYG{+w}{    }\PYG{n}{digitalWrite}\PYG{p}{(}\PYG{n}{ledPin}\PYG{p}{,}\PYG{n}{HIGH}\PYG{p}{)}\PYG{p}{;}
\PYG{+w}{    }\PYG{n}{delay}\PYG{p}{(}\PYG{n}{delayTime}\PYG{p}{)}\PYG{p}{;}
\PYG{+w}{    }\PYG{n}{digitalWrite}\PYG{p}{(}\PYG{n}{ledPin}\PYG{p}{,}\PYG{n}{LOW}\PYG{p}{)}\PYG{p}{;}
\PYG{+w}{    }\PYG{n}{delay}\PYG{p}{(}\PYG{n}{delayTime}\PYG{p}{)}\PYG{p}{;}
\PYG{p}{\PYGZcb{}}
\end{sphinxVerbatim}

\sphinxAtStartPar
This sketch is very similar to the one we used to control the blinking of the on\sphinxhyphen{}board LED, the difference is that the value of \sphinxcode{\sphinxupquote{ledPin}} has been changed to 9.
This is because we are trying to control the level of pin 9 this time.

\sphinxAtStartPar
Now you can see the LED on the breadboard blinking.

\sphinxstepscope


\section{How to add libraries? (Important)}
\label{\detokenize{Get_Started_with_Arduino/How_to_add_Libraries:how-to-add-libraries-important}}\label{\detokenize{Get_Started_with_Arduino/How_to_add_Libraries:add-libraries}}\label{\detokenize{Get_Started_with_Arduino/How_to_add_Libraries::doc}}
\sphinxAtStartPar
A library is a set of pre\sphinxhyphen{}written code or functions designed to enhance the functionality of the Arduino IDE. By utilizing libraries, you can easily incorporate complex features into your projects without having to write the code from scratch, thereby saving both time and effort.


\subsection{Using the Library Manager}
\label{\detokenize{Get_Started_with_Arduino/How_to_add_Libraries:using-the-library-manager}}
\sphinxAtStartPar
Many libraries are available directly through the Arduino Library Manager. You can access the Library Manager by following these steps:
\begin{enumerate}
\sphinxsetlistlabels{\arabic}{enumi}{enumii}{}{.}%
\item {} 
\sphinxAtStartPar
In the \sphinxstylestrong{Library Manager}, you can search for the desired library by name or browse through different categories.

\begin{sphinxadmonition}{note}{Note:}
\sphinxAtStartPar
In projects where library installation is required, there will be prompts indicating which libraries to install. Follow the instructions provided, such as “The DHT sensor library library is used here, you can install it from the Library Manager.” Simply install the recommended libraries as prompted.
\end{sphinxadmonition}

\noindent\sphinxincludegraphics{{Library1}.png}

\item {} 
\sphinxAtStartPar
Once you find the library you want to install, click on it and then click the \sphinxstylestrong{Install} button.

\noindent\sphinxincludegraphics{{Library2}.png}

\item {} 
\sphinxAtStartPar
The Arduino IDE will automatically download and install the library for you.

\end{enumerate}


\subsection{Manual Installation}
\label{\detokenize{Get_Started_with_Arduino/How_to_add_Libraries:manual-installation}}\label{\detokenize{Get_Started_with_Arduino/How_to_add_Libraries:manual-install-lib}}
\sphinxAtStartPar
Some libraries are not available through the \sphinxstylestrong{Library Manager} and need to be manually installed. To install these libraries, follow these steps:
\begin{enumerate}
\sphinxsetlistlabels{\arabic}{enumi}{enumii}{}{.}%
\item {} 
\sphinxAtStartPar
Open the Arduino IDE and go to \sphinxstylestrong{Sketch} \sphinxhyphen{}\textgreater{} \sphinxstylestrong{Include Library} \sphinxhyphen{}\textgreater{} \sphinxstylestrong{Add .ZIP Library}.

\noindent\sphinxincludegraphics{{Library3}.png}

\item {} 
\sphinxAtStartPar
Navigate to the directory where the library files are located, such as the \sphinxcode{\sphinxupquote{elite\sphinxhyphen{}explorer\sphinxhyphen{}kit\sphinxhyphen{}main/library/}} folder, and select the library file and click \sphinxstylestrong{Open}.

\noindent\sphinxincludegraphics{{Library4}.png}

\item {} 
\sphinxAtStartPar
Once the installation is complete, you will receive a notification confirming that the library has been successfully added to your Arduino IDE. The next time you need to use this library, you won’t need to repeat the installation process.

\noindent\sphinxincludegraphics{{Library5}.png}

\item {} 
\sphinxAtStartPar
Repeat the same process to add other libraries.

\end{enumerate}


\subsection{Library Location}
\label{\detokenize{Get_Started_with_Arduino/How_to_add_Libraries:library-location}}
\sphinxAtStartPar
The libraries installed using either of the above methods can be found in the default library directory of the Arduino IDE, which is usually located at \sphinxcode{\sphinxupquote{C:\textbackslash{}Users\textbackslash{}xxx\textbackslash{}Documents\textbackslash{}Arduino\textbackslash{}libraries}}.

\sphinxAtStartPar
If your library directory is different, you can check it by going to \sphinxstylestrong{File} \sphinxhyphen{}\textgreater{} \sphinxstylestrong{Preferences}.

\noindent\sphinxincludegraphics{{Library6}.png}

\sphinxAtStartPar
\sphinxstylestrong{Reference}
\begin{itemize}
\item {} 
\sphinxAtStartPar
\sphinxhref{https://docs.arduino.cc/software/ide-v2/tutorials/ide-v2-installing-a-library}{Installing libraries in Arduino IDE 2}

\end{itemize}

\sphinxstepscope


\chapter{Download the Code}
\label{\detokenize{Download_the_Code/Download_the_Code:download-the-code}}\label{\detokenize{Download_the_Code/Download_the_Code::doc}}
\sphinxAtStartPar
Download the relevant code from the link below.
\begin{itemize}
\item {} 
\sphinxAtStartPar
\sphinxcode{\sphinxupquote{Basic\sphinxhyphen{}Starter\sphinxhyphen{}Kit\sphinxhyphen{}for\sphinxhyphen{}Arduino\sphinxhyphen{}Uno\sphinxhyphen{}R4\sphinxhyphen{}WiFi Code}}

\end{itemize}

\sphinxstepscope


\chapter{Components Introduction}
\label{\detokenize{Components_Kit/Components_Kit:components-introduction}}\label{\detokenize{Components_Kit/Components_Kit::doc}}
\noindent{\hspace*{\fill}\sphinxincludegraphics[width=600\sphinxpxdimen]{{list}.jpg}\hspace*{\fill}}

\sphinxAtStartPar
Below is the introduction to each component, which contains the operating principle of the component and the corresponding projects.

\sphinxstepscope


\section{Arduino Uno R4 WiFi}
\label{\detokenize{Components_Kit/component_UNO_R4:arduino-uno-r4-wifi}}\label{\detokenize{Components_Kit/component_UNO_R4:components-uno-r4-wifi}}\label{\detokenize{Components_Kit/component_UNO_R4::doc}}
\sphinxAtStartPar
\sphinxstylestrong{Overview}

\sphinxAtStartPar
The Arduino UNO R4 WiFi stands at the forefront of IoT and wireless technology. It features the powerful RA4M1 microcontroller from Renesas, complemented by an ESP32\sphinxhyphen{}S3 coprocessor, making it an ideal choice for today’s innovative creators. This board is designed to cater to both beginners and experienced enthusiasts, delivering exceptional performance while preserving the familiar form factor and 5 V operating voltage of its predecessors.

\sphinxAtStartPar
As you explore the Arduino landscape, the UNO R4 WiFi exemplifies connectivity, efficiency, and creativity.

\noindent{\hspace*{\fill}\sphinxincludegraphics[width=0.700\linewidth]{{unor4}.jpg}\hspace*{\fill}}

\sphinxAtStartPar
Here’s what the Arduino UNO R4 WiFi offers:
\begin{itemize}
\item {} 
\sphinxAtStartPar
\sphinxstylestrong{Seamless Integration with UNO Ecosystem:} Staying true to its heritage, the UNO R4 WiFi guarantees compatibility with the iconic UNO form factor, pinout, and 5 V operating voltage. Transitioning from previous versions is effortless, thanks to the harmonious design and the expansive Arduino UNO ecosystem.

\item {} 
\sphinxAtStartPar
\sphinxstylestrong{Supercharged Memory and Processing:} Step into a realm of faster computations and intricate projects. The UNO R4 WiFi not only boasts enhanced memory but also operates with a clock speed that’s three times quicker, ensuring your projects run smoothly and efficiently.

\item {} 
\sphinxAtStartPar
\sphinxstylestrong{Diverse On\sphinxhyphen{}Board Peripherals:} From a 12\sphinxhyphen{}bit DAC and CAN BUS to an OP AMP and a unique SWD port, the UNO R4 WiFi is equipped with features that elevate your project capabilities. Dive into a realm of endless possibilities and unleash your creativity.

\item {} 
\sphinxAtStartPar
\sphinxstylestrong{Connectivity at its Best:} With integrated Wi\sphinxhyphen{}Fi® and Bluetooth® Low Energy, the UNO R4 WiFi paves the way to the Internet of Things. Whether crafting a smart home system or an interactive dashboard, this board has your back.

\item {} 
\sphinxAtStartPar
\sphinxstylestrong{Interactive 12×8 LED Matrix:} Illuminate your projects with dynamic animations or real\sphinxhyphen{}time sensor data visualization, all without the need for external hardware.

\item {} 
\sphinxAtStartPar
\sphinxstylestrong{Advanced Safety Mechanisms:} The board’s innate ability to detect and prevent potentially harmful operations, such as division by zero, ensures a seamless experience. Plus, with detailed feedback on the serial monitor, you’re always in the loop.

\item {} 
\sphinxAtStartPar
\sphinxstylestrong{Qwiic Connector for Rapid Prototyping:} Broaden your project scope with the Qwiic connector. With a vast range of I2C\sphinxhyphen{}compatible modules available, prototyping becomes a breeze.

\end{itemize}

\sphinxAtStartPar
Step into the future of making with the Arduino UNO R4 WiFi. Whether you’re aiming to integrate wireless functionalities, explore the vast IoT landscape, or simply upgrade your existing setup, this board is the ideal partner for your upcoming ventures.

\sphinxAtStartPar
\sphinxstylestrong{Tech specs}


\begin{savenotes}\sphinxattablestart
\sphinxthistablewithglobalstyle
\centering
\begin{tabulary}{\linewidth}[t]{TTT}
\sphinxtoprule
\sphinxstyletheadfamily 
\sphinxAtStartPar
Board
&\sphinxstyletheadfamily 
\sphinxAtStartPar
Name
&\sphinxstyletheadfamily 
\sphinxAtStartPar
Arduino® UNO R4 WiFi
\\
\sphinxmidrule
\sphinxtableatstartofbodyhook
\sphinxAtStartPar
Microcontroller
&\sphinxstartmulticolumn{2}%
\begin{varwidth}[t]{\sphinxcolwidth{2}{3}}
\sphinxAtStartPar
Renesas RA4M1 (Arm® Cortex®\sphinxhyphen{}M4)
\par
\vskip-\baselineskip\vbox{\hbox{\strut}}\end{varwidth}%
\sphinxstopmulticolumn
\\
\sphinxhline
\sphinxAtStartPar
USB
&
\sphinxAtStartPar
USB\sphinxhyphen{}C®
&
\sphinxAtStartPar
Programming Port
\\
\sphinxhline
\sphinxAtStartPar
Pins
&
\sphinxAtStartPar
Digital I/O Pins
&
\sphinxAtStartPar
14
\\
\sphinxhline\sphinxmultirow{3}{12}{%
\begin{varwidth}[t]{\sphinxcolwidth{1}{3}}
\sphinxAtStartPar
Pins
\par
\vskip-\baselineskip\vbox{\hbox{\strut}}\end{varwidth}%
}%
&
\sphinxAtStartPar
Analog input pins
&
\sphinxAtStartPar
6
\\
\sphinxcline{2-3}\sphinxfixclines{3}\sphinxtablestrut{12}&
\sphinxAtStartPar
DAC
&
\sphinxAtStartPar
1
\\
\sphinxcline{2-3}\sphinxfixclines{3}\sphinxtablestrut{12}&
\sphinxAtStartPar
PWM pins
&
\sphinxAtStartPar
6
\\
\sphinxhline\sphinxmultirow{4}{19}{%
\begin{varwidth}[t]{\sphinxcolwidth{1}{3}}
\sphinxAtStartPar
Communication
\par
\vskip-\baselineskip\vbox{\hbox{\strut}}\end{varwidth}%
}%
&
\sphinxAtStartPar
UART
&
\sphinxAtStartPar
Yes, 1x
\\
\sphinxcline{2-3}\sphinxfixclines{3}\sphinxtablestrut{19}&
\sphinxAtStartPar
I2C
&
\sphinxAtStartPar
Yes, 1x
\\
\sphinxcline{2-3}\sphinxfixclines{3}\sphinxtablestrut{19}&
\sphinxAtStartPar
SPI
&
\sphinxAtStartPar
Yes, 1x
\\
\sphinxcline{2-3}\sphinxfixclines{3}\sphinxtablestrut{19}&
\sphinxAtStartPar
CAN
&
\sphinxAtStartPar
Yes 1 CAN Bus
\\
\sphinxhline\sphinxmultirow{3}{28}{%
\begin{varwidth}[t]{\sphinxcolwidth{1}{3}}
\sphinxAtStartPar
Power
\par
\vskip-\baselineskip\vbox{\hbox{\strut}}\end{varwidth}%
}%
&
\sphinxAtStartPar
Circuit operating voltage
&
\sphinxAtStartPar
5 V (ESP32\sphinxhyphen{}S3 is 3.3 V)
\\
\sphinxcline{2-3}\sphinxfixclines{3}\sphinxtablestrut{28}&
\sphinxAtStartPar
Input voltage (VIN)
&
\sphinxAtStartPar
6\sphinxhyphen{}24 V
\\
\sphinxcline{2-3}\sphinxfixclines{3}\sphinxtablestrut{28}&
\sphinxAtStartPar
DC Current per I/O Pin
&
\sphinxAtStartPar
8 mA
\\
\sphinxhline\sphinxmultirow{2}{35}{%
\begin{varwidth}[t]{\sphinxcolwidth{1}{3}}
\sphinxAtStartPar
Clock speed
\par
\vskip-\baselineskip\vbox{\hbox{\strut}}\end{varwidth}%
}%
&
\sphinxAtStartPar
Main core
&
\sphinxAtStartPar
48 MHz
\\
\sphinxcline{2-3}\sphinxfixclines{3}\sphinxtablestrut{35}&
\sphinxAtStartPar
ESP32\sphinxhyphen{}S3
&
\sphinxAtStartPar
up to 240 MHz
\\
\sphinxhline\sphinxmultirow{2}{40}{%
\begin{varwidth}[t]{\sphinxcolwidth{1}{3}}
\sphinxAtStartPar
Memory
\par
\vskip-\baselineskip\vbox{\hbox{\strut}}\end{varwidth}%
}%
&
\sphinxAtStartPar
RA4M1
&
\sphinxAtStartPar
256 kB Flash, 32 kB RAM
\\
\sphinxcline{2-3}\sphinxfixclines{3}\sphinxtablestrut{40}&
\sphinxAtStartPar
ESP32\sphinxhyphen{}S3
&
\sphinxAtStartPar
384 kB ROM, 512 kB SRAM
\\
\sphinxhline\sphinxmultirow{2}{45}{%
\begin{varwidth}[t]{\sphinxcolwidth{1}{3}}
\sphinxAtStartPar
Dimensions
\par
\vskip-\baselineskip\vbox{\hbox{\strut}}\end{varwidth}%
}%
&
\sphinxAtStartPar
Width
&
\sphinxAtStartPar
68.85 mm
\\
\sphinxcline{2-3}\sphinxfixclines{3}\sphinxtablestrut{45}&
\sphinxAtStartPar
Length
&
\sphinxAtStartPar
53.34 mm
\\
\sphinxbottomrule
\end{tabulary}
\sphinxtableafterendhook\par
\sphinxattableend\end{savenotes}

\sphinxAtStartPar
\sphinxstylestrong{Pinout}

\noindent\sphinxincludegraphics[width=1.000\linewidth]{{unor4_wifi_pinout}.png}
\begin{itemize}
\item {} 
\sphinxAtStartPar
\sphinxhref{https://docs.arduino.cc/resources/datasheets/ABX00087-datasheet.pdf}{Arduino UNO R4 WiFi Datasheet}

\item {} 
\sphinxAtStartPar
\sphinxhref{https://docs.arduino.cc/resources/schematics/ABX00087-schematics.pdf}{Arduino UNO R4 WiFi Schematic}

\item {} 
\sphinxAtStartPar
\sphinxhref{https://docs.arduino.cc/hardware/uno-r4-wifi}{Arduino UNO R4 WiFi Documentation}

\item {} 
\sphinxAtStartPar
\sphinxhref{https://docs.arduino.cc/tutorials/uno-r4-wifi/cheat-sheet}{Arduino UNO R4 WiFi Cheat Sheet}

\end{itemize}

\sphinxstepscope


\section{Breadboard}
\label{\detokenize{Components_Kit/component_breadboard:breadboard}}\label{\detokenize{Components_Kit/component_breadboard:cpn-breadboard}}\label{\detokenize{Components_Kit/component_breadboard::doc}}
\noindent\sphinxincludegraphics[width=600\sphinxpxdimen]{{breadboard}.png}

\sphinxAtStartPar
A breadboard is a construction base for prototyping of electronics. Originally the word referred to a literal bread board, a polished piece of wood used for slicing bread. In the 1970s the solderless breadboard (a.k.a. plugboard, a terminal array board) became available and nowadays the term “breadboard” is commonly used to refer to these.

\sphinxAtStartPar
It is used to build and test circuits quickly before finishing any circuit design.
And it has many holes into which components mentioned above can be inserted like ICs and resistors as well as jumper wires.
The breadboard allows you to plug in and remove components easily.

\sphinxAtStartPar
The picture shows the internal structure of a breadboard.
Although these holes on the breadboard appear to be independent of each other, they are actually connected to each other through metal strips internally.

\noindent\sphinxincludegraphics[width=600\sphinxpxdimen]{{breadboard_internal}.png}

\begin{sphinxadmonition}{note}{Note:}
\sphinxAtStartPar
G1 and G2, V1 and V2 in the breadboard are not connected. Usually you need to use jumper wires to connect them.
\end{sphinxadmonition}

\noindent{\hspace*{\fill}\sphinxincludegraphics[width=0.850\linewidth]{{breadboard_internal1}.png}\hspace*{\fill}}

\sphinxAtStartPar
If you want to know more about breadboard, refer to: \sphinxhref{https://www.sciencebuddies.org/science-fair-projects/references/how-to-use-a-breadboard}{How to Use a Breadboard for Electronics and Circuits}

\sphinxstepscope


\section{Jumper Wires}
\label{\detokenize{Components_Kit/component_wires:jumper-wires}}\label{\detokenize{Components_Kit/component_wires:cpn-wires}}\label{\detokenize{Components_Kit/component_wires::doc}}
\sphinxAtStartPar
Wires that link two terminals are known as jumper wires. There are various types of jumper wires, but here we will focus on those commonly used with breadboards. They serve to transmit electrical signals from any point on the breadboard to the input/output pins of a microcontroller.

\sphinxAtStartPar
Jumper wires are connected by inserting their “end connectors” into the slots of the breadboard. Underneath the breadboard’s surface, there are sets of parallel plates that connect these slots in groups of rows or columns, depending on the section of the board. The “end connectors” are placed into the desired slots on the breadboard without soldering, making connections as needed for the specific prototype.

\sphinxAtStartPar
There are three main types of jumper wires: Female\sphinxhyphen{}to\sphinxhyphen{}Female, Male\sphinxhyphen{}to\sphinxhyphen{}Male, and Male\sphinxhyphen{}to\sphinxhyphen{}Female. The names describe the connectors on each end. Male\sphinxhyphen{}to\sphinxhyphen{}Female wires have a protruding pin on one end and a recessed female connector on the other. Male\sphinxhyphen{}to\sphinxhyphen{}Male wires have pins on both ends, while Female\sphinxhyphen{}to\sphinxhyphen{}Female wires have sockets on both ends.

\noindent\sphinxincludegraphics{{Jumper_Wires}.png}

\sphinxAtStartPar
Multiple types of jumper wires can be utilized in a single project. Although jumper wires come in various colors, these colors do not indicate differences in their function. The colors are simply a design feature to help distinguish and identify the connections between different parts of the circuit more easily.

\sphinxstepscope


\section{Resistor}
\label{\detokenize{Components_Kit/component_resistor:resistor}}\label{\detokenize{Components_Kit/component_resistor:component-resistor}}\label{\detokenize{Components_Kit/component_resistor::doc}}
\noindent\sphinxincludegraphics[width=300\sphinxpxdimen]{{resistor}.png}

\sphinxAtStartPar
A resistor is an electronic component that restricts the flow of current in a circuit. A fixed resistor has a set resistance value that cannot be altered, whereas a potentiometer or variable resistor has an adjustable resistance.

\sphinxAtStartPar
There are two commonly used symbols to represent resistors in circuit diagrams. The resistance value is typically indicated on the resistor itself. When you encounter these symbols in a circuit schematic, they denote a resistor.

\noindent\sphinxincludegraphics[width=400\sphinxpxdimen]{{resistor_symbol}.png}

\sphinxAtStartPar
\sphinxstylestrong{Ω} is the unit of resistance and the larger units include KΩ, MΩ, etc.
Their relationship can be shown as follows: 1 MΩ=1000 KΩ, 1 KΩ = 1000 Ω. Normally, the value of resistance is marked on it.

\sphinxAtStartPar
When using a resistor, we need to know its resistance first. Here are two methods: you can observe the bands on the resistor, or use a multimeter to measure the resistance. You are recommended to use the first method as it is more convenient and faster.

\noindent\sphinxincludegraphics{{resistance_card}.jpg}

\sphinxAtStartPar
As shown in the card, each color stands for a number.


\begin{savenotes}\sphinxattablestart
\sphinxthistablewithglobalstyle
\centering
\begin{tabulary}{\linewidth}[t]{TTTTTTTTTTTT}
\sphinxtoprule
\sphinxtableatstartofbodyhook
\sphinxAtStartPar
Black
&
\sphinxAtStartPar
Brown
&
\sphinxAtStartPar
Red
&
\sphinxAtStartPar
Orange
&
\sphinxAtStartPar
Yellow
&
\sphinxAtStartPar
Green
&
\sphinxAtStartPar
Blue
&
\sphinxAtStartPar
Violet
&
\sphinxAtStartPar
Grey
&
\sphinxAtStartPar
White
&
\sphinxAtStartPar
Gold
&
\sphinxAtStartPar
Silver
\\
\sphinxhline
\sphinxAtStartPar
0
&
\sphinxAtStartPar
1
&
\sphinxAtStartPar
2
&
\sphinxAtStartPar
3
&
\sphinxAtStartPar
4
&
\sphinxAtStartPar
5
&
\sphinxAtStartPar
6
&
\sphinxAtStartPar
7
&
\sphinxAtStartPar
8
&
\sphinxAtStartPar
9
&
\sphinxAtStartPar
0.1
&
\sphinxAtStartPar
0.01
\\
\sphinxbottomrule
\end{tabulary}
\sphinxtableafterendhook\par
\sphinxattableend\end{savenotes}

\sphinxAtStartPar
The 4\sphinxhyphen{} and 5\sphinxhyphen{}band resistors are frequently used, on which there are 4 and 5 chromatic bands.

\sphinxAtStartPar
Normally, when you get a resistor, you may find it hard to decide which end to start for reading the color.
The tip is that the gap between the 4th and 5th band will be comparatively larger.

\sphinxAtStartPar
Therefore, you can observe the gap between the two chromatic bands at one end of the resistor;
if it’s larger than any other band gaps, then you can read from the opposite side.

\sphinxAtStartPar
Let’s see how to read the resistance value of a 5\sphinxhyphen{}band resistor as shown below.

\noindent\sphinxincludegraphics[width=500\sphinxpxdimen]{{220ohm}.jpg}

\sphinxAtStartPar
So for this resistor, the resistance should be read from left to right.
The value should be in this format: 1st Band 2nd Band 3rd Band x 10\textasciicircum{}Multiplier (Ω) and the permissible error is \(\pm\)Tolerance\%.
So the resistance value of this resistor is 2(red) 2(red) 0(black) x 10\textasciicircum{}0(black) Ω = 220 Ω,
and the permissible error is \(\pm\) 1\% (brown).

\sphinxAtStartPar
You can learn more about resistor from Wiki: \sphinxhref{https://en.wikipedia.org/wiki/Resistor}{Resistor \sphinxhyphen{} Wikipedia}.

\sphinxstepscope


\section{LED}
\label{\detokenize{Components_Kit/component_led:led}}\label{\detokenize{Components_Kit/component_led:cpn-led}}\label{\detokenize{Components_Kit/component_led::doc}}
\noindent\sphinxincludegraphics[width=400\sphinxpxdimen]{{LED}.png}

\sphinxAtStartPar
A semiconductor light\sphinxhyphen{}emitting diode (LED) is a component that converts electrical energy into light energy through PN junctions. LEDs can be classified by wavelength into laser diodes, infrared LEDs, and visible LEDs, the latter being commonly referred to simply as LEDs.

\sphinxAtStartPar
Due to the diode’s unidirectional conductivity, current flows in the direction of the arrow shown in its circuit symbol. To illuminate the LED, you must supply positive voltage to the anode and negative voltage to the cathode.

\noindent\sphinxincludegraphics{{led_symbol}.png}

\sphinxAtStartPar
An LED has two pins. The longer one is the anode, and shorter one, the cathode. Pay attention not to connect them inversely. There is fixed forward voltage drop in the LED, so it cannot be connected with the circuit directly because the supply voltage can outweigh this drop and cause the LED to be burnt. The forward voltage of the red, yellow, and green LED is 1.8 V and that of the white one is 2.6 V. Most LEDs can withstand a maximum current of 20 mA, so we need to connect a current limiting resistor in series.

\sphinxAtStartPar
The formula of the resistance value is as follows:
\begin{quote}

\sphinxAtStartPar
R = (Vsupply \textendash{} VD)/I
\end{quote}

\sphinxAtStartPar
\sphinxstylestrong{R} stands for the resistance value of the current limiting resistor, \sphinxstylestrong{Vsupply} for voltage supply, \sphinxstylestrong{VD} for voltage drop and \sphinxstylestrong{I} for the working current of the LED.

\sphinxAtStartPar
Here is the detailed introduction for the LED: \sphinxhref{https://en.wikipedia.org/wiki/Light-emitting\_diode}{led\_wiki}.

\sphinxAtStartPar
\sphinxstylestrong{Example}
\begin{itemize}
\item {} 
\sphinxAtStartPar
{\hyperref[\detokenize{Basic_Project/LED_Blink:basic-led-blink}]{\sphinxcrossref{\DUrole{std}{\DUrole{std-ref}{LED Blink}}}}} (Basic Project)

\item {} 
\sphinxAtStartPar
{\hyperref[\detokenize{Basic_Project/Button_LED:basic-button-led}]{\sphinxcrossref{\DUrole{std}{\DUrole{std-ref}{Button Control LED}}}}} (Basic Project)

\item {} 
\sphinxAtStartPar
{\hyperref[\detokenize{Basic_Project/2_Channel_Relay_Module:basic-2-channel-relay-module}]{\sphinxcrossref{\DUrole{std}{\DUrole{std-ref}{2 Channel Relay Module}}}}} (Basic Project)

\end{itemize}

\sphinxstepscope


\section{RGB LED}
\label{\detokenize{Components_Kit/component_rgb_led:rgb-led}}\label{\detokenize{Components_Kit/component_rgb_led:cpn-rgb-led}}\label{\detokenize{Components_Kit/component_rgb_led::doc}}
\noindent\sphinxincludegraphics[width=100\sphinxpxdimen]{{rgb_led}.png}

\sphinxAtStartPar
RGB LEDs can emit light in a range of colors. They combine three LEDs—red, green, and blue—within a single transparent or semitransparent plastic casing. By adjusting the input voltage to each of the three pins, you can mix these colors to produce a wide spectrum of hues. According to statistics, this allows for the creation of up to 16,777,216 different colors.

\noindent\sphinxincludegraphics[width=600\sphinxpxdimen]{{rgb_light}.png}

\sphinxAtStartPar
RGB LEDs are available in two main types: common anode and common cathode. This kit includes the common cathode type. In a \sphinxstylestrong{common cathode} (CC) configuration, the cathodes of the three LEDs are connected together. When you connect this shared cathode to GND and apply voltage to the three individual pins, the LED will emit the corresponding colors.

\sphinxAtStartPar
Its circuit symbol is shown as figure.

\noindent\sphinxincludegraphics[width=300\sphinxpxdimen]{{rgb_symbol}.png}

\sphinxAtStartPar
An RGB LED has 4 pins: the longest one is GND; the others are Red, Green and Blue. Touch its plastic shell and you will find a cut. The pin closest to the cut is the first pin, marked as Red, then GND, Green and Blue in turn.

\noindent\sphinxincludegraphics[width=200\sphinxpxdimen]{{rgb_pin}.jpg}

\sphinxAtStartPar
\sphinxstylestrong{Example}
\begin{itemize}
\item {} 
\sphinxAtStartPar
{\hyperref[\detokenize{Basic_Project/RGB_LED:basic-rgb-led}]{\sphinxcrossref{\DUrole{std}{\DUrole{std-ref}{RGB LED}}}}} (Basic Project)

\end{itemize}

\sphinxstepscope


\section{Transistor}
\label{\detokenize{Components_Kit/component_transistor:transistor}}\label{\detokenize{Components_Kit/component_transistor:cpn-transistor}}\label{\detokenize{Components_Kit/component_transistor::doc}}
\noindent\sphinxincludegraphics[width=300\sphinxpxdimen]{{npn_pnp}.png}

\sphinxAtStartPar
A transistor is a semiconductor device that regulates current by using another current. It amplifies weak signals into stronger ones and can also serve as a non\sphinxhyphen{}contact switch.

\sphinxAtStartPar
A transistor consists of a three\sphinxhyphen{}layer structure made of P\sphinxhyphen{}type and N\sphinxhyphen{}type semiconductors, creating three distinct regions. The thin middle layer is the base region, while the other two layers are either N\sphinxhyphen{}type or P\sphinxhyphen{}type semiconductors. The region with a high concentration of majority carriers is the emitter region, and the other region is the collector. This structure allows the transistor to function as an amplifier. The three regions correspond to three terminals: the base (b), emitter (e), and collector (c). These regions form two P\sphinxhyphen{}N junctions, known as the emitter junction and the collector junction. The arrow in the transistor circuit symbol indicates the direction of the emitter junction.
\begin{itemize}
\item {} 
\sphinxAtStartPar
\sphinxhref{https://en.wikipedia.org/wiki/P-n\_junction}{P\textendash{}N junction \sphinxhyphen{} Wikipedia}

\end{itemize}

\sphinxAtStartPar
Based on the semiconductor type, transistors can be divided into two groups, the NPN and PNP ones. From the abbreviation, we can tell that the former is made of two N\sphinxhyphen{}type semiconductors and one P\sphinxhyphen{}type and that the latter is the opposite. See the figure below.

\begin{sphinxadmonition}{note}{Note:}
\sphinxAtStartPar
s8550 is PNP transistor and the s8050 is the NPN one, They look very similar, and we need to check carefully to see their labels.
\end{sphinxadmonition}

\noindent\sphinxincludegraphics[width=600\sphinxpxdimen]{{transistor_symbol}.png}

\sphinxAtStartPar
When a High level signal goes through an NPN transistor, it is energized. But a PNP one needs a Low level signal to manage it. Both types of transistor are frequently used for contactless switches, just like in this experiment.

\sphinxAtStartPar
Put the label side facing us and the pins facing down. The pins from left to right are emitter(e), base(b), and collector(c).

\noindent\sphinxincludegraphics[width=150\sphinxpxdimen]{{ebc}.png}
\begin{itemize}
\item {} 
\sphinxAtStartPar
\sphinxhref{https://datasheet4u.com/datasheet-pdf/WeitronTechnology/S8050/pdf.php?id=576670}{S8050 Transistor Datasheet}

\item {} 
\sphinxAtStartPar
\sphinxhref{https://www.mouser.com/datasheet/2/149/SS8550-118608.pdf}{S8550 Transistor Datasheet}

\end{itemize}

\sphinxAtStartPar
\sphinxstylestrong{Example}
\begin{itemize}
\item {} 
\sphinxAtStartPar
{\hyperref[\detokenize{Basic_Project/Active_Buzzer:basic-active-buzzer}]{\sphinxcrossref{\DUrole{std}{\DUrole{std-ref}{Active Buzzer}}}}} (Basic Project)

\item {} 
\sphinxAtStartPar
{\hyperref[\detokenize{Basic_Project/Passive_Buzzer:basic-passive-buzzer}]{\sphinxcrossref{\DUrole{std}{\DUrole{std-ref}{Passive Buzzer}}}}} (Basic Project)

\end{itemize}

\sphinxstepscope


\section{Buzzer}
\label{\detokenize{Components_Kit/component_buzzer:buzzer}}\label{\detokenize{Components_Kit/component_buzzer:cpn-buzzer}}\label{\detokenize{Components_Kit/component_buzzer::doc}}
\noindent{\hspace*{\fill}\sphinxincludegraphics[width=0.600\linewidth]{{buzzer2}.png}\hspace*{\fill}}

\sphinxAtStartPar
Electronic buzzers, featuring a compact design, are versatile components powered by direct current (DC) and are commonly found across a variety of electronic devices. They serve as auditory indicators in applications such as computers, printers, photocopiers, security systems, electronic playthings, vehicular electronics, telephones, and timing devices, among others.

\sphinxAtStartPar
Classification of buzzers is based on their functionality, distinguishing between active and passive types (as illustrated in the accompanying diagram). When you position the buzzer with its terminals facing upwards, the one featuring a green printed circuit board is identified as a passive type, whereas the one with an outer layer of black tape is classified as an active buzzer.

\noindent{\hspace*{\fill}\sphinxincludegraphics[width=0.500\linewidth]{{buzzer1}.png}\hspace*{\fill}}

\sphinxAtStartPar
The operational distinction between active and passive buzzers is characterized by their internal circuitry and activation requirements:
\begin{itemize}
\item {} 
\sphinxAtStartPar
\sphinxstylestrong{Active Buzzers} are self\sphinxhyphen{}contained, featuring an integrated oscillating circuit that produces sound immediately when powered by DC. Their design eliminates the need for external signal modulation to generate audible tones.

\item {} 
\sphinxAtStartPar
\sphinxstylestrong{Passive Buzzers} lack an internal oscillating circuit, and thus, are non\sphinxhyphen{}responsive to DC signals in terms of sound output. To induce sound from a passive buzzer, a specific frequency range of square wave signals, typically between 2 kHz and 5 kHz, must be supplied.

\end{itemize}

\sphinxAtStartPar
The presence of these internal oscillating circuits in active buzzers contributes to their higher cost in comparison to passive models, which require additional circuitry for sound emission.

\sphinxAtStartPar
Regarding the schematic representation, a buzzer is depicted with a simple symbol that includes two terminals to differentiate the positive and negative poles. The terminal indicated with a “+” is recognized as the anode, while its counterpart is identified as the cathode, guiding the correct orientation during circuit assembly.

\noindent\sphinxincludegraphics[width=150\sphinxpxdimen]{{buzzer_symbol}.png}

\sphinxAtStartPar
You can check the pins of the buzzer, the longer one is the anode and the shorter one is the cathode. Please don’t mix them up when connecting, otherwise the buzzer will not make sound.

\sphinxAtStartPar
\sphinxhref{https://en.wikipedia.org/wiki/Buzzer}{Buzzer Wiki}

\sphinxAtStartPar
\sphinxstylestrong{Example}
\begin{itemize}
\item {} 
\sphinxAtStartPar
{\hyperref[\detokenize{Basic_Project/Active_Buzzer:basic-active-buzzer}]{\sphinxcrossref{\DUrole{std}{\DUrole{std-ref}{Active Buzzer}}}}} (Basic Project)

\item {} 
\sphinxAtStartPar
{\hyperref[\detokenize{Basic_Project/Passive_Buzzer:basic-passive-buzzer}]{\sphinxcrossref{\DUrole{std}{\DUrole{std-ref}{Passive Buzzer}}}}} (Basic Project)

\end{itemize}

\sphinxstepscope


\section{Button}
\label{\detokenize{Components_Kit/component_button:button}}\label{\detokenize{Components_Kit/component_button:cpn-button}}\label{\detokenize{Components_Kit/component_button::doc}}
\noindent{\hspace*{\fill}\sphinxincludegraphics[width=400\sphinxpxdimen]{{button}.png}\hspace*{\fill}}

\sphinxAtStartPar
Buttons are essential components used to control electronic devices, typically functioning as switches to either complete or interrupt circuits. Despite their variety in sizes and shapes, the 6mm mini\sphinxhyphen{}button depicted in the accompanying images is our focus here. Within this button, pin 1 is connected to pin 2, and pin 3 is connected to pin 4. Therefore, you only need to establish a connection between pin 1 (or pin 2) and pin 3 (or pin 4) to operate it.

\sphinxAtStartPar
The internal structure of such a button is illustrated below. The symbol shown on the right is commonly used to represent a button in circuit diagrams.

\noindent{\hspace*{\fill}\sphinxincludegraphics[width=400\sphinxpxdimen]{{button_symbol}.png}\hspace*{\fill}}

\sphinxAtStartPar
Since the pin 1 is connected to pin 2, and pin 3 to pin 4, when the button is pressed, the 4 pins are connected, thus closing the circuit.

\sphinxAtStartPar
In this kit, we provide two types of buttons. The one mentioned earlier is a small button, and there is also a large button. They have the same principle, only different in size.

\noindent{\hspace*{\fill}\sphinxincludegraphics[width=400\sphinxpxdimen]{{button3}.png}\hspace*{\fill}}

\sphinxAtStartPar
\sphinxstylestrong{Example}
\begin{itemize}
\item {} 
\sphinxAtStartPar
{\hyperref[\detokenize{Basic_Project/Button_LED:basic-button-led}]{\sphinxcrossref{\DUrole{std}{\DUrole{std-ref}{Button Control LED}}}}} (Basic Project)

\item {} 
\sphinxAtStartPar
{\hyperref[\detokenize{Extension_Project/Ping-Pong_Game:ext-ping-pong-game}]{\sphinxcrossref{\DUrole{std}{\DUrole{std-ref}{Ping\sphinxhyphen{}Pong Game}}}}} (Extension\_Project)

\end{itemize}

\sphinxstepscope


\section{Potentiometer}
\label{\detokenize{Components_Kit/component_potentiometer:potentiometer}}\label{\detokenize{Components_Kit/component_potentiometer:cpn-potentiometer}}\label{\detokenize{Components_Kit/component_potentiometer::doc}}
\noindent{\hspace*{\fill}\sphinxincludegraphics[width=150\sphinxpxdimen]{{potentiometer}.png}\hspace*{\fill}}

\sphinxAtStartPar
A potentiometer is another type of resistance component featuring three terminals, with an adjustable resistance value that follows a specific pattern of variation.

\sphinxAtStartPar
Regardless of their different shapes, sizes, and resistance values, all potentiometers share the following characteristics:
\begin{enumerate}
\sphinxsetlistlabels{\arabic}{enumi}{enumii}{}{.}%
\item {} 
\sphinxAtStartPar
They possess three terminals (or connection points).

\item {} 
\sphinxAtStartPar
They include a knob, screw, or slider that can be adjusted to change the resistance between the middle terminal and either of the outer terminals.

\item {} 
\sphinxAtStartPar
As the knob, screw, or slider is moved, the resistance between the middle terminal and either outer terminal can vary from 0 Ω to the potentiometer’s maximum resistance.

\end{enumerate}

\sphinxAtStartPar
Below is the circuit symbol typically used to represent a potentiometer.

\noindent{\hspace*{\fill}\sphinxincludegraphics[width=400\sphinxpxdimen]{{potentiometer_symbol}.png}\hspace*{\fill}}

\sphinxAtStartPar
The functions of a potentiometer in a circuit include:
\begin{enumerate}
\sphinxsetlistlabels{\arabic}{enumi}{enumii}{}{.}%
\item {} 
\sphinxAtStartPar
\sphinxstylestrong{Serving as a Voltage Divider}
\begin{quote}

\sphinxAtStartPar
A potentiometer acts as a continuously adjustable resistor. When you adjust the shaft or sliding handle, the movable contact slides along the resistor. This allows the output voltage to vary depending on the applied voltage and the position of the movable arm. This function is commonly used to derive a specific voltage from a larger range.
\end{quote}

\item {} 
\sphinxAtStartPar
\sphinxstylestrong{Serving as a Rheostat}
\begin{quote}

\sphinxAtStartPar
When used as a rheostat, the potentiometer can be connected to the circuit by using the middle pin and one of the other two pins. This configuration enables you to obtain a smoothly and continuously variable resistance value within the range of the moving contact’s travel. This function is often used for adjusting current or resistance in a circuit.
\end{quote}

\item {} 
\sphinxAtStartPar
\sphinxstylestrong{Serving as a Current Controller}
\begin{quote}

\sphinxAtStartPar
For a potentiometer to function as a current controller, the sliding contact terminal must be used as one of the output terminals. This setup allows for the regulation of the current flowing through the circuit by adjusting the position of the sliding contact.

\sphinxAtStartPar
If you want to know more about potentiometer, refer to: \sphinxhref{https://en.wikipedia.org/wiki/Potentiometer}{Potentiometer Wiki}
\end{quote}

\end{enumerate}

\sphinxAtStartPar
\sphinxstylestrong{Example}
\begin{itemize}
\item {} 
\sphinxAtStartPar
{\hyperref[\detokenize{Basic_Project/Potentiometer:basic-potentiometer}]{\sphinxcrossref{\DUrole{std}{\DUrole{std-ref}{Potentiometer}}}}} (Basic Project)

\end{itemize}

\sphinxstepscope


\section{Photoresistor}
\label{\detokenize{Components_Kit/component_photoresistor:photoresistor}}\label{\detokenize{Components_Kit/component_photoresistor:cpn-photoresistor}}\label{\detokenize{Components_Kit/component_photoresistor::doc}}
\noindent{\hspace*{\fill}\sphinxincludegraphics[width=200\sphinxpxdimen]{{photoresistor}.png}\hspace*{\fill}}

\sphinxAtStartPar
Photoresistor is simply a light sensitive resistor. It is an active component that decreases resistance with respect to receiving luminosity (light) on the component’s light sensitive surface. Photoresistor’s resistance value will change in proportion to the ambient light detected. With this characteristic, we can use a Photoresistor to detect light intensity. The Photoresistor and its electronic symbol are as follows.

\noindent{\hspace*{\fill}\sphinxincludegraphics[width=200\sphinxpxdimen]{{photoresistor_symbol}.png}\hspace*{\fill}}
\begin{itemize}
\item {} 
\sphinxAtStartPar
\sphinxhref{https://en.wikipedia.org/wiki/Photoresistor}{Photoresistor Wiki}

\end{itemize}

\sphinxAtStartPar
\sphinxstylestrong{Example}
\begin{itemize}
\item {} 
\sphinxAtStartPar
{\hyperref[\detokenize{Basic_Project/Photoresistor:basic-photoresistor}]{\sphinxcrossref{\DUrole{std}{\DUrole{std-ref}{Photoresistor}}}}} (Basic Project)

\end{itemize}

\sphinxstepscope


\section{Thermistor}
\label{\detokenize{Components_Kit/component_thermistor:thermistor}}\label{\detokenize{Components_Kit/component_thermistor:component-thermistor}}\label{\detokenize{Components_Kit/component_thermistor::doc}}
\noindent{\hspace*{\fill}\sphinxincludegraphics[width=150\sphinxpxdimen]{{thermistor}.png}\hspace*{\fill}}

\sphinxAtStartPar
A Thermistor is a temperature sensitive resistor. When it senses a change in temperature, the resistance of the Thermistor will change. We can take advantage of this characteristic by using a Thermistor to detect temperature intensity. A Thermistor and its electronic symbol are shown below.
\begin{itemize}
\item {} 
\sphinxAtStartPar
\sphinxhref{https://en.wikipedia.org/wiki/Thermistor}{Thermistor Wiki}

\end{itemize}

\sphinxAtStartPar
Here is the electronic symbol of thermistor.

\noindent{\hspace*{\fill}\sphinxincludegraphics[width=300\sphinxpxdimen]{{thermistor_symbol}.png}\hspace*{\fill}}

\sphinxAtStartPar
Thermistors are of two opposite fundamental types:
\begin{itemize}
\item {} 
\sphinxAtStartPar
With NTC thermistors, resistance decreases as temperature rises usually due to an increase in conduction electrons bumped up by thermal agitation from valency band. An NTC is commonly used as a temperature sensor, or in series with a circuit as an inrush current limiter.

\item {} 
\sphinxAtStartPar
With PTC thermistors, resistance increases as temperature rises usually due to increased thermal lattice agitations particularly those of impurities and imperfections. PTC thermistors are commonly installed in series with a circuit, and used to protect against overcurrent conditions, as resettable fuses.

\end{itemize}

\sphinxAtStartPar
In this kit we use an NTC one. Each thermistor has a normal resistance. Here it is 10k ohm, which is measured under 25 degree Celsius.

\sphinxAtStartPar
Here is the relation between the resistance and temperature:
\begin{quote}

\sphinxAtStartPar
RT = RN * expB(1/TK \textendash{} 1/TN)
\end{quote}
\begin{itemize}
\item {} 
\sphinxAtStartPar
\sphinxstylestrong{RT} is the resistance of the NTC thermistor when the temperature is TK.

\item {} 
\sphinxAtStartPar
\sphinxstylestrong{RN} is the resistance of the NTC thermistor under the rated temperature TN. Here, the numerical value of RN is 10k.

\item {} 
\sphinxAtStartPar
\sphinxstylestrong{TK} is a Kelvin temperature and the unit is K. Here, the numerical value of TK is 273.15 + degree Celsius.

\item {} 
\sphinxAtStartPar
\sphinxstylestrong{TN} is a rated Kelvin temperature; the unit is K too. Here, the numerical value of TN is 273.15+25.

\item {} 
\sphinxAtStartPar
And \sphinxstylestrong{B(beta)}, the material constant of NTC thermistor, is also called heat sensitivity index with a numerical value 3950.

\item {} 
\sphinxAtStartPar
\sphinxstylestrong{exp} is the abbreviation of exponential, and the base number e is a natural number and equals 2.7 approximately.

\end{itemize}

\sphinxAtStartPar
Convert this formula TK=1/(ln(RT/RN)/B+1/TN) to get Kelvin temperature that minus 273.15 equals degree Celsius.

\sphinxAtStartPar
This relation is an empirical formula. It is accurate only when the temperature and resistance are within the effective range.

\sphinxAtStartPar
\sphinxstylestrong{Example}
\begin{itemize}
\item {} 
\sphinxAtStartPar
{\hyperref[\detokenize{Basic_Project/Thermistor:basic-thermistor}]{\sphinxcrossref{\DUrole{std}{\DUrole{std-ref}{Thermistor}}}}} (Basic Project)

\end{itemize}

\sphinxstepscope


\section{Tilt Switch}
\label{\detokenize{Components_Kit/component_tilt_switch:tilt-switch}}\label{\detokenize{Components_Kit/component_tilt_switch:cpn-tilt-switch}}\label{\detokenize{Components_Kit/component_tilt_switch::doc}}
\noindent{\hspace*{\fill}\sphinxincludegraphics[width=80\sphinxpxdimen]{{tilt_switch}.png}\hspace*{\fill}}

\sphinxAtStartPar
The tilt switch used here is a ball\sphinxhyphen{}type switch containing a metal ball inside. It is designed to detect slight inclinations.

\sphinxAtStartPar
The operating principle is straightforward. When the switch is tilted at a specific angle, the internal ball rolls and makes contact with the two terminals connected to the external pins, thereby completing the circuit. If the switch is not tilted, the ball stays away from the contacts, breaking the circuit.

\noindent\sphinxincludegraphics[width=600\sphinxpxdimen]{{tilt_symbol}.png}

\sphinxAtStartPar
\sphinxstylestrong{Example}
\begin{itemize}
\item {} 
\sphinxAtStartPar
{\hyperref[\detokenize{Extension_Project/Digital_Dice_LED_Matrix:ext-digital-dice-led-matrix}]{\sphinxcrossref{\DUrole{std}{\DUrole{std-ref}{Digital Dice LED Matrix}}}}} (Extension\_Project)

\end{itemize}

\sphinxstepscope


\section{Audio Module and Speaker}
\label{\detokenize{Components_Kit/component_audio_speaker:audio-module-and-speaker}}\label{\detokenize{Components_Kit/component_audio_speaker:cpn-audio-speaker}}\label{\detokenize{Components_Kit/component_audio_speaker::doc}}
\sphinxAtStartPar
\sphinxstylestrong{Speaker}

\noindent{\hspace*{\fill}\sphinxincludegraphics[width=300\sphinxpxdimen]{{speaker_pic}.png}\hspace*{\fill}}
\begin{itemize}
\item {} 
\sphinxAtStartPar
\sphinxstylestrong{Size}: 20x30x7mm

\item {} 
\sphinxAtStartPar
\sphinxstylestrong{Impedance}:8ohm

\item {} 
\sphinxAtStartPar
\sphinxstylestrong{Max Input Power}: 2.0W

\item {} 
\sphinxAtStartPar
\sphinxstylestrong{Wire Length}: 10cm

\end{itemize}

\noindent\sphinxincludegraphics{{2030_speaker}.png}

\sphinxAtStartPar
\sphinxstylestrong{Audio Amplifier Module}

\noindent{\hspace*{\fill}\sphinxincludegraphics[width=500\sphinxpxdimen]{{audio_module}.jpg}\hspace*{\fill}}

\sphinxAtStartPar
The Audio Amplifier Module incorporates the HXJ8002 audio power amplifier chip, known for its low power supply requirements. This chip is capable of delivering an average audio power of 3W to a 3Ω BTL load, maintaining low harmonic distortion (below 10\% at 1KHz) when powered by a 5V DC supply. It amplifies audio signals efficiently without the need for coupling or bootstrap capacitors.

\sphinxAtStartPar
The module can be powered by a DC source ranging from 2.0V to 5.5V, with an operating current of 10mA and a typical standby current of just 0.6uA. It effectively drives speakers with impedances of 3Ω, 4Ω, or 8Ω, producing robust sound output. Enhanced pop and click circuitry minimizes noise during power transitions, making this module ideal for portable and battery\sphinxhyphen{}operated projects as well as microcontroller applications due to its compact size, high efficiency, and low power consumption.
\begin{itemize}
\item {} 
\sphinxAtStartPar
\sphinxstylestrong{IC}: HXJ8002

\item {} 
\sphinxAtStartPar
\sphinxstylestrong{Input Voltage}: 2V \textasciitilde{} 5.5V

\item {} 
\sphinxAtStartPar
\sphinxstylestrong{Standby Mode Current}: 0.6uA (typical value)

\item {} 
\sphinxAtStartPar
\sphinxstylestrong{Output Power}: 3W (3Ω load) , 2.5W (4Ω load) , 1.5W (8Ω load)

\item {} 
\sphinxAtStartPar
\sphinxstylestrong{Output Speaker Impedance}: 3Ω, 4Ω, 8Ω

\item {} 
\sphinxAtStartPar
\sphinxstylestrong{Size}: 19.8mm x 14.2mm

\end{itemize}

\sphinxAtStartPar
\sphinxstylestrong{Example}
\begin{itemize}
\item {} 
\sphinxAtStartPar
{\hyperref[\detokenize{Basic_Project/Audio_Module_Speaker:basic-audio-module-speaker}]{\sphinxcrossref{\DUrole{std}{\DUrole{std-ref}{Audio Module and Speaker}}}}} (Basic Project)

\end{itemize}

\sphinxstepscope


\section{2 Channel Relay Module}
\label{\detokenize{Components_Kit/component_relay:channel-relay-module}}\label{\detokenize{Components_Kit/component_relay:cpn-realy}}\label{\detokenize{Components_Kit/component_relay::doc}}
\noindent{\hspace*{\fill}\sphinxincludegraphics[width=400\sphinxpxdimen]{{relay_1}.png}\hspace*{\fill}}

\sphinxAtStartPar
On the left side, there are two sets of three sockets to connect high voltages, and the pins on the right side (low\sphinxhyphen{}voltage) connect to the Arduino GPIOs.

\sphinxAtStartPar
\sphinxstylestrong{Mains Voltage Connections}

\noindent{\hspace*{\fill}\sphinxincludegraphics[width=400\sphinxpxdimen]{{relay_2}.png}\hspace*{\fill}}

\sphinxAtStartPar
The relay module shown in the previous photo has two connectors, each with three sockets: common (COM), Normally Closed (NC), and Normally Open (NO).
\begin{itemize}
\item {} 
\sphinxAtStartPar
COM: connect the current you want to control (mains voltage).

\item {} 
\sphinxAtStartPar
NC (Normally Closed): the normally closed configuration is used when you want the relay to be closed by default. The NC are COM pins are connected, meaning the current is flowing unless you send a signal from the Arduino to the relay module to open the circuit and stop the current flow.

\item {} 
\sphinxAtStartPar
NO (Normally Open): the normally open configuration works the other way around: there is no connection between the NO and COM pins, so the circuit is broken unless you send a signal from the Arduino to close the circuit.

\end{itemize}

\sphinxAtStartPar
\sphinxstylestrong{Control Pins}

\noindent{\hspace*{\fill}\sphinxincludegraphics[width=400\sphinxpxdimen]{{relay_3}.png}\hspace*{\fill}}

\sphinxAtStartPar
The low\sphinxhyphen{}voltage side has a set of four pins and a set of three pins. The first set consists of VCC and GND to power up the module, and input 1 (IN1) and input 2 (IN2) to control the bottom and top relays, respectively.
If your relay module only has one channel, you’ll have just one IN pin. If you have four channels, you’ll have four IN pins, and so on.
The signal you send to the IN pins, determines whether the relay is active or not. The relay is triggered when the input goes below about 2V. This means that you’ll have the following scenarios:
\begin{itemize}
\item {} 
\sphinxAtStartPar
Normally Closed configuration (NC):

\item {} 
\sphinxAtStartPar
HIGH signal \textendash{} current is flowing

\item {} 
\sphinxAtStartPar
LOW signal \textendash{} current is not flowing

\item {} 
\sphinxAtStartPar
Normally Open configuration (NO):

\item {} 
\sphinxAtStartPar
HIGH signal \sphinxhyphen{} current is not flowing

\item {} 
\sphinxAtStartPar
LOW signal \textendash{} current in flowing

\end{itemize}

\sphinxAtStartPar
You should use a normally closed configuration when the current should be flowing most of the times, and you only want to stop it occasionally.
Use a normally open configuration when you want the current to flow occasionally (for example, turn on a lamp occasionally).

\sphinxAtStartPar
\sphinxstylestrong{Power Supply Selection}

\noindent{\hspace*{\fill}\sphinxincludegraphics[width=400\sphinxpxdimen]{{relay_4}.png}\hspace*{\fill}}

\sphinxAtStartPar
The second set of pins consists of GND, VCC, and JD\sphinxhyphen{}VCC pins. The JD\sphinxhyphen{}VCC pin powers the electromagnet of the relay. Notice that the module has a jumper cap connecting the VCC and JD\sphinxhyphen{}VCC pins; the one shown here is yellow, but yours may be a different color.
With the jumper cap on, the VCC and JD\sphinxhyphen{}VCC pins are connected. That means the relay electromagnet is directly powered from the Arduino power pin, so the relay module and the Arduino circuits are not physically isolated from each other.
Without the jumper cap, you need to provide an independent power source to power up the relay’s electromagnet through the JD\sphinxhyphen{}VCC pin. That configuration physically isolates the relays from the Arduino with the module’s built\sphinxhyphen{}in optocoupler, which prevents damage to the Arduino in case of electrical spikes.

\sphinxAtStartPar
\sphinxstylestrong{Example}
\begin{itemize}
\item {} 
\sphinxAtStartPar
{\hyperref[\detokenize{Basic_Project/2_Channel_Relay_Module:basic-2-channel-relay-module}]{\sphinxcrossref{\DUrole{std}{\DUrole{std-ref}{2 Channel Relay Module}}}}} (Basic Project)

\end{itemize}

\sphinxstepscope


\section{Joystick Module}
\label{\detokenize{Components_Kit/Components_joystick_module:joystick-module}}\label{\detokenize{Components_Kit/Components_joystick_module:cpn-joystick}}\label{\detokenize{Components_Kit/Components_joystick_module::doc}}
\noindent{\hspace*{\fill}\sphinxincludegraphics[width=600\sphinxpxdimen]{{joystick_pic}.png}\hspace*{\fill}}

\sphinxAtStartPar
A joystick functions by converting the manual displacement of a lever into digital signals that computers can interpret.

\sphinxAtStartPar
For comprehensive motion tracking, the joystick captures the lever’s displacement across two perpendicular planes: the horizontal X\sphinxhyphen{}axis and the vertical Y\sphinxhyphen{}axis. Utilizing the principles of Cartesian coordinates, these axes work in tandem to define the lever’s precise location.

\noindent{\hspace*{\fill}\sphinxincludegraphics[width=600\sphinxpxdimen]{{joystick318}.png}\hspace*{\fill}}

\sphinxAtStartPar
The joystick’s mechanism assesses the lever’s placement by tracking the shafts’ orientation. Traditional analog joysticks achieve this through the use of potentiometers, which are adjustable resistors that provide variable output based on their position.

\sphinxAtStartPar
Additionally, the joystick is equipped with a digital button that triggers input when it is depressed, offering an additional layer of interaction with the computer system.

\sphinxAtStartPar
\sphinxstylestrong{Example}
\begin{itemize}
\item {} 
\sphinxAtStartPar
{\hyperref[\detokenize{Basic_Project/Joystick_Module:basic-joystick-module}]{\sphinxcrossref{\DUrole{std}{\DUrole{std-ref}{Joystick Module}}}}} (Basic Project)

\item {} 
\sphinxAtStartPar
{\hyperref[\detokenize{Extension_Project/Greedy_Snake_Game:ext-greedy-snake-game}]{\sphinxcrossref{\DUrole{std}{\DUrole{std-ref}{Greedy Snake Game}}}}} (Extension\_Project)

\end{itemize}

\sphinxstepscope


\section{PIR Motion Sensor Module}
\label{\detokenize{Components_Kit/component_pir:pir-motion-sensor-module}}\label{\detokenize{Components_Kit/component_pir:cpn-pir}}\label{\detokenize{Components_Kit/component_pir::doc}}
\sphinxAtStartPar
Passive infrared sensor (PIR sensor) is a common sensor that can measure infrared (IR) light emitted by objects in its field of view. Simply put, it will receive infrared radiation emitted from the body, thereby detecting the movement of people and other animals. More specifically, it tells the main control board that someone has entered your room.

\noindent{\hspace*{\fill}\sphinxincludegraphics[width=500\sphinxpxdimen]{{PIR1}.png}\hspace*{\fill}}

\sphinxAtStartPar
The PIR sensor detects infrared heat radiation that can be used to detect the presence of organisms that emit infrared heat radiation.

\sphinxAtStartPar
The PIR sensor is split into two slots that are connected to a differential amplifier. Whenever a stationary object is in front of the sensor, the two slots receive the same amount of radiation and the output is zero. Whenever a moving object is in front of the sensor, one of the slots receives more radiation than the other , which makes the output fluctuate high or low. This change in output voltage is a result of detection of motion.

\noindent\sphinxincludegraphics[width=800\sphinxpxdimen]{{PIR2}.png}

\begin{sphinxadmonition}{note}{Note:}
\sphinxAtStartPar
\sphinxstylestrong{importance}

\sphinxAtStartPar
After the sensing module is wired, there is a one\sphinxhyphen{}minute initialization.After initialization,then the module will be in the standby mode.
\sphinxstylestrong{During the initialization,do not let any triggered infrared signal appear in the PIR monitoring range, including your hand. Otherwise in standby mode, it may cause false trigger detection}.
During the initialization, module will output for 0\textasciitilde{}3 times at intervals.This is not a real trigger result and you can ignore it until standby mode .
\end{sphinxadmonition}

\noindent{\hspace*{\fill}\sphinxincludegraphics[width=500\sphinxpxdimen]{{PIR3}.png}\hspace*{\fill}}

\sphinxAtStartPar
\sphinxstylestrong{Distance Adjustment}

\sphinxAtStartPar
Turning the knob of the distance adjustment potentiometer clockwise, the range of sensing distance increases, and the maximum sensing distance range is about 0\sphinxhyphen{}7 meters. If turn it anticlockwise, the range of sensing distance is reduced, and the minimum sensing distance range is about 0\sphinxhyphen{}3 meters.

\sphinxAtStartPar
\sphinxstylestrong{Delay adjustment}

\sphinxAtStartPar
Rotate the knob of the delay adjustment potentiometer clockwise, you can also see the sensing delay increasing. The maximum of the sensing delay can reach up to 300s. On the contrary, if rotate it anticlockwise, you can shorten the delay with a minimum of 5s.

\sphinxAtStartPar
\sphinxstylestrong{Two Trigger Modes}

\sphinxAtStartPar
Choosing different modes by using the jumper cap.
\begin{itemize}
\item {} 
\sphinxAtStartPar
\sphinxstylestrong{H}: Repeatable trigger mode, after sensing the human body, the module outputs high level. During the subsequent delay period, if somebody enters the sensing range, the output will keep being the high level.

\item {} 
\sphinxAtStartPar
\sphinxstylestrong{L}: Non\sphinxhyphen{}repeatable trigger mode, outputs high level when it senses the human body. After the delay, the output will change from high level into low level automatically.

\end{itemize}

\sphinxAtStartPar
\sphinxstylestrong{Example}
\begin{itemize}
\item {} 
\sphinxAtStartPar
{\hyperref[\detokenize{Basic_Project/PIR_Motion_Sensor:basic-pir-motion-sensor}]{\sphinxcrossref{\DUrole{std}{\DUrole{std-ref}{PIR Motion Sensor Module}}}}} (Basic Project)

\end{itemize}

\sphinxstepscope


\section{DHT11 Module}
\label{\detokenize{Components_Kit/component_humiture_sensor:dht11-module}}\label{\detokenize{Components_Kit/component_humiture_sensor:cpn-dht11}}\label{\detokenize{Components_Kit/component_humiture_sensor::doc}}
\sphinxAtStartPar
This DHT11 Temperature and Humidity Sensor features calibrated digital signal output with the temperature and humidity sensor complex. Its technology ensures high reliability and excellent long\sphinxhyphen{}term stability. Ahigh\sphinxhyphen{}performance 8\sphinxhyphen{}bit microcontroller is connected. This sensor includes a resistive element and a sense of wet NTC temperature measuring devices. It has excellent quality,fast response, anti\sphinxhyphen{}interference ability and high cost performance advantages.Each DHT11 sensor features extremely accurate calibration data of humidity calibration chamber. The calibration coefficients stored in the OTP program memory, internal sensors detect signals in the process, and we should call these calibration coefficients. The single\sphinxhyphen{}wire serial interface system is integrated to make it quick and easy. Qualities of small size, low power, and 20\sphinxhyphen{}meter signal transmission distance make it a wide applied application and even the most demanding one. Convenient connection, special packages can be provided according to users need.

\noindent{\hspace*{\fill}\sphinxincludegraphics[width=200\sphinxpxdimen]{{dht11_pic}.png}\hspace*{\fill}}

\sphinxAtStartPar
\sphinxstylestrong{Example}
\begin{itemize}
\item {} 
\sphinxAtStartPar
{\hyperref[\detokenize{Basic_Project/DHT11_Module:basic-dht11-module}]{\sphinxcrossref{\DUrole{std}{\DUrole{std-ref}{DHT11 Module}}}}} (Basic Project)

\end{itemize}

\sphinxstepscope


\section{0.96 inch OLED Display Module}
\label{\detokenize{Components_Kit/component_oled:inch-oled-display-module}}\label{\detokenize{Components_Kit/component_oled:cpn-oled}}\label{\detokenize{Components_Kit/component_oled::doc}}
\noindent{\hspace*{\fill}\sphinxincludegraphics[width=300\sphinxpxdimen]{{oled}.png}\hspace*{\fill}}


\subsection{Introduction}
\label{\detokenize{Components_Kit/component_oled:introduction}}
\sphinxAtStartPar
An OLED (Organic Light\sphinxhyphen{}Emitting Diode) display module is a device that uses organic compounds to produce light when an electric current passes through them, allowing it to display text, graphics, and images on a thin and flexible screen.

\sphinxAtStartPar
One of the primary benefits of an OLED display is that it generates its own light and does not require an additional backlight source. This results in OLED displays often having superior contrast, brightness, and viewing angles compared to LCD displays.

\sphinxAtStartPar
Another key characteristic of OLED displays is their ability to achieve deep black levels. In an OLED display, each pixel emits its own light, so to create a black color, the individual pixel can be completely turned off.

\sphinxAtStartPar
OLED displays also have lower power consumption since only the illuminated pixels draw current. This makes them particularly suitable for battery\sphinxhyphen{}powered devices like smartwatches, health trackers, and other portable electronics.


\subsection{Principle}
\label{\detokenize{Components_Kit/component_oled:principle}}
\sphinxAtStartPar
An OLED display module consists of an OLED panel and an OLED driver chip. The panel has tiny pixels made of organic materials that emit light when an electric current passes through electrodes. The driver chip uses the I2C protocol to control these pixels by converting signals from the Arduino into commands. Libraries like Adafruit SSD1306 help initialize the display, set brightness, and render text or images.

\sphinxAtStartPar
In this tutorial, we will use the SSD1306 model, a monochrome, 0.96\sphinxhyphen{}inch display with 128x64 pixels, as shown in the figure below.

\sphinxAtStartPar
\sphinxstylestrong{Example}
\begin{itemize}
\item {} 
\sphinxAtStartPar
{\hyperref[\detokenize{Basic_Project/0.96_inch_OLED:basic-0-96-inch-oled}]{\sphinxcrossref{\DUrole{std}{\DUrole{std-ref}{0.96 inch IIC OLED}}}}} (Basic Project)

\item {} 
\sphinxAtStartPar
{\hyperref[\detokenize{Extension_Project/Ping-Pong_Game:ext-ping-pong-game}]{\sphinxcrossref{\DUrole{std}{\DUrole{std-ref}{Ping\sphinxhyphen{}Pong Game}}}}} (Extension\_Project)

\item {} 
\sphinxAtStartPar
{\hyperref[\detokenize{Extension_Project/Real-time_Weather_OLED:ext-real-time-weather-oled}]{\sphinxcrossref{\DUrole{std}{\DUrole{std-ref}{Real\sphinxhyphen{}time Weather OLED}}}}} (Extension\_Project)

\end{itemize}

\sphinxstepscope


\chapter{Basic Project}
\label{\detokenize{Basic_Project/Basic_Project:basic-project}}\label{\detokenize{Basic_Project/Basic_Project::doc}}
\sphinxstepscope


\section{LED Blink}
\label{\detokenize{Basic_Project/LED_Blink:led-blink}}\label{\detokenize{Basic_Project/LED_Blink:basic-led-blink}}\label{\detokenize{Basic_Project/LED_Blink::doc}}
\sphinxAtStartPar
Just as printing “Hello, world!” is the first step in learning to program, using a program to drive an LED is the traditional introduction to learning physical programming.


\subsection{Wiring}
\label{\detokenize{Basic_Project/LED_Blink:wiring}}
\noindent{\hspace*{\fill}\sphinxincludegraphics[width=0.600\linewidth]{{Led_Blink_Wiring}.png}\hspace*{\fill}}


\subsection{Schematic Diagram}
\label{\detokenize{Basic_Project/LED_Blink:schematic-diagram}}
\noindent{\hspace*{\fill}\sphinxincludegraphics[width=0.800\linewidth]{{Led_Blink_Wiring1}.png}\hspace*{\fill}}


\subsection{Code}
\label{\detokenize{Basic_Project/LED_Blink:code}}
\begin{sphinxadmonition}{note}{Note:}\begin{itemize}
\item {} 
\sphinxAtStartPar
You can open the file \sphinxcode{\sphinxupquote{01\_LED\_Blink.ino}} under the path of \sphinxcode{\sphinxupquote{Basic\sphinxhyphen{}Starter\sphinxhyphen{}Kit\sphinxhyphen{}for\sphinxhyphen{}Arduino\sphinxhyphen{}Uno\sphinxhyphen{}R4\sphinxhyphen{}WiFi\sphinxhyphen{}main\textbackslash{}Code}}.

\end{itemize}
\end{sphinxadmonition}

\sphinxAtStartPar
After the code is uploaded successfully, you will see the LED connected to digital pin 9 of the Arduino board start to blink. The LED will turn on for half a second and then turn off for another half a second, repeating this cycle continuously as the program runs.


\subsection{Code Analysis}
\label{\detokenize{Basic_Project/LED_Blink:code-analysis}}
\sphinxAtStartPar
Here, we connect the LED to the digital pin 9, so we need to declare an \sphinxcode{\sphinxupquote{int}} variable called \sphinxcode{\sphinxupquote{ledpin}} at the beginning of the program and assign a value of 9.

\begin{sphinxVerbatim}[commandchars=\\\{\}]
\PYG{k+kr}{const}\PYG{+w}{ }\PYG{k+kr}{int}\PYG{+w}{ }\PYG{n}{ledPin}\PYG{+w}{ }\PYG{o}{=}\PYG{+w}{ }\PYG{l+m+mi}{9}\PYG{p}{;}
\end{sphinxVerbatim}

\sphinxAtStartPar
Now, initialize the pin in the \sphinxcode{\sphinxupquote{setup()}} function, where you need to initialize the pin to \sphinxcode{\sphinxupquote{OUTPUT}} mode.

\begin{sphinxVerbatim}[commandchars=\\\{\}]
\PYG{k+kr}{void}\PYG{+w}{ }\PYG{n+nb}{setup}\PYG{p}{(}\PYG{p}{)}\PYG{+w}{ }\PYG{p}{\PYGZob{}}
\PYG{+w}{    }\PYG{n+nf}{pinMode}\PYG{p}{(}\PYG{n}{ledPin}\PYG{p}{,}\PYG{+w}{ }\PYG{k+kr}{OUTPUT}\PYG{p}{)}\PYG{p}{;}
\PYG{p}{\PYGZcb{}}
\end{sphinxVerbatim}

\sphinxAtStartPar
In \sphinxcode{\sphinxupquote{loop()}}, \sphinxcode{\sphinxupquote{digitalWrite()}} is used to provide 5V high level signal for ledpin, which will cause voltage difference between LED pins and light LED up.

\begin{sphinxVerbatim}[commandchars=\\\{\}]
\PYG{n+nf}{digitalWrite}\PYG{p}{(}\PYG{n}{ledPin}\PYG{p}{,}\PYG{+w}{ }\PYG{k+kr}{HIGH}\PYG{p}{)}\PYG{p}{;}
\end{sphinxVerbatim}

\sphinxAtStartPar
If the level signal is changed to LOW, the ledPin’s signal will be returned to 0 V to turn LED off.

\begin{sphinxVerbatim}[commandchars=\\\{\}]
\PYG{n+nf}{digitalWrite}\PYG{p}{(}\PYG{n}{ledPin}\PYG{p}{,}\PYG{+w}{ }\PYG{k+kr}{LOW}\PYG{p}{)}\PYG{p}{;}
\end{sphinxVerbatim}

\sphinxAtStartPar
An interval between on and off is required to allow people to see the change,
so we use a \sphinxcode{\sphinxupquote{delay(1000)}} code to let the controller do nothing for 1000 ms.

\begin{sphinxVerbatim}[commandchars=\\\{\}]
\PYG{n+nf}{delay}\PYG{p}{(}\PYG{l+m+mi}{1000}\PYG{p}{)}\PYG{p}{;}
\end{sphinxVerbatim}

\sphinxstepscope


\section{Button Control LED}
\label{\detokenize{Basic_Project/Button_LED:button-control-led}}\label{\detokenize{Basic_Project/Button_LED:basic-button-led}}\label{\detokenize{Basic_Project/Button_LED::doc}}

\subsection{Overview}
\label{\detokenize{Basic_Project/Button_LED:overview}}
\sphinxAtStartPar
In this lesson, you will learn about controlling an LED using a button with Arduino. Buttons and LEDs are fundamental components in a wide range of electronic devices, such as remote controls, flashlights, and interactive installations. In this setup, a button is used as an input device to control the state of an LED, which serves as an output device.

\sphinxAtStartPar
The button is connected to pin 12 on the Arduino Uno R4 board, and the LED is connected to pin 13. When the button is pressed, a signal is sent to the Arduino, triggering the LED to turn on. Conversely, when the button is released, the LED turns off. This simple yet effective mechanism can be the basis for more complex projects, such as home automation systems, interactive displays, and much more.

\sphinxAtStartPar
By the end of this lesson, you will understand how to read input from a button and use it to control an LED, thereby gaining a foundational understanding of input/output operations with Arduino.


\subsection{Wiring}
\label{\detokenize{Basic_Project/Button_LED:wiring}}
\noindent{\hspace*{\fill}\sphinxincludegraphics[width=0.900\linewidth]{{Button_LED_Wiring}.png}\hspace*{\fill}}

\sphinxAtStartPar
Connect one end of the buttons to pin 12 which connects with a pull\sphinxhyphen{}down resistor. Connect the other end of the resistor to GND and one of the pins at the other end of the button to 5V. When the button is pressed, pin 12 is 5V (HIGH) and set pin 13 (integrated with an LED) as High at the same time. Then release the button (pin 12 changes to LOW) and pin 13 is Low. So we will see the LED lights up and goes out alternately as the button is pressed and released.


\subsection{Code}
\label{\detokenize{Basic_Project/Button_LED:code}}
\begin{sphinxadmonition}{note}{Note:}\begin{itemize}
\item {} 
\sphinxAtStartPar
You can open the file \sphinxcode{\sphinxupquote{02\_Button\_LED.ino}} under the path of \sphinxcode{\sphinxupquote{Basic\sphinxhyphen{}Starter\sphinxhyphen{}Kit\sphinxhyphen{}for\sphinxhyphen{}Arduino\sphinxhyphen{}Uno\sphinxhyphen{}R4\sphinxhyphen{}WiFi\sphinxhyphen{}main\textbackslash{}Code}} directly.

\end{itemize}
\end{sphinxadmonition}


\subsection{Code Analysis}
\label{\detokenize{Basic_Project/Button_LED:code-analysis}}\begin{enumerate}
\sphinxsetlistlabels{\arabic}{enumi}{enumii}{}{.}%
\item {} 
\sphinxAtStartPar
Define Constants and Variables

\sphinxAtStartPar
In this segment, the pin numbers for the button and the LED are defined. Also, a variable \sphinxcode{\sphinxupquote{buttonState}} is declared to hold the current state of the button.

\begin{sphinxVerbatim}[commandchars=\\\{\}]
\PYG{k+kr}{const}\PYG{+w}{ }\PYG{k+kr}{int}\PYG{+w}{ }\PYG{n}{buttonPin}\PYG{+w}{ }\PYG{o}{=}\PYG{+w}{ }\PYG{l+m+mi}{12}\PYG{p}{;}
\PYG{k+kr}{const}\PYG{+w}{ }\PYG{k+kr}{int}\PYG{+w}{ }\PYG{n}{ledPin}\PYG{+w}{ }\PYG{o}{=}\PYG{+w}{ }\PYG{l+m+mi}{13}\PYG{p}{;}
\PYG{k+kr}{int}\PYG{+w}{ }\PYG{n}{buttonState}\PYG{+w}{ }\PYG{o}{=}\PYG{+w}{ }\PYG{l+m+mi}{0}\PYG{p}{;}
\end{sphinxVerbatim}

\item {} 
\sphinxAtStartPar
Setup Function

\sphinxAtStartPar
The \sphinxcode{\sphinxupquote{setup()}} function runs once when the Arduino board starts. The pin modes for the button and the LED are set using the \sphinxcode{\sphinxupquote{pinMode}} function.

\begin{sphinxVerbatim}[commandchars=\\\{\}]
\PYG{k+kr}{void}\PYG{+w}{ }\PYG{n+nb}{setup}\PYG{p}{(}\PYG{p}{)}\PYG{+w}{ }\PYG{p}{\PYGZob{}}
\PYG{+w}{  }\PYG{n+nf}{pinMode}\PYG{p}{(}\PYG{n}{buttonPin}\PYG{p}{,}\PYG{+w}{ }\PYG{k+kr}{INPUT}\PYG{p}{)}\PYG{p}{;}
\PYG{+w}{  }\PYG{n+nf}{pinMode}\PYG{p}{(}\PYG{n}{ledPin}\PYG{p}{,}\PYG{+w}{ }\PYG{k+kr}{OUTPUT}\PYG{p}{)}\PYG{p}{;}
\PYG{p}{\PYGZcb{}}
\end{sphinxVerbatim}

\item {} 
\sphinxAtStartPar
Main Loop

\sphinxAtStartPar
The \sphinxcode{\sphinxupquote{loop()}} function runs repeatedly. Inside this loop, the \sphinxcode{\sphinxupquote{digitalRead()}} function is used to read the state of the button. Depending on whether the button is pressed or not, the LED is turned on or off.

\begin{sphinxVerbatim}[commandchars=\\\{\}]
\PYG{k+kr}{void}\PYG{+w}{ }\PYG{n+nb}{loop}\PYG{p}{(}\PYG{p}{)}\PYG{+w}{ }\PYG{p}{\PYGZob{}}
\PYG{+w}{  }\PYG{n}{buttonState}\PYG{+w}{ }\PYG{o}{=}\PYG{+w}{ }\PYG{n+nf}{digitalRead}\PYG{p}{(}\PYG{n}{buttonPin}\PYG{p}{)}\PYG{p}{;}
\PYG{+w}{  }\PYG{k}{if}\PYG{+w}{ }\PYG{p}{(}\PYG{n}{buttonState}\PYG{+w}{ }\PYG{o}{=}\PYG{o}{=}\PYG{+w}{ }\PYG{k+kr}{HIGH}\PYG{p}{)}\PYG{+w}{ }\PYG{p}{\PYGZob{}}
\PYG{+w}{    }\PYG{n+nf}{digitalWrite}\PYG{p}{(}\PYG{n}{ledPin}\PYG{p}{,}\PYG{+w}{ }\PYG{k+kr}{HIGH}\PYG{p}{)}\PYG{p}{;}
\PYG{+w}{  }\PYG{p}{\PYGZcb{}}\PYG{+w}{ }\PYG{k}{else}\PYG{+w}{ }\PYG{p}{\PYGZob{}}
\PYG{+w}{    }\PYG{n+nf}{digitalWrite}\PYG{p}{(}\PYG{n}{ledPin}\PYG{p}{,}\PYG{+w}{ }\PYG{k+kr}{LOW}\PYG{p}{)}\PYG{p}{;}
\PYG{+w}{  }\PYG{p}{\PYGZcb{}}
\PYG{p}{\PYGZcb{}}
\end{sphinxVerbatim}

\end{enumerate}

\sphinxstepscope


\section{RGB LED}
\label{\detokenize{Basic_Project/RGB_LED:rgb-led}}\label{\detokenize{Basic_Project/RGB_LED:basic-rgb-led}}\label{\detokenize{Basic_Project/RGB_LED::doc}}

\subsection{Overview}
\label{\detokenize{Basic_Project/RGB_LED:overview}}
\sphinxAtStartPar
In this lesson, we will use PWM to control an RGB LED to flash various kinds of color. When different PWM values are set to the R, G, and B pins of the LED, its brightness will be different. When the three different colors are mixed, we can see that the RGB LED flashes different colors.


\subsection{PWM}
\label{\detokenize{Basic_Project/RGB_LED:pwm}}
\sphinxAtStartPar
Pulse width modulation, or PWM, is a technique for getting analog results with digital means. Digital control is used to create a square wave, a signal switched between on and off. This on\sphinxhyphen{}off pattern can simulate voltages in between full on (5 Volts) and off (0 Volts) by changing the portion of the time the signal spends on versus the time that the signal spends off. The duration of “on time” is called pulse width. To get varying analog values, you change, or modulate, that width. If you repeat this on\sphinxhyphen{}off pattern fast enough with some device, an LED for example, it would be like this: the signal is a steady voltage between 0 and 5V controlling the brightness of the LED. (See the PWM description on the official website of Arduino).

\sphinxAtStartPar
In the graphic below, the green lines represent a regular time period. This duration or period is the inverse of the PWM frequency. In other words, with Arduino’s PWM frequency at about 500Hz, the green lines would measure 2 milliseconds each.

\noindent{\hspace*{\fill}\sphinxincludegraphics[width=0.600\linewidth]{{RGB_LED_PWM}.png}\hspace*{\fill}}

\sphinxAtStartPar
A call to analogWrite() is on a scale of 0 \sphinxhyphen{} 255, such that analogWrite(255) requests a 100\% duty cycle (always on), and analogWrite(127) is a 50\% duty cycle (on half the time) for example.

\sphinxAtStartPar
You will find that the smaller the PWM value is, the smaller the value will be after being converted into voltage. Then the LED becomes dimmer accordingly. Therefore, we can control the brightness of the LED by controlling the PWM value.


\subsection{Wiring}
\label{\detokenize{Basic_Project/RGB_LED:wiring}}
\noindent{\hspace*{\fill}\sphinxincludegraphics[width=0.700\linewidth]{{RGB_LED_Wiring}.png}\hspace*{\fill}}


\subsection{Schematic Diagram}
\label{\detokenize{Basic_Project/RGB_LED:schematic-diagram}}
\noindent{\hspace*{\fill}\sphinxincludegraphics[width=0.800\linewidth]{{RGB_LED_Wiring1}.png}\hspace*{\fill}}


\subsection{Code}
\label{\detokenize{Basic_Project/RGB_LED:code}}
\begin{sphinxadmonition}{note}{Note:}\begin{itemize}
\item {} 
\sphinxAtStartPar
You can open the file \sphinxcode{\sphinxupquote{03\_RGB\_LED.ino}} under the path of \sphinxcode{\sphinxupquote{Basic\sphinxhyphen{}Starter\sphinxhyphen{}Kit\sphinxhyphen{}for\sphinxhyphen{}Arduino\sphinxhyphen{}Uno\sphinxhyphen{}R4\sphinxhyphen{}WiFi\sphinxhyphen{}main\textbackslash{}Code}} directly.

\end{itemize}
\end{sphinxadmonition}

\sphinxAtStartPar
Once the code is successfully uploaded, you will observe the RGB LED flashing in a circular pattern of red, green, and blue initially. It will then proceed to flash in the sequence of red, orange, yellow, green, blue, indigo, and purple.


\subsection{Code Analysis}
\label{\detokenize{Basic_Project/RGB_LED:code-analysis}}
\sphinxAtStartPar
\sphinxstylestrong{Set the color}

\sphinxAtStartPar
Here use the \sphinxcode{\sphinxupquote{color()}} function to set the color of the RGB LED. In the
code, it is set to flash 7 different colors.

\sphinxAtStartPar
You can use the paint tool on your computer to get the RGB value.
\begin{enumerate}
\sphinxsetlistlabels{\arabic}{enumi}{enumii}{}{.}%
\item {} 
\sphinxAtStartPar
Open the paint tool on your computer and click to Edit colors.

\noindent{\hspace*{\fill}\sphinxincludegraphics{{RGB_LED_Code}.png}\hspace*{\fill}}

\item {} 
\sphinxAtStartPar
Select one color, then you can see the RGB value of this color. Fill them in the code.

\begin{sphinxadmonition}{note}{Note:}
\sphinxAtStartPar
Due to hardware and environmental factors, the colors displayed on computer screens and RGB LEDs may vary even when using the same RGB values.
\end{sphinxadmonition}

\noindent{\hspace*{\fill}\sphinxincludegraphics{{RGB_LED_Code1}.png}\hspace*{\fill}}



\begin{sphinxVerbatim}[commandchars=\\\{\}]
\PYG{k+kr}{void}\PYG{+w}{ }\PYG{n+nb}{loop}\PYG{p}{(}\PYG{p}{)}\PYG{+w}{ }\PYG{c+c1}{// run over and over again}

\PYG{p}{\PYGZob{}}

\PYG{+w}{  }\PYG{c+c1}{// Basic colors:}

\PYG{+w}{  }\PYG{n}{color}\PYG{p}{(}\PYG{l+m+mi}{255}\PYG{p}{,}\PYG{+w}{ }\PYG{l+m+mi}{0}\PYG{p}{,}\PYG{+w}{ }\PYG{l+m+mi}{0}\PYG{p}{)}\PYG{p}{;}\PYG{+w}{ }\PYG{c+c1}{// turn the RGB LED red}

\PYG{+w}{  }\PYG{n+nf}{delay}\PYG{p}{(}\PYG{l+m+mi}{1000}\PYG{p}{)}\PYG{p}{;}\PYG{+w}{ }\PYG{c+c1}{// delay for 1 second}

\PYG{+w}{  }\PYG{n}{color}\PYG{p}{(}\PYG{l+m+mi}{0}\PYG{p}{,}\PYG{l+m+mi}{255}\PYG{p}{,}\PYG{+w}{ }\PYG{l+m+mi}{0}\PYG{p}{)}\PYG{p}{;}\PYG{+w}{ }\PYG{c+c1}{// turn the RGB LED green}

\PYG{+w}{  }\PYG{n+nf}{delay}\PYG{p}{(}\PYG{l+m+mi}{1000}\PYG{p}{)}\PYG{p}{;}\PYG{+w}{ }\PYG{c+c1}{// delay for 1 second}

\PYG{+w}{  }\PYG{n}{color}\PYG{p}{(}\PYG{l+m+mi}{0}\PYG{p}{,}\PYG{+w}{ }\PYG{l+m+mi}{0}\PYG{p}{,}\PYG{+w}{ }\PYG{l+m+mi}{255}\PYG{p}{)}\PYG{p}{;}\PYG{+w}{ }\PYG{c+c1}{// turn the RGB LED blue}

\PYG{+w}{  }\PYG{n+nf}{delay}\PYG{p}{(}\PYG{l+m+mi}{1000}\PYG{p}{)}\PYG{p}{;}\PYG{+w}{ }\PYG{c+c1}{// delay for 1 second}

\PYG{+w}{  }\PYG{c+c1}{// Example blended colors:}

\PYG{+w}{  }\PYG{n}{color}\PYG{p}{(}\PYG{l+m+mi}{255}\PYG{p}{,}\PYG{l+m+mi}{0}\PYG{p}{,}\PYG{l+m+mi}{252}\PYG{p}{)}\PYG{p}{;}\PYG{+w}{ }\PYG{c+c1}{// turn the RGB LED red}

\PYG{+w}{  }\PYG{n+nf}{delay}\PYG{p}{(}\PYG{l+m+mi}{1000}\PYG{p}{)}\PYG{p}{;}\PYG{+w}{ }\PYG{c+c1}{// delay for 1 second}

\PYG{+w}{  }\PYG{n}{color}\PYG{p}{(}\PYG{l+m+mi}{237}\PYG{p}{,}\PYG{l+m+mi}{109}\PYG{p}{,}\PYG{l+m+mi}{0}\PYG{p}{)}\PYG{p}{;}\PYG{+w}{ }\PYG{c+c1}{// turn the RGB LED orange}

\PYG{+w}{  }\PYG{n+nf}{delay}\PYG{p}{(}\PYG{l+m+mi}{1000}\PYG{p}{)}\PYG{p}{;}\PYG{+w}{ }\PYG{c+c1}{// delay for 1 second}

\PYG{+w}{  }\PYG{n}{color}\PYG{p}{(}\PYG{l+m+mi}{255}\PYG{p}{,}\PYG{l+m+mi}{215}\PYG{p}{,}\PYG{l+m+mi}{0}\PYG{p}{)}\PYG{p}{;}\PYG{+w}{ }\PYG{c+c1}{// turn the RGB LED yellow}

\PYG{+w}{  }\PYG{p}{.}\PYG{p}{.}\PYG{p}{.}\PYG{p}{.}\PYG{p}{.}\PYG{p}{.}
\end{sphinxVerbatim}

\end{enumerate}

\sphinxAtStartPar
\sphinxstylestrong{color() function}

\begin{sphinxVerbatim}[commandchars=\\\{\}]
\PYG{k+kr}{void}\PYG{+w}{ }\PYG{n+nf}{color}\PYG{+w}{ }\PYG{p}{(}\PYG{k+kr}{int}\PYG{+w}{ }\PYG{n}{red}\PYG{p}{,}\PYG{+w}{ }\PYG{k+kr}{int}\PYG{+w}{ }\PYG{n}{green}\PYG{p}{,}\PYG{+w}{ }\PYG{k+kr}{int}\PYG{+w}{ }\PYG{n}{blue}\PYG{p}{)}
\PYG{c+c1}{// the color generating function}

\PYG{p}{\PYGZob{}}

\PYG{+w}{  }\PYG{n+nf}{analogWrite}\PYG{p}{(}\PYG{n}{redPin}\PYG{p}{,}\PYG{+w}{ }\PYG{n}{red}\PYG{p}{)}\PYG{p}{;}

\PYG{+w}{  }\PYG{n+nf}{analogWrite}\PYG{p}{(}\PYG{n}{greenPin}\PYG{p}{,}\PYG{+w}{ }\PYG{n}{green}\PYG{p}{)}\PYG{p}{;}

\PYG{+w}{  }\PYG{n+nf}{analogWrite}\PYG{p}{(}\PYG{n}{bluePin}\PYG{p}{,}\PYG{+w}{ }\PYG{n}{blue}\PYG{p}{)}\PYG{p}{;}

\PYG{p}{\PYGZcb{}}
\end{sphinxVerbatim}

\sphinxAtStartPar
Define three unsigned char variables, red, green and blue. Write their values to \sphinxcode{\sphinxupquote{redPin}}, \sphinxcode{\sphinxupquote{greenPin}} and \sphinxcode{\sphinxupquote{bluePin}}. For example, color(128,0,128) is to write 128 to \sphinxcode{\sphinxupquote{redPin}}, 0 to \sphinxcode{\sphinxupquote{greenPin}} and 128 to \sphinxcode{\sphinxupquote{bluePin}}. Then the result is the LED flashing purple.

\sphinxAtStartPar
\sphinxstylestrong{analogWrite()}: Writes an analog value (PWM wave) to a pin. It has nothing to do with an analog pin, but is just for PWM pins. You do not need to call the \sphinxcode{\sphinxupquote{pinMode()}} to set the pin as output before calling \sphinxcode{\sphinxupquote{analogWrite()}}.

\sphinxstepscope


\section{Active Buzzer}
\label{\detokenize{Basic_Project/Active_Buzzer:active-buzzer}}\label{\detokenize{Basic_Project/Active_Buzzer:basic-active-buzzer}}\label{\detokenize{Basic_Project/Active_Buzzer::doc}}

\subsection{Overview}
\label{\detokenize{Basic_Project/Active_Buzzer:overview}}
\sphinxAtStartPar
The active buzzer is a typical digital output device that is as easy to use as lighting up an LED!

\sphinxAtStartPar
Two types of buzzers are included in the kit.
We need to use active buzzer. Turn them around, the sealed back (not the exposed PCB) is the one we want.

\noindent{\hspace*{\fill}\sphinxincludegraphics[width=0.700\linewidth]{{Active_Buzzer}.png}\hspace*{\fill}}


\subsection{Wiring}
\label{\detokenize{Basic_Project/Active_Buzzer:wiring}}
\begin{sphinxadmonition}{note}{Note:}
\sphinxAtStartPar
When connecting the buzzer, make sure to check its pins. The longer pin is the anode and the shorter one is the cathode. It’s important not to mix them up, as doing so will prevent the buzzer from producing any sound.
\end{sphinxadmonition}

\noindent{\hspace*{\fill}\sphinxincludegraphics[width=0.700\linewidth]{{Active_Buzzer_Wiring}.png}\hspace*{\fill}}


\subsection{Schematic Diagram}
\label{\detokenize{Basic_Project/Active_Buzzer:schematic-diagram}}
\noindent{\hspace*{\fill}\sphinxincludegraphics[width=0.800\linewidth]{{Active_Buzzer_Wiring1}.png}\hspace*{\fill}}


\subsection{Code}
\label{\detokenize{Basic_Project/Active_Buzzer:code}}
\begin{sphinxadmonition}{note}{Note:}\begin{itemize}
\item {} 
\sphinxAtStartPar
You can open the file \sphinxcode{\sphinxupquote{04\_Active\_Buzzer.ino}} under the path of \sphinxcode{\sphinxupquote{Basic\sphinxhyphen{}Starter\sphinxhyphen{}Kit\sphinxhyphen{}for\sphinxhyphen{}Arduino\sphinxhyphen{}Uno\sphinxhyphen{}R4\sphinxhyphen{}WiFi\sphinxhyphen{}main\textbackslash{}Code}} directly.

\end{itemize}
\end{sphinxadmonition}

\sphinxAtStartPar
After the code is uploaded successfully, you will hear a beep every second.

\sphinxstepscope


\section{Passive Buzzer}
\label{\detokenize{Basic_Project/Passive_Buzzer:passive-buzzer}}\label{\detokenize{Basic_Project/Passive_Buzzer:basic-passive-buzzer}}\label{\detokenize{Basic_Project/Passive_Buzzer::doc}}

\subsection{Overview}
\label{\detokenize{Basic_Project/Passive_Buzzer:overview}}
\sphinxAtStartPar
In this project, use these two functions to make the passive buzzer vibrate and produce sound. The function \sphinxcode{\sphinxupquote{tone()}} generates a square wave with a specified frequency (and 50\% duty cycle) on a pin. A duration can be specified, or the wave continues until \sphinxcode{\sphinxupquote{noTone()}} is called.
Similar to the active buzzer, the passive buzzer also utilizes electromagnetic induction to operate.
The difference is that a passive buzzer does not have its own oscillating source, so it will not emit sound if DC signals are used.However, this allows the passive buzzer to adjust its own oscillation frequency and produce different notes such as “do, re, mi, fa, sol, la, ti”.


\subsection{Wiring}
\label{\detokenize{Basic_Project/Passive_Buzzer:wiring}}
\begin{sphinxadmonition}{note}{Note:}
\sphinxAtStartPar
When connecting the buzzer, make sure to check its pins. The longer pin is the anode and the shorter one is the cathode. It’s important not to mix them up, as doing so will prevent the buzzer from producing any sound.
\end{sphinxadmonition}

\noindent{\hspace*{\fill}\sphinxincludegraphics[width=0.700\linewidth]{{Passive_Buzzer_Wiring}.png}\hspace*{\fill}}


\subsection{Schematic Diagram}
\label{\detokenize{Basic_Project/Passive_Buzzer:schematic-diagram}}
\noindent{\hspace*{\fill}\sphinxincludegraphics[width=0.800\linewidth]{{Passive_Buzzer_Wiring1}.png}\hspace*{\fill}}


\subsection{Code}
\label{\detokenize{Basic_Project/Passive_Buzzer:code}}
\begin{sphinxadmonition}{note}{Note:}\begin{itemize}
\item {} 
\sphinxAtStartPar
You can open the file \sphinxcode{\sphinxupquote{05\_Passive\_Buzzer.ino}} under the path of \sphinxcode{\sphinxupquote{Basic\sphinxhyphen{}Starter\sphinxhyphen{}Kit\sphinxhyphen{}for\sphinxhyphen{}Arduino\sphinxhyphen{}Uno\sphinxhyphen{}R4\sphinxhyphen{}WiFi\sphinxhyphen{}main\textbackslash{}Code}} directly.

\end{itemize}
\end{sphinxadmonition}

\sphinxAtStartPar
At the time when you finish uploading the codes to the R4 board, you can hear a melody containing seven notes.


\subsection{Code Analysis}
\label{\detokenize{Basic_Project/Passive_Buzzer:code-analysis}}\begin{enumerate}
\sphinxsetlistlabels{\arabic}{enumi}{enumii}{}{.}%
\item {} 
\sphinxAtStartPar
Including the pitches library:
This library provides the frequency values for various musical notes, allowing you to use musical notation in your code.

\begin{sphinxadmonition}{note}{Note:}
\sphinxAtStartPar
Please place the \sphinxcode{\sphinxupquote{pitches.h}} file in the same directory as the code to ensure proper functioning. \sphinxhref{https://raw.githubusercontent.com/lafvintech/Basic-Starter-Kit-for-Arduino-Uno-R4-WiFi/main/Code/05\_Passive\_Buzzer/pitches.h}{pitches.h}

\noindent\sphinxincludegraphics{{Passive_Buzzer_Code}.png}
\end{sphinxadmonition}

\begin{sphinxVerbatim}[commandchars=\\\{\}]
\PYG{c+cp}{\PYGZsh{}}\PYG{c+cp}{include}\PYG{+w}{ }\PYG{c+cpf}{\PYGZdq{}pitches.h\PYGZdq{}}
\end{sphinxVerbatim}

\item {} 
\sphinxAtStartPar
Defining constants and arrays:
\begin{itemize}
\item {} 
\sphinxAtStartPar
\sphinxcode{\sphinxupquote{buzzerPin}} is the digital pin on the Arduino where the buzzer is connected.

\item {} 
\sphinxAtStartPar
\sphinxcode{\sphinxupquote{melody{[}{]}}} is an array that stores the sequence of notes to be played.

\item {} 
\sphinxAtStartPar
\sphinxcode{\sphinxupquote{noteDurations{[}{]}}} is an array that stores the duration of each note in the melody.

\end{itemize}

\begin{sphinxVerbatim}[commandchars=\\\{\}]
\PYG{k+kr}{const}\PYG{+w}{ }\PYG{k+kr}{int}\PYG{+w}{ }\PYG{n}{buzzerPin}\PYG{+w}{ }\PYG{o}{=}\PYG{+w}{ }\PYG{l+m+mi}{8}\PYG{p}{;}
\PYG{k+kr}{int}\PYG{+w}{ }\PYG{n}{melody}\PYG{p}{[}\PYG{p}{]}\PYG{+w}{ }\PYG{o}{=}\PYG{+w}{ }\PYG{p}{\PYGZob{}}
\PYG{+w}{  }\PYG{n}{NOTE\PYGZus{}C4}\PYG{p}{,}\PYG{+w}{ }\PYG{n}{NOTE\PYGZus{}G3}\PYG{p}{,}\PYG{+w}{ }\PYG{n}{NOTE\PYGZus{}G3}\PYG{p}{,}\PYG{+w}{ }\PYG{n}{NOTE\PYGZus{}A3}\PYG{p}{,}\PYG{+w}{ }\PYG{n}{NOTE\PYGZus{}G3}\PYG{p}{,}\PYG{+w}{ }\PYG{l+m+mi}{0}\PYG{p}{,}\PYG{+w}{ }\PYG{n}{NOTE\PYGZus{}B3}\PYG{p}{,}\PYG{+w}{ }\PYG{n}{NOTE\PYGZus{}C4}
\PYG{p}{\PYGZcb{}}\PYG{p}{;}
\PYG{k+kr}{int}\PYG{+w}{ }\PYG{n}{noteDurations}\PYG{p}{[}\PYG{p}{]}\PYG{+w}{ }\PYG{o}{=}\PYG{+w}{ }\PYG{p}{\PYGZob{}}
\PYG{+w}{  }\PYG{l+m+mi}{4}\PYG{p}{,}\PYG{+w}{ }\PYG{l+m+mi}{8}\PYG{p}{,}\PYG{+w}{ }\PYG{l+m+mi}{8}\PYG{p}{,}\PYG{+w}{ }\PYG{l+m+mi}{4}\PYG{p}{,}\PYG{+w}{ }\PYG{l+m+mi}{4}\PYG{p}{,}\PYG{+w}{ }\PYG{l+m+mi}{4}\PYG{p}{,}\PYG{+w}{ }\PYG{l+m+mi}{4}\PYG{p}{,}\PYG{+w}{ }\PYG{l+m+mi}{4}
\PYG{p}{\PYGZcb{}}\PYG{p}{;}
\end{sphinxVerbatim}

\item {} 
\sphinxAtStartPar
Playing the melody:
\begin{itemize}
\item {} 
\sphinxAtStartPar
The \sphinxcode{\sphinxupquote{for}} loop iterates over each note in the melody.

\item {} 
\sphinxAtStartPar
The \sphinxcode{\sphinxupquote{tone()}} function plays a note on the buzzer for a specific duration.

\item {} 
\sphinxAtStartPar
A delay is added between notes to distinguish them.

\item {} 
\sphinxAtStartPar
The \sphinxcode{\sphinxupquote{noTone()}} function stops the sound.

\end{itemize}

\begin{sphinxVerbatim}[commandchars=\\\{\}]
\PYG{k+kr}{void}\PYG{+w}{ }\PYG{n+nb}{setup}\PYG{p}{(}\PYG{p}{)}\PYG{+w}{ }\PYG{p}{\PYGZob{}}
\PYG{+w}{  }\PYG{k}{for}\PYG{+w}{ }\PYG{p}{(}\PYG{k+kr}{int}\PYG{+w}{ }\PYG{n}{thisNote}\PYG{+w}{ }\PYG{o}{=}\PYG{+w}{ }\PYG{l+m+mi}{0}\PYG{p}{;}\PYG{+w}{ }\PYG{n}{thisNote}\PYG{+w}{ }\PYG{o}{\PYGZlt{}}\PYG{+w}{ }\PYG{l+m+mi}{8}\PYG{p}{;}\PYG{+w}{ }\PYG{n}{thisNote}\PYG{o}{+}\PYG{o}{+}\PYG{p}{)}\PYG{+w}{ }\PYG{p}{\PYGZob{}}
\PYG{+w}{    }\PYG{k+kr}{int}\PYG{+w}{ }\PYG{n}{noteDuration}\PYG{+w}{ }\PYG{o}{=}\PYG{+w}{ }\PYG{l+m+mi}{1000}\PYG{+w}{ }\PYG{o}{/}\PYG{+w}{ }\PYG{n}{noteDurations}\PYG{p}{[}\PYG{n}{thisNote}\PYG{p}{]}\PYG{p}{;}
\PYG{+w}{    }\PYG{n+nf}{tone}\PYG{p}{(}\PYG{n}{buzzerPin}\PYG{p}{,}\PYG{+w}{ }\PYG{n}{melody}\PYG{p}{[}\PYG{n}{thisNote}\PYG{p}{]}\PYG{p}{,}\PYG{+w}{ }\PYG{n}{noteDuration}\PYG{p}{)}\PYG{p}{;}
\PYG{+w}{    }\PYG{k+kr}{int}\PYG{+w}{ }\PYG{n}{pauseBetweenNotes}\PYG{+w}{ }\PYG{o}{=}\PYG{+w}{ }\PYG{n}{noteDuration}\PYG{+w}{ }\PYG{o}{*}\PYG{+w}{ }\PYG{l+m+mf}{1.30}\PYG{p}{;}
\PYG{+w}{    }\PYG{n+nf}{delay}\PYG{p}{(}\PYG{n}{pauseBetweenNotes}\PYG{p}{)}\PYG{p}{;}
\PYG{+w}{    }\PYG{n+nf}{noTone}\PYG{p}{(}\PYG{n}{buzzerPin}\PYG{p}{)}\PYG{p}{;}
\PYG{+w}{  }\PYG{p}{\PYGZcb{}}
\PYG{p}{\PYGZcb{}}
\end{sphinxVerbatim}

\item {} 
\sphinxAtStartPar
Empty loop function:
Since the melody is played only once in the setup, there’s no code in the loop function.

\item {} 
\sphinxAtStartPar
Feel free to experiment with altering the notes and durations in the \sphinxcode{\sphinxupquote{melody{[}{]}}} and \sphinxcode{\sphinxupquote{noteDurations{[}{]}}} arrays to create your own melodies. If you’re interested, there is a GitHub repository (\sphinxhref{https://github.com/robsoncouto/arduino-songs}{arduino\sphinxhyphen{}songs}) that offers Arduino code for playing various songs. While their approach may differ from this project, you can consult their notes and durations for reference.

\end{enumerate}

\sphinxstepscope


\section{Photoresistor}
\label{\detokenize{Basic_Project/Photoresistor:photoresistor}}\label{\detokenize{Basic_Project/Photoresistor:basic-photoresistor}}\label{\detokenize{Basic_Project/Photoresistor::doc}}

\subsection{Overview}
\label{\detokenize{Basic_Project/Photoresistor:overview}}
\sphinxAtStartPar
In this lesson, you will learn about Photoresistor. Photoresistor is applied in many electronic goods, such as the camera meter, clock radio, alarm device (as beam detector), small night lights, outdoor clock, solar street lamps and etc. Photoresistor is placed in a street lamp to control when the light is turned on. Ambient light falling on the photoresistor causes street lamps to turn on or off.


\subsection{Wiring}
\label{\detokenize{Basic_Project/Photoresistor:wiring}}
\sphinxAtStartPar
In this example, we use analog pin ( A0 ) to read the value of photoresistor. One pin of photoresistor is connected to 5V, the other is wired up to A0. Besides, a 10kΩ resistor is needed before the other pin is connected to GND.

\noindent{\hspace*{\fill}\sphinxincludegraphics[width=0.800\linewidth]{{Photoresistor_Wiring}.png}\hspace*{\fill}}


\subsection{Schematic Diagram}
\label{\detokenize{Basic_Project/Photoresistor:schematic-diagram}}
\noindent{\hspace*{\fill}\sphinxincludegraphics[width=0.700\linewidth]{{Photoresistor_Wiring1}.png}\hspace*{\fill}}


\subsection{Code}
\label{\detokenize{Basic_Project/Photoresistor:code}}
\begin{sphinxadmonition}{note}{Note:}\begin{itemize}
\item {} 
\sphinxAtStartPar
You can open the file \sphinxcode{\sphinxupquote{06\_Photoresistor.ino}} under the path of \sphinxcode{\sphinxupquote{Basic\sphinxhyphen{}Starter\sphinxhyphen{}Kit\sphinxhyphen{}for\sphinxhyphen{}Arduino\sphinxhyphen{}Uno\sphinxhyphen{}R4\sphinxhyphen{}WiFi\sphinxhyphen{}main\textbackslash{}Code}} directly.

\end{itemize}
\end{sphinxadmonition}

\sphinxAtStartPar
After uploading the codes to the uno board, you can open the serial monitor to see the read value of the pin. When the ambient light becomes stronger, the reading will increase correspondingly, and the pin reading range is 「0」\textasciitilde{}「1023」.  However, according to the environmental conditions and the characteristics of the photoresistor, the actual reading range may be smaller than the theoretical range.

\sphinxstepscope


\section{Thermistor}
\label{\detokenize{Basic_Project/Thermistor:thermistor}}\label{\detokenize{Basic_Project/Thermistor:basic-thermistor}}\label{\detokenize{Basic_Project/Thermistor::doc}}

\subsection{Overview}
\label{\detokenize{Basic_Project/Thermistor:overview}}
\sphinxAtStartPar
In this lesson, you will learn how to use thermistor. Thermistor can be used as electronic circuit components for temperature compensation of instrument circuits. In the current meter, flowmeter, gas analyzer, and other devices. It can also be used for overheating protection, contactless relay, constant temperature, automatic gain control, motor start, time delay, color TV automatic degaussing, fire alarm and temperature compensation.


\subsection{Wiring}
\label{\detokenize{Basic_Project/Thermistor:wiring}}
\sphinxAtStartPar
In this example, we use the analog pin A0 to get the value of Thermistor. One pin of thermistor is connected to 5V, and the other is wired up to A0. At the same time, a 10kΩ resistor is connected with the other pin before connecting to GND.

\noindent{\hspace*{\fill}\sphinxincludegraphics[width=0.700\linewidth]{{Thermistor_Wiring}.png}\hspace*{\fill}}


\subsection{Schematic Diagram}
\label{\detokenize{Basic_Project/Thermistor:schematic-diagram}}
\noindent{\hspace*{\fill}\sphinxincludegraphics[width=0.700\linewidth]{{Thermistor_Wiring1}.png}\hspace*{\fill}}


\subsection{Code}
\label{\detokenize{Basic_Project/Thermistor:code}}
\begin{sphinxadmonition}{note}{Note:}\begin{itemize}
\item {} 
\sphinxAtStartPar
You can open the file \sphinxcode{\sphinxupquote{07\_Thermistor.ino}} under the path of \sphinxcode{\sphinxupquote{Basic\sphinxhyphen{}Starter\sphinxhyphen{}Kit\sphinxhyphen{}for\sphinxhyphen{}Arduino\sphinxhyphen{}Uno\sphinxhyphen{}R4\sphinxhyphen{}WiFi\sphinxhyphen{}main\textbackslash{}Code}} directly.

\end{itemize}
\end{sphinxadmonition}

\sphinxAtStartPar
After uploading the code to the uno r4 board, you can open the serial monitor to check the current temperature.

\sphinxAtStartPar
The Kelvin temperature is calculated using the formula \sphinxstylestrong{TK=1/(ln(RT/RN)/B+1/TN)}. This equation is derived from the \sphinxhref{https://en.wikipedia.org/wiki/Steinhart\%E2\%80\%93Hart\_equation}{steinhart\_hart}  and simplifies calculations. You can also find more information about this formula on the detailed introduction page of the {\hyperref[\detokenize{Components_Kit/component_thermistor:component-thermistor}]{\sphinxcrossref{\DUrole{std}{\DUrole{std-ref}{Thermistor}}}}}.

\sphinxstepscope


\section{Potentiometer}
\label{\detokenize{Basic_Project/Potentiometer:potentiometer}}\label{\detokenize{Basic_Project/Potentiometer:basic-potentiometer}}\label{\detokenize{Basic_Project/Potentiometer::doc}}

\subsection{Overview}
\label{\detokenize{Basic_Project/Potentiometer:overview}}
\sphinxAtStartPar
In this lesson, let’s see how to change the luminance of an LED by a potentiometer, and receive the data of the potentiometer in Serial Monitor to see its value change.


\subsection{Wiring}
\label{\detokenize{Basic_Project/Potentiometer:wiring}}
\noindent{\hspace*{\fill}\sphinxincludegraphics[width=0.700\linewidth]{{Potentiometer_Wiring}.png}\hspace*{\fill}}


\subsection{Schematic Diagram}
\label{\detokenize{Basic_Project/Potentiometer:schematic-diagram}}
\sphinxAtStartPar
In this experiment, the potentiometer is used as voltage divider, meaning connecting devices to all of its three pins. Connect the middle pin of the potentiometer to pin A0 and the other two pins to 5V and GND respectively. Therefore, the voltage of the potentiometer is 0\sphinxhyphen{}5V. Spin the knob of the potentiometer, and the voltage at pin A0 will change. Then convert that voltage into a digital value (0\sphinxhyphen{}1024) with the AD converter in the control board. Through programming, we can use the converted digital value to control the brightness of the LED on the
control board.

\noindent{\hspace*{\fill}\sphinxincludegraphics[width=0.700\linewidth]{{Potentiometer_Wiring1}.png}\hspace*{\fill}}


\subsection{Code}
\label{\detokenize{Basic_Project/Potentiometer:code}}
\begin{sphinxadmonition}{note}{Note:}\begin{itemize}
\item {} 
\sphinxAtStartPar
You can open the file \sphinxcode{\sphinxupquote{08\_Potentiometer.ino}} under the path of \sphinxcode{\sphinxupquote{Basic\sphinxhyphen{}Starter\sphinxhyphen{}Kit\sphinxhyphen{}for\sphinxhyphen{}Arduino\sphinxhyphen{}Uno\sphinxhyphen{}R4\sphinxhyphen{}WiFi\sphinxhyphen{}main\textbackslash{}Code}} directly.

\item {} 
\sphinxAtStartPar
Or copy this code into Arduino IDE.

\end{itemize}
\end{sphinxadmonition}

\sphinxAtStartPar
After uploading the code to the Uno board, you can open the serial monitor to observe the potentiometer’s read values. As you turn the potentiometer knob, the read value will change accordingly. The raw analog reading from the potentiometer will range from (0) to (1023). Simultaneously, the code scales this value to a range of (0) to (255), which is also displayed on the serial monitor. This scaled value is then used to control the brightness of the connected LED. The LED will become brighter or dimmer based on the scaled value. It’s worth noting that while the theoretical range of the potentiometer is (0) to (1023), the actual range may vary slightly due to hardware tolerances.


\subsection{Code Analysis}
\label{\detokenize{Basic_Project/Potentiometer:code-analysis}}\begin{enumerate}
\sphinxsetlistlabels{\arabic}{enumi}{enumii}{}{.}%
\item {} 
\sphinxAtStartPar
Initialization and Setup (Setting Pin Modes and Initializing Serial Communication)

\sphinxAtStartPar
Before we get into the loop, we define which pins we’re using and initialize the serial communication.

\begin{sphinxVerbatim}[commandchars=\\\{\}]
\PYG{k+kr}{const}\PYG{+w}{ }\PYG{k+kr}{int}\PYG{+w}{ }\PYG{n}{analogPin}\PYG{+w}{ }\PYG{o}{=}\PYG{+w}{ }\PYG{l+m+mi}{0}\PYG{p}{;}\PYG{+w}{  }\PYG{c+c1}{// Analog input pin connected to the potentiometer}
\PYG{k+kr}{const}\PYG{+w}{ }\PYG{k+kr}{int}\PYG{+w}{ }\PYG{n}{ledPin}\PYG{+w}{ }\PYG{o}{=}\PYG{+w}{ }\PYG{l+m+mi}{9}\PYG{p}{;}\PYG{+w}{     }\PYG{c+c1}{// Digital output pin connected to the LED}

\PYG{k+kr}{void}\PYG{+w}{ }\PYG{n+nb}{setup}\PYG{p}{(}\PYG{p}{)}\PYG{+w}{ }\PYG{p}{\PYGZob{}}
\PYG{+w}{  }\PYG{n+nf}{Serial}\PYG{p}{.}\PYG{n+nf}{begin}\PYG{p}{(}\PYG{l+m+mi}{9600}\PYG{p}{)}\PYG{p}{;}\PYG{+w}{  }\PYG{c+c1}{// Initialize serial communication with a baud rate of 9600}
\PYG{p}{\PYGZcb{}}
\end{sphinxVerbatim}

\item {} 
\sphinxAtStartPar
Reading Analog Input (Getting Data from Potentiometer)

\sphinxAtStartPar
In this segment, we read the analog data from the potentiometer and print it to the serial monitor.

\begin{sphinxVerbatim}[commandchars=\\\{\}]
\PYG{n}{inputValue}\PYG{+w}{ }\PYG{o}{=}\PYG{+w}{ }\PYG{n+nf}{analogRead}\PYG{p}{(}\PYG{n}{analogPin}\PYG{p}{)}\PYG{p}{;}\PYG{+w}{  }\PYG{c+c1}{// Read the analog value from the potentiometer}
\PYG{n+nf}{Serial}\PYG{p}{.}\PYG{n+nf}{print}\PYG{p}{(}\PYG{l+s}{\PYGZdq{}}\PYG{l+s}{Input: }\PYG{l+s}{\PYGZdq{}}\PYG{p}{)}\PYG{p}{;}\PYG{+w}{             }\PYG{c+c1}{// Print \PYGZdq{}Input: \PYGZdq{} to the serial monitor}
\PYG{n+nf}{Serial}\PYG{p}{.}\PYG{n+nf}{println}\PYG{p}{(}\PYG{n}{inputValue}\PYG{p}{)}\PYG{p}{;}\PYG{+w}{          }\PYG{c+c1}{// Print the raw input value to the serial monitor}
\end{sphinxVerbatim}

\item {} 
\sphinxAtStartPar
Mapping and Scaling (Converting Potentiometer Data)

\sphinxAtStartPar
We scale the raw data from the potentiometer, which is in the range of 0\sphinxhyphen{}1023, to a new range of 0\sphinxhyphen{}255.

\sphinxAtStartPar
\sphinxcode{\sphinxupquote{map(value, fromLow, fromHigh, toLow, toHigh)}} is used to convert a number from one range to another. For example, if the value is within the range of \sphinxcode{\sphinxupquote{fromLow}} and \sphinxcode{\sphinxupquote{fromHigh}}, it will be converted to a corresponding value within the range of \sphinxcode{\sphinxupquote{toLow}} and \sphinxcode{\sphinxupquote{toHigh}}, maintaining proportionality between the two ranges.

\sphinxAtStartPar
In this case, since the LED pin (pin 9) has a range of 0\sphinxhyphen{}255, we need to map values in the range of 0\sphinxhyphen{}1023 to match that same scale of 0\sphinxhyphen{}255.

\begin{sphinxVerbatim}[commandchars=\\\{\}]
\PYG{n}{outputValue}\PYG{+w}{ }\PYG{o}{=}\PYG{+w}{ }\PYG{n+nf}{map}\PYG{p}{(}\PYG{n}{inputValue}\PYG{p}{,}\PYG{+w}{ }\PYG{l+m+mi}{0}\PYG{p}{,}\PYG{+w}{ }\PYG{l+m+mi}{1023}\PYG{p}{,}\PYG{+w}{ }\PYG{l+m+mi}{0}\PYG{p}{,}\PYG{+w}{ }\PYG{l+m+mi}{255}\PYG{p}{)}\PYG{p}{;}\PYG{+w}{  }\PYG{c+c1}{// Map the input value to a new range}
\end{sphinxVerbatim}

\item {} 
\sphinxAtStartPar
Controlling LED and Serial Output

\sphinxAtStartPar
Finally, we control the LED’s brightness based on the scaled value and print the scaled value for monitoring.

\begin{sphinxVerbatim}[commandchars=\\\{\}]
\PYG{n+nf}{Serial}\PYG{p}{.}\PYG{n+nf}{print}\PYG{p}{(}\PYG{l+s}{\PYGZdq{}}\PYG{l+s}{Output: }\PYG{l+s}{\PYGZdq{}}\PYG{p}{)}\PYG{p}{;}\PYG{+w}{                        }\PYG{c+c1}{// Print \PYGZdq{}Output: \PYGZdq{} to the serial monitor}
\PYG{n+nf}{Serial}\PYG{p}{.}\PYG{n+nf}{println}\PYG{p}{(}\PYG{n}{outputValue}\PYG{p}{)}\PYG{p}{;}\PYG{+w}{                     }\PYG{c+c1}{// Print the scaled output value to the serial monitor}
\PYG{n+nf}{analogWrite}\PYG{p}{(}\PYG{n}{ledPin}\PYG{p}{,}\PYG{+w}{ }\PYG{n}{outputValue}\PYG{p}{)}\PYG{p}{;}\PYG{+w}{                }\PYG{c+c1}{// Control the LED brightness based on the scaled value}
\PYG{n+nf}{delay}\PYG{p}{(}\PYG{l+m+mi}{1000}\PYG{p}{)}\PYG{p}{;}
\end{sphinxVerbatim}

\end{enumerate}

\sphinxstepscope


\section{DHT11 Module}
\label{\detokenize{Basic_Project/DHT11_Module:dht11-module}}\label{\detokenize{Basic_Project/DHT11_Module:basic-dht11-module}}\label{\detokenize{Basic_Project/DHT11_Module::doc}}

\subsection{Overview}
\label{\detokenize{Basic_Project/DHT11_Module:overview}}
\sphinxAtStartPar
Humidity and temperature are closely related from the physical quantity itself to the actual people’s life.
The temperature and humidity of human environment will directly affect the thermoregulatory function and heat transfer effect of human body.
It will further affect the thinking activity and mental state, thus affecting the efficiency of our study and work.

\sphinxAtStartPar
Temperature is one of the seven basic physical quantities in the International System of Units, which is used to measure the degree of hot and cold of an object.
Celsius is one of the more widely used temperature scales in the world, expressed by the symbol “℃”.

\sphinxAtStartPar
Humidity is the concentration of water vapor present in the air.
The relative humidity of air is commonly used in life and is expressed in \%RH. Relative humidity is closely related to temperature.
For a certain volume of sealed gas, the higher the temperature, the lower the relative humidity, and the lower the temperature, the higher the relative humidity.

\sphinxAtStartPar
The dht11, a digital temperature and humidity sensor, is provided in this kit. It uses a capacitive humidity sensor and thermistor to measure the surrounding air and outputs a digital signal on the data pin.


\subsection{Wiring}
\label{\detokenize{Basic_Project/DHT11_Module:wiring}}
\noindent{\hspace*{\fill}\sphinxincludegraphics{{DHT11_Module_Wiring}.png}\hspace*{\fill}}


\subsection{Schematic Diagram}
\label{\detokenize{Basic_Project/DHT11_Module:schematic-diagram}}
\noindent{\hspace*{\fill}\sphinxincludegraphics[width=0.400\linewidth]{{DHT11_Module_Wiring1}.png}\hspace*{\fill}}


\subsection{Code}
\label{\detokenize{Basic_Project/DHT11_Module:code}}
\begin{sphinxadmonition}{note}{Note:}\begin{itemize}
\item {} 
\sphinxAtStartPar
You can open the file \sphinxcode{\sphinxupquote{09\_DHT11\_Module.ino}} under the path of \sphinxcode{\sphinxupquote{Basic\sphinxhyphen{}Starter\sphinxhyphen{}Kit\sphinxhyphen{}for\sphinxhyphen{}Arduino\sphinxhyphen{}Uno\sphinxhyphen{}R4\sphinxhyphen{}WiFi\sphinxhyphen{}main\textbackslash{}Code}} directly.

\item {} 
\sphinxAtStartPar
To install the library, use the Arduino Library Manager and search for \sphinxstylestrong{“DHT sensor library”} and install it.

\end{itemize}
\end{sphinxadmonition}

\sphinxAtStartPar
After the code is uploaded successfully, you will see the Serial Monitor continuously print out the temperature and humidity, and as the program runs steadily, these two values will become more and more accurate.


\subsection{Code Analysis}
\label{\detokenize{Basic_Project/DHT11_Module:code-analysis}}\begin{enumerate}
\sphinxsetlistlabels{\arabic}{enumi}{enumii}{}{.}%
\item {} 
\sphinxAtStartPar
Inclusion of necessary libraries and definition of constants.
This part of the code includes the DHT sensor library and defines the pin number and sensor type used in this project.

\begin{sphinxadmonition}{note}{Note:}
\sphinxAtStartPar
To install the library, use the Arduino Library Manager and search for \sphinxstylestrong{“DHT sensor library”} and install it.
\end{sphinxadmonition}

\begin{sphinxVerbatim}[commandchars=\\\{\}]
\PYG{c+cp}{\PYGZsh{}}\PYG{c+cp}{include}\PYG{+w}{ }\PYG{c+cpf}{\PYGZlt{}DHT.h\PYGZgt{}}
\PYG{c+cp}{\PYGZsh{}}\PYG{c+cp}{define DHTPIN 2       }\PYG{c+c1}{// Define the pin used to connect the sensor}
\PYG{c+cp}{\PYGZsh{}}\PYG{c+cp}{define DHTTYPE DHT11  }\PYG{c+c1}{// Define the sensor type}
\end{sphinxVerbatim}

\item {} 
\sphinxAtStartPar
Creation of DHT object.
Here we create a DHT object using the defined pin number and sensor type.

\begin{sphinxVerbatim}[commandchars=\\\{\}]
\PYG{n}{DHT}\PYG{+w}{ }\PYG{n+nf}{dht}\PYG{p}{(}\PYG{n}{DHTPIN}\PYG{p}{,}\PYG{+w}{ }\PYG{n}{DHTTYPE}\PYG{p}{)}\PYG{p}{;}\PYG{+w}{  }\PYG{c+c1}{// Create a DHT object}
\end{sphinxVerbatim}

\item {} 
\sphinxAtStartPar
This function is executed once when the Arduino starts. We initialize the serial communication and the DHT sensor in this function.

\begin{sphinxVerbatim}[commandchars=\\\{\}]
\PYG{k+kr}{void}\PYG{+w}{ }\PYG{n+nb}{setup}\PYG{p}{(}\PYG{p}{)}\PYG{+w}{ }\PYG{p}{\PYGZob{}}
\PYG{+w}{  }\PYG{n+nf}{Serial}\PYG{p}{.}\PYG{n+nf}{begin}\PYG{p}{(}\PYG{l+m+mi}{9600}\PYG{p}{)}\PYG{p}{;}
\PYG{+w}{  }\PYG{n+nf}{Serial}\PYG{p}{.}\PYG{n+nf}{println}\PYG{p}{(}\PYG{n}{F}\PYG{p}{(}\PYG{l+s}{\PYGZdq{}}\PYG{l+s}{DHT11 test!}\PYG{l+s}{\PYGZdq{}}\PYG{p}{)}\PYG{p}{)}\PYG{p}{;}
\PYG{+w}{  }\PYG{n}{dht}\PYG{p}{.}\PYG{n+nf}{begin}\PYG{p}{(}\PYG{p}{)}\PYG{p}{;}\PYG{+w}{  }\PYG{c+c1}{// Initialize the DHT sensor}
\PYG{p}{\PYGZcb{}}
\end{sphinxVerbatim}

\item {} 
\sphinxAtStartPar
Main loop.
The \sphinxcode{\sphinxupquote{loop()}} function runs continuously after the setup function. Here, we read the humidity and temperature values, calculate the heat index, and print these values to the serial monitor.  If the sensor read fails (returns NaN), it prints an error message.

\begin{sphinxadmonition}{note}{Note:}
\sphinxAtStartPar
The \sphinxhref{https://en.wikipedia.org/wiki/Heat\_index}{heat\_index} is a way to measure how hot it feels outside by combining the air temperature and the humidity. It is also called the “felt air temperature” or “apparent temperature”.
\end{sphinxadmonition}

\begin{sphinxVerbatim}[commandchars=\\\{\}]
\PYG{k+kr}{void}\PYG{+w}{ }\PYG{n+nb}{loop}\PYG{p}{(}\PYG{p}{)}\PYG{+w}{ }\PYG{p}{\PYGZob{}}
\PYG{+w}{  }\PYG{n+nf}{delay}\PYG{p}{(}\PYG{l+m+mi}{2000}\PYG{p}{)}\PYG{p}{;}
\PYG{+w}{  }\PYG{k+kr}{float}\PYG{+w}{ }\PYG{n}{h}\PYG{+w}{ }\PYG{o}{=}\PYG{+w}{ }\PYG{n}{dht}\PYG{p}{.}\PYG{n}{readHumidity}\PYG{p}{(}\PYG{p}{)}\PYG{p}{;}
\PYG{+w}{  }\PYG{k+kr}{float}\PYG{+w}{ }\PYG{n}{t}\PYG{+w}{ }\PYG{o}{=}\PYG{+w}{ }\PYG{n}{dht}\PYG{p}{.}\PYG{n+nf}{readTemperature}\PYG{p}{(}\PYG{p}{)}\PYG{p}{;}
\PYG{+w}{  }\PYG{k+kr}{float}\PYG{+w}{ }\PYG{n}{f}\PYG{+w}{ }\PYG{o}{=}\PYG{+w}{ }\PYG{n}{dht}\PYG{p}{.}\PYG{n+nf}{readTemperature}\PYG{p}{(}\PYG{k+kr}{true}\PYG{p}{)}\PYG{p}{;}
\PYG{+w}{  }\PYG{k}{if}\PYG{+w}{ }\PYG{p}{(}\PYG{n}{isnan}\PYG{p}{(}\PYG{n}{h}\PYG{p}{)}\PYG{+w}{ }\PYG{o}{|}\PYG{o}{|}\PYG{+w}{ }\PYG{n}{isnan}\PYG{p}{(}\PYG{n}{t}\PYG{p}{)}\PYG{+w}{ }\PYG{o}{|}\PYG{o}{|}\PYG{+w}{ }\PYG{n}{isnan}\PYG{p}{(}\PYG{n}{f}\PYG{p}{)}\PYG{p}{)}\PYG{+w}{ }\PYG{p}{\PYGZob{}}
\PYG{+w}{    }\PYG{n+nf}{Serial}\PYG{p}{.}\PYG{n+nf}{println}\PYG{p}{(}\PYG{n}{F}\PYG{p}{(}\PYG{l+s}{\PYGZdq{}}\PYG{l+s}{Failed to read from DHT sensor!}\PYG{l+s}{\PYGZdq{}}\PYG{p}{)}\PYG{p}{)}\PYG{p}{;}
\PYG{+w}{    }\PYG{k}{return}\PYG{p}{;}
\PYG{+w}{  }\PYG{p}{\PYGZcb{}}
\PYG{+w}{  }\PYG{k+kr}{float}\PYG{+w}{ }\PYG{n}{hif}\PYG{+w}{ }\PYG{o}{=}\PYG{+w}{ }\PYG{n}{dht}\PYG{p}{.}\PYG{n}{computeHeatIndex}\PYG{p}{(}\PYG{n}{f}\PYG{p}{,}\PYG{+w}{ }\PYG{n}{h}\PYG{p}{)}\PYG{p}{;}
\PYG{+w}{  }\PYG{k+kr}{float}\PYG{+w}{ }\PYG{n}{hic}\PYG{+w}{ }\PYG{o}{=}\PYG{+w}{ }\PYG{n}{dht}\PYG{p}{.}\PYG{n}{computeHeatIndex}\PYG{p}{(}\PYG{n}{t}\PYG{p}{,}\PYG{+w}{ }\PYG{n}{h}\PYG{p}{,}\PYG{+w}{ }\PYG{k+kr}{false}\PYG{p}{)}\PYG{p}{;}
\PYG{+w}{  }\PYG{n+nf}{Serial}\PYG{p}{.}\PYG{n+nf}{print}\PYG{p}{(}\PYG{n}{F}\PYG{p}{(}\PYG{l+s}{\PYGZdq{}}\PYG{l+s}{Humidity: }\PYG{l+s}{\PYGZdq{}}\PYG{p}{)}\PYG{p}{)}\PYG{p}{;}
\PYG{+w}{  }\PYG{n+nf}{Serial}\PYG{p}{.}\PYG{n+nf}{print}\PYG{p}{(}\PYG{n}{h}\PYG{p}{)}\PYG{p}{;}
\PYG{+w}{  }\PYG{n+nf}{Serial}\PYG{p}{.}\PYG{n+nf}{print}\PYG{p}{(}\PYG{n}{F}\PYG{p}{(}\PYG{l+s}{\PYGZdq{}}\PYG{l+s}{\PYGZpc{}  Temperature: }\PYG{l+s}{\PYGZdq{}}\PYG{p}{)}\PYG{p}{)}\PYG{p}{;}
\PYG{+w}{  }\PYG{n+nf}{Serial}\PYG{p}{.}\PYG{n+nf}{print}\PYG{p}{(}\PYG{n}{t}\PYG{p}{)}\PYG{p}{;}
\PYG{+w}{  }\PYG{n+nf}{Serial}\PYG{p}{.}\PYG{n+nf}{print}\PYG{p}{(}\PYG{n}{F}\PYG{p}{(}\PYG{l+s}{\PYGZdq{}}\PYG{l+s}{°C }\PYG{l+s}{\PYGZdq{}}\PYG{p}{)}\PYG{p}{)}\PYG{p}{;}
\PYG{+w}{  }\PYG{n+nf}{Serial}\PYG{p}{.}\PYG{n+nf}{print}\PYG{p}{(}\PYG{n}{f}\PYG{p}{)}\PYG{p}{;}
\PYG{+w}{  }\PYG{n+nf}{Serial}\PYG{p}{.}\PYG{n+nf}{print}\PYG{p}{(}\PYG{n}{F}\PYG{p}{(}\PYG{l+s}{\PYGZdq{}}\PYG{l+s}{°F  Heat index: }\PYG{l+s}{\PYGZdq{}}\PYG{p}{)}\PYG{p}{)}\PYG{p}{;}
\PYG{+w}{  }\PYG{n+nf}{Serial}\PYG{p}{.}\PYG{n+nf}{print}\PYG{p}{(}\PYG{n}{hic}\PYG{p}{)}\PYG{p}{;}
\PYG{+w}{  }\PYG{n+nf}{Serial}\PYG{p}{.}\PYG{n+nf}{print}\PYG{p}{(}\PYG{n}{F}\PYG{p}{(}\PYG{l+s}{\PYGZdq{}}\PYG{l+s}{°C }\PYG{l+s}{\PYGZdq{}}\PYG{p}{)}\PYG{p}{)}\PYG{p}{;}
\PYG{+w}{  }\PYG{n+nf}{Serial}\PYG{p}{.}\PYG{n+nf}{print}\PYG{p}{(}\PYG{n}{hif}\PYG{p}{)}\PYG{p}{;}
\PYG{+w}{  }\PYG{n+nf}{Serial}\PYG{p}{.}\PYG{n+nf}{println}\PYG{p}{(}\PYG{n}{F}\PYG{p}{(}\PYG{l+s}{\PYGZdq{}}\PYG{l+s}{°F}\PYG{l+s}{\PYGZdq{}}\PYG{p}{)}\PYG{p}{)}\PYG{p}{;}
\PYG{p}{\PYGZcb{}}
\end{sphinxVerbatim}

\end{enumerate}

\sphinxstepscope


\section{Joystick Module}
\label{\detokenize{Basic_Project/Joystick_Module:joystick-module}}\label{\detokenize{Basic_Project/Joystick_Module:basic-joystick-module}}\label{\detokenize{Basic_Project/Joystick_Module::doc}}

\subsection{Overview}
\label{\detokenize{Basic_Project/Joystick_Module:overview}}
\sphinxAtStartPar
A joystick is an input device consisting of a stick that pivots on a base and reports its angle or direction to the device it is controlling. Joysticks are often used to control video games and robots. A Joystick PS2 is used here.


\subsection{Wiring}
\label{\detokenize{Basic_Project/Joystick_Module:wiring}}
\noindent{\hspace*{\fill}\sphinxincludegraphics[width=0.700\linewidth]{{Joystick_Wiring}.png}\hspace*{\fill}}


\subsection{Schematic Diagram}
\label{\detokenize{Basic_Project/Joystick_Module:schematic-diagram}}
\sphinxAtStartPar
This module has two analog outputs (corresponding to X,Y biaxial offsets).

\sphinxAtStartPar
In this experiment, we use the Uno board to detect the moving direction of the Joystick knob.

\noindent{\hspace*{\fill}\sphinxincludegraphics[width=0.700\linewidth]{{Joystick_Wiring1}.png}\hspace*{\fill}}


\subsection{Code}
\label{\detokenize{Basic_Project/Joystick_Module:code}}
\begin{sphinxadmonition}{note}{Note:}\begin{itemize}
\item {} 
\sphinxAtStartPar
You can open the file \sphinxcode{\sphinxupquote{10\_Joystick\_Module.ino}} under the path of \sphinxcode{\sphinxupquote{Basic\sphinxhyphen{}Starter\sphinxhyphen{}Kit\sphinxhyphen{}for\sphinxhyphen{}Arduino\sphinxhyphen{}Uno\sphinxhyphen{}R4\sphinxhyphen{}WiFi\sphinxhyphen{}main\textbackslash{}Code}} directly.

\end{itemize}
\end{sphinxadmonition}

\sphinxAtStartPar
Now, when you push the rocker, the coordinates of the X and Y axes displayed on the Serial Monitor will change accordingly.


\subsection{Code Analysis}
\label{\detokenize{Basic_Project/Joystick_Module:code-analysis}}
\sphinxAtStartPar
The code is use the serial monitor to print the value of the VRX and VRY pins of the joystick ps2.

\begin{sphinxVerbatim}[commandchars=\\\{\}]
\PYG{k+kr}{void}\PYG{+w}{ }\PYG{n+nb}{loop}\PYG{p}{(}\PYG{p}{)}
\PYG{p}{\PYGZob{}}
\PYG{+w}{    }\PYG{n+nf}{Serial}\PYG{p}{.}\PYG{n+nf}{print}\PYG{p}{(}\PYG{l+s}{\PYGZdq{}}\PYG{l+s}{X: }\PYG{l+s}{\PYGZdq{}}\PYG{p}{)}\PYG{p}{;}
\PYG{+w}{    }\PYG{n+nf}{Serial}\PYG{p}{.}\PYG{n+nf}{print}\PYG{p}{(}\PYG{n+nf}{analogRead}\PYG{p}{(}\PYG{n}{xPin}\PYG{p}{)}\PYG{p}{,}\PYG{+w}{ }\PYG{n}{DEC}\PYG{p}{)}\PYG{p}{;}\PYG{+w}{  }\PYG{c+c1}{// print the value of VRX in DEC}
\PYG{+w}{    }\PYG{n+nf}{Serial}\PYG{p}{.}\PYG{n+nf}{print}\PYG{p}{(}\PYG{l+s}{\PYGZdq{}}\PYG{l+s}{|Y: }\PYG{l+s}{\PYGZdq{}}\PYG{p}{)}\PYG{p}{;}
\PYG{+w}{    }\PYG{n+nf}{Serial}\PYG{p}{.}\PYG{n+nf}{print}\PYG{p}{(}\PYG{n+nf}{analogRead}\PYG{p}{(}\PYG{n}{yPin}\PYG{p}{)}\PYG{p}{,}\PYG{+w}{ }\PYG{n}{DEC}\PYG{p}{)}\PYG{p}{;}\PYG{+w}{  }\PYG{c+c1}{// print the value of VRX in DEC}
\PYG{+w}{    }\PYG{n+nf}{delay}\PYG{p}{(}\PYG{l+m+mi}{50}\PYG{p}{)}\PYG{p}{;}
\PYG{p}{\PYGZcb{}}
\end{sphinxVerbatim}

\sphinxstepscope


\section{2 Channel Relay Module}
\label{\detokenize{Basic_Project/2_Channel_Relay_Module:channel-relay-module}}\label{\detokenize{Basic_Project/2_Channel_Relay_Module:basic-2-channel-relay-module}}\label{\detokenize{Basic_Project/2_Channel_Relay_Module::doc}}

\subsection{Overview}
\label{\detokenize{Basic_Project/2_Channel_Relay_Module:overview}}
\sphinxAtStartPar
As we may know, relay is a device which is used to provide connection between two or more points or devices in response to the input signal applied. In other words, relays provide isolation between the controller and the device as devices may work on AC as well as on DC. However, they receive signals from a micro\sphinxhyphen{}controller which works on DC hence requiring a relay to bridge the gap. Relay is extremely useful when you need to control a large amount of current or voltage with small electrical signal.


\subsection{Wiring}
\label{\detokenize{Basic_Project/2_Channel_Relay_Module:wiring}}
\noindent{\hspace*{\fill}\sphinxincludegraphics[width=0.900\linewidth]{{Realy_Wiring}.png}\hspace*{\fill}}

\sphinxAtStartPar
In this experiment, when the relay closes, the LED will light up; when the relay opens, the LED will go out.


\subsection{Code}
\label{\detokenize{Basic_Project/2_Channel_Relay_Module:code}}
\begin{sphinxadmonition}{note}{Note:}\begin{itemize}
\item {} 
\sphinxAtStartPar
You can open the file \sphinxcode{\sphinxupquote{11\_Realy.ino}} under the path of \sphinxcode{\sphinxupquote{Basic\sphinxhyphen{}Starter\sphinxhyphen{}Kit\sphinxhyphen{}for\sphinxhyphen{}Arduino\sphinxhyphen{}Uno\sphinxhyphen{}R4\sphinxhyphen{}WiFi\sphinxhyphen{}main\textbackslash{}Code}} directly.

\end{itemize}
\end{sphinxadmonition}

\sphinxAtStartPar
Now, send a High level signal, and the relay will close and the LED will light up; send a low one, and it will open and the LED will go out. In addition, you can hear a tick\sphinxhyphen{}tock caused by breaking the normally close contact and closing the normally open one.


\subsection{Code Analysis}
\label{\detokenize{Basic_Project/2_Channel_Relay_Module:code-analysis}}
\begin{sphinxVerbatim}[commandchars=\\\{\}]
\PYG{k+kr}{void}\PYG{+w}{ }\PYG{n+nb}{loop}\PYG{p}{(}\PYG{p}{)}\PYG{+w}{ }\PYG{p}{\PYGZob{}}
\PYG{+w}{  }\PYG{n+nf}{digitalWrite}\PYG{p}{(}\PYG{n}{relayPin}\PYG{p}{,}\PYG{+w}{ }\PYG{k+kr}{HIGH}\PYG{p}{)}\PYG{p}{;}\PYG{+w}{  }\PYG{c+c1}{// Turn the relay on}
\PYG{+w}{  }\PYG{n+nf}{delay}\PYG{p}{(}\PYG{l+m+mi}{1000}\PYG{p}{)}\PYG{p}{;}\PYG{+w}{                   }\PYG{c+c1}{// Wait for one second}
\PYG{+w}{  }\PYG{n+nf}{digitalWrite}\PYG{p}{(}\PYG{n}{relayPin}\PYG{p}{,}\PYG{+w}{ }\PYG{k+kr}{LOW}\PYG{p}{)}\PYG{p}{;}\PYG{+w}{   }\PYG{c+c1}{// Turn the relay off}
\PYG{+w}{  }\PYG{n+nf}{delay}\PYG{p}{(}\PYG{l+m+mi}{1000}\PYG{p}{)}\PYG{p}{;}\PYG{+w}{                   }\PYG{c+c1}{// Wait for one second}
\PYG{p}{\PYGZcb{}}
\end{sphinxVerbatim}

\sphinxAtStartPar
The code in this experiment is simple. First, set relayPin as HIGH level and the LED connected to the relay will light up. Then set relayPin as LOW level and the LED goes out.

\sphinxstepscope


\section{PIR Motion Sensor Module}
\label{\detokenize{Basic_Project/PIR_Motion_Sensor:pir-motion-sensor-module}}\label{\detokenize{Basic_Project/PIR_Motion_Sensor:basic-pir-motion-sensor}}\label{\detokenize{Basic_Project/PIR_Motion_Sensor::doc}}

\subsection{Overview}
\label{\detokenize{Basic_Project/PIR_Motion_Sensor:overview}}
\sphinxAtStartPar
In this lesson, you will learn about PIR motion sensor module. The Passive Infrared(PIR) Motion Sensor is a sensor that detects motion. It is commonly used in security systems and automatic lighting systems. The sensor has two slots that detect infrared radiation. When an object, such as a person, passes in front of the sensor, it detects a change in the amount of infrared radiation and triggers an output signal.


\subsection{Wiring}
\label{\detokenize{Basic_Project/PIR_Motion_Sensor:wiring}}
\noindent{\hspace*{\fill}\sphinxincludegraphics[width=1.000\linewidth]{{PIR_Wiring}.png}\hspace*{\fill}}


\subsection{Schematic Diagram}
\label{\detokenize{Basic_Project/PIR_Motion_Sensor:schematic-diagram}}
\noindent{\hspace*{\fill}\sphinxincludegraphics[width=0.500\linewidth]{{PIR_Wiring1}.png}\hspace*{\fill}}


\subsection{Code}
\label{\detokenize{Basic_Project/PIR_Motion_Sensor:code}}
\begin{sphinxadmonition}{note}{Note:}\begin{itemize}
\item {} 
\sphinxAtStartPar
You can open the file \sphinxcode{\sphinxupquote{12\_PIR\_Motion\_Sensor.ino}} under the path of \sphinxcode{\sphinxupquote{Basic\sphinxhyphen{}Starter\sphinxhyphen{}Kit\sphinxhyphen{}for\sphinxhyphen{}Arduino\sphinxhyphen{}Uno\sphinxhyphen{}R4\sphinxhyphen{}WiFi\sphinxhyphen{}main\textbackslash{}Code}} directly.

\end{itemize}
\end{sphinxadmonition}

\sphinxAtStartPar
After uploading the code to the Arduino Uno board, you can open the serial monitor to observe the sensor’s output. When the PIR (passive infrared) motion sensor detects movement, the serial monitor will display the message “Somebody here!” to indicate that motion has been detected. If no motion is detected, the message “Monitoring…” will be shown instead.

\sphinxAtStartPar
The PIR sensor outputs a digital HIGH or LOW signal, corresponding to detected or undetected motion, respectively. Unlike an analog sensor that provides a range of values, the digital output from this PIR sensor will either be HIGH (typically represented as ‘1’) or LOW (typically represented as ‘0’).

\sphinxAtStartPar
Note that the actual sensitivity and range of detection can vary based on the PIR sensor’s characteristics and the environmental conditions. Therefore, it is advisable to calibrate the sensor according to your specific needs.

\sphinxstepscope


\section{Audio Module and Speaker}
\label{\detokenize{Basic_Project/Audio_Module_Speaker:audio-module-and-speaker}}\label{\detokenize{Basic_Project/Audio_Module_Speaker:basic-audio-module-speaker}}\label{\detokenize{Basic_Project/Audio_Module_Speaker::doc}}

\subsection{Overview}
\label{\detokenize{Basic_Project/Audio_Module_Speaker:overview}}
\sphinxAtStartPar
In this lesson, you will learn about the Audio Module and Speaker when used with an Arduino Uno board. These components are widely utilized in various electronic applications, including musical toys, DIY sound systems, alarms, and even sophisticated musical instruments. By combining an Arduino with an Audio Module and Speaker, you can create a simple yet effective melody player.


\subsection{Wiring}
\label{\detokenize{Basic_Project/Audio_Module_Speaker:wiring}}
\sphinxAtStartPar
As this is a mono amplifier, you can connect pin 8 to the L or R pin of the audio amplifier module.

\sphinxAtStartPar
The 10K resistor is used to reduce high\sphinxhyphen{}frequency noise and lower the audio volume. It forms an RC low\sphinxhyphen{}pass filter with the parasitic capacitance of the DAC and audio amplifier. This filter decreases the amplitude of high\sphinxhyphen{}frequency signals, effectively reducing high\sphinxhyphen{}frequency noise. So, adding the 10K resistor makes the music sound softer and eliminates unwanted high\sphinxhyphen{}frequency noise.

\noindent{\hspace*{\fill}\sphinxincludegraphics[width=1.000\linewidth]{{Audio_Speaker_Wiring}.png}\hspace*{\fill}}


\subsection{Schematic Diagram}
\label{\detokenize{Basic_Project/Audio_Module_Speaker:schematic-diagram}}
\noindent{\hspace*{\fill}\sphinxincludegraphics[width=0.800\linewidth]{{Audio_Speaker_Wiring1}.png}\hspace*{\fill}}


\subsection{Code}
\label{\detokenize{Basic_Project/Audio_Module_Speaker:code}}
\begin{sphinxadmonition}{note}{Note:}\begin{itemize}
\item {} 
\sphinxAtStartPar
You can open the file \sphinxcode{\sphinxupquote{13\_Audio\_Module\_Speaker.ino}} under the path of \sphinxcode{\sphinxupquote{Basic\sphinxhyphen{}Starter\sphinxhyphen{}Kit\sphinxhyphen{}for\sphinxhyphen{}Arduino\sphinxhyphen{}Uno\sphinxhyphen{}R4\sphinxhyphen{}WiFi\sphinxhyphen{}main\textbackslash{}Code}} directly.

\end{itemize}
\end{sphinxadmonition}

\sphinxAtStartPar
At the time when you finish uploading the codes to the R4 board, you can hear a melody containing seven notes.


\subsection{Code Analysis}
\label{\detokenize{Basic_Project/Audio_Module_Speaker:code-analysis}}\begin{enumerate}
\sphinxsetlistlabels{\arabic}{enumi}{enumii}{}{.}%
\item {} 
\sphinxAtStartPar
Including the pitches library:
This library provides the frequency values for various musical notes, allowing you to use musical notation in your code.

\begin{sphinxadmonition}{note}{Note:}
\sphinxAtStartPar
Please place the \sphinxcode{\sphinxupquote{pitches.h}} file in the same directory as the code to ensure proper functioning. \sphinxhref{https://raw.githubusercontent.com/lafvintech/Basic-Starter-Kit-for-Arduino-Uno-R4-WiFi/main/Code/05\_Passive\_Buzzer/pitches.h}{pitches.h}

\noindent\sphinxincludegraphics{{Audio_Speaker_Code}.png}
\end{sphinxadmonition}

\begin{sphinxVerbatim}[commandchars=\\\{\}]
\PYG{c+cp}{\PYGZsh{}}\PYG{c+cp}{include}\PYG{+w}{ }\PYG{c+cpf}{\PYGZdq{}pitches.h\PYGZdq{}}
\end{sphinxVerbatim}

\item {} 
\sphinxAtStartPar
Defining constants and arrays:
\begin{itemize}
\item {} 
\sphinxAtStartPar
\sphinxcode{\sphinxupquote{speakerPin}} is the digital pin on the Arduino where the speaker is connected.

\item {} 
\sphinxAtStartPar
\sphinxcode{\sphinxupquote{melody{[}{]}}} is an array that stores the sequence of notes to be played.

\item {} 
\sphinxAtStartPar
\sphinxcode{\sphinxupquote{noteDurations{[}{]}}} is an array that stores the duration of each note in the melody.

\end{itemize}

\begin{sphinxVerbatim}[commandchars=\\\{\}]
\PYG{k+kr}{const}\PYG{+w}{ }\PYG{k+kr}{int}\PYG{+w}{ }\PYG{n}{speakerPin}\PYG{+w}{ }\PYG{o}{=}\PYG{+w}{ }\PYG{l+m+mi}{8}\PYG{p}{;}
\PYG{k+kr}{int}\PYG{+w}{ }\PYG{n}{melody}\PYG{p}{[}\PYG{p}{]}\PYG{+w}{ }\PYG{o}{=}\PYG{+w}{ }\PYG{p}{\PYGZob{}}
\PYG{+w}{  }\PYG{n}{NOTE\PYGZus{}C4}\PYG{p}{,}\PYG{+w}{ }\PYG{n}{NOTE\PYGZus{}G3}\PYG{p}{,}\PYG{+w}{ }\PYG{n}{NOTE\PYGZus{}G3}\PYG{p}{,}\PYG{+w}{ }\PYG{n}{NOTE\PYGZus{}A3}\PYG{p}{,}\PYG{+w}{ }\PYG{n}{NOTE\PYGZus{}G3}\PYG{p}{,}\PYG{+w}{ }\PYG{l+m+mi}{0}\PYG{p}{,}\PYG{+w}{ }\PYG{n}{NOTE\PYGZus{}B3}\PYG{p}{,}\PYG{+w}{ }\PYG{n}{NOTE\PYGZus{}C4}
\PYG{p}{\PYGZcb{}}\PYG{p}{;}
\PYG{k+kr}{int}\PYG{+w}{ }\PYG{n}{noteDurations}\PYG{p}{[}\PYG{p}{]}\PYG{+w}{ }\PYG{o}{=}\PYG{+w}{ }\PYG{p}{\PYGZob{}}
\PYG{+w}{  }\PYG{l+m+mi}{4}\PYG{p}{,}\PYG{+w}{ }\PYG{l+m+mi}{8}\PYG{p}{,}\PYG{+w}{ }\PYG{l+m+mi}{8}\PYG{p}{,}\PYG{+w}{ }\PYG{l+m+mi}{4}\PYG{p}{,}\PYG{+w}{ }\PYG{l+m+mi}{4}\PYG{p}{,}\PYG{+w}{ }\PYG{l+m+mi}{4}\PYG{p}{,}\PYG{+w}{ }\PYG{l+m+mi}{4}\PYG{p}{,}\PYG{+w}{ }\PYG{l+m+mi}{4}
\PYG{p}{\PYGZcb{}}\PYG{p}{;}
\end{sphinxVerbatim}

\item {} 
\sphinxAtStartPar
Playing the melody:
\begin{itemize}
\item {} 
\sphinxAtStartPar
The \sphinxcode{\sphinxupquote{for}} loop iterates over each note in the melody.

\item {} 
\sphinxAtStartPar
The \sphinxcode{\sphinxupquote{tone()}} function plays a note on the spekaer for a specific duration.

\item {} 
\sphinxAtStartPar
A delay is added between notes to distinguish them.

\item {} 
\sphinxAtStartPar
The \sphinxcode{\sphinxupquote{noTone()}} function stops the sound.

\end{itemize}

\begin{sphinxVerbatim}[commandchars=\\\{\}]
\PYG{k+kr}{void}\PYG{+w}{ }\PYG{n+nb}{setup}\PYG{p}{(}\PYG{p}{)}\PYG{+w}{ }\PYG{p}{\PYGZob{}}
\PYG{+w}{  }\PYG{k}{for}\PYG{+w}{ }\PYG{p}{(}\PYG{k+kr}{int}\PYG{+w}{ }\PYG{n}{thisNote}\PYG{+w}{ }\PYG{o}{=}\PYG{+w}{ }\PYG{l+m+mi}{0}\PYG{p}{;}\PYG{+w}{ }\PYG{n}{thisNote}\PYG{+w}{ }\PYG{o}{\PYGZlt{}}\PYG{+w}{ }\PYG{l+m+mi}{8}\PYG{p}{;}\PYG{+w}{ }\PYG{n}{thisNote}\PYG{o}{+}\PYG{o}{+}\PYG{p}{)}\PYG{+w}{ }\PYG{p}{\PYGZob{}}
\PYG{+w}{    }\PYG{k+kr}{int}\PYG{+w}{ }\PYG{n}{noteDuration}\PYG{+w}{ }\PYG{o}{=}\PYG{+w}{ }\PYG{l+m+mi}{1000}\PYG{+w}{ }\PYG{o}{/}\PYG{+w}{ }\PYG{n}{noteDurations}\PYG{p}{[}\PYG{n}{thisNote}\PYG{p}{]}\PYG{p}{;}
\PYG{+w}{    }\PYG{n+nf}{tone}\PYG{p}{(}\PYG{n}{speakerPin}\PYG{p}{,}\PYG{+w}{ }\PYG{n}{melody}\PYG{p}{[}\PYG{n}{thisNote}\PYG{p}{]}\PYG{p}{,}\PYG{+w}{ }\PYG{n}{noteDuration}\PYG{p}{)}\PYG{p}{;}
\PYG{+w}{    }\PYG{k+kr}{int}\PYG{+w}{ }\PYG{n}{pauseBetweenNotes}\PYG{+w}{ }\PYG{o}{=}\PYG{+w}{ }\PYG{n}{noteDuration}\PYG{+w}{ }\PYG{o}{*}\PYG{+w}{ }\PYG{l+m+mf}{1.30}\PYG{p}{;}
\PYG{+w}{    }\PYG{n+nf}{delay}\PYG{p}{(}\PYG{n}{pauseBetweenNotes}\PYG{p}{)}\PYG{p}{;}
\PYG{+w}{    }\PYG{n+nf}{noTone}\PYG{p}{(}\PYG{n}{speakerPin}\PYG{p}{)}\PYG{p}{;}
\PYG{+w}{  }\PYG{p}{\PYGZcb{}}
\PYG{p}{\PYGZcb{}}
\end{sphinxVerbatim}

\item {} 
\sphinxAtStartPar
Empty loop function:
Since the melody is played only once in the setup, there’s no code in the loop function.

\item {} 
\sphinxAtStartPar
Feel free to experiment with altering the notes and durations in the \sphinxcode{\sphinxupquote{melody{[}{]}}} and \sphinxcode{\sphinxupquote{noteDurations{[}{]}}} arrays to create your own melodies. If you’re interested, there is a GitHub repository (\sphinxhref{https://github.com/robsoncouto/arduino-songs}{arduino\sphinxhyphen{}songs}) that offers Arduino code for playing various songs. While their approach may differ from this project, you can consult their notes and durations for reference.

\end{enumerate}

\sphinxstepscope


\section{0.96 inch IIC OLED}
\label{\detokenize{Basic_Project/0.96_inch_OLED:inch-iic-oled}}\label{\detokenize{Basic_Project/0.96_inch_OLED:basic-0-96-inch-oled}}\label{\detokenize{Basic_Project/0.96_inch_OLED::doc}}

\subsection{Overview}
\label{\detokenize{Basic_Project/0.96_inch_OLED:overview}}
\sphinxAtStartPar
In this lesson, you will learn about OLED Displays using the SSD1306 driver. OLED (Organic Light\sphinxhyphen{}Emitting Diodes) displays are widely used in various electronic devices such as smartwatches, mobile phones, and even televisions. The SSD1306 is a single\sphinxhyphen{}chip CMOS OLED/PLED driver with controller for organic/polymer light emitting diode dot\sphinxhyphen{}matrix graphic display system. It offers a crisp and clear visual output through the means of organic material\sphinxhyphen{}based diodes that emit light when an electric current passes through them.

\sphinxAtStartPar
In the code provided, an OLED display is interfaced with an Arduino board via the I2C protocol. The code uses the Adafruit SSD1306 library to control the display. The program covers various functionalities such as:
\begin{enumerate}
\sphinxsetlistlabels{\arabic}{enumi}{enumii}{}{.}%
\item {} 
\sphinxAtStartPar
Displaying text: “Hello world!” is printed on the screen.

\item {} 
\sphinxAtStartPar
Inverted text: The text “Hello world!” is displayed in an inverted color scheme.

\item {} 
\sphinxAtStartPar
Numerical Display: The numbers 123456789 are displayed.

\item {} 
\sphinxAtStartPar
ASCII Characters: A set of ASCII characters are displayed.

\item {} 
\sphinxAtStartPar
Font Size: The text “Hello!” is displayed with an increased font size.

\item {} 
\sphinxAtStartPar
Scrolling: Text is scrolled horizontally across the display.

\item {} 
\sphinxAtStartPar
Bitmap Display: A predefined bitmap image is displayed on the OLED screen.

\end{enumerate}

\sphinxAtStartPar
This OLED display can be used in a multitude of applications including digital clocks, mini game consoles, information displays, and so on. It offers a great way to provide a user interface in compact and portable devices.


\subsection{Wiring}
\label{\detokenize{Basic_Project/0.96_inch_OLED:wiring}}
\noindent{\hspace*{\fill}\sphinxincludegraphics{{0.96_inch_OLED_Wiring}.png}\hspace*{\fill}}


\subsection{Schematic Diagram}
\label{\detokenize{Basic_Project/0.96_inch_OLED:schematic-diagram}}
\noindent{\hspace*{\fill}\sphinxincludegraphics[width=0.700\linewidth]{{0.96_inch_OLED_Wiring1}.png}\hspace*{\fill}}


\subsection{Code}
\label{\detokenize{Basic_Project/0.96_inch_OLED:code}}
\begin{sphinxadmonition}{note}{Note:}\begin{itemize}
\item {} 
\sphinxAtStartPar
You can open the file \sphinxcode{\sphinxupquote{14\_IIC\_OLED.ino}} under the path of \sphinxcode{\sphinxupquote{Basic\sphinxhyphen{}Starter\sphinxhyphen{}Kit\sphinxhyphen{}for\sphinxhyphen{}Arduino\sphinxhyphen{}Uno\sphinxhyphen{}R4\sphinxhyphen{}WiFi\sphinxhyphen{}main\textbackslash{}Code}} directly.

\item {} 
\sphinxAtStartPar
To install the library, use the Arduino Library Manager and search for \sphinxstylestrong{“Adafruit SSD1306”} and \sphinxstylestrong{“Adafruit GFX”} and install it.

\end{itemize}
\end{sphinxadmonition}


\subsection{Code Analysis}
\label{\detokenize{Basic_Project/0.96_inch_OLED:code-analysis}}\begin{enumerate}
\sphinxsetlistlabels{\arabic}{enumi}{enumii}{}{.}%
\item {} 
\sphinxAtStartPar
\sphinxstylestrong{Library Inclusion and Initial Definitions}:
The necessary libraries for interfacing with the OLED are included. Following that, definitions regarding the OLED’s dimensions and I2C address are provided.
\begin{itemize}
\item {} 
\sphinxAtStartPar
\sphinxstylestrong{Adafruit SSD1306}: This library is designed to help with the interfacing of the SSD1306 OLED display. It provides methods to initialize the display, control its settings, and display content.

\item {} 
\sphinxAtStartPar
\sphinxstylestrong{Adafruit GFX Library}: This is a core graphics library for displaying text, producing colors, drawing shapes, etc., on various screens including OLEDs.

\end{itemize}

\begin{sphinxadmonition}{note}{Note:}
\sphinxAtStartPar
To install the library, use the Arduino Library Manager and search for \sphinxstylestrong{“Adafruit SSD1306”} and \sphinxstylestrong{“Adafruit GFX”} and install it.
\end{sphinxadmonition}

\begin{sphinxVerbatim}[commandchars=\\\{\}]
\PYG{c+cp}{\PYGZsh{}}\PYG{c+cp}{include}\PYG{+w}{ }\PYG{c+cpf}{\PYGZlt{}SPI.h\PYGZgt{}}
\PYG{c+cp}{\PYGZsh{}}\PYG{c+cp}{include}\PYG{+w}{ }\PYG{c+cpf}{\PYGZlt{}Wire.h\PYGZgt{}}
\PYG{c+cp}{\PYGZsh{}}\PYG{c+cp}{include}\PYG{+w}{ }\PYG{c+cpf}{\PYGZlt{}Adafruit\PYGZus{}GFX.h\PYGZgt{}}
\PYG{c+cp}{\PYGZsh{}}\PYG{c+cp}{include}\PYG{+w}{ }\PYG{c+cpf}{\PYGZlt{}Adafruit\PYGZus{}SSD1306.h\PYGZgt{}}

\PYG{c+cp}{\PYGZsh{}}\PYG{c+cp}{define SCREEN\PYGZus{}WIDTH 128  }\PYG{c+c1}{// OLED display width, in pixels}
\PYG{c+cp}{\PYGZsh{}}\PYG{c+cp}{define SCREEN\PYGZus{}HEIGHT 64  }\PYG{c+c1}{// OLED display height, in pixels}

\PYG{c+cp}{\PYGZsh{}}\PYG{c+cp}{define OLED\PYGZus{}RESET \PYGZhy{}1}
\PYG{c+cp}{\PYGZsh{}}\PYG{c+cp}{define SCREEN\PYGZus{}ADDRESS 0x3C}
\end{sphinxVerbatim}

\item {} 
\sphinxAtStartPar
\sphinxstylestrong{Bitmap Data}:
Bitmap data for displaying a custom icon on the OLED screen. This data represents an image in a format that the OLED can interpret.

\sphinxAtStartPar
You can use this online tool called \sphinxhref{https://javl.github.io/image2cpp/}{image2cpp} that can turn your image into an array.

\sphinxAtStartPar
The \sphinxcode{\sphinxupquote{PROGMEM}} keyword denotes that the array is stored in the program memory of the Arduino microcontroller. Storing data in program memory(PROGMEM) instead of RAM can be helpful for large amounts of data, which would otherwise take up too much space in RAM.

\begin{sphinxVerbatim}[commandchars=\\\{\}]
\PYG{k+kr}{static}\PYG{+w}{ }\PYG{k+kr}{const}\PYG{+w}{ }\PYG{k+kr}{unsigned}\PYG{+w}{ }\PYG{k+kr}{char}\PYG{+w}{ }\PYG{k+kr}{PROGMEM}\PYG{+w}{ }\PYG{n}{L\PYGZus{}Icon}\PYG{p}{[}\PYG{p}{]}\PYG{+w}{ }\PYG{o}{=}\PYG{+w}{ }\PYG{p}{\PYGZob{}}\PYG{p}{.}\PYG{p}{.}\PYG{p}{.}\PYG{p}{\PYGZcb{}}\PYG{p}{;}
\end{sphinxVerbatim}

\item {} 
\sphinxAtStartPar
\sphinxstylestrong{Setup Function (Initialization and Display)}:
The \sphinxcode{\sphinxupquote{setup()}} function initializes the OLED and displays a series of patterns, texts, and animations.

\begin{sphinxVerbatim}[commandchars=\\\{\}]
\PYG{k+kr}{void}\PYG{+w}{ }\PYG{n+nb}{setup}\PYG{p}{(}\PYG{p}{)}\PYG{+w}{ }\PYG{p}{\PYGZob{}}
\PYG{+w}{   }\PYG{p}{.}\PYG{p}{.}\PYG{p}{.}\PYG{+w}{  }\PYG{c+c1}{// Serial initialization and OLED object initialization}
\PYG{+w}{   }\PYG{p}{.}\PYG{p}{.}\PYG{p}{.}\PYG{+w}{  }\PYG{c+c1}{// Displaying various text, numbers, and animations}
\PYG{p}{\PYGZcb{}}
\end{sphinxVerbatim}

\end{enumerate}

\sphinxstepscope


\chapter{Extension\_Project}
\label{\detokenize{Extension_Project/Extension_Project:extension-project}}\label{\detokenize{Extension_Project/Extension_Project::doc}}
\sphinxstepscope


\section{12x8 LED Matrix}
\label{\detokenize{Extension_Project/12x8_LED_Matrix:x8-led-matrix}}\label{\detokenize{Extension_Project/12x8_LED_Matrix:ext-12x8-led-matrix}}\label{\detokenize{Extension_Project/12x8_LED_Matrix::doc}}
\sphinxAtStartPar
The Arduino UNO R4 WiFi comes with an integrated 12x8 LED Matrix that can be programmed to display a variety of graphics, animations, act as an interface, or even facilitate gaming experiences.

\noindent\sphinxincludegraphics[width=0.600\linewidth]{{LED_Matrix}.png}

\sphinxAtStartPar
In this guide, we provide a straightforward example to help you display your desired pattern on the LED Matrix.


\subsection{How to store LED matrix data in Arduino}
\label{\detokenize{Extension_Project/12x8_LED_Matrix:how-to-store-led-matrix-data-in-arduino}}
\sphinxAtStartPar
To utilize the LED matrix, you’ll need the \sphinxcode{\sphinxupquote{Arduino\_LED\_Matrix}} library, which is installed along with the Renesas core.

\sphinxAtStartPar
The LED Matrix library for the UNO R4 WiFi operates by creating and loading frames into a buffer to display them. A frame, also known as an “image,” represents what is currently shown on the matrix. In an animation consisting of multiple images, each image is considered a frame.

\sphinxAtStartPar
To control the 12x8 LED matrix on the UNO R4 WiFi, a minimum of 96 bits of memory space is required. The library offers two approaches for this.

\sphinxAtStartPar
\sphinxstylestrong{One approach uses a two\sphinxhyphen{}dimensional array}, with zeros and ones to represent whether the corresponding LED is off or on.  Each number corresponds to an LED on the LED matrix. The following array illustrates a heart\sphinxhyphen{}shaped pattern.

\begin{sphinxVerbatim}[commandchars=\\\{\}]
\PYG{c+c1}{// Use a two\PYGZhy{}dimensional array to represent a 12x8 LED matrix.}
\PYG{k+kr}{byte}\PYG{+w}{ }\PYG{n}{frame}\PYG{p}{[}\PYG{l+m+mi}{8}\PYG{p}{]}\PYG{p}{[}\PYG{l+m+mi}{12}\PYG{p}{]}\PYG{+w}{ }\PYG{o}{=}\PYG{+w}{ }\PYG{p}{\PYGZob{}}
\PYG{+w}{  }\PYG{p}{\PYGZob{}}\PYG{+w}{ }\PYG{l+m+mi}{0}\PYG{p}{,}\PYG{+w}{ }\PYG{l+m+mi}{0}\PYG{p}{,}\PYG{+w}{ }\PYG{l+m+mi}{1}\PYG{p}{,}\PYG{+w}{ }\PYG{l+m+mi}{1}\PYG{p}{,}\PYG{+w}{ }\PYG{l+m+mi}{0}\PYG{p}{,}\PYG{+w}{ }\PYG{l+m+mi}{0}\PYG{p}{,}\PYG{+w}{ }\PYG{l+m+mi}{0}\PYG{p}{,}\PYG{+w}{ }\PYG{l+m+mi}{1}\PYG{p}{,}\PYG{+w}{ }\PYG{l+m+mi}{1}\PYG{p}{,}\PYG{+w}{ }\PYG{l+m+mi}{0}\PYG{p}{,}\PYG{+w}{ }\PYG{l+m+mi}{0}\PYG{p}{,}\PYG{+w}{ }\PYG{l+m+mi}{0}\PYG{+w}{ }\PYG{p}{\PYGZcb{}}\PYG{p}{,}
\PYG{+w}{  }\PYG{p}{\PYGZob{}}\PYG{+w}{ }\PYG{l+m+mi}{0}\PYG{p}{,}\PYG{+w}{ }\PYG{l+m+mi}{1}\PYG{p}{,}\PYG{+w}{ }\PYG{l+m+mi}{0}\PYG{p}{,}\PYG{+w}{ }\PYG{l+m+mi}{0}\PYG{p}{,}\PYG{+w}{ }\PYG{l+m+mi}{1}\PYG{p}{,}\PYG{+w}{ }\PYG{l+m+mi}{0}\PYG{p}{,}\PYG{+w}{ }\PYG{l+m+mi}{1}\PYG{p}{,}\PYG{+w}{ }\PYG{l+m+mi}{0}\PYG{p}{,}\PYG{+w}{ }\PYG{l+m+mi}{0}\PYG{p}{,}\PYG{+w}{ }\PYG{l+m+mi}{1}\PYG{p}{,}\PYG{+w}{ }\PYG{l+m+mi}{0}\PYG{p}{,}\PYG{+w}{ }\PYG{l+m+mi}{0}\PYG{+w}{ }\PYG{p}{\PYGZcb{}}\PYG{p}{,}
\PYG{+w}{  }\PYG{p}{\PYGZob{}}\PYG{+w}{ }\PYG{l+m+mi}{0}\PYG{p}{,}\PYG{+w}{ }\PYG{l+m+mi}{1}\PYG{p}{,}\PYG{+w}{ }\PYG{l+m+mi}{0}\PYG{p}{,}\PYG{+w}{ }\PYG{l+m+mi}{0}\PYG{p}{,}\PYG{+w}{ }\PYG{l+m+mi}{0}\PYG{p}{,}\PYG{+w}{ }\PYG{l+m+mi}{1}\PYG{p}{,}\PYG{+w}{ }\PYG{l+m+mi}{0}\PYG{p}{,}\PYG{+w}{ }\PYG{l+m+mi}{0}\PYG{p}{,}\PYG{+w}{ }\PYG{l+m+mi}{0}\PYG{p}{,}\PYG{+w}{ }\PYG{l+m+mi}{1}\PYG{p}{,}\PYG{+w}{ }\PYG{l+m+mi}{0}\PYG{p}{,}\PYG{+w}{ }\PYG{l+m+mi}{0}\PYG{+w}{ }\PYG{p}{\PYGZcb{}}\PYG{p}{,}
\PYG{+w}{  }\PYG{p}{\PYGZob{}}\PYG{+w}{ }\PYG{l+m+mi}{0}\PYG{p}{,}\PYG{+w}{ }\PYG{l+m+mi}{0}\PYG{p}{,}\PYG{+w}{ }\PYG{l+m+mi}{1}\PYG{p}{,}\PYG{+w}{ }\PYG{l+m+mi}{0}\PYG{p}{,}\PYG{+w}{ }\PYG{l+m+mi}{0}\PYG{p}{,}\PYG{+w}{ }\PYG{l+m+mi}{0}\PYG{p}{,}\PYG{+w}{ }\PYG{l+m+mi}{0}\PYG{p}{,}\PYG{+w}{ }\PYG{l+m+mi}{0}\PYG{p}{,}\PYG{+w}{ }\PYG{l+m+mi}{1}\PYG{p}{,}\PYG{+w}{ }\PYG{l+m+mi}{0}\PYG{p}{,}\PYG{+w}{ }\PYG{l+m+mi}{0}\PYG{p}{,}\PYG{+w}{ }\PYG{l+m+mi}{0}\PYG{+w}{ }\PYG{p}{\PYGZcb{}}\PYG{p}{,}
\PYG{+w}{  }\PYG{p}{\PYGZob{}}\PYG{+w}{ }\PYG{l+m+mi}{0}\PYG{p}{,}\PYG{+w}{ }\PYG{l+m+mi}{0}\PYG{p}{,}\PYG{+w}{ }\PYG{l+m+mi}{0}\PYG{p}{,}\PYG{+w}{ }\PYG{l+m+mi}{1}\PYG{p}{,}\PYG{+w}{ }\PYG{l+m+mi}{0}\PYG{p}{,}\PYG{+w}{ }\PYG{l+m+mi}{0}\PYG{p}{,}\PYG{+w}{ }\PYG{l+m+mi}{0}\PYG{p}{,}\PYG{+w}{ }\PYG{l+m+mi}{1}\PYG{p}{,}\PYG{+w}{ }\PYG{l+m+mi}{0}\PYG{p}{,}\PYG{+w}{ }\PYG{l+m+mi}{0}\PYG{p}{,}\PYG{+w}{ }\PYG{l+m+mi}{0}\PYG{p}{,}\PYG{+w}{ }\PYG{l+m+mi}{0}\PYG{+w}{ }\PYG{p}{\PYGZcb{}}\PYG{p}{,}
\PYG{+w}{  }\PYG{p}{\PYGZob{}}\PYG{+w}{ }\PYG{l+m+mi}{0}\PYG{p}{,}\PYG{+w}{ }\PYG{l+m+mi}{0}\PYG{p}{,}\PYG{+w}{ }\PYG{l+m+mi}{0}\PYG{p}{,}\PYG{+w}{ }\PYG{l+m+mi}{0}\PYG{p}{,}\PYG{+w}{ }\PYG{l+m+mi}{1}\PYG{p}{,}\PYG{+w}{ }\PYG{l+m+mi}{0}\PYG{p}{,}\PYG{+w}{ }\PYG{l+m+mi}{1}\PYG{p}{,}\PYG{+w}{ }\PYG{l+m+mi}{0}\PYG{p}{,}\PYG{+w}{ }\PYG{l+m+mi}{0}\PYG{p}{,}\PYG{+w}{ }\PYG{l+m+mi}{0}\PYG{p}{,}\PYG{+w}{ }\PYG{l+m+mi}{0}\PYG{p}{,}\PYG{+w}{ }\PYG{l+m+mi}{0}\PYG{+w}{ }\PYG{p}{\PYGZcb{}}\PYG{p}{,}
\PYG{+w}{  }\PYG{p}{\PYGZob{}}\PYG{+w}{ }\PYG{l+m+mi}{0}\PYG{p}{,}\PYG{+w}{ }\PYG{l+m+mi}{0}\PYG{p}{,}\PYG{+w}{ }\PYG{l+m+mi}{0}\PYG{p}{,}\PYG{+w}{ }\PYG{l+m+mi}{0}\PYG{p}{,}\PYG{+w}{ }\PYG{l+m+mi}{0}\PYG{p}{,}\PYG{+w}{ }\PYG{l+m+mi}{1}\PYG{p}{,}\PYG{+w}{ }\PYG{l+m+mi}{0}\PYG{p}{,}\PYG{+w}{ }\PYG{l+m+mi}{0}\PYG{p}{,}\PYG{+w}{ }\PYG{l+m+mi}{0}\PYG{p}{,}\PYG{+w}{ }\PYG{l+m+mi}{0}\PYG{p}{,}\PYG{+w}{ }\PYG{l+m+mi}{0}\PYG{p}{,}\PYG{+w}{ }\PYG{l+m+mi}{0}\PYG{+w}{ }\PYG{p}{\PYGZcb{}}\PYG{p}{,}
\PYG{+w}{  }\PYG{p}{\PYGZob{}}\PYG{+w}{ }\PYG{l+m+mi}{0}\PYG{p}{,}\PYG{+w}{ }\PYG{l+m+mi}{0}\PYG{p}{,}\PYG{+w}{ }\PYG{l+m+mi}{0}\PYG{p}{,}\PYG{+w}{ }\PYG{l+m+mi}{0}\PYG{p}{,}\PYG{+w}{ }\PYG{l+m+mi}{0}\PYG{p}{,}\PYG{+w}{ }\PYG{l+m+mi}{0}\PYG{p}{,}\PYG{+w}{ }\PYG{l+m+mi}{0}\PYG{p}{,}\PYG{+w}{ }\PYG{l+m+mi}{0}\PYG{p}{,}\PYG{+w}{ }\PYG{l+m+mi}{0}\PYG{p}{,}\PYG{+w}{ }\PYG{l+m+mi}{0}\PYG{p}{,}\PYG{+w}{ }\PYG{l+m+mi}{0}\PYG{p}{,}\PYG{+w}{ }\PYG{l+m+mi}{0}\PYG{+w}{ }\PYG{p}{\PYGZcb{}}
\PYG{p}{\PYGZcb{}}\PYG{p}{;}
\end{sphinxVerbatim}

\sphinxAtStartPar
\sphinxstylestrong{Another approach employs an array of 32\sphinxhyphen{}bit integers} to maintain the LED matrix status. This method is more compact but slightly more complex. Each \sphinxcode{\sphinxupquote{unsigned long}} stores 32 bits. Hence, for a 12x8 LED matrix, which contains 96 LEDs, you’ll need at least three \sphinxcode{\sphinxupquote{unsigned long}} variables.
\begin{enumerate}
\sphinxsetlistlabels{\arabic}{enumi}{enumii}{}{.}%
\item {} 
\sphinxAtStartPar
Each \sphinxcode{\sphinxupquote{unsigned long}} contains 32 bits, and you can think of these bits as the state of a certain part in an LED matrix.

\item {} 
\sphinxAtStartPar
These \sphinxcode{\sphinxupquote{unsigned long}} variables form an array that encapsulates the complete LED matrix state.

\end{enumerate}

\sphinxAtStartPar
Here’s a code snippet using three \sphinxtitleref{unsigned long} variables to represent a 12x8 LED matrix.

\begin{sphinxVerbatim}[commandchars=\\\{\}]
\PYG{c+c1}{// Use an array of 32\PYGZhy{}bit integers to store the LED matrix.}
\PYG{k+kr}{unsigned}\PYG{+w}{ }\PYG{k+kr}{long}\PYG{+w}{ }\PYG{n}{frame}\PYG{p}{[}\PYG{p}{]}\PYG{+w}{ }\PYG{o}{=}\PYG{+w}{ }\PYG{p}{\PYGZob{}}
\PYG{+w}{  }\PYG{l+m+mh}{0x3184a444}\PYG{p}{,}\PYG{+w}{ }\PYG{c+c1}{// State of the first 32 LEDs}
\PYG{+w}{  }\PYG{l+m+mh}{0x42081100}\PYG{p}{,}\PYG{+w}{ }\PYG{c+c1}{// State of the next 32 LEDs}
\PYG{+w}{  }\PYG{l+m+mh}{0xa0040000}\PYG{+w}{  }\PYG{c+c1}{// State of the last 32 LEDs}
\PYG{p}{\PYGZcb{}}\PYG{p}{;}
\end{sphinxVerbatim}

\sphinxAtStartPar
To better visualize the LED statuses, these values can be converted to binary form, where each bit sequentially represents each LED state from left to right and top to bottom. A 0 indicates off, and a 1 indicates on.

\begin{sphinxVerbatim}[commandchars=\\\{\}]
\PYG{l+m+mh}{0x3184a444}\PYG{+w}{ }\PYG{o}{\PYGZhy{}}\PYG{o}{\PYGZgt{}}\PYG{+w}{ }\PYG{l+m+mi}{110001100001001010010001000100}
\PYG{l+m+mh}{0x42081100}\PYG{+w}{ }\PYG{o}{\PYGZhy{}}\PYG{o}{\PYGZgt{}}\PYG{+w}{ }\PYG{l+m+mi}{1000010000010000001000100000000}
\PYG{l+m+mh}{0xa0040000}\PYG{+w}{ }\PYG{o}{\PYGZhy{}}\PYG{o}{\PYGZgt{}}\PYG{+w}{ }\PYG{l+m+mi}{10100000000001000000000000000000}
\end{sphinxVerbatim}


\subsection{Display pattern on LED matrix}
\label{\detokenize{Extension_Project/12x8_LED_Matrix:display-pattern-on-led-matrix}}
\sphinxAtStartPar
Once your pattern is ready, the next step is to transmit this data to the 12x8 LED Matrix. This usually involves invoking library functions and passing the array or variables containing the LED states to these functions.
\begin{enumerate}
\sphinxsetlistlabels{\arabic}{enumi}{enumii}{}{.}%
\item {} 
\sphinxAtStartPar
Using a two\sphinxhyphen{}dimensional Array

\sphinxAtStartPar
To display the pattern stored in a 2D array, you can use the following code:

\begin{sphinxVerbatim}[commandchars=\\\{\}]
\PYG{c+cp}{\PYGZsh{}}\PYG{c+cp}{include}\PYG{+w}{ }\PYG{c+cpf}{\PYGZlt{}Arduino\PYGZus{}LED\PYGZus{}Matrix.h\PYGZgt{}}

\PYG{n}{ArduinoLEDMatrix}\PYG{+w}{ }\PYG{n}{matrix}\PYG{p}{;}

\PYG{c+c1}{// Pre\PYGZhy{}defined 2D array}
\PYG{k+kr}{byte}\PYG{+w}{ }\PYG{n}{frame}\PYG{p}{[}\PYG{l+m+mi}{8}\PYG{p}{]}\PYG{p}{[}\PYG{l+m+mi}{12}\PYG{p}{]}\PYG{+w}{ }\PYG{o}{=}\PYG{+w}{ }\PYG{p}{\PYGZob{}}
\PYG{+w}{     }\PYG{p}{\PYGZob{}}\PYG{+w}{ }\PYG{l+m+mi}{0}\PYG{p}{,}\PYG{+w}{ }\PYG{l+m+mi}{0}\PYG{p}{,}\PYG{+w}{ }\PYG{l+m+mi}{1}\PYG{p}{,}\PYG{+w}{ }\PYG{l+m+mi}{1}\PYG{p}{,}\PYG{+w}{ }\PYG{l+m+mi}{0}\PYG{p}{,}\PYG{+w}{ }\PYG{l+m+mi}{0}\PYG{p}{,}\PYG{+w}{ }\PYG{l+m+mi}{0}\PYG{p}{,}\PYG{+w}{ }\PYG{l+m+mi}{1}\PYG{p}{,}\PYG{+w}{ }\PYG{l+m+mi}{1}\PYG{p}{,}\PYG{+w}{ }\PYG{l+m+mi}{0}\PYG{p}{,}\PYG{+w}{ }\PYG{l+m+mi}{0}\PYG{p}{,}\PYG{+w}{ }\PYG{l+m+mi}{0}\PYG{+w}{ }\PYG{p}{\PYGZcb{}}\PYG{p}{,}
\PYG{+w}{     }\PYG{p}{\PYGZob{}}\PYG{+w}{ }\PYG{l+m+mi}{0}\PYG{p}{,}\PYG{+w}{ }\PYG{l+m+mi}{1}\PYG{p}{,}\PYG{+w}{ }\PYG{l+m+mi}{0}\PYG{p}{,}\PYG{+w}{ }\PYG{l+m+mi}{0}\PYG{p}{,}\PYG{+w}{ }\PYG{l+m+mi}{1}\PYG{p}{,}\PYG{+w}{ }\PYG{l+m+mi}{0}\PYG{p}{,}\PYG{+w}{ }\PYG{l+m+mi}{1}\PYG{p}{,}\PYG{+w}{ }\PYG{l+m+mi}{0}\PYG{p}{,}\PYG{+w}{ }\PYG{l+m+mi}{0}\PYG{p}{,}\PYG{+w}{ }\PYG{l+m+mi}{1}\PYG{p}{,}\PYG{+w}{ }\PYG{l+m+mi}{0}\PYG{p}{,}\PYG{+w}{ }\PYG{l+m+mi}{0}\PYG{+w}{ }\PYG{p}{\PYGZcb{}}\PYG{p}{,}
\PYG{+w}{     }\PYG{p}{\PYGZob{}}\PYG{+w}{ }\PYG{l+m+mi}{0}\PYG{p}{,}\PYG{+w}{ }\PYG{l+m+mi}{1}\PYG{p}{,}\PYG{+w}{ }\PYG{l+m+mi}{0}\PYG{p}{,}\PYG{+w}{ }\PYG{l+m+mi}{0}\PYG{p}{,}\PYG{+w}{ }\PYG{l+m+mi}{0}\PYG{p}{,}\PYG{+w}{ }\PYG{l+m+mi}{1}\PYG{p}{,}\PYG{+w}{ }\PYG{l+m+mi}{0}\PYG{p}{,}\PYG{+w}{ }\PYG{l+m+mi}{0}\PYG{p}{,}\PYG{+w}{ }\PYG{l+m+mi}{0}\PYG{p}{,}\PYG{+w}{ }\PYG{l+m+mi}{1}\PYG{p}{,}\PYG{+w}{ }\PYG{l+m+mi}{0}\PYG{p}{,}\PYG{+w}{ }\PYG{l+m+mi}{0}\PYG{+w}{ }\PYG{p}{\PYGZcb{}}\PYG{p}{,}
\PYG{+w}{     }\PYG{p}{\PYGZob{}}\PYG{+w}{ }\PYG{l+m+mi}{0}\PYG{p}{,}\PYG{+w}{ }\PYG{l+m+mi}{0}\PYG{p}{,}\PYG{+w}{ }\PYG{l+m+mi}{1}\PYG{p}{,}\PYG{+w}{ }\PYG{l+m+mi}{0}\PYG{p}{,}\PYG{+w}{ }\PYG{l+m+mi}{0}\PYG{p}{,}\PYG{+w}{ }\PYG{l+m+mi}{0}\PYG{p}{,}\PYG{+w}{ }\PYG{l+m+mi}{0}\PYG{p}{,}\PYG{+w}{ }\PYG{l+m+mi}{0}\PYG{p}{,}\PYG{+w}{ }\PYG{l+m+mi}{1}\PYG{p}{,}\PYG{+w}{ }\PYG{l+m+mi}{0}\PYG{p}{,}\PYG{+w}{ }\PYG{l+m+mi}{0}\PYG{p}{,}\PYG{+w}{ }\PYG{l+m+mi}{0}\PYG{+w}{ }\PYG{p}{\PYGZcb{}}\PYG{p}{,}
\PYG{+w}{     }\PYG{p}{\PYGZob{}}\PYG{+w}{ }\PYG{l+m+mi}{0}\PYG{p}{,}\PYG{+w}{ }\PYG{l+m+mi}{0}\PYG{p}{,}\PYG{+w}{ }\PYG{l+m+mi}{0}\PYG{p}{,}\PYG{+w}{ }\PYG{l+m+mi}{1}\PYG{p}{,}\PYG{+w}{ }\PYG{l+m+mi}{0}\PYG{p}{,}\PYG{+w}{ }\PYG{l+m+mi}{0}\PYG{p}{,}\PYG{+w}{ }\PYG{l+m+mi}{0}\PYG{p}{,}\PYG{+w}{ }\PYG{l+m+mi}{1}\PYG{p}{,}\PYG{+w}{ }\PYG{l+m+mi}{0}\PYG{p}{,}\PYG{+w}{ }\PYG{l+m+mi}{0}\PYG{p}{,}\PYG{+w}{ }\PYG{l+m+mi}{0}\PYG{p}{,}\PYG{+w}{ }\PYG{l+m+mi}{0}\PYG{+w}{ }\PYG{p}{\PYGZcb{}}\PYG{p}{,}
\PYG{+w}{     }\PYG{p}{\PYGZob{}}\PYG{+w}{ }\PYG{l+m+mi}{0}\PYG{p}{,}\PYG{+w}{ }\PYG{l+m+mi}{0}\PYG{p}{,}\PYG{+w}{ }\PYG{l+m+mi}{0}\PYG{p}{,}\PYG{+w}{ }\PYG{l+m+mi}{0}\PYG{p}{,}\PYG{+w}{ }\PYG{l+m+mi}{1}\PYG{p}{,}\PYG{+w}{ }\PYG{l+m+mi}{0}\PYG{p}{,}\PYG{+w}{ }\PYG{l+m+mi}{1}\PYG{p}{,}\PYG{+w}{ }\PYG{l+m+mi}{0}\PYG{p}{,}\PYG{+w}{ }\PYG{l+m+mi}{0}\PYG{p}{,}\PYG{+w}{ }\PYG{l+m+mi}{0}\PYG{p}{,}\PYG{+w}{ }\PYG{l+m+mi}{0}\PYG{p}{,}\PYG{+w}{ }\PYG{l+m+mi}{0}\PYG{+w}{ }\PYG{p}{\PYGZcb{}}\PYG{p}{,}
\PYG{+w}{     }\PYG{p}{\PYGZob{}}\PYG{+w}{ }\PYG{l+m+mi}{0}\PYG{p}{,}\PYG{+w}{ }\PYG{l+m+mi}{0}\PYG{p}{,}\PYG{+w}{ }\PYG{l+m+mi}{0}\PYG{p}{,}\PYG{+w}{ }\PYG{l+m+mi}{0}\PYG{p}{,}\PYG{+w}{ }\PYG{l+m+mi}{0}\PYG{p}{,}\PYG{+w}{ }\PYG{l+m+mi}{1}\PYG{p}{,}\PYG{+w}{ }\PYG{l+m+mi}{0}\PYG{p}{,}\PYG{+w}{ }\PYG{l+m+mi}{0}\PYG{p}{,}\PYG{+w}{ }\PYG{l+m+mi}{0}\PYG{p}{,}\PYG{+w}{ }\PYG{l+m+mi}{0}\PYG{p}{,}\PYG{+w}{ }\PYG{l+m+mi}{0}\PYG{p}{,}\PYG{+w}{ }\PYG{l+m+mi}{0}\PYG{+w}{ }\PYG{p}{\PYGZcb{}}\PYG{p}{,}
\PYG{+w}{     }\PYG{p}{\PYGZob{}}\PYG{+w}{ }\PYG{l+m+mi}{0}\PYG{p}{,}\PYG{+w}{ }\PYG{l+m+mi}{0}\PYG{p}{,}\PYG{+w}{ }\PYG{l+m+mi}{0}\PYG{p}{,}\PYG{+w}{ }\PYG{l+m+mi}{0}\PYG{p}{,}\PYG{+w}{ }\PYG{l+m+mi}{0}\PYG{p}{,}\PYG{+w}{ }\PYG{l+m+mi}{0}\PYG{p}{,}\PYG{+w}{ }\PYG{l+m+mi}{0}\PYG{p}{,}\PYG{+w}{ }\PYG{l+m+mi}{0}\PYG{p}{,}\PYG{+w}{ }\PYG{l+m+mi}{0}\PYG{p}{,}\PYG{+w}{ }\PYG{l+m+mi}{0}\PYG{p}{,}\PYG{+w}{ }\PYG{l+m+mi}{0}\PYG{p}{,}\PYG{+w}{ }\PYG{l+m+mi}{0}\PYG{+w}{ }\PYG{p}{\PYGZcb{}}
\PYG{p}{\PYGZcb{}}\PYG{p}{;}

\PYG{k+kr}{void}\PYG{+w}{ }\PYG{n+nb}{setup}\PYG{p}{(}\PYG{p}{)}\PYG{+w}{ }\PYG{p}{\PYGZob{}}
\PYG{+w}{  }\PYG{c+c1}{// Initialize LED matrix}
\PYG{+w}{  }\PYG{n}{matrix}\PYG{p}{.}\PYG{n+nf}{begin}\PYG{p}{(}\PYG{p}{)}\PYG{p}{;}
\PYG{p}{\PYGZcb{}}

\PYG{k+kr}{void}\PYG{+w}{ }\PYG{n+nb}{loop}\PYG{p}{(}\PYG{p}{)}\PYG{+w}{ }\PYG{p}{\PYGZob{}}
\PYG{+w}{  }\PYG{c+c1}{// Display pattern on the LED matrix}
\PYG{+w}{  }\PYG{n}{matrix}\PYG{p}{.}\PYG{n}{renderBitmap}\PYG{p}{(}\PYG{n}{frame}\PYG{p}{,}\PYG{+w}{ }\PYG{l+m+mi}{8}\PYG{p}{,}\PYG{+w}{ }\PYG{l+m+mi}{12}\PYG{p}{)}\PYG{p}{;}
\PYG{+w}{  }\PYG{n+nf}{delay}\PYG{p}{(}\PYG{l+m+mi}{1000}\PYG{p}{)}\PYG{p}{;}
\PYG{p}{\PYGZcb{}}
\end{sphinxVerbatim}

\sphinxAtStartPar
In this code, we use the \sphinxcode{\sphinxupquote{matrix.renderBitmap(frame, 8, 12);}} function to display the LED matrix. Here, 8 and 12 respectively represent the rows and columns of the LED matrix.

\item {} 
\sphinxAtStartPar
Using an Array of 32\sphinxhyphen{}bit integers

\sphinxAtStartPar
To display the pattern stored in an array of \sphinxcode{\sphinxupquote{unsigned long}}, use the following code:

\begin{sphinxVerbatim}[commandchars=\\\{\}]
\PYG{c+cp}{\PYGZsh{}}\PYG{c+cp}{include}\PYG{+w}{ }\PYG{c+cpf}{\PYGZdq{}Arduino\PYGZus{}LED\PYGZus{}Matrix.h\PYGZdq{}}

\PYG{n}{ArduinoLEDMatrix}\PYG{+w}{ }\PYG{n}{matrix}\PYG{p}{;}

\PYG{k+kr}{void}\PYG{+w}{ }\PYG{n+nb}{setup}\PYG{p}{(}\PYG{p}{)}\PYG{+w}{ }\PYG{p}{\PYGZob{}}
\PYG{+w}{  }\PYG{n}{matrix}\PYG{p}{.}\PYG{n+nf}{begin}\PYG{p}{(}\PYG{p}{)}\PYG{p}{;}
\PYG{p}{\PYGZcb{}}

\PYG{k+kr}{const}\PYG{+w}{ }\PYG{k+kr}{uint32\PYGZus{}t}\PYG{+w}{ }\PYG{n}{heart}\PYG{p}{[}\PYG{p}{]}\PYG{+w}{ }\PYG{o}{=}\PYG{+w}{ }\PYG{p}{\PYGZob{}}
\PYG{+w}{    }\PYG{l+m+mh}{0x3184a444}\PYG{p}{,}
\PYG{+w}{    }\PYG{l+m+mh}{0x44042081}\PYG{p}{,}
\PYG{+w}{    }\PYG{l+m+mh}{0x100a0040}
\PYG{p}{\PYGZcb{}}\PYG{p}{;}

\PYG{k+kr}{void}\PYG{+w}{ }\PYG{n+nb}{loop}\PYG{p}{(}\PYG{p}{)}\PYG{p}{\PYGZob{}}
\PYG{+w}{  }\PYG{n}{matrix}\PYG{p}{.}\PYG{n}{loadFrame}\PYG{p}{(}\PYG{n}{heart}\PYG{p}{)}\PYG{p}{;}
\PYG{+w}{  }\PYG{n+nf}{delay}\PYG{p}{(}\PYG{l+m+mi}{500}\PYG{p}{)}\PYG{p}{;}
\PYG{p}{\PYGZcb{}}
\end{sphinxVerbatim}

\sphinxAtStartPar
In this case, we need to use the \sphinxcode{\sphinxupquote{matrix.loadFrame(heart)}} function to display the pattern on the LED matrix.

\end{enumerate}


\subsection{Arduino LED Matrix Editor}
\label{\detokenize{Extension_Project/12x8_LED_Matrix:arduino-led-matrix-editor}}
\sphinxAtStartPar
I recommend using an \sphinxcode{\sphinxupquote{unsigned long}} array to store the state of the LED matrix, as it saves memory on the Arduino. Though this method might not be very intuitive, you can use the \sphinxhref{https://ledmatrix-editor.arduino.cc/}{Arduino LED Matrix Editor} as an aid, which helps you generate an \sphinxcode{\sphinxupquote{unsigned long}} array.

\sphinxAtStartPar
With the \sphinxhref{https://ledmatrix-editor.arduino.cc/}{Arduino LED Matrix Editor} and the \sphinxcode{\sphinxupquote{Arduino\_LED\_Matrix}} library, you can conveniently create icons or animations and display them on the UNO R4 WiFi board. All you have to do is draw, download the \sphinxcode{\sphinxupquote{.h}} file, and call the \sphinxcode{\sphinxupquote{matrix.play()}} function in your sketch to easily build your next project.
\begin{enumerate}
\sphinxsetlistlabels{\arabic}{enumi}{enumii}{}{.}%
\item {} 
\sphinxAtStartPar
Open the LED Matrix Editor

\noindent\sphinxincludegraphics[width=0.800\linewidth]{{LED_Matrix_editor}.png}

\item {} 
\sphinxAtStartPar
Draw your pattern on the center canvas

\noindent\sphinxincludegraphics[width=0.800\linewidth]{{LED_Matrix_editor1}.png}

\item {} 
\sphinxAtStartPar
Set the frame interval in milliseconds

\noindent\sphinxincludegraphics[width=0.800\linewidth]{{LED_Matrix_editor2}.png}

\item {} 
\sphinxAtStartPar
You can create a new blank frame or copy and create a new frame from the current frame.

\noindent\sphinxincludegraphics[width=0.800\linewidth]{{LED_Matrix_editor3}.png}

\item {} 
\sphinxAtStartPar
Export the \sphinxcode{\sphinxupquote{.h}} header file

\noindent\sphinxincludegraphics[width=0.800\linewidth]{{LED_Matrix_editor4}.png}

\noindent\sphinxincludegraphics{{LED_Matrix_editor5}.png}

\sphinxAtStartPar
After clicking OK, you’ll receive a file named \sphinxcode{\sphinxupquote{animation.h}}.

\end{enumerate}


\subsection{Display Animations}
\label{\detokenize{Extension_Project/12x8_LED_Matrix:display-animations}}
\sphinxAtStartPar
In the previous steps, we obtained a \sphinxcode{\sphinxupquote{.h}} file that stores a series of frames along with their durations. Next, let’s display them on the LED matrix.
\begin{enumerate}
\sphinxsetlistlabels{\arabic}{enumi}{enumii}{}{.}%
\item {} 
\sphinxAtStartPar
First, create a sketch. You can either open the \sphinxcode{\sphinxupquote{15\_LED\_Matrix.ino}} file located under the path \sphinxcode{\sphinxupquote{Basic\sphinxhyphen{}Starter\sphinxhyphen{}Kit\sphinxhyphen{}for\sphinxhyphen{}Arduino\sphinxhyphen{}Uno\sphinxhyphen{}R4\sphinxhyphen{}WiFi\sphinxhyphen{}main\textbackslash{}Code}}.

\item {} 
\sphinxAtStartPar
If you are using code from the \sphinxcode{\sphinxupquote{Basic\sphinxhyphen{}Starter\sphinxhyphen{}Kit\sphinxhyphen{}for\sphinxhyphen{}Arduino\sphinxhyphen{}Uno\sphinxhyphen{}R4\sphinxhyphen{}WiFi\sphinxhyphen{}main\textbackslash{}Code}} path, you’ll find a tab named \sphinxcode{\sphinxupquote{animation.h}} in the Arduino IDE. Open it and replace the existing code with the .h file you obtained from the website.

\noindent\sphinxincludegraphics[width=0.800\linewidth]{{LED_Matrix_code}.png}

\item {} 
\sphinxAtStartPar
If you have created your own sketch, you need to copy the \sphinxcode{\sphinxupquote{.h}} file obtained from the webpage to the same directory of the sketch.

\item {} 
\sphinxAtStartPar
After setting up your preferred code in the Arduino IDE and uploading it to your Arduino UNO R4 WiFi, your LED matrix should now display the pattern you defined.

\sphinxAtStartPar
Congratulations! You’ve successfully programmed your Arduino UNO R4 WiFi’s 12x8 LED Matrix!

\end{enumerate}

\sphinxAtStartPar
\sphinxstylestrong{Reference}

\sphinxAtStartPar
\sphinxhref{https://docs.arduino.cc/tutorials/uno-r4-wifi/led-matrix/}{Using the Arduino UNO R4 WiFi LED Matrix}

\sphinxAtStartPar
\sphinxstylestrong{More Projects}
\begin{itemize}
\item {} 
\sphinxAtStartPar
\DUrole{xref}{\DUrole{std}{\DUrole{std-ref}{fun\_snake}}} (Fun Project)

\end{itemize}

\sphinxstepscope


\section{WiFi Connect Test}
\label{\detokenize{Extension_Project/WiFi_Connect_Test:wifi-connect-test}}\label{\detokenize{Extension_Project/WiFi_Connect_Test:ext-wifi-connect-test}}\label{\detokenize{Extension_Project/WiFi_Connect_Test::doc}}
\sphinxAtStartPar
This tutorial will guide you through the essential steps to connect your Arduino board to a Wi\sphinxhyphen{}Fi network. You’ll learn how to initialize the Wi\sphinxhyphen{}Fi module, verify its firmware, and securely join a network using its SSID and password. Once connected, you’ll discover how to monitor important network details like your device’s IP and MAC addresses, as well as the network’s signal strength, directly from the serial console. This tutorial serves as both a practical guide to Wi\sphinxhyphen{}Fi connectivity and an introduction to network monitoring with Arduino, helping you establish and maintain a reliable Wi\sphinxhyphen{}Fi connection.


\subsection{1. Upload the code}
\label{\detokenize{Extension_Project/WiFi_Connect_Test:upload-the-code}}
\sphinxAtStartPar
Open the \sphinxcode{\sphinxupquote{16\_WiFi\_Connect\_Test.ino}} file under the path of \sphinxcode{\sphinxupquote{Basic\sphinxhyphen{}Starter\sphinxhyphen{}Kit\sphinxhyphen{}for\sphinxhyphen{}Arduino\sphinxhyphen{}Uno\sphinxhyphen{}R4\sphinxhyphen{}WiFi\sphinxhyphen{}main\textbackslash{}Code}}, or copy this code into \sphinxstylestrong{Arduino IDE}.

\begin{sphinxadmonition}{note}{Note:}
\sphinxAtStartPar
Wi\sphinxhyphen{}Fi® support is enabled via the built\sphinxhyphen{}in \sphinxcode{\sphinxupquote{WiFiS3}} library that is shipped with the Arduino UNO R4 Core. Installing the core automatically installs the \sphinxcode{\sphinxupquote{WiFiS3}} library.
\end{sphinxadmonition}

\sphinxAtStartPar
You still need to create or modify \sphinxcode{\sphinxupquote{arduino\_secrets.h}}, replace \sphinxcode{\sphinxupquote{SECRET\_SSID}} and \sphinxcode{\sphinxupquote{SECRET\_PASS}} with the name and password of the wifi you want to connect to. The file should contain:

\begin{sphinxVerbatim}[commandchars=\\\{\}]
\PYG{c+c1}{//arduino\PYGZus{}secrets.h header file}
\PYG{c+cp}{\PYGZsh{}}\PYG{c+cp}{define SECRET\PYGZus{}SSID \PYGZdq{}yournetwork\PYGZdq{}}
\PYG{c+cp}{\PYGZsh{}}\PYG{c+cp}{define SECRET\PYGZus{}PASS \PYGZdq{}yourpassword\PYGZdq{}}
\end{sphinxVerbatim}

\sphinxAtStartPar
Open the serial monitor, and you will see similar content as follows. Arduino will output your device’s IP and MAC addresses, as well as the network’s signal strength.

\noindent\sphinxincludegraphics[width=1.000\linewidth]{{WiFi_Connect_Test}.png}


\subsection{2. Code explanation}
\label{\detokenize{Extension_Project/WiFi_Connect_Test:code-explanation}}\begin{enumerate}
\sphinxsetlistlabels{\arabic}{enumi}{enumii}{}{.}%
\item {} 
\sphinxAtStartPar
Including Libraries and Secret Data

\begin{sphinxVerbatim}[commandchars=\\\{\}]
\PYG{c+cp}{\PYGZsh{}}\PYG{c+cp}{include}\PYG{+w}{ }\PYG{c+cpf}{\PYGZlt{}WiFiS3.h\PYGZgt{}}
\PYG{c+cp}{\PYGZsh{}}\PYG{c+cp}{include}\PYG{+w}{ }\PYG{c+cpf}{\PYGZdq{}arduino\PYGZus{}secrets.h\PYGZdq{}}
\end{sphinxVerbatim}
\begin{itemize}
\item {} 
\sphinxAtStartPar
\sphinxcode{\sphinxupquote{WiFiS3}} is a library that provides functions for Wi\sphinxhyphen{}Fi connectivity. Installing the R4 core automatically installs the WiFiS3 library.

\item {} 
\sphinxAtStartPar
\sphinxcode{\sphinxupquote{arduino\_secrets.h}} is a separate file where you keep your SSID and password so they’re not exposed in your main code. Storing network and password separately reduces accidental sharing of Wi\sphinxhyphen{}Fi credentials.

\end{itemize}



\item {} 
\sphinxAtStartPar
Declaring Global Variables

\begin{sphinxVerbatim}[commandchars=\\\{\}]
\PYG{k+kr}{char}\PYG{+w}{ }\PYG{n}{ssid}\PYG{p}{[}\PYG{p}{]}\PYG{+w}{ }\PYG{o}{=}\PYG{+w}{ }\PYG{n}{SECRET\PYGZus{}SSID}\PYG{p}{;}
\PYG{k+kr}{char}\PYG{+w}{ }\PYG{n}{pass}\PYG{p}{[}\PYG{p}{]}\PYG{+w}{ }\PYG{o}{=}\PYG{+w}{ }\PYG{n}{SECRET\PYGZus{}PASS}\PYG{p}{;}
\PYG{k+kr}{int}\PYG{+w}{ }\PYG{n}{status}\PYG{+w}{ }\PYG{o}{=}\PYG{+w}{ }\PYG{n}{WL\PYGZus{}IDLE\PYGZus{}STATUS}\PYG{p}{;}
\end{sphinxVerbatim}
\begin{itemize}
\item {} 
\sphinxAtStartPar
\sphinxcode{\sphinxupquote{ssid}} and \sphinxcode{\sphinxupquote{pass}} contain your network name and password.

\item {} 
\sphinxAtStartPar
\sphinxcode{\sphinxupquote{status}} will store the current status of your Wi\sphinxhyphen{}Fi connection.

\end{itemize}



\item {} 
\sphinxAtStartPar
\sphinxcode{\sphinxupquote{setup()}} Function

\sphinxAtStartPar
The Serial interface is initialized with a baud rate of 9600. The \sphinxcode{\sphinxupquote{while (!Serial);}} line makes sure that the program waits until the Serial connection is established.

\begin{sphinxVerbatim}[commandchars=\\\{\}]
\PYG{k+kr}{void}\PYG{+w}{ }\PYG{n+nb}{setup}\PYG{p}{(}\PYG{p}{)}\PYG{+w}{ }\PYG{p}{\PYGZob{}}
\PYG{+w}{    }\PYG{c+c1}{//Initialize serial and wait for port to open:}
\PYG{+w}{    }\PYG{n+nf}{Serial}\PYG{p}{.}\PYG{n+nf}{begin}\PYG{p}{(}\PYG{l+m+mi}{9600}\PYG{p}{)}\PYG{p}{;}
\PYG{+w}{    }\PYG{k}{while}\PYG{+w}{ }\PYG{p}{(}\PYG{o}{!}\PYG{n+nf}{Serial}\PYG{p}{)}\PYG{+w}{ }\PYG{p}{\PYGZob{}}
\PYG{+w}{      }\PYG{p}{;}\PYG{+w}{ }\PYG{c+c1}{// wait for serial port to connect. Needed for native USB port only}
\PYG{+w}{    }\PYG{p}{\PYGZcb{}}
\PYG{+w}{    }\PYG{p}{.}\PYG{p}{.}\PYG{p}{.}
\PYG{p}{\PYGZcb{}}
\end{sphinxVerbatim}

\sphinxAtStartPar
And then, the code checks whether the Wi\sphinxhyphen{}Fi module is available or not. If not, the program will halt, effectively stopping any further execution.

\begin{sphinxVerbatim}[commandchars=\\\{\}]
\PYG{p}{.}\PYG{p}{.}\PYG{p}{.}
\PYG{c+c1}{// check for the WiFi module:}
\PYG{k}{if}\PYG{+w}{ }\PYG{p}{(}\PYG{n+nf}{WiFi}\PYG{p}{.}\PYG{n}{status}\PYG{p}{(}\PYG{p}{)}\PYG{+w}{ }\PYG{o}{=}\PYG{o}{=}\PYG{+w}{ }\PYG{n}{WL\PYGZus{}NO\PYGZus{}MODULE}\PYG{p}{)}\PYG{+w}{ }\PYG{p}{\PYGZob{}}
\PYG{+w}{    }\PYG{n+nf}{Serial}\PYG{p}{.}\PYG{n+nf}{println}\PYG{p}{(}\PYG{l+s}{\PYGZdq{}}\PYG{l+s}{Communication with WiFi module failed!}\PYG{l+s}{\PYGZdq{}}\PYG{p}{)}\PYG{p}{;}
\PYG{+w}{    }\PYG{c+c1}{// don\PYGZsq{}t continue}
\PYG{+w}{    }\PYG{k}{while}\PYG{+w}{ }\PYG{p}{(}\PYG{k+kr}{true}\PYG{p}{)}\PYG{p}{;}
\PYG{p}{\PYGZcb{}}
\PYG{p}{.}\PYG{p}{.}\PYG{p}{.}
\end{sphinxVerbatim}

\sphinxAtStartPar
In this part of the code, we check if the firmware version of uno R4 wifi is up to date. If it is not the latest version, a prompt for upgrade will be displayed. You can refer to \DUrole{xref}{\DUrole{std}{\DUrole{std-ref}{update\_firmware}}} for firmware upgrade.

\begin{sphinxVerbatim}[commandchars=\\\{\}]
\PYG{p}{.}\PYG{p}{.}\PYG{p}{.}
\PYG{k+kr}{String}\PYG{+w}{ }\PYG{n}{fv}\PYG{+w}{ }\PYG{o}{=}\PYG{+w}{ }\PYG{n+nf}{WiFi}\PYG{p}{.}\PYG{n}{firmwareVersion}\PYG{p}{(}\PYG{p}{)}\PYG{p}{;}
\PYG{k}{if}\PYG{+w}{ }\PYG{p}{(}\PYG{n}{fv}\PYG{+w}{ }\PYG{o}{\PYGZlt{}}\PYG{+w}{ }\PYG{n}{WIFI\PYGZus{}FIRMWARE\PYGZus{}LATEST\PYGZus{}VERSION}\PYG{p}{)}\PYG{+w}{ }\PYG{p}{\PYGZob{}}
\PYG{+w}{    }\PYG{n+nf}{Serial}\PYG{p}{.}\PYG{n+nf}{println}\PYG{p}{(}\PYG{l+s}{\PYGZdq{}}\PYG{l+s}{Please upgrade the firmware}\PYG{l+s}{\PYGZdq{}}\PYG{p}{)}\PYG{p}{;}
\PYG{p}{\PYGZcb{}}
\PYG{p}{.}\PYG{p}{.}\PYG{p}{.}
\end{sphinxVerbatim}

\item {} 
\sphinxAtStartPar
\sphinxcode{\sphinxupquote{loop()}} Function

\begin{sphinxVerbatim}[commandchars=\\\{\}]
\PYG{k+kr}{void}\PYG{+w}{ }\PYG{n+nb}{loop}\PYG{p}{(}\PYG{p}{)}\PYG{+w}{ }\PYG{p}{\PYGZob{}}
\PYG{+w}{  }\PYG{c+c1}{// check the network connection once every 10 seconds:}
\PYG{+w}{  }\PYG{n+nf}{delay}\PYG{p}{(}\PYG{l+m+mi}{10000}\PYG{p}{)}\PYG{p}{;}
\PYG{+w}{  }\PYG{n}{printCurrentNet}\PYG{p}{(}\PYG{p}{)}\PYG{p}{;}
\PYG{p}{\PYGZcb{}}
\end{sphinxVerbatim}
\begin{itemize}
\item {} 
\sphinxAtStartPar
Every 10 seconds, the function \sphinxcode{\sphinxupquote{printCurrentNet()}} is called to print the current network details.

\end{itemize}

\end{enumerate}

\sphinxAtStartPar
\sphinxstylestrong{Reference}

\sphinxAtStartPar
\sphinxhref{https://docs.arduino.cc/tutorials/uno-r4-wifi/wifi-examples}{UNO R4 WiFi Network Examples}

\sphinxstepscope


\section{Web Control LED (WiFi Access Point)}
\label{\detokenize{Extension_Project/Web_Control_LED:web-control-led-wifi-access-point}}\label{\detokenize{Extension_Project/Web_Control_LED:ext-web-control-led}}\label{\detokenize{Extension_Project/Web_Control_LED::doc}}
\sphinxAtStartPar
This project allows you to control an LED light through a web interface. The Arduino board acts as a WiFi access point, creating its own local network that you can connect to with a web browser. Once connected, you can navigate to the device’s IP address using the web browser, where you’ll find options to turn an LED (connected to the board’s pin 13) on and off. The project provides real\sphinxhyphen{}time feedback on the LED status via the Serial Monitor, making it easier to debug and understand the flow of operations.


\subsection{1. Upload the code}
\label{\detokenize{Extension_Project/Web_Control_LED:upload-the-code}}
\sphinxAtStartPar
Open the \sphinxcode{\sphinxupquote{17\_Web\_Control\_LED}} file under the path of \sphinxcode{\sphinxupquote{Basic\sphinxhyphen{}Starter\sphinxhyphen{}Kit\sphinxhyphen{}for\sphinxhyphen{}Arduino\sphinxhyphen{}Uno\sphinxhyphen{}R4\sphinxhyphen{}WiFi\sphinxhyphen{}main\textbackslash{}Code}}, or copy this code into \sphinxstylestrong{Arduino IDE}.

\begin{sphinxadmonition}{note}{Note:}
\sphinxAtStartPar
Wi\sphinxhyphen{}Fi® support is enabled via the built\sphinxhyphen{}in \sphinxcode{\sphinxupquote{WiFiS3}} library that is shipped with the Arduino UNO R4 Core. Installing the core automatically installs the \sphinxcode{\sphinxupquote{WiFiS3}} library.
\end{sphinxadmonition}

\sphinxAtStartPar
You still need to create or modify \sphinxcode{\sphinxupquote{arduino\_secrets.h}}, replace \sphinxcode{\sphinxupquote{SECRET\_SSID}} and \sphinxcode{\sphinxupquote{SECRET\_PASS}} with the name and password of your WiFi access point. The file should contain:

\begin{sphinxVerbatim}[commandchars=\\\{\}]
\PYG{c+c1}{//arduino\PYGZus{}secrets.h header file}
\PYG{c+cp}{\PYGZsh{}}\PYG{c+cp}{define SECRET\PYGZus{}SSID \PYGZdq{}yournetwork\PYGZdq{}}
\PYG{c+cp}{\PYGZsh{}}\PYG{c+cp}{define SECRET\PYGZus{}PASS \PYGZdq{}yourpassword\PYGZdq{}}
\end{sphinxVerbatim}

\sphinxAtStartPar
Note: This is the SSID and Password you created for UNO R4 WIFI. It is not the SSID and Password of your home router’s WiFi access point.

\noindent{\hspace*{\fill}\sphinxincludegraphics[width=0.800\linewidth]{{Web_Control_LED}.png}\hspace*{\fill}}


\subsection{2. Code explanation}
\label{\detokenize{Extension_Project/Web_Control_LED:code-explanation}}\begin{enumerate}
\sphinxsetlistlabels{\arabic}{enumi}{enumii}{}{.}%
\item {} 
\sphinxAtStartPar
Importing Required Libraries

\sphinxAtStartPar
Importing the \sphinxcode{\sphinxupquote{WiFiS3}} library for WiFi functionalities and \sphinxcode{\sphinxupquote{arduino\_secrets.h}} for sensitive data like passwords.

\begin{sphinxVerbatim}[commandchars=\\\{\}]
\PYG{c+cp}{\PYGZsh{}}\PYG{c+cp}{include}\PYG{+w}{ }\PYG{c+cpf}{\PYGZdq{}WiFiS3.h\PYGZdq{}}
\PYG{c+cp}{\PYGZsh{}}\PYG{c+cp}{include}\PYG{+w}{ }\PYG{c+cpf}{\PYGZdq{}arduino\PYGZus{}secrets.h\PYGZdq{}}
\end{sphinxVerbatim}

\item {} 
\sphinxAtStartPar
Configuration and Variable Initialization

\sphinxAtStartPar
Define WiFi SSID, password, and key index along with the LED pin and WiFi status.

\begin{sphinxVerbatim}[commandchars=\\\{\}]
\PYG{k+kr}{char}\PYG{+w}{ }\PYG{n}{ssid}\PYG{p}{[}\PYG{p}{]}\PYG{+w}{ }\PYG{o}{=}\PYG{+w}{ }\PYG{n}{SECRET\PYGZus{}SSID}\PYG{p}{;}
\PYG{k+kr}{char}\PYG{+w}{ }\PYG{n}{pass}\PYG{p}{[}\PYG{p}{]}\PYG{+w}{ }\PYG{o}{=}\PYG{+w}{ }\PYG{n}{SECRET\PYGZus{}PASS}\PYG{p}{;}
\PYG{k+kr}{int}\PYG{+w}{ }\PYG{n}{keyIndex}\PYG{+w}{ }\PYG{o}{=}\PYG{+w}{ }\PYG{l+m+mi}{0}\PYG{p}{;}
\PYG{k+kr}{int}\PYG{+w}{ }\PYG{n}{led}\PYG{+w}{ }\PYG{o}{=}\PYG{+w}{  }\PYG{k+kr}{LED\PYGZus{}BUILTIN}\PYG{p}{;}
\PYG{k+kr}{int}\PYG{+w}{ }\PYG{n}{status}\PYG{+w}{ }\PYG{o}{=}\PYG{+w}{ }\PYG{n}{WL\PYGZus{}IDLE\PYGZus{}STATUS}\PYG{p}{;}
\PYG{n+nf}{WiFiServer}\PYG{+w}{ }\PYG{n+nf}{server}\PYG{p}{(}\PYG{l+m+mi}{80}\PYG{p}{)}\PYG{p}{;}
\end{sphinxVerbatim}

\item {} 
\sphinxAtStartPar
\sphinxcode{\sphinxupquote{setup()}} Function

\sphinxAtStartPar
Initialize the serial communication and configure the WiFi module.

\begin{sphinxVerbatim}[commandchars=\\\{\}]
\PYG{k+kr}{void}\PYG{+w}{ }\PYG{n+nb}{setup}\PYG{p}{(}\PYG{p}{)}\PYG{+w}{ }\PYG{p}{\PYGZob{}}

\PYG{+w}{  }\PYG{c+c1}{// ... setup code ...}
\PYG{+w}{  }\PYG{c+c1}{// Create access point}
\PYG{+w}{  }\PYG{n}{status}\PYG{+w}{ }\PYG{o}{=}\PYG{+w}{ }\PYG{n+nf}{WiFi}\PYG{p}{.}\PYG{n}{beginAP}\PYG{p}{(}\PYG{n}{ssid}\PYG{p}{,}\PYG{+w}{ }\PYG{n}{pass}\PYG{p}{)}\PYG{p}{;}
\PYG{+w}{  }\PYG{c+c1}{// ... error handling ...}
\PYG{+w}{  }\PYG{c+c1}{// start the web server on port 80}
\PYG{+w}{  }\PYG{n}{server}\PYG{p}{.}\PYG{n+nf}{begin}\PYG{p}{(}\PYG{p}{)}\PYG{p}{;}
\PYG{p}{\PYGZcb{}}
\end{sphinxVerbatim}

\end{enumerate}
\begin{quote}

\sphinxAtStartPar
We also check if the firmware version of uno R4 wifi is up to date. If it is not the latest version, a prompt for upgrade will be displayed. You can refer to \DUrole{xref}{\DUrole{std}{\DUrole{std-ref}{update\_firmware}}} for firmware upgrade.
\begin{quote}

\begin{sphinxVerbatim}[commandchars=\\\{\}]
\PYG{p}{.}\PYG{p}{.}\PYG{p}{.}
\PYG{k+kr}{String}\PYG{+w}{ }\PYG{n}{fv}\PYG{+w}{ }\PYG{o}{=}\PYG{+w}{ }\PYG{n+nf}{WiFi}\PYG{p}{.}\PYG{n}{firmwareVersion}\PYG{p}{(}\PYG{p}{)}\PYG{p}{;}
\PYG{k}{if}\PYG{+w}{ }\PYG{p}{(}\PYG{n}{fv}\PYG{+w}{ }\PYG{o}{\PYGZlt{}}\PYG{+w}{ }\PYG{n}{WIFI\PYGZus{}FIRMWARE\PYGZus{}LATEST\PYGZus{}VERSION}\PYG{p}{)}\PYG{+w}{ }\PYG{p}{\PYGZob{}}
\PYG{+w}{    }\PYG{n+nf}{Serial}\PYG{p}{.}\PYG{n+nf}{println}\PYG{p}{(}\PYG{l+s}{\PYGZdq{}}\PYG{l+s}{Please upgrade the firmware}\PYG{l+s}{\PYGZdq{}}\PYG{p}{)}\PYG{p}{;}
\PYG{p}{\PYGZcb{}}
\PYG{p}{.}\PYG{p}{.}\PYG{p}{.}
\end{sphinxVerbatim}

\sphinxAtStartPar
You may want to modify the following code in order to be able to change the default IP of Arduino.

\begin{sphinxVerbatim}[commandchars=\\\{\}]
\PYG{n+nf}{WiFi}\PYG{p}{.}\PYG{n+nf}{config}\PYG{p}{(}\PYG{n+nf}{IPAddress}\PYG{p}{(}\PYG{l+m+mi}{192}\PYG{p}{,}\PYG{l+m+mi}{168}\PYG{p}{,}\PYG{l+m+mi}{4}\PYG{p}{,}\PYG{l+m+mi}{11}\PYG{p}{)}\PYG{p}{)}\PYG{p}{;}
\end{sphinxVerbatim}
\end{quote}
\end{quote}
\begin{enumerate}
\sphinxsetlistlabels{\arabic}{enumi}{enumii}{}{.}%
\item {} 
\sphinxAtStartPar
Main \sphinxcode{\sphinxupquote{loop()}} Function

\sphinxAtStartPar
The \sphinxcode{\sphinxupquote{loop()}} function in the Arduino code performs several key operations, specifically:
\begin{enumerate}
\sphinxsetlistlabels{\arabic}{enumii}{enumiii}{}{.}%
\item {} 
\sphinxAtStartPar
Checking if a device has connected or disconnected from the access point.

\item {} 
\sphinxAtStartPar
Listening for incoming clients who make HTTP requests.

\item {} 
\sphinxAtStartPar
Reading client data and executing actions based on that data—like turning an LED on or off.

\end{enumerate}

\sphinxAtStartPar
Here, let’s break down the \sphinxcode{\sphinxupquote{loop()}} function to make these steps more understandable.
\begin{enumerate}
\sphinxsetlistlabels{\arabic}{enumii}{enumiii}{}{.}%
\item {} 
\sphinxAtStartPar
Checking WiFi Status

\sphinxAtStartPar
The code first checks if the WiFi status has changed. If a device has connected or disconnected, the serial monitor will display the information accordingly.

\begin{sphinxVerbatim}[commandchars=\\\{\}]
\PYG{k}{if}\PYG{+w}{ }\PYG{p}{(}\PYG{n}{status}\PYG{+w}{ }\PYG{o}{!}\PYG{o}{=}\PYG{+w}{ }\PYG{n+nf}{WiFi}\PYG{p}{.}\PYG{n}{status}\PYG{p}{(}\PYG{p}{)}\PYG{p}{)}\PYG{+w}{ }\PYG{p}{\PYGZob{}}
\PYG{+w}{  }\PYG{n}{status}\PYG{+w}{ }\PYG{o}{=}\PYG{+w}{ }\PYG{n+nf}{WiFi}\PYG{p}{.}\PYG{n}{status}\PYG{p}{(}\PYG{p}{)}\PYG{p}{;}
\PYG{+w}{  }\PYG{k}{if}\PYG{+w}{ }\PYG{p}{(}\PYG{n}{status}\PYG{+w}{ }\PYG{o}{=}\PYG{o}{=}\PYG{+w}{ }\PYG{n}{WL\PYGZus{}AP\PYGZus{}CONNECTED}\PYG{p}{)}\PYG{+w}{ }\PYG{p}{\PYGZob{}}
\PYG{+w}{    }\PYG{n+nf}{Serial}\PYG{p}{.}\PYG{n+nf}{println}\PYG{p}{(}\PYG{l+s}{\PYGZdq{}}\PYG{l+s}{Device connected to AP}\PYG{l+s}{\PYGZdq{}}\PYG{p}{)}\PYG{p}{;}
\PYG{+w}{  }\PYG{p}{\PYGZcb{}}\PYG{+w}{ }\PYG{k}{else}\PYG{+w}{ }\PYG{p}{\PYGZob{}}
\PYG{+w}{    }\PYG{n+nf}{Serial}\PYG{p}{.}\PYG{n+nf}{println}\PYG{p}{(}\PYG{l+s}{\PYGZdq{}}\PYG{l+s}{Device disconnected from AP}\PYG{l+s}{\PYGZdq{}}\PYG{p}{)}\PYG{p}{;}
\PYG{+w}{  }\PYG{p}{\PYGZcb{}}
\PYG{p}{\PYGZcb{}}
\end{sphinxVerbatim}

\item {} 
\sphinxAtStartPar
Listening for Incoming Clients

\sphinxAtStartPar
\sphinxcode{\sphinxupquote{WiFiClient client = server.available();}} waits for incoming clients.

\begin{sphinxVerbatim}[commandchars=\\\{\}]
\PYG{n+nf}{WiFiClient}\PYG{+w}{ }\PYG{n}{client}\PYG{+w}{ }\PYG{o}{=}\PYG{+w}{ }\PYG{n}{server}\PYG{p}{.}\PYG{n+nf}{available}\PYG{p}{(}\PYG{p}{)}\PYG{p}{;}
\end{sphinxVerbatim}

\item {} 
\sphinxAtStartPar
Handling Client Requests

\sphinxAtStartPar
Listens for incoming clients and serves them the HTML web page. When a user clicks on the “Click here to turn the LED on” or “Click here to turn the LED off” links on the served webpage, an HTTP GET request is sent to the Arduino server. Specifically, the URLs “\sphinxurl{http://yourAddress/H}” for turning on the LED and “\sphinxurl{http://yourAddress/L}” for turning it off will be accessed.

\begin{sphinxVerbatim}[commandchars=\\\{\}]
\PYG{n+nf}{WiFiClient}\PYG{+w}{ }\PYG{n}{client}\PYG{+w}{ }\PYG{o}{=}\PYG{+w}{ }\PYG{n}{server}\PYG{p}{.}\PYG{n+nf}{available}\PYG{p}{(}\PYG{p}{)}\PYG{p}{;}
\PYG{k}{if}\PYG{+w}{ }\PYG{p}{(}\PYG{n}{client}\PYG{p}{)}\PYG{+w}{ }\PYG{p}{\PYGZob{}}
\PYG{+w}{  }\PYG{c+c1}{// ...}
\PYG{+w}{  }\PYG{n}{client}\PYG{p}{.}\PYG{n+nf}{println}\PYG{p}{(}\PYG{l+s}{\PYGZdq{}}\PYG{l+s}{HTTP/1.1 200 OK}\PYG{l+s}{\PYGZdq{}}\PYG{p}{)}\PYG{p}{;}
\PYG{+w}{  }\PYG{n}{client}\PYG{p}{.}\PYG{n+nf}{println}\PYG{p}{(}\PYG{l+s}{\PYGZdq{}}\PYG{l+s}{Content\PYGZhy{}type:text/html}\PYG{l+s}{\PYGZdq{}}\PYG{p}{)}\PYG{p}{;}
\PYG{+w}{  }\PYG{n}{client}\PYG{p}{.}\PYG{n+nf}{println}\PYG{p}{(}\PYG{p}{)}\PYG{p}{;}
\PYG{+w}{  }\PYG{n}{client}\PYG{p}{.}\PYG{n+nf}{print}\PYG{p}{(}\PYG{l+s}{\PYGZdq{}}\PYG{l+s}{\PYGZlt{}p style=}\PYG{l+s+se}{\PYGZbs{}\PYGZdq{}}\PYG{l+s}{font\PYGZhy{}size:7vw;}\PYG{l+s+se}{\PYGZbs{}\PYGZdq{}}\PYG{l+s}{\PYGZgt{}Click \PYGZlt{}a href=}\PYG{l+s+se}{\PYGZbs{}\PYGZdq{}}\PYG{l+s}{/H}\PYG{l+s+se}{\PYGZbs{}\PYGZdq{}}\PYG{l+s}{\PYGZgt{}here\PYGZlt{}/a\PYGZgt{} turn the LED on\PYGZlt{}br\PYGZgt{}\PYGZlt{}/p\PYGZgt{}}\PYG{l+s}{\PYGZdq{}}\PYG{p}{)}\PYG{p}{;}
\PYG{+w}{  }\PYG{n}{client}\PYG{p}{.}\PYG{n+nf}{print}\PYG{p}{(}\PYG{l+s}{\PYGZdq{}}\PYG{l+s}{\PYGZlt{}p style=}\PYG{l+s+se}{\PYGZbs{}\PYGZdq{}}\PYG{l+s}{font\PYGZhy{}size:7vw;}\PYG{l+s+se}{\PYGZbs{}\PYGZdq{}}\PYG{l+s}{\PYGZgt{}Click \PYGZlt{}a href=}\PYG{l+s+se}{\PYGZbs{}\PYGZdq{}}\PYG{l+s}{/L}\PYG{l+s+se}{\PYGZbs{}\PYGZdq{}}\PYG{l+s}{\PYGZgt{}here\PYGZlt{}/a\PYGZgt{} turn the LED off\PYGZlt{}br\PYGZgt{}\PYGZlt{}/p\PYGZgt{}}\PYG{l+s}{\PYGZdq{}}\PYG{p}{)}\PYG{p}{;}
\PYG{+w}{  }\PYG{c+c1}{// ...}
\PYG{p}{\PYGZcb{}}
\end{sphinxVerbatim}

\sphinxAtStartPar
The Arduino code listens for these incoming GET requests. When it detects \sphinxcode{\sphinxupquote{GET /H}} at the end of an incoming line of text (HTTP header), it sets the LED connected to pin 13 to HIGH, effectively turning it on. Similarly, if it detects \sphinxcode{\sphinxupquote{GET /L}}, it sets the LED to LOW, turning it off.

\begin{sphinxVerbatim}[commandchars=\\\{\}]
\PYG{k}{while}\PYG{+w}{ }\PYG{p}{(}\PYG{n}{client}\PYG{p}{.}\PYG{n+nf}{connected}\PYG{p}{(}\PYG{p}{)}\PYG{p}{)}\PYG{+w}{ }\PYG{p}{\PYGZob{}}\PYG{+w}{            }\PYG{c+c1}{// loop while the client\PYGZsq{}s connected}
\PYG{+w}{  }\PYG{n+nf}{delayMicroseconds}\PYG{p}{(}\PYG{l+m+mi}{10}\PYG{p}{)}\PYG{p}{;}\PYG{+w}{                }\PYG{c+c1}{// This is required for the Arduino Nano RP2040 Connect \PYGZhy{} otherwise it will loop so fast that SPI will never be served.}
\PYG{+w}{  }\PYG{k}{if}\PYG{+w}{ }\PYG{p}{(}\PYG{n}{client}\PYG{p}{.}\PYG{n+nf}{available}\PYG{p}{(}\PYG{p}{)}\PYG{p}{)}\PYG{+w}{ }\PYG{p}{\PYGZob{}}\PYG{+w}{             }\PYG{c+c1}{// if there\PYGZsq{}s bytes to read from the client,}
\PYG{+w}{    }\PYG{k+kr}{char}\PYG{+w}{ }\PYG{n}{c}\PYG{+w}{ }\PYG{o}{=}\PYG{+w}{ }\PYG{n}{client}\PYG{p}{.}\PYG{n+nf}{read}\PYG{p}{(}\PYG{p}{)}\PYG{p}{;}\PYG{+w}{             }\PYG{c+c1}{// read a byte, then}
\PYG{+w}{    }\PYG{n+nf}{Serial}\PYG{p}{.}\PYG{n+nf}{write}\PYG{p}{(}\PYG{n}{c}\PYG{p}{)}\PYG{p}{;}\PYG{+w}{                    }\PYG{c+c1}{// print it out to the serial monitor}
\PYG{+w}{    }\PYG{k}{if}\PYG{+w}{ }\PYG{p}{(}\PYG{n}{c}\PYG{+w}{ }\PYG{o}{=}\PYG{o}{=}\PYG{+w}{ }\PYG{l+s+sc}{\PYGZsq{}}\PYG{l+s+sc}{\PYGZbs{}n}\PYG{l+s+sc}{\PYGZsq{}}\PYG{p}{)}\PYG{+w}{ }\PYG{p}{\PYGZob{}}\PYG{+w}{                    }\PYG{c+c1}{// if the byte is a newline character}
\PYG{+w}{      }\PYG{p}{.}\PYG{p}{.}\PYG{p}{.}
\PYG{+w}{      }\PYG{p}{\PYGZcb{}}
\PYG{+w}{      }\PYG{k}{else}\PYG{+w}{ }\PYG{p}{\PYGZob{}}\PYG{+w}{      }\PYG{c+c1}{// if you got a newline, then clear currentLine:}
\PYG{+w}{        }\PYG{n}{currentLine}\PYG{+w}{ }\PYG{o}{=}\PYG{+w}{ }\PYG{l+s}{\PYGZdq{}}\PYG{l+s}{\PYGZdq{}}\PYG{p}{;}
\PYG{+w}{      }\PYG{p}{\PYGZcb{}}
\PYG{+w}{    }\PYG{p}{\PYGZcb{}}
\PYG{+w}{    }\PYG{k}{else}\PYG{+w}{ }\PYG{k}{if}\PYG{+w}{ }\PYG{p}{(}\PYG{n}{c}\PYG{+w}{ }\PYG{o}{!}\PYG{o}{=}\PYG{+w}{ }\PYG{l+s+sc}{\PYGZsq{}}\PYG{l+s+sc}{\PYGZbs{}r}\PYG{l+s+sc}{\PYGZsq{}}\PYG{p}{)}\PYG{+w}{ }\PYG{p}{\PYGZob{}}\PYG{+w}{    }\PYG{c+c1}{// if you got anything else but a carriage return character,}
\PYG{+w}{      }\PYG{n}{currentLine}\PYG{+w}{ }\PYG{o}{+}\PYG{o}{=}\PYG{+w}{ }\PYG{n}{c}\PYG{p}{;}\PYG{+w}{      }\PYG{c+c1}{// add it to the end of the currentLine}
\PYG{+w}{    }\PYG{p}{\PYGZcb{}}

\PYG{+w}{    }\PYG{c+c1}{// Check to see if the client request was \PYGZdq{}GET /H\PYGZdq{} or \PYGZdq{}GET /L\PYGZdq{}:}
\PYG{+w}{    }\PYG{k}{if}\PYG{+w}{ }\PYG{p}{(}\PYG{n}{currentLine}\PYG{p}{.}\PYG{n}{endsWith}\PYG{p}{(}\PYG{l+s}{\PYGZdq{}}\PYG{l+s}{GET /H}\PYG{l+s}{\PYGZdq{}}\PYG{p}{)}\PYG{p}{)}\PYG{+w}{ }\PYG{p}{\PYGZob{}}
\PYG{+w}{      }\PYG{n+nf}{digitalWrite}\PYG{p}{(}\PYG{n}{led}\PYG{p}{,}\PYG{+w}{ }\PYG{k+kr}{HIGH}\PYG{p}{)}\PYG{p}{;}\PYG{+w}{               }\PYG{c+c1}{// GET /H turns the LED on}
\PYG{+w}{    }\PYG{p}{\PYGZcb{}}
\PYG{+w}{    }\PYG{k}{if}\PYG{+w}{ }\PYG{p}{(}\PYG{n}{currentLine}\PYG{p}{.}\PYG{n}{endsWith}\PYG{p}{(}\PYG{l+s}{\PYGZdq{}}\PYG{l+s}{GET /L}\PYG{l+s}{\PYGZdq{}}\PYG{p}{)}\PYG{p}{)}\PYG{+w}{ }\PYG{p}{\PYGZob{}}
\PYG{+w}{      }\PYG{n+nf}{digitalWrite}\PYG{p}{(}\PYG{n}{led}\PYG{p}{,}\PYG{+w}{ }\PYG{k+kr}{LOW}\PYG{p}{)}\PYG{p}{;}\PYG{+w}{                }\PYG{c+c1}{// GET /L turns the LED off}
\PYG{+w}{    }\PYG{p}{\PYGZcb{}}
\PYG{+w}{  }\PYG{p}{\PYGZcb{}}
\end{sphinxVerbatim}

\end{enumerate}

\end{enumerate}

\sphinxAtStartPar
\sphinxstylestrong{Reference}

\sphinxAtStartPar
\sphinxhref{https://docs.arduino.cc/tutorials/uno-r4-wifi/wifi-examples}{UNO R4 WiFi Network Examples}

\sphinxstepscope


\section{Web Control Relay (Station Mode)}
\label{\detokenize{Extension_Project/Web_Control_Relay:web-control-relay-station-mode}}\label{\detokenize{Extension_Project/Web_Control_Relay:ext-web-control-relay}}\label{\detokenize{Extension_Project/Web_Control_Relay::doc}}

\subsection{Overview}
\label{\detokenize{Extension_Project/Web_Control_Relay:overview}}
\sphinxAtStartPar
In this Article we are going to make a very useful and very easy home\sphinxhyphen{}automation project.

\sphinxAtStartPar
And for making of this home\sphinxhyphen{}automation system, we don’t required any kind Internet and IOT Platform like blynk, ESP\sphinxhyphen{}Rainmeker, Arduino IOT Cloud.

\sphinxAtStartPar
We need only a Router or hotspot for making a local server, here router will work as a bridge between web\sphinxhyphen{}page and Arduino UNO R4 WIFI board.

\noindent{\hspace*{\fill}\sphinxincludegraphics[width=0.800\linewidth]{{Web_Control_Realy}.png}\hspace*{\fill}}


\subsection{Wiring}
\label{\detokenize{Extension_Project/Web_Control_Relay:wiring}}
\begin{sphinxadmonition}{note}{Note:}
\sphinxAtStartPar
In the experiment, a relay is used to control the closing and opening of a 5mm LED. If you have enough experience and knowledge in using electricity, you can try to control the lighting bulb with a relay while ensuring safety.
\end{sphinxadmonition}

\noindent{\hspace*{\fill}\sphinxincludegraphics[width=0.700\linewidth]{{Web_Control_Realy_Wiring}.png}\hspace*{\fill}}

\noindent{\hspace*{\fill}\sphinxincludegraphics[width=0.800\linewidth]{{Web_Control_Realy_Wiring}.png}\hspace*{\fill}}


\subsection{1. Upload the code}
\label{\detokenize{Extension_Project/Web_Control_Relay:upload-the-code}}
\sphinxAtStartPar
Open the \sphinxcode{\sphinxupquote{18\_Web\_Control\_Relay}} file under the path of \sphinxcode{\sphinxupquote{Basic\sphinxhyphen{}Starter\sphinxhyphen{}Kit\sphinxhyphen{}for\sphinxhyphen{}Arduino\sphinxhyphen{}Uno\sphinxhyphen{}R4\sphinxhyphen{}WiFi\sphinxhyphen{}main\textbackslash{}Code}}, or copy this code into \sphinxstylestrong{Arduino IDE}.

\begin{sphinxadmonition}{note}{Note:}
\sphinxAtStartPar
Wi\sphinxhyphen{}Fi® support is enabled via the built\sphinxhyphen{}in \sphinxcode{\sphinxupquote{WiFiS3}} library that is shipped with the Arduino UNO R4 Core. Installing the core automatically installs the \sphinxcode{\sphinxupquote{WiFiS3}} library.
\end{sphinxadmonition}

\sphinxAtStartPar
You still need to create or modify \sphinxcode{\sphinxupquote{arduino\_secrets.h}}, replace \sphinxcode{\sphinxupquote{SECRET\_SSID}} and \sphinxcode{\sphinxupquote{SECRET\_PASS}} with the name and password of your home router WiFi access point. The file should contain:

\begin{sphinxVerbatim}[commandchars=\\\{\}]
\PYG{c+c1}{//arduino\PYGZus{}secrets.h header file}
\PYG{c+cp}{\PYGZsh{}}\PYG{c+cp}{define SECRET\PYGZus{}SSID \PYGZdq{}yournetwork\PYGZdq{}}
\PYG{c+cp}{\PYGZsh{}}\PYG{c+cp}{define SECRET\PYGZus{}PASS \PYGZdq{}yourpassword\PYGZdq{}}
\end{sphinxVerbatim}

\begin{sphinxadmonition}{note}{Note:}\begin{quote}

\sphinxAtStartPar
After uploading the code, check the IP address of UNO R4’s Wifi in the serial monitor of Arduino IDE.

\noindent{\hspace*{\fill}\sphinxincludegraphics[width=1.000\linewidth]{{Web_Control_Realy_Code}.png}\hspace*{\fill}}
\end{quote}

\sphinxAtStartPar
Then use your mobile phone to connect to your home router’s WiFi (the mobile phone and UNO R4 are connected to the same LAN), and access the returned IP address in the serial monitor in the mobile browser. The browser will get the control page.
\end{sphinxadmonition}

\noindent{\hspace*{\fill}\sphinxincludegraphics[width=0.600\linewidth]{{Web_Control_Realy_Code1}.png}\hspace*{\fill}}


\subsection{2. Code explanation}
\label{\detokenize{Extension_Project/Web_Control_Relay:code-explanation}}\begin{enumerate}
\sphinxsetlistlabels{\arabic}{enumi}{enumii}{}{.}%
\item {} 
\sphinxAtStartPar
Importing Required Libraries

\sphinxAtStartPar
Importing the \sphinxcode{\sphinxupquote{Arduino\_LED\_Matrix}} library for controlling LED matrix, \sphinxcode{\sphinxupquote{WiFiS3}} library for WiFi functionalities, and \sphinxcode{\sphinxupquote{arduino\_secrets.h}} for sensitive data like passwords.

\begin{sphinxVerbatim}[commandchars=\\\{\}]
\PYG{c+cp}{\PYGZsh{}}\PYG{c+cp}{include}\PYG{+w}{ }\PYG{c+cpf}{\PYGZdq{}Arduino\PYGZus{}LED\PYGZus{}Matrix.h\PYGZdq{}}
\PYG{c+cp}{\PYGZsh{}}\PYG{c+cp}{include}\PYG{+w}{ }\PYG{c+cpf}{\PYGZdq{}WiFiS3.h\PYGZdq{}}
\PYG{c+cp}{\PYGZsh{}}\PYG{c+cp}{include}\PYG{+w}{ }\PYG{c+cpf}{\PYGZdq{}arduino\PYGZus{}secrets.h\PYGZdq{}}
\end{sphinxVerbatim}

\item {} 
\sphinxAtStartPar
Configuration and Variable Initialization

\sphinxAtStartPar
Define WiFi SSID, password, key index, LED pins, WiFi status, and create a WiFi server.

\begin{sphinxVerbatim}[commandchars=\\\{\}]
\PYG{c+cp}{\PYGZsh{}}\PYG{c+cp}{define LED1 7}
\PYG{c+cp}{\PYGZsh{}}\PYG{c+cp}{define LED2 8}

\PYG{k+kr}{char}\PYG{+w}{ }\PYG{n}{ssid}\PYG{p}{[}\PYG{p}{]}\PYG{+w}{ }\PYG{o}{=}\PYG{+w}{ }\PYG{n}{SECRET\PYGZus{}SSID}\PYG{p}{;}
\PYG{k+kr}{char}\PYG{+w}{ }\PYG{n}{pass}\PYG{p}{[}\PYG{p}{]}\PYG{+w}{ }\PYG{o}{=}\PYG{+w}{ }\PYG{n}{SECRET\PYGZus{}PASS}\PYG{p}{;}
\PYG{k+kr}{int}\PYG{+w}{ }\PYG{n}{keyIndex}\PYG{+w}{ }\PYG{o}{=}\PYG{+w}{ }\PYG{l+m+mi}{0}\PYG{p}{;}
\PYG{k+kr}{int}\PYG{+w}{ }\PYG{n}{status}\PYG{+w}{ }\PYG{o}{=}\PYG{+w}{ }\PYG{n}{WL\PYGZus{}IDLE\PYGZus{}STATUS}\PYG{p}{;}
\PYG{n+nf}{WiFiServer}\PYG{+w}{ }\PYG{n+nf}{server}\PYG{p}{(}\PYG{l+m+mi}{80}\PYG{p}{)}\PYG{p}{;}
\end{sphinxVerbatim}

\item {} 
\sphinxAtStartPar
\sphinxcode{\sphinxupquote{setup()}} Function

\sphinxAtStartPar
Initialize serial communication, configure the WiFi module, start the server, and initialize LED pins.

\begin{sphinxVerbatim}[commandchars=\\\{\}]
\PYG{k+kr}{void}\PYG{+w}{ }\PYG{n+nb}{setup}\PYG{p}{(}\PYG{p}{)}\PYG{+w}{ }\PYG{p}{\PYGZob{}}
\PYG{+w}{  }\PYG{n+nf}{Serial}\PYG{p}{.}\PYG{n+nf}{begin}\PYG{p}{(}\PYG{l+m+mi}{9600}\PYG{p}{)}\PYG{p}{;}
\PYG{+w}{  }\PYG{n}{matrix}\PYG{p}{.}\PYG{n+nf}{begin}\PYG{p}{(}\PYG{p}{)}\PYG{p}{;}
\PYG{+w}{  }\PYG{n+nf}{pinMode}\PYG{p}{(}\PYG{n}{LED1}\PYG{p}{,}\PYG{+w}{ }\PYG{k+kr}{OUTPUT}\PYG{p}{)}\PYG{p}{;}
\PYG{+w}{  }\PYG{n+nf}{pinMode}\PYG{p}{(}\PYG{n}{LED2}\PYG{p}{,}\PYG{+w}{ }\PYG{k+kr}{OUTPUT}\PYG{p}{)}\PYG{p}{;}
\PYG{+w}{  }\PYG{n+nf}{digitalWrite}\PYG{p}{(}\PYG{n}{LED1}\PYG{p}{,}\PYG{+w}{ }\PYG{k+kr}{HIGH}\PYG{p}{)}\PYG{p}{;}
\PYG{+w}{  }\PYG{n+nf}{digitalWrite}\PYG{p}{(}\PYG{n}{LED2}\PYG{p}{,}\PYG{+w}{ }\PYG{k+kr}{HIGH}\PYG{p}{)}\PYG{p}{;}

\PYG{+w}{  }\PYG{k}{if}\PYG{+w}{ }\PYG{p}{(}\PYG{n+nf}{WiFi}\PYG{p}{.}\PYG{n}{status}\PYG{p}{(}\PYG{p}{)}\PYG{+w}{ }\PYG{o}{=}\PYG{o}{=}\PYG{+w}{ }\PYG{n}{WL\PYGZus{}NO\PYGZus{}MODULE}\PYG{p}{)}\PYG{+w}{ }\PYG{p}{\PYGZob{}}
\PYG{+w}{    }\PYG{n+nf}{Serial}\PYG{p}{.}\PYG{n+nf}{println}\PYG{p}{(}\PYG{l+s}{\PYGZdq{}}\PYG{l+s}{Communication with WiFi module failed!}\PYG{l+s}{\PYGZdq{}}\PYG{p}{)}\PYG{p}{;}
\PYG{+w}{    }\PYG{k}{while}\PYG{+w}{ }\PYG{p}{(}\PYG{k+kr}{true}\PYG{p}{)}\PYG{p}{;}
\PYG{+w}{  }\PYG{p}{\PYGZcb{}}

\PYG{+w}{  }\PYG{k+kr}{String}\PYG{+w}{ }\PYG{n}{fv}\PYG{+w}{ }\PYG{o}{=}\PYG{+w}{ }\PYG{n+nf}{WiFi}\PYG{p}{.}\PYG{n}{firmwareVersion}\PYG{p}{(}\PYG{p}{)}\PYG{p}{;}
\PYG{+w}{  }\PYG{k}{if}\PYG{+w}{ }\PYG{p}{(}\PYG{n}{fv}\PYG{+w}{ }\PYG{o}{\PYGZlt{}}\PYG{+w}{ }\PYG{n}{WIFI\PYGZus{}FIRMWARE\PYGZus{}LATEST\PYGZus{}VERSION}\PYG{p}{)}\PYG{+w}{ }\PYG{p}{\PYGZob{}}
\PYG{+w}{    }\PYG{n+nf}{Serial}\PYG{p}{.}\PYG{n+nf}{println}\PYG{p}{(}\PYG{l+s}{\PYGZdq{}}\PYG{l+s}{Please upgrade the firmware}\PYG{l+s}{\PYGZdq{}}\PYG{p}{)}\PYG{p}{;}
\PYG{+w}{  }\PYG{p}{\PYGZcb{}}

\PYG{+w}{  }\PYG{k}{while}\PYG{+w}{ }\PYG{p}{(}\PYG{n}{status}\PYG{+w}{ }\PYG{o}{!}\PYG{o}{=}\PYG{+w}{ }\PYG{n}{WL\PYGZus{}CONNECTED}\PYG{p}{)}\PYG{+w}{ }\PYG{p}{\PYGZob{}}
\PYG{+w}{    }\PYG{n+nf}{Serial}\PYG{p}{.}\PYG{n+nf}{print}\PYG{p}{(}\PYG{l+s}{\PYGZdq{}}\PYG{l+s}{Network named: }\PYG{l+s}{\PYGZdq{}}\PYG{p}{)}\PYG{p}{;}
\PYG{+w}{    }\PYG{n+nf}{Serial}\PYG{p}{.}\PYG{n+nf}{println}\PYG{p}{(}\PYG{n}{ssid}\PYG{p}{)}\PYG{p}{;}
\PYG{+w}{    }\PYG{n}{status}\PYG{+w}{ }\PYG{o}{=}\PYG{+w}{ }\PYG{n+nf}{WiFi}\PYG{p}{.}\PYG{n+nf}{begin}\PYG{p}{(}\PYG{n}{ssid}\PYG{p}{,}\PYG{+w}{ }\PYG{n}{pass}\PYG{p}{)}\PYG{p}{;}
\PYG{+w}{    }\PYG{n+nf}{delay}\PYG{p}{(}\PYG{l+m+mi}{10000}\PYG{p}{)}\PYG{p}{;}
\PYG{+w}{  }\PYG{p}{\PYGZcb{}}
\PYG{+w}{  }\PYG{n}{server}\PYG{p}{.}\PYG{n+nf}{begin}\PYG{p}{(}\PYG{p}{)}\PYG{p}{;}
\PYG{+w}{  }\PYG{n}{printWifiStatus}\PYG{p}{(}\PYG{p}{)}\PYG{p}{;}
\PYG{p}{\PYGZcb{}}
\end{sphinxVerbatim}

\item {} 
\sphinxAtStartPar
Main \sphinxcode{\sphinxupquote{loop()}} Function

\sphinxAtStartPar
The \sphinxcode{\sphinxupquote{loop()}} function performs two key operations: running the web server and updating the LED matrix.

\begin{sphinxVerbatim}[commandchars=\\\{\}]
\PYG{k+kr}{void}\PYG{+w}{ }\PYG{n+nb}{loop}\PYG{p}{(}\PYG{p}{)}\PYG{+w}{ }\PYG{p}{\PYGZob{}}
\PYG{+w}{  }\PYG{n}{webServer}\PYG{p}{(}\PYG{p}{)}\PYG{p}{;}
\PYG{+w}{  }\PYG{n}{LEDMatrix}\PYG{p}{(}\PYG{p}{)}\PYG{p}{;}
\PYG{p}{\PYGZcb{}}
\end{sphinxVerbatim}

\item {} 
\sphinxAtStartPar
\sphinxcode{\sphinxupquote{LEDMatrix()}} Function

\sphinxAtStartPar
Load a frame to the LED matrix.

\begin{sphinxVerbatim}[commandchars=\\\{\}]
\PYG{k+kr}{void}\PYG{+w}{ }\PYG{n+nf}{LEDMatrix}\PYG{p}{(}\PYG{p}{)}\PYG{+w}{ }\PYG{p}{\PYGZob{}}
\PYG{+w}{  }\PYG{n}{matrix}\PYG{p}{.}\PYG{n}{loadFrame}\PYG{p}{(}\PYG{n}{hi}\PYG{p}{)}\PYG{p}{;}
\PYG{p}{\PYGZcb{}}
\end{sphinxVerbatim}

\item {} 
\sphinxAtStartPar
\sphinxcode{\sphinxupquote{webServer()}} Function

\sphinxAtStartPar
Listen for incoming clients, parse their requests, and control LEDs based on the received commands.

\begin{sphinxVerbatim}[commandchars=\\\{\}]
\PYG{k+kr}{void}\PYG{+w}{ }\PYG{n+nf}{webServer}\PYG{p}{(}\PYG{p}{)}\PYG{+w}{ }\PYG{p}{\PYGZob{}}
\PYG{+w}{  }\PYG{n+nf}{WiFiClient}\PYG{+w}{ }\PYG{n}{client}\PYG{+w}{ }\PYG{o}{=}\PYG{+w}{ }\PYG{n}{server}\PYG{p}{.}\PYG{n+nf}{available}\PYG{p}{(}\PYG{p}{)}\PYG{p}{;}
\PYG{+w}{  }\PYG{k}{if}\PYG{+w}{ }\PYG{p}{(}\PYG{n}{client}\PYG{p}{)}\PYG{+w}{ }\PYG{p}{\PYGZob{}}
\PYG{+w}{    }\PYG{n+nf}{Serial}\PYG{p}{.}\PYG{n+nf}{println}\PYG{p}{(}\PYG{l+s}{\PYGZdq{}}\PYG{l+s}{New Client.}\PYG{l+s}{\PYGZdq{}}\PYG{p}{)}\PYG{p}{;}
\PYG{+w}{    }\PYG{k+kr}{String}\PYG{+w}{ }\PYG{n}{currentLine}\PYG{+w}{ }\PYG{o}{=}\PYG{+w}{ }\PYG{l+s}{\PYGZdq{}}\PYG{l+s}{\PYGZdq{}}\PYG{p}{;}
\PYG{+w}{    }\PYG{n}{currentTime}\PYG{+w}{ }\PYG{o}{=}\PYG{+w}{ }\PYG{n+nf}{millis}\PYG{p}{(}\PYG{p}{)}\PYG{p}{;}
\PYG{+w}{    }\PYG{n}{previousTime}\PYG{+w}{ }\PYG{o}{=}\PYG{+w}{ }\PYG{n}{currentTime}\PYG{p}{;}
\PYG{+w}{    }\PYG{k}{while}\PYG{+w}{ }\PYG{p}{(}\PYG{n}{client}\PYG{p}{.}\PYG{n+nf}{connected}\PYG{p}{(}\PYG{p}{)}\PYG{+w}{ }\PYG{o}{\PYGZam{}}\PYG{o}{\PYGZam{}}\PYG{+w}{ }\PYG{n}{currentTime}\PYG{+w}{ }\PYG{o}{\PYGZhy{}}\PYG{+w}{ }\PYG{n}{previousTime}\PYG{+w}{ }\PYG{o}{\PYGZlt{}}\PYG{o}{=}\PYG{+w}{ }\PYG{n}{timeoutTime}\PYG{p}{)}\PYG{+w}{ }\PYG{p}{\PYGZob{}}
\PYG{+w}{      }\PYG{n}{currentTime}\PYG{+w}{ }\PYG{o}{=}\PYG{+w}{ }\PYG{n+nf}{millis}\PYG{p}{(}\PYG{p}{)}\PYG{p}{;}
\PYG{+w}{      }\PYG{k}{if}\PYG{+w}{ }\PYG{p}{(}\PYG{n}{client}\PYG{p}{.}\PYG{n+nf}{available}\PYG{p}{(}\PYG{p}{)}\PYG{p}{)}\PYG{+w}{ }\PYG{p}{\PYGZob{}}
\PYG{+w}{        }\PYG{k+kr}{char}\PYG{+w}{ }\PYG{n}{c}\PYG{+w}{ }\PYG{o}{=}\PYG{+w}{ }\PYG{n}{client}\PYG{p}{.}\PYG{n+nf}{read}\PYG{p}{(}\PYG{p}{)}\PYG{p}{;}
\PYG{+w}{        }\PYG{n+nf}{Serial}\PYG{p}{.}\PYG{n+nf}{write}\PYG{p}{(}\PYG{n}{c}\PYG{p}{)}\PYG{p}{;}
\PYG{+w}{        }\PYG{n}{header}\PYG{+w}{ }\PYG{o}{+}\PYG{o}{=}\PYG{+w}{ }\PYG{n}{c}\PYG{p}{;}
\PYG{+w}{        }\PYG{k}{if}\PYG{+w}{ }\PYG{p}{(}\PYG{n}{c}\PYG{+w}{ }\PYG{o}{=}\PYG{o}{=}\PYG{+w}{ }\PYG{l+s+sc}{\PYGZsq{}}\PYG{l+s+sc}{\PYGZbs{}n}\PYG{l+s+sc}{\PYGZsq{}}\PYG{p}{)}\PYG{+w}{ }\PYG{p}{\PYGZob{}}
\PYG{+w}{          }\PYG{k}{if}\PYG{+w}{ }\PYG{p}{(}\PYG{n}{currentLine}\PYG{p}{.}\PYG{n}{length}\PYG{p}{(}\PYG{p}{)}\PYG{+w}{ }\PYG{o}{=}\PYG{o}{=}\PYG{+w}{ }\PYG{l+m+mi}{0}\PYG{p}{)}\PYG{+w}{ }\PYG{p}{\PYGZob{}}
\PYG{+w}{            }\PYG{n}{client}\PYG{p}{.}\PYG{n+nf}{println}\PYG{p}{(}\PYG{l+s}{\PYGZdq{}}\PYG{l+s}{HTTP/1.1 200 OK}\PYG{l+s}{\PYGZdq{}}\PYG{p}{)}\PYG{p}{;}
\PYG{+w}{            }\PYG{n}{client}\PYG{p}{.}\PYG{n+nf}{println}\PYG{p}{(}\PYG{l+s}{\PYGZdq{}}\PYG{l+s}{Content\PYGZhy{}type:text/html}\PYG{l+s}{\PYGZdq{}}\PYG{p}{)}\PYG{p}{;}
\PYG{+w}{            }\PYG{n}{client}\PYG{p}{.}\PYG{n+nf}{println}\PYG{p}{(}\PYG{l+s}{\PYGZdq{}}\PYG{l+s}{Connection: close}\PYG{l+s}{\PYGZdq{}}\PYG{p}{)}\PYG{p}{;}
\PYG{+w}{            }\PYG{n}{client}\PYG{p}{.}\PYG{n+nf}{println}\PYG{p}{(}\PYG{p}{)}\PYG{p}{;}
\PYG{+w}{            }\PYG{k}{if}\PYG{+w}{ }\PYG{p}{(}\PYG{n}{header}\PYG{p}{.}\PYG{n}{indexOf}\PYG{p}{(}\PYG{l+s}{\PYGZdq{}}\PYG{l+s}{GET /ledon1}\PYG{l+s}{\PYGZdq{}}\PYG{p}{)}\PYG{+w}{ }\PYG{o}{\PYGZgt{}}\PYG{o}{=}\PYG{+w}{ }\PYG{l+m+mi}{0}\PYG{p}{)}\PYG{+w}{ }\PYG{p}{\PYGZob{}}
\PYG{+w}{              }\PYG{n}{output1}\PYG{+w}{ }\PYG{o}{=}\PYG{+w}{ }\PYG{l+s}{\PYGZdq{}}\PYG{l+s}{on}\PYG{l+s}{\PYGZdq{}}\PYG{p}{;}
\PYG{+w}{              }\PYG{n+nf}{digitalWrite}\PYG{p}{(}\PYG{n}{LED1}\PYG{p}{,}\PYG{+w}{ }\PYG{k+kr}{LOW}\PYG{p}{)}\PYG{p}{;}
\PYG{+w}{            }\PYG{p}{\PYGZcb{}}\PYG{+w}{ }\PYG{k}{else}\PYG{+w}{ }\PYG{k}{if}\PYG{+w}{ }\PYG{p}{(}\PYG{n}{header}\PYG{p}{.}\PYG{n}{indexOf}\PYG{p}{(}\PYG{l+s}{\PYGZdq{}}\PYG{l+s}{GET /ledoff1}\PYG{l+s}{\PYGZdq{}}\PYG{p}{)}\PYG{+w}{ }\PYG{o}{\PYGZgt{}}\PYG{o}{=}\PYG{+w}{ }\PYG{l+m+mi}{0}\PYG{p}{)}\PYG{+w}{ }\PYG{p}{\PYGZob{}}
\PYG{+w}{              }\PYG{n}{output1}\PYG{+w}{ }\PYG{o}{=}\PYG{+w}{ }\PYG{l+s}{\PYGZdq{}}\PYG{l+s}{off}\PYG{l+s}{\PYGZdq{}}\PYG{p}{;}
\PYG{+w}{              }\PYG{n+nf}{digitalWrite}\PYG{p}{(}\PYG{n}{LED1}\PYG{p}{,}\PYG{+w}{ }\PYG{k+kr}{HIGH}\PYG{p}{)}\PYG{p}{;}
\PYG{+w}{            }\PYG{p}{\PYGZcb{}}
\PYG{+w}{            }\PYG{k}{if}\PYG{+w}{ }\PYG{p}{(}\PYG{n}{header}\PYG{p}{.}\PYG{n}{indexOf}\PYG{p}{(}\PYG{l+s}{\PYGZdq{}}\PYG{l+s}{GET /ledon2}\PYG{l+s}{\PYGZdq{}}\PYG{p}{)}\PYG{+w}{ }\PYG{o}{\PYGZgt{}}\PYG{o}{=}\PYG{+w}{ }\PYG{l+m+mi}{0}\PYG{p}{)}\PYG{+w}{ }\PYG{p}{\PYGZob{}}
\PYG{+w}{              }\PYG{n}{output2}\PYG{+w}{ }\PYG{o}{=}\PYG{+w}{ }\PYG{l+s}{\PYGZdq{}}\PYG{l+s}{on}\PYG{l+s}{\PYGZdq{}}\PYG{p}{;}
\PYG{+w}{              }\PYG{n+nf}{digitalWrite}\PYG{p}{(}\PYG{n}{LED2}\PYG{p}{,}\PYG{+w}{ }\PYG{k+kr}{LOW}\PYG{p}{)}\PYG{p}{;}
\PYG{+w}{            }\PYG{p}{\PYGZcb{}}\PYG{+w}{ }\PYG{k}{else}\PYG{+w}{ }\PYG{k}{if}\PYG{+w}{ }\PYG{p}{(}\PYG{n}{header}\PYG{p}{.}\PYG{n}{indexOf}\PYG{p}{(}\PYG{l+s}{\PYGZdq{}}\PYG{l+s}{GET /ledoff2}\PYG{l+s}{\PYGZdq{}}\PYG{p}{)}\PYG{+w}{ }\PYG{o}{\PYGZgt{}}\PYG{o}{=}\PYG{+w}{ }\PYG{l+m+mi}{0}\PYG{p}{)}\PYG{+w}{ }\PYG{p}{\PYGZob{}}
\PYG{+w}{              }\PYG{n}{output2}\PYG{+w}{ }\PYG{o}{=}\PYG{+w}{ }\PYG{l+s}{\PYGZdq{}}\PYG{l+s}{off}\PYG{l+s}{\PYGZdq{}}\PYG{p}{;}
\PYG{+w}{              }\PYG{n+nf}{digitalWrite}\PYG{p}{(}\PYG{n}{LED2}\PYG{p}{,}\PYG{+w}{ }\PYG{k+kr}{HIGH}\PYG{p}{)}\PYG{p}{;}
\PYG{+w}{            }\PYG{p}{\PYGZcb{}}
\PYG{+w}{            }\PYG{n}{client}\PYG{p}{.}\PYG{n+nf}{println}\PYG{p}{(}\PYG{l+s}{\PYGZdq{}}\PYG{l+s}{\PYGZlt{}!DOCTYPE html\PYGZgt{}\PYGZlt{}html\PYGZgt{}}\PYG{l+s}{\PYGZdq{}}\PYG{p}{)}\PYG{p}{;}
\PYG{+w}{            }\PYG{n}{client}\PYG{p}{.}\PYG{n+nf}{println}\PYG{p}{(}\PYG{l+s}{\PYGZdq{}}\PYG{l+s}{\PYGZlt{}head\PYGZgt{}\PYGZlt{}meta name=}\PYG{l+s+se}{\PYGZbs{}\PYGZdq{}}\PYG{l+s}{viewport}\PYG{l+s+se}{\PYGZbs{}\PYGZdq{}}\PYG{l+s}{ content=}\PYG{l+s+se}{\PYGZbs{}\PYGZdq{}}\PYG{l+s}{width=device\PYGZhy{}width, initial\PYGZhy{}scale=1}\PYG{l+s+se}{\PYGZbs{}\PYGZdq{}}\PYG{l+s}{\PYGZgt{}}\PYG{l+s}{\PYGZdq{}}\PYG{p}{)}\PYG{p}{;}
\PYG{+w}{            }\PYG{n}{client}\PYG{p}{.}\PYG{n+nf}{println}\PYG{p}{(}\PYG{l+s}{\PYGZdq{}}\PYG{l+s}{\PYGZlt{}link rel=}\PYG{l+s+se}{\PYGZbs{}\PYGZdq{}}\PYG{l+s}{icon}\PYG{l+s+se}{\PYGZbs{}\PYGZdq{}}\PYG{l+s}{ href=}\PYG{l+s+se}{\PYGZbs{}\PYGZdq{}}\PYG{l+s}{data:,}\PYG{l+s+se}{\PYGZbs{}\PYGZdq{}}\PYG{l+s}{\PYGZgt{}}\PYG{l+s}{\PYGZdq{}}\PYG{p}{)}\PYG{p}{;}
\PYG{+w}{            }\PYG{n}{client}\PYG{p}{.}\PYG{n+nf}{println}\PYG{p}{(}\PYG{l+s}{\PYGZdq{}}\PYG{l+s}{\PYGZlt{}style\PYGZgt{}html \PYGZob{} font\PYGZhy{}family: Helvetica; display: inline\PYGZhy{}block; margin: 0px auto; text\PYGZhy{}align: center;\PYGZcb{}}\PYG{l+s}{\PYGZdq{}}\PYG{p}{)}\PYG{p}{;}
\PYG{+w}{            }\PYG{n}{client}\PYG{p}{.}\PYG{n+nf}{println}\PYG{p}{(}\PYG{l+s}{\PYGZdq{}}\PYG{l+s}{.on \PYGZob{} background\PYGZhy{}color: \PYGZsh{}FF0000; border: 5px; color: white; padding: 16px 40px; border\PYGZhy{}radius: 20px;}\PYG{l+s}{\PYGZdq{}}\PYG{p}{)}\PYG{p}{;}
\PYG{+w}{            }\PYG{n}{client}\PYG{p}{.}\PYG{n+nf}{println}\PYG{p}{(}\PYG{l+s}{\PYGZdq{}}\PYG{l+s}{text\PYGZhy{}decoration: none; font\PYGZhy{}size: 30px; margin: 2px; cursor: pointer;\PYGZcb{}}\PYG{l+s}{\PYGZdq{}}\PYG{p}{)}\PYG{p}{;}
\PYG{+w}{            }\PYG{n}{client}\PYG{p}{.}\PYG{n+nf}{println}\PYG{p}{(}\PYG{l+s}{\PYGZdq{}}\PYG{l+s}{.off \PYGZob{}background\PYGZhy{}color: \PYGZsh{}000000;border: 5px; color: white; padding: 16px 40px; border\PYGZhy{}radius: 20px;}\PYG{l+s}{\PYGZdq{}}\PYG{p}{)}\PYG{p}{;}
\PYG{+w}{            }\PYG{n}{client}\PYG{p}{.}\PYG{n+nf}{println}\PYG{p}{(}\PYG{l+s}{\PYGZdq{}}\PYG{l+s}{text\PYGZhy{}decoration: none; font\PYGZhy{}size: 30px; margin: 2px; cursor: pointer;\PYGZcb{}\PYGZlt{}/style\PYGZgt{}\PYGZlt{}/head\PYGZgt{}}\PYG{l+s}{\PYGZdq{}}\PYG{p}{)}\PYG{p}{;}
\PYG{+w}{            }\PYG{n}{client}\PYG{p}{.}\PYG{n+nf}{println}\PYG{p}{(}\PYG{l+s}{\PYGZdq{}}\PYG{l+s}{\PYGZlt{}body\PYGZgt{}\PYGZlt{}h1\PYGZgt{}Web Relay Control\PYGZlt{}/h1\PYGZgt{}}\PYG{l+s}{\PYGZdq{}}\PYG{p}{)}\PYG{p}{;}
\PYG{+w}{            }\PYG{n}{client}\PYG{p}{.}\PYG{n+nf}{println}\PYG{p}{(}\PYG{l+s}{\PYGZdq{}}\PYG{l+s}{\PYGZlt{}p\PYGZgt{}Relay1 }\PYG{l+s}{\PYGZdq{}}\PYG{+w}{ }\PYG{o}{+}\PYG{+w}{ }\PYG{n}{output1}\PYG{+w}{ }\PYG{o}{+}\PYG{+w}{ }\PYG{l+s}{\PYGZdq{}}\PYG{l+s}{\PYGZlt{}/p\PYGZgt{}}\PYG{l+s}{\PYGZdq{}}\PYG{p}{)}\PYG{p}{;}
\PYG{+w}{            }\PYG{k}{if}\PYG{+w}{ }\PYG{p}{(}\PYG{n}{output1}\PYG{+w}{ }\PYG{o}{=}\PYG{o}{=}\PYG{+w}{ }\PYG{l+s}{\PYGZdq{}}\PYG{l+s}{off}\PYG{l+s}{\PYGZdq{}}\PYG{p}{)}\PYG{+w}{ }\PYG{p}{\PYGZob{}}
\PYG{+w}{              }\PYG{n}{client}\PYG{p}{.}\PYG{n+nf}{println}\PYG{p}{(}\PYG{l+s}{\PYGZdq{}}\PYG{l+s}{\PYGZlt{}p\PYGZgt{}\PYGZlt{}a href=}\PYG{l+s+se}{\PYGZbs{}\PYGZdq{}}\PYG{l+s}{/ledon1}\PYG{l+s+se}{\PYGZbs{}\PYGZdq{}}\PYG{l+s}{\PYGZgt{}\PYGZlt{}button class=}\PYG{l+s+se}{\PYGZbs{}\PYGZdq{}}\PYG{l+s}{off}\PYG{l+s+se}{\PYGZbs{}\PYGZdq{}}\PYG{l+s}{\PYGZgt{}Turn ON\PYGZlt{}/button\PYGZgt{}\PYGZlt{}/a\PYGZgt{}\PYGZlt{}/p\PYGZgt{}}\PYG{l+s}{\PYGZdq{}}\PYG{p}{)}\PYG{p}{;}
\PYG{+w}{            }\PYG{p}{\PYGZcb{}}\PYG{+w}{ }\PYG{k}{else}\PYG{+w}{ }\PYG{p}{\PYGZob{}}
\PYG{+w}{              }\PYG{n}{client}\PYG{p}{.}\PYG{n+nf}{println}\PYG{p}{(}\PYG{l+s}{\PYGZdq{}}\PYG{l+s}{\PYGZlt{}p\PYGZgt{}\PYGZlt{}a href=}\PYG{l+s+se}{\PYGZbs{}\PYGZdq{}}\PYG{l+s}{/ledoff1}\PYG{l+s+se}{\PYGZbs{}\PYGZdq{}}\PYG{l+s}{\PYGZgt{}\PYGZlt{}button class=}\PYG{l+s+se}{\PYGZbs{}\PYGZdq{}}\PYG{l+s}{on}\PYG{l+s+se}{\PYGZbs{}\PYGZdq{}}\PYG{l+s}{\PYGZgt{}Turn OFF\PYGZlt{}/button\PYGZgt{}\PYGZlt{}/a\PYGZgt{}\PYGZlt{}/p\PYGZgt{}}\PYG{l+s}{\PYGZdq{}}\PYG{p}{)}\PYG{p}{;}
\PYG{+w}{            }\PYG{p}{\PYGZcb{}}
\PYG{+w}{            }\PYG{n}{client}\PYG{p}{.}\PYG{n+nf}{println}\PYG{p}{(}\PYG{l+s}{\PYGZdq{}}\PYG{l+s}{\PYGZlt{}p\PYGZgt{}Relay2 }\PYG{l+s}{\PYGZdq{}}\PYG{+w}{ }\PYG{o}{+}\PYG{+w}{ }\PYG{n}{output2}\PYG{+w}{ }\PYG{o}{+}\PYG{+w}{ }\PYG{l+s}{\PYGZdq{}}\PYG{l+s}{\PYGZlt{}/p\PYGZgt{}}\PYG{l+s}{\PYGZdq{}}\PYG{p}{)}\PYG{p}{;}
\PYG{+w}{            }\PYG{k}{if}\PYG{+w}{ }\PYG{p}{(}\PYG{n}{output2}\PYG{+w}{ }\PYG{o}{=}\PYG{o}{=}\PYG{+w}{ }\PYG{l+s}{\PYGZdq{}}\PYG{l+s}{off}\PYG{l+s}{\PYGZdq{}}\PYG{p}{)}\PYG{+w}{ }\PYG{p}{\PYGZob{}}
\PYG{+w}{              }\PYG{n}{client}\PYG{p}{.}\PYG{n+nf}{println}\PYG{p}{(}\PYG{l+s}{\PYGZdq{}}\PYG{l+s}{\PYGZlt{}p\PYGZgt{}\PYGZlt{}a href=}\PYG{l+s+se}{\PYGZbs{}\PYGZdq{}}\PYG{l+s}{/ledon2}\PYG{l+s+se}{\PYGZbs{}\PYGZdq{}}\PYG{l+s}{\PYGZgt{}\PYGZlt{}button class=}\PYG{l+s+se}{\PYGZbs{}\PYGZdq{}}\PYG{l+s}{off}\PYG{l+s+se}{\PYGZbs{}\PYGZdq{}}\PYG{l+s}{\PYGZgt{}Turn ON\PYGZlt{}/button\PYGZgt{}\PYGZlt{}/a\PYGZgt{}\PYGZlt{}/p\PYGZgt{}}\PYG{l+s}{\PYGZdq{}}\PYG{p}{)}\PYG{p}{;}
\PYG{+w}{            }\PYG{p}{\PYGZcb{}}\PYG{+w}{ }\PYG{k}{else}\PYG{+w}{ }\PYG{p}{\PYGZob{}}
\PYG{+w}{              }\PYG{n}{client}\PYG{p}{.}\PYG{n+nf}{println}\PYG{p}{(}\PYG{l+s}{\PYGZdq{}}\PYG{l+s}{\PYGZlt{}p\PYGZgt{}\PYGZlt{}a href=}\PYG{l+s+se}{\PYGZbs{}\PYGZdq{}}\PYG{l+s}{/ledoff2}\PYG{l+s+se}{\PYGZbs{}\PYGZdq{}}\PYG{l+s}{\PYGZgt{}\PYGZlt{}button class=}\PYG{l+s+se}{\PYGZbs{}\PYGZdq{}}\PYG{l+s}{on}\PYG{l+s+se}{\PYGZbs{}\PYGZdq{}}\PYG{l+s}{\PYGZgt{}Turn OFF\PYGZlt{}/button\PYGZgt{}\PYGZlt{}/a\PYGZgt{}\PYGZlt{}/p\PYGZgt{}}\PYG{l+s}{\PYGZdq{}}\PYG{p}{)}\PYG{p}{;}
\PYG{+w}{            }\PYG{p}{\PYGZcb{}}
\PYG{+w}{            }\PYG{n}{client}\PYG{p}{.}\PYG{n+nf}{println}\PYG{p}{(}\PYG{l+s}{\PYGZdq{}}\PYG{l+s}{\PYGZlt{}/body\PYGZgt{}\PYGZlt{}/html\PYGZgt{}}\PYG{l+s}{\PYGZdq{}}\PYG{p}{)}\PYG{p}{;}
\PYG{+w}{            }\PYG{n}{client}\PYG{p}{.}\PYG{n+nf}{println}\PYG{p}{(}\PYG{p}{)}\PYG{p}{;}
\PYG{+w}{            }\PYG{k}{break}\PYG{p}{;}
\PYG{+w}{          }\PYG{p}{\PYGZcb{}}\PYG{+w}{ }\PYG{k}{else}\PYG{+w}{ }\PYG{p}{\PYGZob{}}
\PYG{+w}{            }\PYG{n}{currentLine}\PYG{+w}{ }\PYG{o}{=}\PYG{+w}{ }\PYG{l+s}{\PYGZdq{}}\PYG{l+s}{\PYGZdq{}}\PYG{p}{;}
\PYG{+w}{          }\PYG{p}{\PYGZcb{}}
\PYG{+w}{        }\PYG{p}{\PYGZcb{}}\PYG{+w}{ }\PYG{k}{else}\PYG{+w}{ }\PYG{k}{if}\PYG{+w}{ }\PYG{p}{(}\PYG{n}{c}\PYG{+w}{ }\PYG{o}{!}\PYG{o}{=}\PYG{+w}{ }\PYG{l+s+sc}{\PYGZsq{}}\PYG{l+s+sc}{\PYGZbs{}r}\PYG{l+s+sc}{\PYGZsq{}}\PYG{p}{)}\PYG{+w}{ }\PYG{p}{\PYGZob{}}
\PYG{+w}{          }\PYG{n}{currentLine}\PYG{+w}{ }\PYG{o}{+}\PYG{o}{=}\PYG{+w}{ }\PYG{n}{c}\PYG{p}{;}
\PYG{+w}{        }\PYG{p}{\PYGZcb{}}
\PYG{+w}{      }\PYG{p}{\PYGZcb{}}
\PYG{+w}{    }\PYG{p}{\PYGZcb{}}
\PYG{+w}{    }\PYG{n}{header}\PYG{+w}{ }\PYG{o}{=}\PYG{+w}{ }\PYG{l+s}{\PYGZdq{}}\PYG{l+s}{\PYGZdq{}}\PYG{p}{;}
\PYG{+w}{    }\PYG{n}{client}\PYG{p}{.}\PYG{n+nf}{stop}\PYG{p}{(}\PYG{p}{)}\PYG{p}{;}
\PYG{+w}{    }\PYG{n+nf}{Serial}\PYG{p}{.}\PYG{n+nf}{println}\PYG{p}{(}\PYG{l+s}{\PYGZdq{}}\PYG{l+s}{Client disconnected.}\PYG{l+s}{\PYGZdq{}}\PYG{p}{)}\PYG{p}{;}
\PYG{+w}{    }\PYG{n+nf}{Serial}\PYG{p}{.}\PYG{n+nf}{println}\PYG{p}{(}\PYG{l+s}{\PYGZdq{}}\PYG{l+s}{\PYGZdq{}}\PYG{p}{)}\PYG{p}{;}
\PYG{+w}{  }\PYG{p}{\PYGZcb{}}
\PYG{p}{\PYGZcb{}}
\end{sphinxVerbatim}

\item {} 
\sphinxAtStartPar
\sphinxcode{\sphinxupquote{printWifiStatus()}} Function

\sphinxAtStartPar
Print the current WiFi status, including SSID, IP address, and signal strength.

\begin{sphinxVerbatim}[commandchars=\\\{\}]
\PYG{k+kr}{void}\PYG{+w}{ }\PYG{n+nf}{printWifiStatus}\PYG{p}{(}\PYG{p}{)}\PYG{+w}{ }\PYG{p}{\PYGZob{}}
\PYG{+w}{  }\PYG{n+nf}{Serial}\PYG{p}{.}\PYG{n+nf}{print}\PYG{p}{(}\PYG{l+s}{\PYGZdq{}}\PYG{l+s}{SSID: }\PYG{l+s}{\PYGZdq{}}\PYG{p}{)}\PYG{p}{;}
\PYG{+w}{  }\PYG{n+nf}{Serial}\PYG{p}{.}\PYG{n+nf}{println}\PYG{p}{(}\PYG{n+nf}{WiFi}\PYG{p}{.}\PYG{n+nf}{SSID}\PYG{p}{(}\PYG{p}{)}\PYG{p}{)}\PYG{p}{;}
\PYG{+w}{  }\PYG{n+nf}{IPAddress}\PYG{+w}{ }\PYG{n}{ip}\PYG{+w}{ }\PYG{o}{=}\PYG{+w}{ }\PYG{n+nf}{WiFi}\PYG{p}{.}\PYG{n+nf}{localIP}\PYG{p}{(}\PYG{p}{)}\PYG{p}{;}
\PYG{+w}{  }\PYG{n+nf}{Serial}\PYG{p}{.}\PYG{n+nf}{print}\PYG{p}{(}\PYG{l+s}{\PYGZdq{}}\PYG{l+s}{IP Address: }\PYG{l+s}{\PYGZdq{}}\PYG{p}{)}\PYG{p}{;}
\PYG{+w}{  }\PYG{n+nf}{Serial}\PYG{p}{.}\PYG{n+nf}{println}\PYG{p}{(}\PYG{n}{ip}\PYG{p}{)}\PYG{p}{;}
\PYG{+w}{  }\PYG{k+kr}{long}\PYG{+w}{ }\PYG{n}{rssi}\PYG{+w}{ }\PYG{o}{=}\PYG{+w}{ }\PYG{n+nf}{WiFi}\PYG{p}{.}\PYG{n+nf}{RSSI}\PYG{p}{(}\PYG{p}{)}\PYG{p}{;}
\PYG{+w}{  }\PYG{n+nf}{Serial}\PYG{p}{.}\PYG{n+nf}{print}\PYG{p}{(}\PYG{l+s}{\PYGZdq{}}\PYG{l+s}{signal strength (RSSI):}\PYG{l+s}{\PYGZdq{}}\PYG{p}{)}\PYG{p}{;}
\PYG{+w}{  }\PYG{n+nf}{Serial}\PYG{p}{.}\PYG{n+nf}{print}\PYG{p}{(}\PYG{n}{rssi}\PYG{p}{)}\PYG{p}{;}
\PYG{+w}{  }\PYG{n+nf}{Serial}\PYG{p}{.}\PYG{n+nf}{println}\PYG{p}{(}\PYG{l+s}{\PYGZdq{}}\PYG{l+s}{ dBm}\PYG{l+s}{\PYGZdq{}}\PYG{p}{)}\PYG{p}{;}
\PYG{+w}{  }\PYG{n+nf}{Serial}\PYG{p}{.}\PYG{n+nf}{print}\PYG{p}{(}\PYG{l+s}{\PYGZdq{}}\PYG{l+s}{Now open this URL on your browser \PYGZhy{}\PYGZhy{}\PYGZgt{} http://}\PYG{l+s}{\PYGZdq{}}\PYG{p}{)}\PYG{p}{;}
\PYG{+w}{  }\PYG{n+nf}{Serial}\PYG{p}{.}\PYG{n+nf}{println}\PYG{p}{(}\PYG{n}{ip}\PYG{p}{)}\PYG{p}{;}
\PYG{p}{\PYGZcb{}}
\end{sphinxVerbatim}

\end{enumerate}

\sphinxstepscope


\section{Digital Dice LED Matrix}
\label{\detokenize{Extension_Project/Digital_Dice_LED_Matrix:digital-dice-led-matrix}}\label{\detokenize{Extension_Project/Digital_Dice_LED_Matrix:ext-digital-dice-led-matrix}}\label{\detokenize{Extension_Project/Digital_Dice_LED_Matrix::doc}}
\sphinxAtStartPar
This project is designed to simulate dice rolling using LED Matrix. The dice rolling simulation can be activated by directly shaking the tilt switch. When this is done, the LED Matrix cycles through random numbers between 1 and 6, simulating the roll of a die. After a short interval, the LED Matrix stops and displays a random number representing the result of the dice roll.


\subsection{Wiring}
\label{\detokenize{Extension_Project/Digital_Dice_LED_Matrix:wiring}}
\noindent{\hspace*{\fill}\sphinxincludegraphics[width=0.800\linewidth]{{Digital_Dice_LED_Matrix_Wiring}.png}\hspace*{\fill}}




\subsection{Code}
\label{\detokenize{Extension_Project/Digital_Dice_LED_Matrix:code}}
\begin{sphinxadmonition}{note}{Note:}\begin{itemize}
\item {} 
\sphinxAtStartPar
You can open the file \sphinxcode{\sphinxupquote{19\_Digital\_Dice\_LED\_Matrix.ino}} under the path of \sphinxcode{\sphinxupquote{Basic\sphinxhyphen{}Starter\sphinxhyphen{}Kit\sphinxhyphen{}for\sphinxhyphen{}Arduino\sphinxhyphen{}Uno\sphinxhyphen{}R4\sphinxhyphen{}WiFi\sphinxhyphen{}main\textbackslash{}Code}} directly.

\end{itemize}
\end{sphinxadmonition}


\subsection{Code explanation}
\label{\detokenize{Extension_Project/Digital_Dice_LED_Matrix:code-explanation}}\begin{enumerate}
\sphinxsetlistlabels{\arabic}{enumi}{enumii}{}{.}%
\item {} 
\sphinxAtStartPar
Importing Required Library

\sphinxAtStartPar
Import the \sphinxcode{\sphinxupquote{Arduino\_LED\_Matrix}} library to control the LED matrix.

\begin{sphinxVerbatim}[commandchars=\\\{\}]
\PYG{c+cp}{\PYGZsh{}}\PYG{c+cp}{include}\PYG{+w}{ }\PYG{c+cpf}{\PYGZlt{}Arduino\PYGZus{}LED\PYGZus{}Matrix.h\PYGZgt{}}
\end{sphinxVerbatim}

\item {} 
\sphinxAtStartPar
Variable Declarations

\sphinxAtStartPar
Declare an instance of \sphinxcode{\sphinxupquote{ArduinoLEDMatrix}} and variables for tilt switch and rolling state.

\begin{sphinxVerbatim}[commandchars=\\\{\}]
\PYG{n}{ArduinoLEDMatrix}\PYG{+w}{ }\PYG{n}{matrix}\PYG{p}{;}
\PYG{k+kr}{const}\PYG{+w}{ }\PYG{k+kr}{int}\PYG{+w}{ }\PYG{n}{tiltPin}\PYG{+w}{ }\PYG{o}{=}\PYG{+w}{ }\PYG{l+m+mi}{2}\PYG{p}{;}
\PYG{k+kr}{volatile}\PYG{+w}{ }\PYG{k+kr}{bool}\PYG{+w}{ }\PYG{n}{rolling}\PYG{+w}{ }\PYG{o}{=}\PYG{+w}{ }\PYG{k+kr}{false}\PYG{p}{;}
\PYG{k+kr}{unsigned}\PYG{+w}{ }\PYG{k+kr}{long}\PYG{+w}{ }\PYG{n}{lastShakeTime}\PYG{+w}{ }\PYG{o}{=}\PYG{+w}{ }\PYG{l+m+mi}{0}\PYG{p}{;}
\end{sphinxVerbatim}

\item {} 
\sphinxAtStartPar
Pre\sphinxhyphen{}defined 2D Arrays

\sphinxAtStartPar
Define 2D arrays representing frames to be displayed on the LED matrix.

\begin{sphinxVerbatim}[commandchars=\\\{\}]
\PYG{k+kr}{byte}\PYG{+w}{ }\PYG{n}{frame}\PYG{p}{[}\PYG{l+m+mi}{8}\PYG{p}{]}\PYG{p}{[}\PYG{l+m+mi}{12}\PYG{p}{]}\PYG{+w}{ }\PYG{o}{=}\PYG{+w}{ }\PYG{p}{\PYGZob{}}\PYG{+w}{ }\PYG{p}{.}\PYG{p}{.}\PYG{p}{.}\PYG{+w}{ }\PYG{p}{\PYGZcb{}}\PYG{p}{;}
\PYG{k+kr}{byte}\PYG{+w}{ }\PYG{n}{one}\PYG{p}{[}\PYG{l+m+mi}{8}\PYG{p}{]}\PYG{p}{[}\PYG{l+m+mi}{12}\PYG{p}{]}\PYG{+w}{ }\PYG{o}{=}\PYG{+w}{ }\PYG{p}{\PYGZob{}}\PYG{+w}{ }\PYG{p}{.}\PYG{p}{.}\PYG{p}{.}\PYG{+w}{ }\PYG{p}{\PYGZcb{}}\PYG{p}{;}
\PYG{k+kr}{byte}\PYG{+w}{ }\PYG{n}{two}\PYG{p}{[}\PYG{l+m+mi}{8}\PYG{p}{]}\PYG{p}{[}\PYG{l+m+mi}{12}\PYG{p}{]}\PYG{+w}{ }\PYG{o}{=}\PYG{+w}{ }\PYG{p}{\PYGZob{}}\PYG{+w}{ }\PYG{p}{.}\PYG{p}{.}\PYG{p}{.}\PYG{+w}{ }\PYG{p}{\PYGZcb{}}\PYG{p}{;}
\PYG{k+kr}{byte}\PYG{+w}{ }\PYG{n}{three}\PYG{p}{[}\PYG{l+m+mi}{8}\PYG{p}{]}\PYG{p}{[}\PYG{l+m+mi}{12}\PYG{p}{]}\PYG{+w}{ }\PYG{o}{=}\PYG{+w}{ }\PYG{p}{\PYGZob{}}\PYG{+w}{ }\PYG{p}{.}\PYG{p}{.}\PYG{p}{.}\PYG{+w}{ }\PYG{p}{\PYGZcb{}}\PYG{p}{;}
\PYG{k+kr}{byte}\PYG{+w}{ }\PYG{n}{four}\PYG{p}{[}\PYG{l+m+mi}{8}\PYG{p}{]}\PYG{p}{[}\PYG{l+m+mi}{12}\PYG{p}{]}\PYG{+w}{ }\PYG{o}{=}\PYG{+w}{ }\PYG{p}{\PYGZob{}}\PYG{+w}{ }\PYG{p}{.}\PYG{p}{.}\PYG{p}{.}\PYG{+w}{ }\PYG{p}{\PYGZcb{}}\PYG{p}{;}
\PYG{k+kr}{byte}\PYG{+w}{ }\PYG{n}{five}\PYG{p}{[}\PYG{l+m+mi}{8}\PYG{p}{]}\PYG{p}{[}\PYG{l+m+mi}{12}\PYG{p}{]}\PYG{+w}{ }\PYG{o}{=}\PYG{+w}{ }\PYG{p}{\PYGZob{}}\PYG{+w}{ }\PYG{p}{.}\PYG{p}{.}\PYG{p}{.}\PYG{+w}{ }\PYG{p}{\PYGZcb{}}\PYG{p}{;}
\PYG{k+kr}{byte}\PYG{+w}{ }\PYG{n}{six}\PYG{p}{[}\PYG{l+m+mi}{8}\PYG{p}{]}\PYG{p}{[}\PYG{l+m+mi}{12}\PYG{p}{]}\PYG{+w}{ }\PYG{o}{=}\PYG{+w}{ }\PYG{p}{\PYGZob{}}\PYG{+w}{ }\PYG{p}{.}\PYG{p}{.}\PYG{p}{.}\PYG{+w}{ }\PYG{p}{\PYGZcb{}}\PYG{p}{;}
\end{sphinxVerbatim}

\item {} 
\sphinxAtStartPar
Pre\sphinxhyphen{}defined Bitmap

\sphinxAtStartPar
Define a pre\sphinxhyphen{}loaded frame for initializing the matrix.

\begin{sphinxVerbatim}[commandchars=\\\{\}]
\PYG{k+kr}{const}\PYG{+w}{ }\PYG{k+kr}{uint32\PYGZus{}t}\PYG{+w}{ }\PYG{n}{hi}\PYG{p}{[}\PYG{p}{]}\PYG{+w}{ }\PYG{o}{=}\PYG{+w}{ }\PYG{p}{\PYGZob{}}
\PYG{+w}{  }\PYG{l+m+mh}{0xcdfcdfcc}\PYG{p}{,}
\PYG{+w}{  }\PYG{l+m+mh}{0x4fc4fc4c}\PYG{p}{,}
\PYG{+w}{  }\PYG{l+m+mh}{0xc4cdfcdf}\PYG{p}{,}
\PYG{+w}{  }\PYG{l+m+mi}{66}
\PYG{p}{\PYGZcb{}}\PYG{p}{;}
\end{sphinxVerbatim}

\item {} 
\sphinxAtStartPar
Display Number Function

\sphinxAtStartPar
Function to display a specific number on the LED matrix.

\begin{sphinxVerbatim}[commandchars=\\\{\}]
\PYG{k+kr}{void}\PYG{+w}{ }\PYG{n+nf}{display\PYGZus{}number}\PYG{p}{(}\PYG{k+kr}{byte}\PYG{+w}{ }\PYG{n}{number}\PYG{p}{)}\PYG{+w}{ }\PYG{p}{\PYGZob{}}
\PYG{+w}{  }\PYG{k}{if}\PYG{+w}{ }\PYG{p}{(}\PYG{n}{number}\PYG{+w}{ }\PYG{o}{=}\PYG{o}{=}\PYG{+w}{ }\PYG{l+m+mi}{1}\PYG{p}{)}\PYG{+w}{ }\PYG{p}{\PYGZob{}}\PYG{+w}{ }\PYG{n}{matrix}\PYG{p}{.}\PYG{n}{renderBitmap}\PYG{p}{(}\PYG{n}{one}\PYG{p}{,}\PYG{+w}{ }\PYG{l+m+mi}{8}\PYG{p}{,}\PYG{+w}{ }\PYG{l+m+mi}{12}\PYG{p}{)}\PYG{p}{;}\PYG{+w}{ }\PYG{p}{\PYGZcb{}}
\PYG{+w}{  }\PYG{k}{else}\PYG{+w}{ }\PYG{k}{if}\PYG{+w}{ }\PYG{p}{(}\PYG{n}{number}\PYG{+w}{ }\PYG{o}{=}\PYG{o}{=}\PYG{+w}{ }\PYG{l+m+mi}{2}\PYG{p}{)}\PYG{+w}{ }\PYG{p}{\PYGZob{}}\PYG{+w}{ }\PYG{n}{matrix}\PYG{p}{.}\PYG{n}{renderBitmap}\PYG{p}{(}\PYG{n}{two}\PYG{p}{,}\PYG{+w}{ }\PYG{l+m+mi}{8}\PYG{p}{,}\PYG{+w}{ }\PYG{l+m+mi}{12}\PYG{p}{)}\PYG{p}{;}\PYG{+w}{ }\PYG{p}{\PYGZcb{}}
\PYG{+w}{  }\PYG{k}{else}\PYG{+w}{ }\PYG{k}{if}\PYG{+w}{ }\PYG{p}{(}\PYG{n}{number}\PYG{+w}{ }\PYG{o}{=}\PYG{o}{=}\PYG{+w}{ }\PYG{l+m+mi}{3}\PYG{p}{)}\PYG{+w}{ }\PYG{p}{\PYGZob{}}\PYG{+w}{ }\PYG{n}{matrix}\PYG{p}{.}\PYG{n}{renderBitmap}\PYG{p}{(}\PYG{n}{three}\PYG{p}{,}\PYG{+w}{ }\PYG{l+m+mi}{8}\PYG{p}{,}\PYG{+w}{ }\PYG{l+m+mi}{12}\PYG{p}{)}\PYG{p}{;}\PYG{+w}{ }\PYG{p}{\PYGZcb{}}
\PYG{+w}{  }\PYG{k}{else}\PYG{+w}{ }\PYG{k}{if}\PYG{+w}{ }\PYG{p}{(}\PYG{n}{number}\PYG{+w}{ }\PYG{o}{=}\PYG{o}{=}\PYG{+w}{ }\PYG{l+m+mi}{4}\PYG{p}{)}\PYG{+w}{ }\PYG{p}{\PYGZob{}}\PYG{+w}{ }\PYG{n}{matrix}\PYG{p}{.}\PYG{n}{renderBitmap}\PYG{p}{(}\PYG{n}{four}\PYG{p}{,}\PYG{+w}{ }\PYG{l+m+mi}{8}\PYG{p}{,}\PYG{+w}{ }\PYG{l+m+mi}{12}\PYG{p}{)}\PYG{p}{;}\PYG{+w}{ }\PYG{p}{\PYGZcb{}}
\PYG{+w}{  }\PYG{k}{else}\PYG{+w}{ }\PYG{k}{if}\PYG{+w}{ }\PYG{p}{(}\PYG{n}{number}\PYG{+w}{ }\PYG{o}{=}\PYG{o}{=}\PYG{+w}{ }\PYG{l+m+mi}{5}\PYG{p}{)}\PYG{+w}{ }\PYG{p}{\PYGZob{}}\PYG{+w}{ }\PYG{n}{matrix}\PYG{p}{.}\PYG{n}{renderBitmap}\PYG{p}{(}\PYG{n}{five}\PYG{p}{,}\PYG{+w}{ }\PYG{l+m+mi}{8}\PYG{p}{,}\PYG{+w}{ }\PYG{l+m+mi}{12}\PYG{p}{)}\PYG{p}{;}\PYG{+w}{ }\PYG{p}{\PYGZcb{}}
\PYG{+w}{  }\PYG{k}{else}\PYG{+w}{ }\PYG{k}{if}\PYG{+w}{ }\PYG{p}{(}\PYG{n}{number}\PYG{+w}{ }\PYG{o}{=}\PYG{o}{=}\PYG{+w}{ }\PYG{l+m+mi}{6}\PYG{p}{)}\PYG{+w}{ }\PYG{p}{\PYGZob{}}\PYG{+w}{ }\PYG{n}{matrix}\PYG{p}{.}\PYG{n}{renderBitmap}\PYG{p}{(}\PYG{n}{six}\PYG{p}{,}\PYG{+w}{ }\PYG{l+m+mi}{8}\PYG{p}{,}\PYG{+w}{ }\PYG{l+m+mi}{12}\PYG{p}{)}\PYG{p}{;}\PYG{+w}{ }\PYG{p}{\PYGZcb{}}
\PYG{+w}{  }\PYG{k}{else}\PYG{+w}{ }\PYG{p}{\PYGZob{}}\PYG{+w}{ }\PYG{n}{matrix}\PYG{p}{.}\PYG{n}{renderBitmap}\PYG{p}{(}\PYG{n}{frame}\PYG{p}{,}\PYG{+w}{ }\PYG{l+m+mi}{8}\PYG{p}{,}\PYG{+w}{ }\PYG{l+m+mi}{12}\PYG{p}{)}\PYG{p}{;}\PYG{+w}{ }\PYG{p}{\PYGZcb{}}
\PYG{p}{\PYGZcb{}}
\end{sphinxVerbatim}

\item {} 
\sphinxAtStartPar
Setup Function

\sphinxAtStartPar
Initialize the LED matrix and configure the tilt switch.

\begin{sphinxVerbatim}[commandchars=\\\{\}]
\PYG{k+kr}{void}\PYG{+w}{ }\PYG{n+nb}{setup}\PYG{p}{(}\PYG{p}{)}\PYG{+w}{ }\PYG{p}{\PYGZob{}}
\PYG{+w}{  }\PYG{n}{matrix}\PYG{p}{.}\PYG{n+nf}{begin}\PYG{p}{(}\PYG{p}{)}\PYG{p}{;}
\PYG{+w}{  }\PYG{n+nf}{pinMode}\PYG{p}{(}\PYG{n}{tiltPin}\PYG{p}{,}\PYG{+w}{ }\PYG{k+kr}{INPUT\PYGZus{}PULLUP}\PYG{p}{)}\PYG{p}{;}
\PYG{+w}{  }\PYG{n+nf}{attachInterrupt}\PYG{p}{(}\PYG{n}{digitalPinToInterrupt}\PYG{p}{(}\PYG{n}{tiltPin}\PYG{p}{)}\PYG{p}{,}\PYG{+w}{ }\PYG{n}{rollDice}\PYG{p}{,}\PYG{+w}{ }\PYG{n}{CHANGE}\PYG{p}{)}\PYG{p}{;}
\PYG{+w}{  }\PYG{n}{matrix}\PYG{p}{.}\PYG{n}{loadFrame}\PYG{p}{(}\PYG{n}{hi}\PYG{p}{)}\PYG{p}{;}
\PYG{p}{\PYGZcb{}}
\end{sphinxVerbatim}

\item {} 
\sphinxAtStartPar
Main Loop

\sphinxAtStartPar
Main loop to check if the dice is rolling and display a random number.

\begin{sphinxVerbatim}[commandchars=\\\{\}]
\PYG{k+kr}{void}\PYG{+w}{ }\PYG{n+nb}{loop}\PYG{p}{(}\PYG{p}{)}\PYG{+w}{ }\PYG{p}{\PYGZob{}}
\PYG{+w}{  }\PYG{k}{if}\PYG{+w}{ }\PYG{p}{(}\PYG{n}{rolling}\PYG{p}{)}\PYG{+w}{ }\PYG{p}{\PYGZob{}}
\PYG{+w}{    }\PYG{k+kr}{byte}\PYG{+w}{ }\PYG{n}{number}\PYG{+w}{ }\PYG{o}{=}\PYG{+w}{ }\PYG{n+nf}{random}\PYG{p}{(}\PYG{l+m+mi}{1}\PYG{p}{,}\PYG{+w}{ }\PYG{l+m+mi}{7}\PYG{p}{)}\PYG{p}{;}
\PYG{+w}{    }\PYG{n}{display\PYGZus{}number}\PYG{p}{(}\PYG{n}{number}\PYG{p}{)}\PYG{p}{;}
\PYG{+w}{    }\PYG{n+nf}{delay}\PYG{p}{(}\PYG{l+m+mi}{80}\PYG{p}{)}\PYG{p}{;}
\PYG{+w}{    }\PYG{k}{if}\PYG{+w}{ }\PYG{p}{(}\PYG{p}{(}\PYG{n+nf}{millis}\PYG{p}{(}\PYG{p}{)}\PYG{+w}{ }\PYG{o}{\PYGZhy{}}\PYG{+w}{ }\PYG{n}{lastShakeTime}\PYG{p}{)}\PYG{+w}{ }\PYG{o}{\PYGZgt{}}\PYG{+w}{ }\PYG{l+m+mi}{1000}\PYG{p}{)}\PYG{+w}{ }\PYG{p}{\PYGZob{}}
\PYG{+w}{      }\PYG{n}{rolling}\PYG{+w}{ }\PYG{o}{=}\PYG{+w}{ }\PYG{k+kr}{false}\PYG{p}{;}
\PYG{+w}{    }\PYG{p}{\PYGZcb{}}
\PYG{+w}{  }\PYG{p}{\PYGZcb{}}
\PYG{p}{\PYGZcb{}}
\end{sphinxVerbatim}

\item {} 
\sphinxAtStartPar
Interrupt Handler

\sphinxAtStartPar
Interrupt handler to detect tilt and start rolling the dice.

\begin{sphinxVerbatim}[commandchars=\\\{\}]
\PYG{k+kr}{void}\PYG{+w}{ }\PYG{n+nf}{rollDice}\PYG{p}{(}\PYG{p}{)}\PYG{+w}{ }\PYG{p}{\PYGZob{}}
\PYG{+w}{  }\PYG{k}{if}\PYG{+w}{ }\PYG{p}{(}\PYG{n+nf}{digitalRead}\PYG{p}{(}\PYG{n}{tiltPin}\PYG{p}{)}\PYG{+w}{ }\PYG{o}{=}\PYG{o}{=}\PYG{+w}{ }\PYG{k+kr}{LOW}\PYG{p}{)}\PYG{+w}{ }\PYG{p}{\PYGZob{}}
\PYG{+w}{    }\PYG{n}{lastShakeTime}\PYG{+w}{ }\PYG{o}{=}\PYG{+w}{ }\PYG{n+nf}{millis}\PYG{p}{(}\PYG{p}{)}\PYG{p}{;}
\PYG{+w}{    }\PYG{n}{rolling}\PYG{+w}{ }\PYG{o}{=}\PYG{+w}{ }\PYG{k+kr}{true}\PYG{p}{;}
\PYG{+w}{  }\PYG{p}{\PYGZcb{}}
\PYG{p}{\PYGZcb{}}
\end{sphinxVerbatim}

\end{enumerate}

\sphinxstepscope


\section{Greedy Snake Game}
\label{\detokenize{Extension_Project/Greedy_Snake_Game:greedy-snake-game}}\label{\detokenize{Extension_Project/Greedy_Snake_Game:ext-greedy-snake-game}}\label{\detokenize{Extension_Project/Greedy_Snake_Game::doc}}
\sphinxAtStartPar
This example implements the classic Snake game on an 8x12 LED matrix using the R4 Wifi board.
Players control the snake’s direction using a dual\sphinxhyphen{}axis joystick.


\subsection{Wiring}
\label{\detokenize{Extension_Project/Greedy_Snake_Game:wiring}}
\noindent{\hspace*{\fill}\sphinxincludegraphics[width=0.800\linewidth]{{Greedy_Snake_Game_Wiring}.png}\hspace*{\fill}}

\sphinxAtStartPar
\sphinxstylestrong{Schematic}

\noindent{\hspace*{\fill}\sphinxincludegraphics[width=0.800\linewidth]{{Greedy_Snake_Game_Wiring1}.png}\hspace*{\fill}}


\subsection{Code}
\label{\detokenize{Extension_Project/Greedy_Snake_Game:code}}
\begin{sphinxadmonition}{note}{Note:}\begin{itemize}
\item {} 
\sphinxAtStartPar
You can open the file \sphinxcode{\sphinxupquote{20\_Greedy\_Snake\_Game.ino}} under the path of \sphinxcode{\sphinxupquote{Basic\sphinxhyphen{}Starter\sphinxhyphen{}Kit\sphinxhyphen{}for\sphinxhyphen{}Arduino\sphinxhyphen{}Uno\sphinxhyphen{}R4\sphinxhyphen{}WiFi\sphinxhyphen{}main\textbackslash{}Code}} directly.

\end{itemize}
\end{sphinxadmonition}


\subsection{How it works?}
\label{\detokenize{Extension_Project/Greedy_Snake_Game:how-it-works}}\begin{enumerate}
\sphinxsetlistlabels{\arabic}{enumi}{enumii}{}{.}%
\item {} 
\sphinxAtStartPar
Include Libraries

\sphinxAtStartPar
Include the necessary library for the LED matrix.

\begin{sphinxVerbatim}[commandchars=\\\{\}]
\PYG{c+cp}{\PYGZsh{}}\PYG{c+cp}{include}\PYG{+w}{ }\PYG{c+cpf}{\PYGZdq{}Arduino\PYGZus{}LED\PYGZus{}Matrix.h\PYGZdq{}}
\end{sphinxVerbatim}

\item {} 
\sphinxAtStartPar
Initialize Variables

\sphinxAtStartPar
Define and initialize variables for the LED matrix, snake, and food.

\begin{sphinxVerbatim}[commandchars=\\\{\}]
\PYG{n}{ArduinoLEDMatrix}\PYG{+w}{ }\PYG{n}{matrix}\PYG{p}{;}
\PYG{k+kr}{byte}\PYG{+w}{ }\PYG{n}{frame}\PYG{p}{[}\PYG{l+m+mi}{8}\PYG{p}{]}\PYG{p}{[}\PYG{l+m+mi}{12}\PYG{p}{]}\PYG{p}{;}
\PYG{k+kr}{byte}\PYG{+w}{ }\PYG{n}{flatFrame}\PYG{p}{[}\PYG{l+m+mi}{8}\PYG{+w}{ }\PYG{o}{*}\PYG{+w}{ }\PYG{l+m+mi}{12}\PYG{p}{]}\PYG{p}{;}

\PYG{k+kr}{struct}\PYG{+w}{ }\PYG{n+nc}{Point}\PYG{+w}{ }\PYG{p}{\PYGZob{}}
\PYG{+w}{  }\PYG{k+kr}{byte}\PYG{+w}{ }\PYG{n}{x}\PYG{p}{;}
\PYG{+w}{  }\PYG{k+kr}{byte}\PYG{+w}{ }\PYG{n}{y}\PYG{p}{;}
\PYG{p}{\PYGZcb{}}\PYG{p}{;}

\PYG{n}{Point}\PYG{+w}{ }\PYG{n}{snake}\PYG{p}{[}\PYG{l+m+mi}{100}\PYG{p}{]}\PYG{p}{;}
\PYG{k+kr}{int}\PYG{+w}{ }\PYG{n}{snakeLength}\PYG{+w}{ }\PYG{o}{=}\PYG{+w}{ }\PYG{l+m+mi}{3}\PYG{p}{;}
\PYG{n}{Point}\PYG{+w}{ }\PYG{n}{food}\PYG{p}{;}
\PYG{k+kr}{int}\PYG{+w}{ }\PYG{n}{direction}\PYG{+w}{ }\PYG{o}{=}\PYG{+w}{ }\PYG{l+m+mi}{0}\PYG{p}{;}
\end{sphinxVerbatim}

\item {} 
\sphinxAtStartPar
Setup Function

\sphinxAtStartPar
Initialize the joystick and LED matrix. Set initial snake position and generate food.

\begin{sphinxVerbatim}[commandchars=\\\{\}]
\PYG{k+kr}{void}\PYG{+w}{ }\PYG{n+nb}{setup}\PYG{p}{(}\PYG{p}{)}\PYG{+w}{ }\PYG{p}{\PYGZob{}}
\PYG{+w}{  }\PYG{n+nf}{pinMode}\PYG{p}{(}\PYG{n}{A0}\PYG{p}{,}\PYG{+w}{ }\PYG{k+kr}{INPUT}\PYG{p}{)}\PYG{p}{;}
\PYG{+w}{  }\PYG{n+nf}{pinMode}\PYG{p}{(}\PYG{n}{A1}\PYG{p}{,}\PYG{+w}{ }\PYG{k+kr}{INPUT}\PYG{p}{)}\PYG{p}{;}

\PYG{+w}{  }\PYG{n}{matrix}\PYG{p}{.}\PYG{n+nf}{begin}\PYG{p}{(}\PYG{p}{)}\PYG{p}{;}

\PYG{+w}{  }\PYG{n}{snake}\PYG{p}{[}\PYG{l+m+mi}{0}\PYG{p}{]}\PYG{+w}{ }\PYG{o}{=}\PYG{+w}{ }\PYG{p}{\PYGZob{}}\PYG{+w}{ }\PYG{l+m+mi}{6}\PYG{p}{,}\PYG{+w}{ }\PYG{l+m+mi}{4}\PYG{+w}{ }\PYG{p}{\PYGZcb{}}\PYG{p}{;}
\PYG{+w}{  }\PYG{n}{snake}\PYG{p}{[}\PYG{l+m+mi}{1}\PYG{p}{]}\PYG{+w}{ }\PYG{o}{=}\PYG{+w}{ }\PYG{p}{\PYGZob{}}\PYG{+w}{ }\PYG{l+m+mi}{6}\PYG{p}{,}\PYG{+w}{ }\PYG{l+m+mi}{5}\PYG{+w}{ }\PYG{p}{\PYGZcb{}}\PYG{p}{;}
\PYG{+w}{  }\PYG{n}{snake}\PYG{p}{[}\PYG{l+m+mi}{2}\PYG{p}{]}\PYG{+w}{ }\PYG{o}{=}\PYG{+w}{ }\PYG{p}{\PYGZob{}}\PYG{+w}{ }\PYG{l+m+mi}{6}\PYG{p}{,}\PYG{+w}{ }\PYG{l+m+mi}{6}\PYG{+w}{ }\PYG{p}{\PYGZcb{}}\PYG{p}{;}

\PYG{+w}{  }\PYG{n}{generateFood}\PYG{p}{(}\PYG{p}{)}\PYG{p}{;}
\PYG{p}{\PYGZcb{}}
\end{sphinxVerbatim}

\item {} 
\sphinxAtStartPar
Main Loop

\sphinxAtStartPar
Read joystick input, update snake direction, move snake, check for collisions, and update the display.

\begin{sphinxVerbatim}[commandchars=\\\{\}]
\PYG{k+kr}{void}\PYG{+w}{ }\PYG{n+nb}{loop}\PYG{p}{(}\PYG{p}{)}\PYG{+w}{ }\PYG{p}{\PYGZob{}}
\PYG{+w}{  }\PYG{k+kr}{int}\PYG{+w}{ }\PYG{n}{x}\PYG{+w}{ }\PYG{o}{=}\PYG{+w}{ }\PYG{n+nf}{analogRead}\PYG{p}{(}\PYG{n}{A0}\PYG{p}{)}\PYG{p}{;}
\PYG{+w}{  }\PYG{k+kr}{int}\PYG{+w}{ }\PYG{n}{y}\PYG{+w}{ }\PYG{o}{=}\PYG{+w}{ }\PYG{n+nf}{analogRead}\PYG{p}{(}\PYG{n}{A1}\PYG{p}{)}\PYG{p}{;}

\PYG{+w}{  }\PYG{k}{if}\PYG{+w}{ }\PYG{p}{(}\PYG{n}{x}\PYG{+w}{ }\PYG{o}{\PYGZgt{}}\PYG{+w}{ }\PYG{l+m+mi}{600}\PYG{+w}{ }\PYG{o}{\PYGZam{}}\PYG{o}{\PYGZam{}}\PYG{+w}{ }\PYG{n}{direction}\PYG{+w}{ }\PYG{o}{!}\PYG{o}{=}\PYG{+w}{ }\PYG{l+m+mi}{3}\PYG{p}{)}\PYG{+w}{ }\PYG{n}{direction}\PYG{+w}{ }\PYG{o}{=}\PYG{+w}{ }\PYG{l+m+mi}{1}\PYG{p}{;}
\PYG{+w}{  }\PYG{k}{else}\PYG{+w}{ }\PYG{k}{if}\PYG{+w}{ }\PYG{p}{(}\PYG{n}{x}\PYG{+w}{ }\PYG{o}{\PYGZlt{}}\PYG{+w}{ }\PYG{l+m+mi}{400}\PYG{+w}{ }\PYG{o}{\PYGZam{}}\PYG{o}{\PYGZam{}}\PYG{+w}{ }\PYG{n}{direction}\PYG{+w}{ }\PYG{o}{!}\PYG{o}{=}\PYG{+w}{ }\PYG{l+m+mi}{1}\PYG{p}{)}\PYG{+w}{ }\PYG{n}{direction}\PYG{+w}{ }\PYG{o}{=}\PYG{+w}{ }\PYG{l+m+mi}{3}\PYG{p}{;}
\PYG{+w}{  }\PYG{k}{else}\PYG{+w}{ }\PYG{k}{if}\PYG{+w}{ }\PYG{p}{(}\PYG{n}{y}\PYG{+w}{ }\PYG{o}{\PYGZgt{}}\PYG{+w}{ }\PYG{l+m+mi}{600}\PYG{+w}{ }\PYG{o}{\PYGZam{}}\PYG{o}{\PYGZam{}}\PYG{+w}{ }\PYG{n}{direction}\PYG{+w}{ }\PYG{o}{!}\PYG{o}{=}\PYG{+w}{ }\PYG{l+m+mi}{0}\PYG{p}{)}\PYG{+w}{ }\PYG{n}{direction}\PYG{+w}{ }\PYG{o}{=}\PYG{+w}{ }\PYG{l+m+mi}{2}\PYG{p}{;}
\PYG{+w}{  }\PYG{k}{else}\PYG{+w}{ }\PYG{k}{if}\PYG{+w}{ }\PYG{p}{(}\PYG{n}{y}\PYG{+w}{ }\PYG{o}{\PYGZlt{}}\PYG{+w}{ }\PYG{l+m+mi}{400}\PYG{+w}{ }\PYG{o}{\PYGZam{}}\PYG{o}{\PYGZam{}}\PYG{+w}{ }\PYG{n}{direction}\PYG{+w}{ }\PYG{o}{!}\PYG{o}{=}\PYG{+w}{ }\PYG{l+m+mi}{2}\PYG{p}{)}\PYG{+w}{ }\PYG{n}{direction}\PYG{+w}{ }\PYG{o}{=}\PYG{+w}{ }\PYG{l+m+mi}{0}\PYG{p}{;}

\PYG{+w}{  }\PYG{n}{moveSnake}\PYG{p}{(}\PYG{p}{)}\PYG{p}{;}

\PYG{+w}{  }\PYG{k}{if}\PYG{+w}{ }\PYG{p}{(}\PYG{n}{snake}\PYG{p}{[}\PYG{l+m+mi}{0}\PYG{p}{]}\PYG{p}{.}\PYG{n}{x}\PYG{+w}{ }\PYG{o}{=}\PYG{o}{=}\PYG{+w}{ }\PYG{n}{food}\PYG{p}{.}\PYG{n}{x}\PYG{+w}{ }\PYG{o}{\PYGZam{}}\PYG{o}{\PYGZam{}}\PYG{+w}{ }\PYG{n}{snake}\PYG{p}{[}\PYG{l+m+mi}{0}\PYG{p}{]}\PYG{p}{.}\PYG{n}{y}\PYG{+w}{ }\PYG{o}{=}\PYG{o}{=}\PYG{+w}{ }\PYG{n}{food}\PYG{p}{.}\PYG{n}{y}\PYG{p}{)}\PYG{+w}{ }\PYG{p}{\PYGZob{}}
\PYG{+w}{    }\PYG{n}{snake}\PYG{p}{[}\PYG{n}{snakeLength}\PYG{p}{]}\PYG{+w}{ }\PYG{o}{=}\PYG{+w}{ }\PYG{n}{snake}\PYG{p}{[}\PYG{n}{snakeLength}\PYG{+w}{ }\PYG{o}{\PYGZhy{}}\PYG{+w}{ }\PYG{l+m+mi}{1}\PYG{p}{]}\PYG{p}{;}
\PYG{+w}{    }\PYG{n}{snakeLength}\PYG{o}{+}\PYG{o}{+}\PYG{p}{;}
\PYG{+w}{    }\PYG{n}{generateFood}\PYG{p}{(}\PYG{p}{)}\PYG{p}{;}
\PYG{+w}{  }\PYG{p}{\PYGZcb{}}

\PYG{+w}{  }\PYG{k}{for}\PYG{+w}{ }\PYG{p}{(}\PYG{k+kr}{int}\PYG{+w}{ }\PYG{n}{i}\PYG{+w}{ }\PYG{o}{=}\PYG{+w}{ }\PYG{l+m+mi}{1}\PYG{p}{;}\PYG{+w}{ }\PYG{n}{i}\PYG{+w}{ }\PYG{o}{\PYGZlt{}}\PYG{+w}{ }\PYG{n}{snakeLength}\PYG{p}{;}\PYG{+w}{ }\PYG{n}{i}\PYG{o}{+}\PYG{o}{+}\PYG{p}{)}\PYG{+w}{ }\PYG{p}{\PYGZob{}}
\PYG{+w}{    }\PYG{k}{if}\PYG{+w}{ }\PYG{p}{(}\PYG{n}{snake}\PYG{p}{[}\PYG{l+m+mi}{0}\PYG{p}{]}\PYG{p}{.}\PYG{n}{x}\PYG{+w}{ }\PYG{o}{=}\PYG{o}{=}\PYG{+w}{ }\PYG{n}{snake}\PYG{p}{[}\PYG{n}{i}\PYG{p}{]}\PYG{p}{.}\PYG{n}{x}\PYG{+w}{ }\PYG{o}{\PYGZam{}}\PYG{o}{\PYGZam{}}\PYG{+w}{ }\PYG{n}{snake}\PYG{p}{[}\PYG{l+m+mi}{0}\PYG{p}{]}\PYG{p}{.}\PYG{n}{y}\PYG{+w}{ }\PYG{o}{=}\PYG{o}{=}\PYG{+w}{ }\PYG{n}{snake}\PYG{p}{[}\PYG{n}{i}\PYG{p}{]}\PYG{p}{.}\PYG{n}{y}\PYG{p}{)}\PYG{+w}{ }\PYG{p}{\PYGZob{}}
\PYG{+w}{      }\PYG{n}{snakeLength}\PYG{+w}{ }\PYG{o}{=}\PYG{+w}{ }\PYG{l+m+mi}{3}\PYG{p}{;}
\PYG{+w}{      }\PYG{n}{snake}\PYG{p}{[}\PYG{l+m+mi}{0}\PYG{p}{]}\PYG{+w}{ }\PYG{o}{=}\PYG{+w}{ }\PYG{p}{\PYGZob{}}\PYG{+w}{ }\PYG{l+m+mi}{6}\PYG{p}{,}\PYG{+w}{ }\PYG{l+m+mi}{4}\PYG{+w}{ }\PYG{p}{\PYGZcb{}}\PYG{p}{;}
\PYG{+w}{      }\PYG{n}{snake}\PYG{p}{[}\PYG{l+m+mi}{1}\PYG{p}{]}\PYG{+w}{ }\PYG{o}{=}\PYG{+w}{ }\PYG{p}{\PYGZob{}}\PYG{+w}{ }\PYG{l+m+mi}{6}\PYG{p}{,}\PYG{+w}{ }\PYG{l+m+mi}{5}\PYG{+w}{ }\PYG{p}{\PYGZcb{}}\PYG{p}{;}
\PYG{+w}{      }\PYG{n}{snake}\PYG{p}{[}\PYG{l+m+mi}{2}\PYG{p}{]}\PYG{+w}{ }\PYG{o}{=}\PYG{+w}{ }\PYG{p}{\PYGZob{}}\PYG{+w}{ }\PYG{l+m+mi}{6}\PYG{p}{,}\PYG{+w}{ }\PYG{l+m+mi}{6}\PYG{+w}{ }\PYG{p}{\PYGZcb{}}\PYG{p}{;}
\PYG{+w}{      }\PYG{n}{direction}\PYG{+w}{ }\PYG{o}{=}\PYG{+w}{ }\PYG{l+m+mi}{0}\PYG{p}{;}
\PYG{+w}{      }\PYG{n}{generateFood}\PYG{p}{(}\PYG{p}{)}\PYG{p}{;}
\PYG{+w}{    }\PYG{p}{\PYGZcb{}}
\PYG{+w}{  }\PYG{p}{\PYGZcb{}}

\PYG{+w}{  }\PYG{n}{drawFrame}\PYG{p}{(}\PYG{p}{)}\PYG{p}{;}
\PYG{+w}{  }\PYG{n+nf}{delay}\PYG{p}{(}\PYG{l+m+mi}{200}\PYG{p}{)}\PYG{p}{;}
\PYG{p}{\PYGZcb{}}
\end{sphinxVerbatim}

\item {} 
\sphinxAtStartPar
Move Snake

\sphinxAtStartPar
Update the snake’s position based on the direction.

\begin{sphinxVerbatim}[commandchars=\\\{\}]
\PYG{k+kr}{void}\PYG{+w}{ }\PYG{n+nf}{moveSnake}\PYG{p}{(}\PYG{p}{)}\PYG{+w}{ }\PYG{p}{\PYGZob{}}
\PYG{+w}{  }\PYG{k}{for}\PYG{+w}{ }\PYG{p}{(}\PYG{k+kr}{int}\PYG{+w}{ }\PYG{n}{i}\PYG{+w}{ }\PYG{o}{=}\PYG{+w}{ }\PYG{n}{snakeLength}\PYG{+w}{ }\PYG{o}{\PYGZhy{}}\PYG{+w}{ }\PYG{l+m+mi}{1}\PYG{p}{;}\PYG{+w}{ }\PYG{n}{i}\PYG{+w}{ }\PYG{o}{\PYGZgt{}}\PYG{+w}{ }\PYG{l+m+mi}{0}\PYG{p}{;}\PYG{+w}{ }\PYG{n}{i}\PYG{o}{\PYGZhy{}}\PYG{o}{\PYGZhy{}}\PYG{p}{)}\PYG{+w}{ }\PYG{p}{\PYGZob{}}
\PYG{+w}{    }\PYG{n}{snake}\PYG{p}{[}\PYG{n}{i}\PYG{p}{]}\PYG{+w}{ }\PYG{o}{=}\PYG{+w}{ }\PYG{n}{snake}\PYG{p}{[}\PYG{n}{i}\PYG{+w}{ }\PYG{o}{\PYGZhy{}}\PYG{+w}{ }\PYG{l+m+mi}{1}\PYG{p}{]}\PYG{p}{;}
\PYG{+w}{  }\PYG{p}{\PYGZcb{}}

\PYG{+w}{  }\PYG{k}{switch}\PYG{+w}{ }\PYG{p}{(}\PYG{n}{direction}\PYG{p}{)}\PYG{+w}{ }\PYG{p}{\PYGZob{}}
\PYG{+w}{    }\PYG{k}{case}\PYG{+w}{ }\PYG{l+m+mi}{0}\PYG{p}{:}
\PYG{+w}{      }\PYG{n}{snake}\PYG{p}{[}\PYG{l+m+mi}{0}\PYG{p}{]}\PYG{p}{.}\PYG{n}{y}\PYG{+w}{ }\PYG{o}{=}\PYG{+w}{ }\PYG{p}{(}\PYG{n}{snake}\PYG{p}{[}\PYG{l+m+mi}{0}\PYG{p}{]}\PYG{p}{.}\PYG{n}{y}\PYG{+w}{ }\PYG{o}{\PYGZhy{}}\PYG{+w}{ }\PYG{l+m+mi}{1}\PYG{+w}{ }\PYG{o}{+}\PYG{+w}{ }\PYG{l+m+mi}{8}\PYG{p}{)}\PYG{+w}{ }\PYG{o}{\PYGZpc{}}\PYG{+w}{ }\PYG{l+m+mi}{8}\PYG{p}{;}
\PYG{+w}{      }\PYG{k}{break}\PYG{p}{;}
\PYG{+w}{    }\PYG{k}{case}\PYG{+w}{ }\PYG{l+m+mi}{1}\PYG{p}{:}
\PYG{+w}{      }\PYG{n}{snake}\PYG{p}{[}\PYG{l+m+mi}{0}\PYG{p}{]}\PYG{p}{.}\PYG{n}{x}\PYG{+w}{ }\PYG{o}{=}\PYG{+w}{ }\PYG{p}{(}\PYG{n}{snake}\PYG{p}{[}\PYG{l+m+mi}{0}\PYG{p}{]}\PYG{p}{.}\PYG{n}{x}\PYG{+w}{ }\PYG{o}{+}\PYG{+w}{ }\PYG{l+m+mi}{1}\PYG{p}{)}\PYG{+w}{ }\PYG{o}{\PYGZpc{}}\PYG{+w}{ }\PYG{l+m+mi}{12}\PYG{p}{;}
\PYG{+w}{      }\PYG{k}{break}\PYG{p}{;}
\PYG{+w}{    }\PYG{k}{case}\PYG{+w}{ }\PYG{l+m+mi}{2}\PYG{p}{:}
\PYG{+w}{      }\PYG{n}{snake}\PYG{p}{[}\PYG{l+m+mi}{0}\PYG{p}{]}\PYG{p}{.}\PYG{n}{y}\PYG{+w}{ }\PYG{o}{=}\PYG{+w}{ }\PYG{p}{(}\PYG{n}{snake}\PYG{p}{[}\PYG{l+m+mi}{0}\PYG{p}{]}\PYG{p}{.}\PYG{n}{y}\PYG{+w}{ }\PYG{o}{+}\PYG{+w}{ }\PYG{l+m+mi}{1}\PYG{p}{)}\PYG{+w}{ }\PYG{o}{\PYGZpc{}}\PYG{+w}{ }\PYG{l+m+mi}{8}\PYG{p}{;}
\PYG{+w}{      }\PYG{k}{break}\PYG{p}{;}
\PYG{+w}{    }\PYG{k}{case}\PYG{+w}{ }\PYG{l+m+mi}{3}\PYG{p}{:}
\PYG{+w}{      }\PYG{n}{snake}\PYG{p}{[}\PYG{l+m+mi}{0}\PYG{p}{]}\PYG{p}{.}\PYG{n}{x}\PYG{+w}{ }\PYG{o}{=}\PYG{+w}{ }\PYG{p}{(}\PYG{n}{snake}\PYG{p}{[}\PYG{l+m+mi}{0}\PYG{p}{]}\PYG{p}{.}\PYG{n}{x}\PYG{+w}{ }\PYG{o}{\PYGZhy{}}\PYG{+w}{ }\PYG{l+m+mi}{1}\PYG{+w}{ }\PYG{o}{+}\PYG{+w}{ }\PYG{l+m+mi}{12}\PYG{p}{)}\PYG{+w}{ }\PYG{o}{\PYGZpc{}}\PYG{+w}{ }\PYG{l+m+mi}{12}\PYG{p}{;}
\PYG{+w}{      }\PYG{k}{break}\PYG{p}{;}
\PYG{+w}{  }\PYG{p}{\PYGZcb{}}
\PYG{p}{\PYGZcb{}}
\end{sphinxVerbatim}

\item {} 
\sphinxAtStartPar
Generate Food

\sphinxAtStartPar
Generate a new food position that doesn’t overlap with the snake.

\begin{sphinxVerbatim}[commandchars=\\\{\}]
\PYG{k+kr}{void}\PYG{+w}{ }\PYG{n+nf}{generateFood}\PYG{p}{(}\PYG{p}{)}\PYG{+w}{ }\PYG{p}{\PYGZob{}}
\PYG{+w}{  }\PYG{n}{Point}\PYG{+w}{ }\PYG{n}{possibleLocations}\PYG{p}{[}\PYG{l+m+mi}{8}\PYG{+w}{ }\PYG{o}{*}\PYG{+w}{ }\PYG{l+m+mi}{12}\PYG{p}{]}\PYG{p}{;}
\PYG{+w}{  }\PYG{k+kr}{int}\PYG{+w}{ }\PYG{n}{idx}\PYG{+w}{ }\PYG{o}{=}\PYG{+w}{ }\PYG{l+m+mi}{0}\PYG{p}{;}

\PYG{+w}{  }\PYG{k}{for}\PYG{+w}{ }\PYG{p}{(}\PYG{k+kr}{int}\PYG{+w}{ }\PYG{n}{y}\PYG{+w}{ }\PYG{o}{=}\PYG{+w}{ }\PYG{l+m+mi}{0}\PYG{p}{;}\PYG{+w}{ }\PYG{n}{y}\PYG{+w}{ }\PYG{o}{\PYGZlt{}}\PYG{+w}{ }\PYG{l+m+mi}{8}\PYG{p}{;}\PYG{+w}{ }\PYG{n}{y}\PYG{o}{+}\PYG{o}{+}\PYG{p}{)}\PYG{+w}{ }\PYG{p}{\PYGZob{}}
\PYG{+w}{    }\PYG{k}{for}\PYG{+w}{ }\PYG{p}{(}\PYG{k+kr}{int}\PYG{+w}{ }\PYG{n}{x}\PYG{+w}{ }\PYG{o}{=}\PYG{+w}{ }\PYG{l+m+mi}{0}\PYG{p}{;}\PYG{+w}{ }\PYG{n}{x}\PYG{+w}{ }\PYG{o}{\PYGZlt{}}\PYG{+w}{ }\PYG{l+m+mi}{12}\PYG{p}{;}\PYG{+w}{ }\PYG{n}{x}\PYG{o}{+}\PYG{o}{+}\PYG{p}{)}\PYG{+w}{ }\PYG{p}{\PYGZob{}}
\PYG{+w}{      }\PYG{k+kr}{bool}\PYG{+w}{ }\PYG{n}{overlap}\PYG{+w}{ }\PYG{o}{=}\PYG{+w}{ }\PYG{k+kr}{false}\PYG{p}{;}

\PYG{+w}{      }\PYG{k}{for}\PYG{+w}{ }\PYG{p}{(}\PYG{k+kr}{int}\PYG{+w}{ }\PYG{n}{i}\PYG{+w}{ }\PYG{o}{=}\PYG{+w}{ }\PYG{l+m+mi}{0}\PYG{p}{;}\PYG{+w}{ }\PYG{n}{i}\PYG{+w}{ }\PYG{o}{\PYGZlt{}}\PYG{+w}{ }\PYG{n}{snakeLength}\PYG{p}{;}\PYG{+w}{ }\PYG{n}{i}\PYG{o}{+}\PYG{o}{+}\PYG{p}{)}\PYG{+w}{ }\PYG{p}{\PYGZob{}}
\PYG{+w}{        }\PYG{k}{if}\PYG{+w}{ }\PYG{p}{(}\PYG{n}{snake}\PYG{p}{[}\PYG{n}{i}\PYG{p}{]}\PYG{p}{.}\PYG{n}{x}\PYG{+w}{ }\PYG{o}{=}\PYG{o}{=}\PYG{+w}{ }\PYG{n}{x}\PYG{+w}{ }\PYG{o}{\PYGZam{}}\PYG{o}{\PYGZam{}}\PYG{+w}{ }\PYG{n}{snake}\PYG{p}{[}\PYG{n}{i}\PYG{p}{]}\PYG{p}{.}\PYG{n}{y}\PYG{+w}{ }\PYG{o}{=}\PYG{o}{=}\PYG{+w}{ }\PYG{n}{y}\PYG{p}{)}\PYG{+w}{ }\PYG{p}{\PYGZob{}}
\PYG{+w}{          }\PYG{n}{overlap}\PYG{+w}{ }\PYG{o}{=}\PYG{+w}{ }\PYG{k+kr}{true}\PYG{p}{;}
\PYG{+w}{          }\PYG{k}{break}\PYG{p}{;}
\PYG{+w}{        }\PYG{p}{\PYGZcb{}}
\PYG{+w}{      }\PYG{p}{\PYGZcb{}}

\PYG{+w}{      }\PYG{k}{if}\PYG{+w}{ }\PYG{p}{(}\PYG{o}{!}\PYG{n}{overlap}\PYG{p}{)}\PYG{+w}{ }\PYG{p}{\PYGZob{}}
\PYG{+w}{        }\PYG{n}{possibleLocations}\PYG{p}{[}\PYG{n}{idx}\PYG{o}{+}\PYG{o}{+}\PYG{p}{]}\PYG{+w}{ }\PYG{o}{=}\PYG{+w}{ }\PYG{p}{\PYGZob{}}\PYG{+w}{ }\PYG{n}{x}\PYG{p}{,}\PYG{+w}{ }\PYG{n}{y}\PYG{+w}{ }\PYG{p}{\PYGZcb{}}\PYG{p}{;}
\PYG{+w}{      }\PYG{p}{\PYGZcb{}}
\PYG{+w}{    }\PYG{p}{\PYGZcb{}}
\PYG{+w}{  }\PYG{p}{\PYGZcb{}}

\PYG{+w}{  }\PYG{k+kr}{int}\PYG{+w}{ }\PYG{n}{choice}\PYG{+w}{ }\PYG{o}{=}\PYG{+w}{ }\PYG{n+nf}{random}\PYG{p}{(}\PYG{l+m+mi}{0}\PYG{p}{,}\PYG{+w}{ }\PYG{n}{idx}\PYG{p}{)}\PYG{p}{;}
\PYG{+w}{  }\PYG{n}{food}\PYG{+w}{ }\PYG{o}{=}\PYG{+w}{ }\PYG{n}{possibleLocations}\PYG{p}{[}\PYG{n}{choice}\PYG{p}{]}\PYG{p}{;}
\PYG{p}{\PYGZcb{}}
\end{sphinxVerbatim}

\item {} 
\sphinxAtStartPar
Draw Frame

\sphinxAtStartPar
Draw the current state of the snake and food on the LED matrix.

\begin{sphinxVerbatim}[commandchars=\\\{\}]
\PYG{k+kr}{void}\PYG{+w}{ }\PYG{n+nf}{drawFrame}\PYG{p}{(}\PYG{p}{)}\PYG{+w}{ }\PYG{p}{\PYGZob{}}
\PYG{+w}{  }\PYG{k}{for}\PYG{+w}{ }\PYG{p}{(}\PYG{k+kr}{int}\PYG{+w}{ }\PYG{n}{y}\PYG{+w}{ }\PYG{o}{=}\PYG{+w}{ }\PYG{l+m+mi}{0}\PYG{p}{;}\PYG{+w}{ }\PYG{n}{y}\PYG{+w}{ }\PYG{o}{\PYGZlt{}}\PYG{+w}{ }\PYG{l+m+mi}{8}\PYG{p}{;}\PYG{+w}{ }\PYG{n}{y}\PYG{o}{+}\PYG{o}{+}\PYG{p}{)}\PYG{+w}{ }\PYG{p}{\PYGZob{}}
\PYG{+w}{    }\PYG{k}{for}\PYG{+w}{ }\PYG{p}{(}\PYG{k+kr}{int}\PYG{+w}{ }\PYG{n}{x}\PYG{+w}{ }\PYG{o}{=}\PYG{+w}{ }\PYG{l+m+mi}{0}\PYG{p}{;}\PYG{+w}{ }\PYG{n}{x}\PYG{+w}{ }\PYG{o}{\PYGZlt{}}\PYG{+w}{ }\PYG{l+m+mi}{12}\PYG{p}{;}\PYG{+w}{ }\PYG{n}{x}\PYG{o}{+}\PYG{o}{+}\PYG{p}{)}\PYG{+w}{ }\PYG{p}{\PYGZob{}}
\PYG{+w}{      }\PYG{n}{frame}\PYG{p}{[}\PYG{n}{y}\PYG{p}{]}\PYG{p}{[}\PYG{n}{x}\PYG{p}{]}\PYG{+w}{ }\PYG{o}{=}\PYG{+w}{ }\PYG{l+m+mi}{0}\PYG{p}{;}
\PYG{+w}{    }\PYG{p}{\PYGZcb{}}
\PYG{+w}{  }\PYG{p}{\PYGZcb{}}

\PYG{+w}{  }\PYG{k}{for}\PYG{+w}{ }\PYG{p}{(}\PYG{k+kr}{int}\PYG{+w}{ }\PYG{n}{i}\PYG{+w}{ }\PYG{o}{=}\PYG{+w}{ }\PYG{l+m+mi}{0}\PYG{p}{;}\PYG{+w}{ }\PYG{n}{i}\PYG{+w}{ }\PYG{o}{\PYGZlt{}}\PYG{+w}{ }\PYG{n}{snakeLength}\PYG{p}{;}\PYG{+w}{ }\PYG{n}{i}\PYG{o}{+}\PYG{o}{+}\PYG{p}{)}\PYG{+w}{ }\PYG{p}{\PYGZob{}}
\PYG{+w}{    }\PYG{n}{frame}\PYG{p}{[}\PYG{n}{snake}\PYG{p}{[}\PYG{n}{i}\PYG{p}{]}\PYG{p}{.}\PYG{n}{y}\PYG{p}{]}\PYG{p}{[}\PYG{n}{snake}\PYG{p}{[}\PYG{n}{i}\PYG{p}{]}\PYG{p}{.}\PYG{n}{x}\PYG{p}{]}\PYG{+w}{ }\PYG{o}{=}\PYG{+w}{ }\PYG{l+m+mi}{1}\PYG{p}{;}
\PYG{+w}{  }\PYG{p}{\PYGZcb{}}

\PYG{+w}{  }\PYG{n}{frame}\PYG{p}{[}\PYG{n}{food}\PYG{p}{.}\PYG{n}{y}\PYG{p}{]}\PYG{p}{[}\PYG{n}{food}\PYG{p}{.}\PYG{n}{x}\PYG{p}{]}\PYG{+w}{ }\PYG{o}{=}\PYG{+w}{ }\PYG{l+m+mi}{1}\PYG{p}{;}

\PYG{+w}{  }\PYG{k+kr}{int}\PYG{+w}{ }\PYG{n}{idx}\PYG{+w}{ }\PYG{o}{=}\PYG{+w}{ }\PYG{l+m+mi}{0}\PYG{p}{;}
\PYG{+w}{  }\PYG{k}{for}\PYG{+w}{ }\PYG{p}{(}\PYG{k+kr}{int}\PYG{+w}{ }\PYG{n}{y}\PYG{+w}{ }\PYG{o}{=}\PYG{+w}{ }\PYG{l+m+mi}{0}\PYG{p}{;}\PYG{+w}{ }\PYG{n}{y}\PYG{+w}{ }\PYG{o}{\PYGZlt{}}\PYG{+w}{ }\PYG{l+m+mi}{8}\PYG{p}{;}\PYG{+w}{ }\PYG{n}{y}\PYG{o}{+}\PYG{o}{+}\PYG{p}{)}\PYG{+w}{ }\PYG{p}{\PYGZob{}}
\PYG{+w}{    }\PYG{k}{for}\PYG{+w}{ }\PYG{p}{(}\PYG{k+kr}{int}\PYG{+w}{ }\PYG{n}{x}\PYG{+w}{ }\PYG{o}{=}\PYG{+w}{ }\PYG{l+m+mi}{0}\PYG{p}{;}\PYG{+w}{ }\PYG{n}{x}\PYG{+w}{ }\PYG{o}{\PYGZlt{}}\PYG{+w}{ }\PYG{l+m+mi}{12}\PYG{p}{;}\PYG{+w}{ }\PYG{n}{x}\PYG{o}{+}\PYG{o}{+}\PYG{p}{)}\PYG{+w}{ }\PYG{p}{\PYGZob{}}
\PYG{+w}{      }\PYG{n}{flatFrame}\PYG{p}{[}\PYG{n}{idx}\PYG{o}{+}\PYG{o}{+}\PYG{p}{]}\PYG{+w}{ }\PYG{o}{=}\PYG{+w}{ }\PYG{n}{frame}\PYG{p}{[}\PYG{n}{y}\PYG{p}{]}\PYG{p}{[}\PYG{n}{x}\PYG{p}{]}\PYG{p}{;}
\PYG{+w}{    }\PYG{p}{\PYGZcb{}}
\PYG{+w}{  }\PYG{p}{\PYGZcb{}}
\PYG{+w}{  }\PYG{n}{matrix}\PYG{p}{.}\PYG{n}{loadPixels}\PYG{p}{(}\PYG{n}{flatFrame}\PYG{p}{,}\PYG{+w}{ }\PYG{l+m+mi}{8}\PYG{+w}{ }\PYG{o}{*}\PYG{+w}{ }\PYG{l+m+mi}{12}\PYG{p}{)}\PYG{p}{;}
\PYG{+w}{  }\PYG{n}{matrix}\PYG{p}{.}\PYG{n}{renderFrame}\PYG{p}{(}\PYG{l+m+mi}{0}\PYG{p}{)}\PYG{p}{;}
\PYG{p}{\PYGZcb{}}
\end{sphinxVerbatim}

\end{enumerate}

\sphinxstepscope


\section{Ping\sphinxhyphen{}Pong Game}
\label{\detokenize{Extension_Project/Ping-Pong_Game:ping-pong-game}}\label{\detokenize{Extension_Project/Ping-Pong_Game:ext-ping-pong-game}}\label{\detokenize{Extension_Project/Ping-Pong_Game::doc}}
\sphinxAtStartPar
This is a simple Pong game designed using an OLED display and an Arduino board.
In the Pong game, players compete against the computer, controlling a vertical paddle to bounce back a bouncing ball.
The goal is to prevent the ball from passing your paddle’s edge, or else the opponent scores.

\sphinxAtStartPar
The game mechanics can be divided into the following parts:
\begin{enumerate}
\sphinxsetlistlabels{\arabic}{enumi}{enumii}{}{.}%
\item {} 
\sphinxAtStartPar
Ball Movement \sphinxhyphen{} The ball moves along its current direction at a set speed. Whenever the ball collides with a paddle, its speed increases, making the game more challenging.

\item {} 
\sphinxAtStartPar
Paddle Movement \sphinxhyphen{} Used to block the ball’s movement, the paddle can move up or down. Players control their own paddle using buttons, while the computer’s paddle automatically follows the ball’s position.

\item {} 
\sphinxAtStartPar
Scoring \sphinxhyphen{} Whenever the ball goes beyond the left or right edge of the screen, the corresponding player or CPU scores.

\end{enumerate}


\subsection{Wiring}
\label{\detokenize{Extension_Project/Ping-Pong_Game:wiring}}
\begin{sphinxadmonition}{note}{Note:}
\sphinxAtStartPar
To protect the \DUrole{xref}{\DUrole{std}{\DUrole{std-ref}{cpn\_power}}}’s battery, please fully charge it before using it for the first time.
\end{sphinxadmonition}

\noindent{\hspace*{\fill}\sphinxincludegraphics[width=1.000\linewidth]{{Ping_Pong_Game_Wiring}.png}\hspace*{\fill}}

\sphinxAtStartPar
\sphinxstylestrong{Schematic}

\noindent{\hspace*{\fill}\sphinxincludegraphics[width=1.000\linewidth]{{Ping_Pong_Game_Wiring1}.png}\hspace*{\fill}}


\subsection{Code}
\label{\detokenize{Extension_Project/Ping-Pong_Game:code}}
\begin{sphinxadmonition}{note}{Note:}\begin{itemize}
\item {} 
\sphinxAtStartPar
You can open the file \sphinxcode{\sphinxupquote{21\_Ping\_Pong\_Game.ino}} under the path of \sphinxcode{\sphinxupquote{Basic\sphinxhyphen{}Starter\sphinxhyphen{}Kit\sphinxhyphen{}for\sphinxhyphen{}Arduino\sphinxhyphen{}Uno\sphinxhyphen{}R4\sphinxhyphen{}WiFi\sphinxhyphen{}main\textbackslash{}Code}} directly.

\end{itemize}
\end{sphinxadmonition}

\begin{sphinxadmonition}{note}{Note:}
\sphinxAtStartPar
To install the library, use the Arduino Library Manager and search for \sphinxstylestrong{“Adafruit SSD1306”} and \sphinxstylestrong{“Adafruit GFX”} and install them.
\end{sphinxadmonition}

\sphinxAtStartPar
\sphinxstylestrong{How it works?}


\subsection{Code Explanation}
\label{\detokenize{Extension_Project/Ping-Pong_Game:code-explanation}}\begin{enumerate}
\sphinxsetlistlabels{\arabic}{enumi}{enumii}{}{.}%
\item {} 
\sphinxAtStartPar
Importing Required Libraries

\sphinxAtStartPar
Import the \sphinxcode{\sphinxupquote{Adafruit\_GFX}}, \sphinxcode{\sphinxupquote{Adafruit\_SSD1306}}, \sphinxcode{\sphinxupquote{SPI}}, and \sphinxcode{\sphinxupquote{Wire}} libraries for handling the OLED display and communication.

\begin{sphinxVerbatim}[commandchars=\\\{\}]
\PYG{c+cp}{\PYGZsh{}}\PYG{c+cp}{include}\PYG{+w}{ }\PYG{c+cpf}{\PYGZlt{}Adafruit\PYGZus{}GFX.h\PYGZgt{}}
\PYG{c+cp}{\PYGZsh{}}\PYG{c+cp}{include}\PYG{+w}{ }\PYG{c+cpf}{\PYGZlt{}Adafruit\PYGZus{}SSD1306.h\PYGZgt{}}
\PYG{c+cp}{\PYGZsh{}}\PYG{c+cp}{include}\PYG{+w}{ }\PYG{c+cpf}{\PYGZlt{}SPI.h\PYGZgt{}}
\PYG{c+cp}{\PYGZsh{}}\PYG{c+cp}{include}\PYG{+w}{ }\PYG{c+cpf}{\PYGZlt{}Wire.h\PYGZgt{}}
\end{sphinxVerbatim}

\item {} 
\sphinxAtStartPar
Configuration and Variable Initialization

\sphinxAtStartPar
Define display dimensions, button pins, and game variables such as ball and paddle settings.

\begin{sphinxVerbatim}[commandchars=\\\{\}]
\PYG{c+cp}{\PYGZsh{}}\PYG{c+cp}{define SCREEN\PYGZus{}WIDTH 128}
\PYG{c+cp}{\PYGZsh{}}\PYG{c+cp}{define SCREEN\PYGZus{}HEIGHT 64}
\PYG{c+cp}{\PYGZsh{}}\PYG{c+cp}{define UP\PYGZus{}BUTTON 3}
\PYG{c+cp}{\PYGZsh{}}\PYG{c+cp}{define DOWN\PYGZus{}BUTTON 2}
\PYG{c+cp}{\PYGZsh{}}\PYG{c+cp}{define OLED\PYGZus{}RESET 4}
\PYG{k+kr}{const}\PYG{+w}{ }\PYG{k+kr}{unsigned}\PYG{+w}{ }\PYG{k+kr}{long}\PYG{+w}{ }\PYG{n}{BALL\PYGZus{}RATE}\PYG{+w}{ }\PYG{o}{=}\PYG{+w}{ }\PYG{l+m+mi}{80}\PYG{p}{;}
\PYG{k+kr}{const}\PYG{+w}{ }\PYG{k+kr}{unsigned}\PYG{+w}{ }\PYG{k+kr}{long}\PYG{+w}{ }\PYG{n}{PADDLE\PYGZus{}RATE}\PYG{+w}{ }\PYG{o}{=}\PYG{+w}{ }\PYG{l+m+mi}{33}\PYG{p}{;}
\PYG{k+kr}{const}\PYG{+w}{ }\PYG{k+kr}{uint8\PYGZus{}t}\PYG{+w}{ }\PYG{n}{PADDLE\PYGZus{}HEIGHT}\PYG{+w}{ }\PYG{o}{=}\PYG{+w}{ }\PYG{l+m+mi}{16}\PYG{p}{;}
\PYG{k+kr}{const}\PYG{+w}{ }\PYG{k+kr}{uint8\PYGZus{}t}\PYG{+w}{ }\PYG{n}{CPU\PYGZus{}X}\PYG{+w}{ }\PYG{o}{=}\PYG{+w}{ }\PYG{l+m+mi}{12}\PYG{p}{;}
\PYG{k+kr}{const}\PYG{+w}{ }\PYG{k+kr}{uint8\PYGZus{}t}\PYG{+w}{ }\PYG{n}{PLAYER\PYGZus{}X}\PYG{+w}{ }\PYG{o}{=}\PYG{+w}{ }\PYG{l+m+mi}{115}\PYG{p}{;}
\end{sphinxVerbatim}

\item {} 
\sphinxAtStartPar
\sphinxcode{\sphinxupquote{setup()}} Function

\sphinxAtStartPar
Initialize serial communication, random seed, display, and button pins. Set up initial ball and paddle positions.

\begin{sphinxVerbatim}[commandchars=\\\{\}]
\PYG{k+kr}{void}\PYG{+w}{ }\PYG{n+nb}{setup}\PYG{p}{(}\PYG{p}{)}\PYG{+w}{ }\PYG{p}{\PYGZob{}}
\PYG{+w}{  }\PYG{n+nf}{Serial}\PYG{p}{.}\PYG{n+nf}{begin}\PYG{p}{(}\PYG{l+m+mi}{115200}\PYG{p}{)}\PYG{p}{;}
\PYG{+w}{  }\PYG{n+nf}{randomSeed}\PYG{p}{(}\PYG{n+nf}{analogRead}\PYG{p}{(}\PYG{n}{A0}\PYG{p}{)}\PYG{p}{)}\PYG{p}{;}
\PYG{+w}{  }\PYG{n+nf}{display}\PYG{p}{.}\PYG{n+nf}{begin}\PYG{p}{(}\PYG{n}{SSD1306\PYGZus{}SWITCHCAPVCC}\PYG{p}{,}\PYG{+w}{ }\PYG{l+m+mh}{0x3C}\PYG{p}{)}\PYG{p}{;}
\PYG{+w}{  }\PYG{n+nf}{pinMode}\PYG{p}{(}\PYG{n}{UP\PYGZus{}BUTTON}\PYG{p}{,}\PYG{+w}{ }\PYG{k+kr}{INPUT}\PYG{p}{)}\PYG{p}{;}
\PYG{+w}{  }\PYG{n+nf}{pinMode}\PYG{p}{(}\PYG{n}{DOWN\PYGZus{}BUTTON}\PYG{p}{,}\PYG{+w}{ }\PYG{k+kr}{INPUT}\PYG{p}{)}\PYG{p}{;}
\PYG{+w}{  }\PYG{n}{drawCourt}\PYG{p}{(}\PYG{p}{)}\PYG{p}{;}
\PYG{+w}{  }\PYG{n}{displayScore}\PYG{p}{(}\PYG{p}{)}\PYG{p}{;}
\PYG{+w}{  }\PYG{n}{ball\PYGZus{}update}\PYG{+w}{ }\PYG{o}{=}\PYG{+w}{ }\PYG{n+nf}{millis}\PYG{p}{(}\PYG{p}{)}\PYG{p}{;}
\PYG{+w}{  }\PYG{n}{paddle\PYGZus{}update}\PYG{+w}{ }\PYG{o}{=}\PYG{+w}{ }\PYG{n}{ball\PYGZus{}update}\PYG{p}{;}
\PYG{p}{\PYGZcb{}}
\end{sphinxVerbatim}

\item {} 
\sphinxAtStartPar
\sphinxcode{\sphinxupquote{loop()}} Function

\sphinxAtStartPar
Main game loop to handle ball movement, paddle updates, and scoring.

\begin{sphinxVerbatim}[commandchars=\\\{\}]
\PYG{k+kr}{void}\PYG{+w}{ }\PYG{n+nb}{loop}\PYG{p}{(}\PYG{p}{)}\PYG{+w}{ }\PYG{p}{\PYGZob{}}
\PYG{+w}{  }\PYG{c+c1}{// Ball and paddle logic}
\PYG{+w}{  }\PYG{k}{if}\PYG{+w}{ }\PYG{p}{(}\PYG{n+nf}{millis}\PYG{p}{(}\PYG{p}{)}\PYG{+w}{ }\PYG{o}{\PYGZgt{}}\PYG{+w}{ }\PYG{n}{ball\PYGZus{}update}\PYG{p}{)}\PYG{+w}{ }\PYG{p}{\PYGZob{}}
\PYG{+w}{    }\PYG{c+c1}{// Ball movement and collision detection}
\PYG{+w}{  }\PYG{p}{\PYGZcb{}}
\PYG{+w}{  }\PYG{k}{if}\PYG{+w}{ }\PYG{p}{(}\PYG{n+nf}{millis}\PYG{p}{(}\PYG{p}{)}\PYG{+w}{ }\PYG{o}{\PYGZgt{}}\PYG{+w}{ }\PYG{n}{paddle\PYGZus{}update}\PYG{p}{)}\PYG{+w}{ }\PYG{p}{\PYGZob{}}
\PYG{+w}{    }\PYG{c+c1}{// Paddle updates}
\PYG{+w}{  }\PYG{p}{\PYGZcb{}}
\PYG{+w}{  }\PYG{n+nf}{display}\PYG{p}{.}\PYG{n+nf}{display}\PYG{p}{(}\PYG{p}{)}\PYG{p}{;}
\PYG{p}{\PYGZcb{}}
\end{sphinxVerbatim}

\item {} 
\sphinxAtStartPar
\sphinxcode{\sphinxupquote{crossesPlayerPaddle()}} and \sphinxcode{\sphinxupquote{crossesCpuPaddle()}} Functions

\sphinxAtStartPar
Check if the ball crosses the player or CPU paddle.

\begin{sphinxVerbatim}[commandchars=\\\{\}]
\PYG{k+kr}{bool}\PYG{+w}{ }\PYG{n+nf}{crossesPlayerPaddle}\PYG{p}{(}\PYG{k+kr}{uint8\PYGZus{}t}\PYG{+w}{ }\PYG{n}{old\PYGZus{}x}\PYG{p}{,}\PYG{+w}{ }\PYG{k+kr}{uint8\PYGZus{}t}\PYG{+w}{ }\PYG{n}{new\PYGZus{}x}\PYG{p}{,}\PYG{+w}{ }\PYG{k+kr}{uint8\PYGZus{}t}\PYG{+w}{ }\PYG{n}{ball\PYGZus{}y}\PYG{p}{)}\PYG{+w}{ }\PYG{p}{\PYGZob{}}
\PYG{+w}{  }\PYG{c+c1}{// Check if ball crosses player paddle}
\PYG{p}{\PYGZcb{}}

\PYG{k+kr}{bool}\PYG{+w}{ }\PYG{n+nf}{crossesCpuPaddle}\PYG{p}{(}\PYG{k+kr}{uint8\PYGZus{}t}\PYG{+w}{ }\PYG{n}{old\PYGZus{}x}\PYG{p}{,}\PYG{+w}{ }\PYG{k+kr}{uint8\PYGZus{}t}\PYG{+w}{ }\PYG{n}{new\PYGZus{}x}\PYG{p}{,}\PYG{+w}{ }\PYG{k+kr}{uint8\PYGZus{}t}\PYG{+w}{ }\PYG{n}{ball\PYGZus{}y}\PYG{p}{)}\PYG{+w}{ }\PYG{p}{\PYGZob{}}
\PYG{+w}{  }\PYG{c+c1}{// Check if ball crosses CPU paddle}
\PYG{p}{\PYGZcb{}}
\end{sphinxVerbatim}

\item {} 
\sphinxAtStartPar
\sphinxcode{\sphinxupquote{drawCourt()}} and \sphinxcode{\sphinxupquote{displayScore()}} Functions

\sphinxAtStartPar
Draw the game court and display the score.

\begin{sphinxVerbatim}[commandchars=\\\{\}]
\PYG{k+kr}{void}\PYG{+w}{ }\PYG{n+nf}{drawCourt}\PYG{p}{(}\PYG{p}{)}\PYG{+w}{ }\PYG{p}{\PYGZob{}}
\PYG{+w}{  }\PYG{n+nf}{display}\PYG{p}{.}\PYG{n}{drawRect}\PYG{p}{(}\PYG{l+m+mi}{0}\PYG{p}{,}\PYG{+w}{ }\PYG{l+m+mi}{0}\PYG{p}{,}\PYG{+w}{ }\PYG{n}{SCREEN\PYGZus{}WIDTH}\PYG{p}{,}\PYG{+w}{ }\PYG{n}{SCREEN\PYGZus{}HEIGHT}\PYG{p}{,}\PYG{+w}{ }\PYG{n}{WHITE}\PYG{p}{)}\PYG{p}{;}
\PYG{p}{\PYGZcb{}}

\PYG{k+kr}{void}\PYG{+w}{ }\PYG{n+nf}{displayScore}\PYG{p}{(}\PYG{p}{)}\PYG{+w}{ }\PYG{p}{\PYGZob{}}
\PYG{+w}{  }\PYG{c+c1}{// Update score display}
\PYG{p}{\PYGZcb{}}
\end{sphinxVerbatim}

\end{enumerate}

\sphinxstepscope


\section{Real\sphinxhyphen{}time Weather OLED}
\label{\detokenize{Extension_Project/Real-time_Weather_OLED:real-time-weather-oled}}\label{\detokenize{Extension_Project/Real-time_Weather_OLED:ext-real-time-weather-oled}}\label{\detokenize{Extension_Project/Real-time_Weather_OLED::doc}}
\sphinxAtStartPar
This sketch connects to a WiFi network, fetches weather data from OpenWeatherMap every minute, retrieves the current time from an NTP server, and displays the day, time, and weather information on an OLED screen.


\subsection{Wiring}
\label{\detokenize{Extension_Project/Real-time_Weather_OLED:wiring}}
\noindent{\hspace*{\fill}\sphinxincludegraphics[width=1.000\linewidth]{{Real_time_Weather_OLED_Wiring}.png}\hspace*{\fill}}

\sphinxAtStartPar
\sphinxstylestrong{Schematic}

\noindent{\hspace*{\fill}\sphinxincludegraphics[width=0.600\linewidth]{{Real_time_Weather_OLED_Wiring1}.png}\hspace*{\fill}}


\subsection{OpenWeather}
\label{\detokenize{Extension_Project/Real-time_Weather_OLED:openweather}}
\sphinxAtStartPar
Get OpenWeather API keys

\sphinxAtStartPar
\sphinxhref{https://openweathermap.org/}{OpenWeather} is an online service, owned by OpenWeather Ltd, that provides global weather data via API, including current weather data, forecasts, nowcasts and historical weather data for any geographical location.
\begin{enumerate}
\sphinxsetlistlabels{\arabic}{enumi}{enumii}{}{.}%
\item {} 
\sphinxAtStartPar
Visit OpenWeather to log in/create an account.

\noindent\sphinxincludegraphics{{Real_time_Weather_OLED_Code}.png}

\item {} 
\sphinxAtStartPar
Click into the API page from the navigation bar.

\noindent\sphinxincludegraphics{{Real_time_Weather_OLED_Code1}.png}

\item {} 
\sphinxAtStartPar
Find \sphinxstylestrong{Current Weather Data} and click Subscribe.

\noindent\sphinxincludegraphics{{Real_time_Weather_OLED_Code2}.png}

\item {} 
\sphinxAtStartPar
Under \sphinxstylestrong{Current weather and forecasts collection}, subscribe to the appropriate service. In our project, Free is good enough.

\noindent\sphinxincludegraphics{{Real_time_Weather_OLED_Code3}.png}

\item {} 
\sphinxAtStartPar
Copy the Key from the \sphinxstylestrong{API keys} page.

\noindent\sphinxincludegraphics{{Real_time_Weather_OLED_Code4}.png}

\item {} 
\sphinxAtStartPar
Copy it to the \sphinxcode{\sphinxupquote{arduino\_secrets.h}} API\_KEY.

\begin{sphinxVerbatim}[commandchars=\\\{\}]
\PYG{c+cp}{\PYGZsh{}}\PYG{c+cp}{define SECRET\PYGZus{}SSID \PYGZdq{}\PYGZlt{}SSID\PYGZgt{}\PYGZdq{}        }\PYG{c+c1}{// your network SSID (name)}
\PYG{c+cp}{\PYGZsh{}}\PYG{c+cp}{define SECRET\PYGZus{}PASS \PYGZdq{}\PYGZlt{}PASSWORD\PYGZgt{}\PYGZdq{}        }\PYG{c+c1}{// your network password}
\PYG{c+cp}{\PYGZsh{}}\PYG{c+cp}{define API\PYGZus{}KEY \PYGZdq{}\PYGZlt{}OpenWeather\PYGZus{}API\PYGZus{}KEY\PYGZgt{}\PYGZdq{}}
\PYG{c+cp}{\PYGZsh{}}\PYG{c+cp}{define LOCATION \PYGZdq{}\PYGZlt{}YOUR CITY\PYGZgt{}\PYGZdq{}}
\end{sphinxVerbatim}

\item {} 
\sphinxAtStartPar
Set the time zone of your location.

\sphinxAtStartPar
Take the capital of United Kingdom, London, as an example. Search “London timezone” on Google.

\noindent\sphinxincludegraphics{{Real_time_Weather_OLED_Code5}.png}

\sphinxAtStartPar
In the search results, you will see “GMT+1”, so you set the parameter of the function below to \sphinxcode{\sphinxupquote{3600 * 1}} seconds.

\begin{sphinxVerbatim}[commandchars=\\\{\}]
\PYG{n}{timeClient}\PYG{p}{.}\PYG{n}{setTimeOffset}\PYG{p}{(}\PYG{l+m+mi}{3600}\PYG{+w}{ }\PYG{o}{*}\PYG{+w}{ }\PYG{l+m+mi}{1}\PYG{p}{)}\PYG{p}{;}\PYG{+w}{  }\PYG{c+c1}{// Adjust for your time zone (this is +1 hour)}
\end{sphinxVerbatim}

\end{enumerate}


\subsection{Install the Library}
\label{\detokenize{Extension_Project/Real-time_Weather_OLED:install-the-library}}
\sphinxAtStartPar
To install the library, use the Arduino Library Manager and search for “ArduinoMqttClient”, “FastLED”, “Adafruit GFX” and “Adafruit SSD1306” and install them.

\sphinxAtStartPar
\sphinxcode{\sphinxupquote{ArduinoJson.h}}: Used for handling JSON data (data obtained from openweathermap).

\sphinxAtStartPar
\sphinxcode{\sphinxupquote{NTPClient.h}}: Used for obtaining real\sphinxhyphen{}time time.

\sphinxAtStartPar
\sphinxcode{\sphinxupquote{Adafruit\_GFX.h}}, \sphinxcode{\sphinxupquote{Adafruit\_SSD1306.h}}: Used for OLED module.


\subsection{Run the Code}
\label{\detokenize{Extension_Project/Real-time_Weather_OLED:run-the-code}}
\begin{sphinxadmonition}{note}{Note:}\begin{itemize}
\item {} 
\sphinxAtStartPar
You can open the file \sphinxcode{\sphinxupquote{22\_Real\_time\_Weather\_OLED.ino}} under the path of \sphinxcode{\sphinxupquote{Basic\sphinxhyphen{}Starter\sphinxhyphen{}Kit\sphinxhyphen{}for\sphinxhyphen{}Arduino\sphinxhyphen{}Uno\sphinxhyphen{}R4\sphinxhyphen{}WiFi\sphinxhyphen{}main\textbackslash{}Code}} directly.

\item {} 
\sphinxAtStartPar
Or copy this code into Arduino IDE.

\end{itemize}
\end{sphinxadmonition}

\begin{sphinxadmonition}{note}{Note:}
\sphinxAtStartPar
In the code, SSID and password are stored in \sphinxcode{\sphinxupquote{arduino\_secrets.h}}. Before uploading this example, you need to modify them with your own WiFi credentials. Additionally, for security purposes, ensure that this information is kept confidential when sharing or storing the code.
\end{sphinxadmonition}


\subsection{How it works?}
\label{\detokenize{Extension_Project/Real-time_Weather_OLED:how-it-works}}
\sphinxAtStartPar
The code fetches and displays weather information such as temperature, humidity, pressure, and wind details from the OpenWeatherMap API. It connects to a WiFi network, retrieves the current time from an NTP server, and regularly updates the weather data on an OLED screen.
\begin{enumerate}
\sphinxsetlistlabels{\arabic}{enumi}{enumii}{}{.}%
\item {} 
\sphinxAtStartPar
Importing Required Libraries

\sphinxAtStartPar
Import necessary libraries for WiFi, JSON handling, NTP client, and OLED display.

\begin{sphinxVerbatim}[commandchars=\\\{\}]
\PYG{c+cp}{\PYGZsh{}}\PYG{c+cp}{include}\PYG{+w}{ }\PYG{c+cpf}{\PYGZdq{}WiFiS3.h\PYGZdq{}}
\PYG{c+cp}{\PYGZsh{}}\PYG{c+cp}{include}\PYG{+w}{ }\PYG{c+cpf}{\PYGZlt{}ArduinoJson.h\PYGZgt{}}
\PYG{c+cp}{\PYGZsh{}}\PYG{c+cp}{include}\PYG{+w}{ }\PYG{c+cpf}{\PYGZlt{}NTPClient.h\PYGZgt{}}
\PYG{c+cp}{\PYGZsh{}}\PYG{c+cp}{include}\PYG{+w}{ }\PYG{c+cpf}{\PYGZlt{}WiFiUdp.h\PYGZgt{}}
\PYG{c+cp}{\PYGZsh{}}\PYG{c+cp}{include}\PYG{+w}{ }\PYG{c+cpf}{\PYGZlt{}SPI.h\PYGZgt{}}
\PYG{c+cp}{\PYGZsh{}}\PYG{c+cp}{include}\PYG{+w}{ }\PYG{c+cpf}{\PYGZlt{}Wire.h\PYGZgt{}}
\PYG{c+cp}{\PYGZsh{}}\PYG{c+cp}{include}\PYG{+w}{ }\PYG{c+cpf}{\PYGZlt{}Adafruit\PYGZus{}GFX.h\PYGZgt{}}
\PYG{c+cp}{\PYGZsh{}}\PYG{c+cp}{include}\PYG{+w}{ }\PYG{c+cpf}{\PYGZlt{}Adafruit\PYGZus{}SSD1306.h\PYGZgt{}}
\end{sphinxVerbatim}

\item {} 
\sphinxAtStartPar
Configuration and Variable Initialization

\sphinxAtStartPar
Define WiFi credentials, server information, display dimensions, and timing variables.

\begin{sphinxVerbatim}[commandchars=\\\{\}]
\PYG{c+cp}{\PYGZsh{}}\PYG{c+cp}{define SCREEN\PYGZus{}WIDTH 128}
\PYG{c+cp}{\PYGZsh{}}\PYG{c+cp}{define SCREEN\PYGZus{}HEIGHT 64}
\PYG{c+cp}{\PYGZsh{}}\PYG{c+cp}{define OLED\PYGZus{}RESET 4}
\PYG{k+kr}{char}\PYG{+w}{ }\PYG{n}{ssid}\PYG{p}{[}\PYG{p}{]}\PYG{+w}{ }\PYG{o}{=}\PYG{+w}{ }\PYG{n}{SECRET\PYGZus{}SSID}\PYG{p}{;}
\PYG{k+kr}{char}\PYG{+w}{ }\PYG{n}{pass}\PYG{p}{[}\PYG{p}{]}\PYG{+w}{ }\PYG{o}{=}\PYG{+w}{ }\PYG{n}{SECRET\PYGZus{}PASS}\PYG{p}{;}
\PYG{k+kr}{char}\PYG{+w}{ }\PYG{n}{server}\PYG{p}{[}\PYG{p}{]}\PYG{+w}{ }\PYG{o}{=}\PYG{+w}{ }\PYG{l+s}{\PYGZdq{}}\PYG{l+s}{api.openweathermap.org}\PYG{l+s}{\PYGZdq{}}\PYG{p}{;}
\PYG{k+kr}{const}\PYG{+w}{ }\PYG{k+kr}{unsigned}\PYG{+w}{ }\PYG{k+kr}{long}\PYG{+w}{ }\PYG{n}{postingInterval}\PYG{+w}{ }\PYG{o}{=}\PYG{+w}{ }\PYG{l+m+mi}{60000}\PYG{p}{;}
\end{sphinxVerbatim}

\item {} 
\sphinxAtStartPar
\sphinxcode{\sphinxupquote{setup()}} Function

\sphinxAtStartPar
Initialize serial communication, connect to WiFi, set up the OLED display, and initialize the NTP client.

\begin{sphinxVerbatim}[commandchars=\\\{\}]
\PYG{k+kr}{void}\PYG{+w}{ }\PYG{n+nb}{setup}\PYG{p}{(}\PYG{p}{)}\PYG{+w}{ }\PYG{p}{\PYGZob{}}
\PYG{+w}{  }\PYG{n+nf}{Serial}\PYG{p}{.}\PYG{n+nf}{begin}\PYG{p}{(}\PYG{l+m+mi}{9600}\PYG{p}{)}\PYG{p}{;}
\PYG{+w}{  }\PYG{k}{while}\PYG{+w}{ }\PYG{p}{(}\PYG{o}{!}\PYG{n+nf}{Serial}\PYG{p}{)}\PYG{+w}{ }\PYG{p}{\PYGZob{}}\PYG{+w}{ }\PYG{p}{;}\PYG{+w}{ }\PYG{p}{\PYGZcb{}}
\PYG{+w}{  }\PYG{k}{if}\PYG{+w}{ }\PYG{p}{(}\PYG{n+nf}{WiFi}\PYG{p}{.}\PYG{n}{status}\PYG{p}{(}\PYG{p}{)}\PYG{+w}{ }\PYG{o}{=}\PYG{o}{=}\PYG{+w}{ }\PYG{n}{WL\PYGZus{}NO\PYGZus{}MODULE}\PYG{p}{)}\PYG{+w}{ }\PYG{p}{\PYGZob{}}
\PYG{+w}{    }\PYG{n+nf}{Serial}\PYG{p}{.}\PYG{n+nf}{println}\PYG{p}{(}\PYG{l+s}{\PYGZdq{}}\PYG{l+s}{Communication with WiFi module failed!}\PYG{l+s}{\PYGZdq{}}\PYG{p}{)}\PYG{p}{;}
\PYG{+w}{    }\PYG{k}{while}\PYG{+w}{ }\PYG{p}{(}\PYG{k+kr}{true}\PYG{p}{)}\PYG{+w}{ }\PYG{p}{\PYGZob{}}\PYG{+w}{ }\PYG{p}{;}\PYG{+w}{ }\PYG{p}{\PYGZcb{}}
\PYG{+w}{  }\PYG{p}{\PYGZcb{}}
\PYG{+w}{  }\PYG{k+kr}{String}\PYG{+w}{ }\PYG{n}{fv}\PYG{+w}{ }\PYG{o}{=}\PYG{+w}{ }\PYG{n+nf}{WiFi}\PYG{p}{.}\PYG{n}{firmwareVersion}\PYG{p}{(}\PYG{p}{)}\PYG{p}{;}
\PYG{+w}{  }\PYG{k}{if}\PYG{+w}{ }\PYG{p}{(}\PYG{n}{fv}\PYG{+w}{ }\PYG{o}{\PYGZlt{}}\PYG{+w}{ }\PYG{n}{WIFI\PYGZus{}FIRMWARE\PYGZus{}LATEST\PYGZus{}VERSION}\PYG{p}{)}\PYG{+w}{ }\PYG{p}{\PYGZob{}}
\PYG{+w}{    }\PYG{n+nf}{Serial}\PYG{p}{.}\PYG{n+nf}{println}\PYG{p}{(}\PYG{l+s}{\PYGZdq{}}\PYG{l+s}{Please upgrade the firmware}\PYG{l+s}{\PYGZdq{}}\PYG{p}{)}\PYG{p}{;}
\PYG{+w}{  }\PYG{p}{\PYGZcb{}}
\PYG{+w}{  }\PYG{k}{while}\PYG{+w}{ }\PYG{p}{(}\PYG{n}{status}\PYG{+w}{ }\PYG{o}{!}\PYG{o}{=}\PYG{+w}{ }\PYG{n}{WL\PYGZus{}CONNECTED}\PYG{p}{)}\PYG{+w}{ }\PYG{p}{\PYGZob{}}
\PYG{+w}{    }\PYG{n+nf}{Serial}\PYG{p}{.}\PYG{n+nf}{print}\PYG{p}{(}\PYG{l+s}{\PYGZdq{}}\PYG{l+s}{Attempting to connect to SSID: }\PYG{l+s}{\PYGZdq{}}\PYG{p}{)}\PYG{p}{;}
\PYG{+w}{    }\PYG{n+nf}{Serial}\PYG{p}{.}\PYG{n+nf}{println}\PYG{p}{(}\PYG{n}{ssid}\PYG{p}{)}\PYG{p}{;}
\PYG{+w}{    }\PYG{n}{status}\PYG{+w}{ }\PYG{o}{=}\PYG{+w}{ }\PYG{n+nf}{WiFi}\PYG{p}{.}\PYG{n+nf}{begin}\PYG{p}{(}\PYG{n}{ssid}\PYG{p}{,}\PYG{+w}{ }\PYG{n}{pass}\PYG{p}{)}\PYG{p}{;}
\PYG{+w}{    }\PYG{n+nf}{delay}\PYG{p}{(}\PYG{l+m+mi}{5000}\PYG{p}{)}\PYG{p}{;}
\PYG{+w}{  }\PYG{p}{\PYGZcb{}}
\PYG{+w}{  }\PYG{n}{printWifiStatus}\PYG{p}{(}\PYG{p}{)}\PYG{p}{;}
\PYG{+w}{  }\PYG{n+nf}{display}\PYG{p}{.}\PYG{n+nf}{begin}\PYG{p}{(}\PYG{n}{SSD1306\PYGZus{}SWITCHCAPVCC}\PYG{p}{,}\PYG{+w}{ }\PYG{l+m+mh}{0x3C}\PYG{p}{)}\PYG{p}{;}
\PYG{+w}{  }\PYG{n+nf}{display}\PYG{p}{.}\PYG{n}{clearDisplay}\PYG{p}{(}\PYG{p}{)}\PYG{p}{;}
\PYG{+w}{  }\PYG{n}{timeClient}\PYG{p}{.}\PYG{n+nf}{begin}\PYG{p}{(}\PYG{p}{)}\PYG{p}{;}
\PYG{+w}{  }\PYG{n}{timeClient}\PYG{p}{.}\PYG{n}{setTimeOffset}\PYG{p}{(}\PYG{l+m+mi}{3600}\PYG{+w}{ }\PYG{o}{*}\PYG{+w}{ }\PYG{l+m+mi}{1}\PYG{p}{)}\PYG{p}{;}
\PYG{p}{\PYGZcb{}}
\end{sphinxVerbatim}

\item {} 
\sphinxAtStartPar
\sphinxcode{\sphinxupquote{loop()}} Function

\sphinxAtStartPar
Main loop to read weather data and update the display periodically.

\begin{sphinxVerbatim}[commandchars=\\\{\}]
\PYG{k+kr}{void}\PYG{+w}{ }\PYG{n+nb}{loop}\PYG{p}{(}\PYG{p}{)}\PYG{+w}{ }\PYG{p}{\PYGZob{}}
\PYG{+w}{  }\PYG{n}{read\PYGZus{}response}\PYG{p}{(}\PYG{p}{)}\PYG{p}{;}
\PYG{+w}{  }\PYG{n}{timeClient}\PYG{p}{.}\PYG{n}{update}\PYG{p}{(}\PYG{p}{)}\PYG{p}{;}
\PYG{+w}{  }\PYG{k}{if}\PYG{+w}{ }\PYG{p}{(}\PYG{o}{!}\PYG{n}{lastConnectionTime}\PYG{+w}{ }\PYG{o}{|}\PYG{o}{|}\PYG{+w}{ }\PYG{n+nf}{millis}\PYG{p}{(}\PYG{p}{)}\PYG{+w}{ }\PYG{o}{\PYGZhy{}}\PYG{+w}{ }\PYG{n}{lastConnectionTime}\PYG{+w}{ }\PYG{o}{\PYGZgt{}}\PYG{+w}{ }\PYG{n}{postingInterval}\PYG{p}{)}\PYG{+w}{ }\PYG{p}{\PYGZob{}}
\PYG{+w}{    }\PYG{n}{httpRequest}\PYG{p}{(}\PYG{p}{)}\PYG{p}{;}
\PYG{+w}{  }\PYG{p}{\PYGZcb{}}
\PYG{p}{\PYGZcb{}}
\end{sphinxVerbatim}

\item {} 
\sphinxAtStartPar
Helper Functions
\begin{itemize}
\item {} 
\sphinxAtStartPar
\sphinxcode{\sphinxupquote{read\_response()}}: Parse JSON response and display weather data.

\item {} 
\sphinxAtStartPar
\sphinxcode{\sphinxupquote{httpRequest()}}: Send HTTP GET request to OpenWeatherMap API.

\item {} 
\sphinxAtStartPar
\sphinxcode{\sphinxupquote{printWifiStatus()}}: Print WiFi status information.

\item {} 
\sphinxAtStartPar
\sphinxcode{\sphinxupquote{displayWeatherData()}}: Display weather data on the OLED screen.

\end{itemize}

\begin{sphinxVerbatim}[commandchars=\\\{\}]
\PYG{k+kr}{void}\PYG{+w}{ }\PYG{n+nf}{read\PYGZus{}response}\PYG{p}{(}\PYG{p}{)}\PYG{+w}{ }\PYG{p}{\PYGZob{}}
\PYG{+w}{  }\PYG{c+c1}{// JSON parsing and display logic}
\PYG{p}{\PYGZcb{}}

\PYG{k+kr}{void}\PYG{+w}{ }\PYG{n+nf}{httpRequest}\PYG{p}{(}\PYG{p}{)}\PYG{+w}{ }\PYG{p}{\PYGZob{}}
\PYG{+w}{  }\PYG{c+c1}{// HTTP request logic}
\PYG{p}{\PYGZcb{}}

\PYG{k+kr}{void}\PYG{+w}{ }\PYG{n+nf}{printWifiStatus}\PYG{p}{(}\PYG{p}{)}\PYG{+w}{ }\PYG{p}{\PYGZob{}}
\PYG{+w}{  }\PYG{c+c1}{// Print WiFi status}
\PYG{p}{\PYGZcb{}}

\PYG{k+kr}{void}\PYG{+w}{ }\PYG{n+nf}{displayWeatherData}\PYG{p}{(}\PYG{k+kr}{String}\PYG{+w}{ }\PYG{n}{weather}\PYG{p}{,}\PYG{+w}{ }\PYG{k+kr}{float}\PYG{+w}{ }\PYG{n}{temp}\PYG{p}{,}\PYG{+w}{ }\PYG{k+kr}{int}\PYG{+w}{ }\PYG{n}{humidity}\PYG{p}{,}\PYG{+w}{ }\PYG{k+kr}{float}\PYG{+w}{ }\PYG{n}{pressure}\PYG{p}{,}\PYG{+w}{ }\PYG{k+kr}{float}\PYG{+w}{ }\PYG{n}{wind\PYGZus{}speed}\PYG{p}{)}\PYG{+w}{ }\PYG{p}{\PYGZob{}}
\PYG{+w}{  }\PYG{c+c1}{// Display weather data on OLED}
\PYG{p}{\PYGZcb{}}
\end{sphinxVerbatim}

\end{enumerate}



\renewcommand{\indexname}{Index}
\printindex
\end{document}